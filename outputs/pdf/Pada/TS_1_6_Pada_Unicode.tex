\documentclass[17pt]{extarticle}
\usepackage{babel}
\usepackage{fontspec}
\usepackage{polyglossia}
\usepackage{extsizes}



\setmainlanguage{sanskrit}
\setotherlanguages{english} %% or other languages
\setlength{\parindent}{0pt}
\pagestyle{myheadings}
\newfontfamily\devanagarifont[Script=Devanagari]{AdishilaVedic}


\newcommand{\VAR}[1]{}
\newcommand{\BLOCK}[1]{}




\begin{document}
\begin{titlepage}
    \begin{center}
 
\begin{sanskrit}
    { \Large
    ॐ नमः परमात्मने, श्री महागणपतये नमः, 
श्री गुरुभ्यो नमः । ह॒रिः॒ ॐ ॥ 
    }
    \\
    \vspace{2.5cm}
    \mbox{ \Huge
    1.6     प्रथमकाण्डे षष्ठः प्रश्नः - (याजमानकाण्डं)   }
\end{sanskrit}
\end{center}

\end{titlepage}
\tableofcontents

ॐ नमः परमात्मने, श्री महागणपतये नमः, 
श्री गुरुभ्यो नमः । ह॒रिः॒ ॐ ॥ \newline
1.6     प्रथमकाण्डे षष्ठः प्रश्नः - (याजमानकाण्डं) \newline

\addcontentsline{toc}{section}{ 1.6     प्रथमकाण्डे षष्ठः प्रश्नः - (याजमानकाण्डं)}
\markright{ 1.6     प्रथमकाण्डे षष्ठः प्रश्नः - (याजमानकाण्डं) \hfill https://www.vedavms.in \hfill}
\section*{ 1.6     प्रथमकाण्डे षष्ठः प्रश्नः - (याजमानकाण्डं) }
                                \textbf{ TS 1.6.1.1} \newline
                  समिति॑ । त्वा॒ । सि॒ञ्चा॒मि॒ । यजु॑षा । प्र॒जामिति॑ प्र-जाम् । आयुः॑ । धन᳚म् । च॒ ॥ बृह॒स्पति॑ प्रसूत॒ इति॒ बृह॒स्पति॑ - प्र॒सू॒तः॒ । यज॑मानः । इ॒ह । मा । रि॒ष॒त् ॥ आज्य᳚म् । अ॒सि॒ । स॒त्यम् । अ॒सि॒ । स॒त्यस्य॑ । अद्ध्य॑क्ष॒मित्यधि॑ - अ॒क्ष॒म् । अ॒सि॒ । ह॒विः । अ॒सि॒ । वै॒श्वा॒न॒रम् । वै॒श्व॒दे॒वमिति॑ वैश्व - दे॒वम् । उत्पू॑तशुष्म॒मित्युत्पू॑त - शु॒ष्म॒म् । स॒त्यौजा॒ इति॑ स॒त्य-ओ॒जाः॒ । सहः॑ । अ॒सि॒ । सह॑मानम् । अ॒सि॒ । सह॑स्व । अरा॑तीः । सह॑स्व । अ॒रा॒ती॒य॒तः । सह॑स्व । पृत॑नाः । सह॑स्व । पृ॒त॒न्य॒तः । स॒हस्र॑वीर्य॒मिति॑ स॒हस्र॑-वि॒र्य॒म् । अ॒सि॒ । तत् । मा॒ । जि॒न्व॒ । आज्य॑स्य । आज्य᳚म् । अ॒सि॒ । स॒त्यस्य॑ । स॒त्यम् । अ॒सि॒ । स॒त्यायु॒रिति॑ स॒त्य - आ॒युः॒ । \textbf{  1} \newline
                  \newline
                                \textbf{ TS 1.6.1.2} \newline
                  अ॒सि॒ । स॒त्यशु॑ष्म॒मिति॑ स॒त्य - शु॒ष्म॒म् । अ॒सि॒ । स॒त्येन॑ । त्वा॒ । अ॒भीति॑ । घा॒र॒या॒मि॒ । तस्य॑ । ते॒ । भ॒क्षी॒य॒ । प॒ञ्चा॒नाम् । त्वा॒ । वाता॑नाम् । य॒न्त्राय॑ । ध॒र्त्राय॑ । गृ॒ह्णा॒मि॒ । प॒ञ्चा॒नाम् । त्वा॒ । ऋ॒तू॒नाम् । य॒न्त्राय॑ । ध॒र्त्राय॑ । गृ॒ह्णा॒मि॒ । प॒ञ्चा॒नाम् । त्वा॒ । दि॒शाम् । य॒न्त्राय॑ । ध॒र्त्राय॑ । गृ॒ह्णा॒मि॒ । प॒ञ्चा॒नाम् । त्वा॒ । प॒ञ्च॒ज॒नाना॒मिति॑ पञ्च - ज॒नाना᳚म् । य॒न्त्राय॑ । ध॒र्त्राय॑ । गृ॒ह्णा॒मि॒ । च॒रोः । त्वा॒ । पञ्च॑बिल॒स्येति॒ पञ्च॑ - बि॒ल॒स्य॒ । य॒न्त्राय॑ । ध॒र्त्राय॑ । गृ॒ह्णा॒मि॒ । ब्रह्म॑णः । त्वा॒ । तेज॑से । य॒न्त्राय॑ । ध॒र्त्राय॑ । गृ॒ह्णा॒मि॒ । क्ष॒त्रस्य॑ । त्वा॒ । ओज॑से । य॒न्त्राय॑ । \textbf{  2} \newline
                  \newline
                                \textbf{ TS 1.6.1.3} \newline
                  ध॒र्त्राय॑ । गृ॒ह्णा॒मि॒ । वि॒शे । त्वा॒ । य॒न्त्राय॑ । ध॒र्त्राय॑ । गृ॒ह्णा॒मि॒ । सु॒वीर्या॒येति॑ सु - वीर्या॑य । त्वा॒ । गृ॒ह्णा॒मि॒ । सु॒प्र॒जा॒स्त्वायेति॑ सुप्रजाः - त्वाय॑ । त्वा॒ । गृ॒ह्णा॒मि॒ । रा॒यः । पोषा॑य । त्वा॒ । गृ॒ह्णा॒मि॒ । ब्र॒ह्म॒व॒र्च॒सायेति॑ ब्रह्म-व॒र्च॒साय॑ । त्वा॒ । गृ॒ह्णा॒मि॒ । भूः । अ॒स्माक᳚म् । ह॒विः । दे॒वाना᳚म् । आ॒शिष॒ इत्या᳚-शिषः॑ । यज॑मानस्य । दे॒वाना᳚म् । त्वा॒ । दे॒वता᳚भ्यः । गृ॒ह्णा॒मि॒ । कामा॑य । त्वा॒ । गृ॒ह्णा॒मि॒ ॥ \textbf{  3} \newline
                  \newline
                      (स॒त्यायु॒-रोज॑से य॒न्त्राय॒-त्रय॑स्त्रिꣳशच्च)  \textbf{(A1)} \newline \newline
                                \textbf{ TS 1.6.2.1} \newline
                  ध्रु॒वः । अ॒सि॒ । ध्रु॒वः । अ॒हम् । स॒जा॒तेष्विति॑ स - जा॒तेषु॑ । भू॒या॒स॒म् । धीरः॑ । चेत्ता᳚ । व॒सु॒विदिति॑ वसु - वित् । उ॒ग्रः । अ॒सि॒ । उ॒ग्रः । अ॒हम् । स॒जा॒तेष्विति॑ स-जा॒तेषु॑ । भू॒या॒स॒म् । उ॒ग्रः । चेत्ता᳚ । व॒सु॒विदिति॑ वसु - वित् । अ॒भि॒भूरित्य॑भि - भूः । अ॒सि॒ । अ॒भि॒भूरित्य॑भि - भूः । अ॒हम् । स॒जा॒तेष्विति॑ स - जा॒तेषु॑ । भू॒या॒स॒म् । अ॒भि॒भूरित्य॑भि - भूः । चेत्ता᳚ । व॒सु॒विदिति॑ वसु-वित् । यु॒नज्मि॑ । त्वा॒ । ब्रह्म॑णा । दैव्ये॑न । ह॒व्याय॑ । अ॒स्मै । वो॒ढ॒वे । जा॒त॒वे॒द॒ इति॑ जात - वे॒दः॒ ॥ इन्धा॑नाः । त्वा॒ । सु॒प्र॒जस॒ इति॑ सु - प्र॒जसः॑ । सु॒वीरा॒ इति॑ सु - वीराः᳚ । ज्योक् । जी॒वे॒म॒ । ब॒लि॒हृत॒ इति॑ बलि - हृतः॑ । व॒यम् । ते॒ ॥ यत् । मे॒ । अ॒ग्ने॒ । अ॒स्य । य॒ज्ञ्स्य॑ । रिष्या᳚त् । \textbf{  4} \newline
                  \newline
                                \textbf{ TS 1.6.2.2} \newline
                  यत् । वा॒ । स्कन्दा᳚त् । आज्य॑स्य । उ॒त । वि॒ष्णो॒ इति॑ ॥ तेन॑ । ह॒न्मि॒ । स॒पत्न᳚म् । दु॒र्म॒रा॒युमिति॑ दुः - म॒रा॒युम् । एति॑ । ए॒न॒म् । द॒धा॒मि॒ । निर्.ऋ॑त्या॒ इति॒ निः-ऋ॒त्याः॒ । उ॒पस्थ॒ इत्यु॒प - स्थे॒ ॥ भूः । भुवः॑ । सुवः॑ । उच्छु॑ष्म॒ इत्युत् - शु॒ष्मः॒ । अ॒ग्ने॒ । यज॑मानाय । ए॒धि॒ । निशु॑ष्म॒ इति॒ नि-शु॒ष्मः॒ । अ॒भि॒दास॑त॒ इत्य॑भि - दास॑ते ॥ अग्ने᳚ । देवे॒द्धेति॒ देव॑ - इ॒द्ध॒ । मन्वि॒द्धेति॒ मनु॑ - इ॒द्ध॒ । मन्द्र॑जि॒ह्वेति॒ मन्द्र॑-जि॒ह्व॒ । अम॑र्त्यस्य । ते॒ । हो॒तः॒ । मू॒र्द्धन्न् । एति॑ । जि॒घ॒र्मि॒ । रा॒यः । पोषा॑य । सु॒प्र॒जा॒स्त्वायेति॑ सुप्रजाः - त्वाय॑ । सु॒वीर्या॒येति॑ सु - वीर्या॑य । मनः॑ । अ॒सि॒ । प्रा॒जा॒प॒त्यमिति॑ प्राजा-प॒त्यम् । मन॑सा । मा॒ । भू॒तेन॑ । एति॑ । वि॒श॒ । वाक् । अ॒सि॒ । ऐ॒न्द्री । स॒प॒त्न॒क्षय॒णीति॑ सपत्न - क्षय॑णी । \textbf{  5} \newline
                  \newline
                                \textbf{ TS 1.6.2.3} \newline
                  वा॒चा । मा॒ । इ॒न्द्रि॒येण॑ । एति॑ । वि॒श॒ । व॒स॒न्तम् । ऋ॒तू॒नाम् । प्री॒णा॒मि॒ । सः । मा॒ । प्री॒तः । प्री॒णा॒तु॒ । ग्री॒ष्मम् । ऋ॒तू॒नाम् । प्री॒णा॒मि॒ । सः । मा॒ । प्री॒तः । प्री॒णा॒तु॒ । व॒र्॒.षा ः । ऋ॒तू॒नाम् । प्री॒णा॒मि॒ । ताः । मा॒ । प्री॒ताः । प्री॒ण॒न्तु॒ । श॒रद᳚म् । ऋ॒तू॒नाम् । प्री॒णा॒मि॒ । सा । मा॒ । प्री॒ता । प्री॒णा॒तु॒ । हे॒म॒न्त॒शि॒शि॒राविति॑ हेमन्त - शि॒शि॒रौ । ऋ॒तू॒नाम् । प्री॒णा॒मि॒ । तौ । मा॒ । प्री॒तौ । प्री॒णी॒ता॒म् । अ॒ग्नीषोम॑यो॒रित्य॒ग्नी - सोम॑योः । अ॒हम् । दे॒व॒य॒ज्ययेति॑ देव- य॒ज्यया᳚ । चक्षु॑ष्मान् । भू॒या॒स॒म् । अ॒ग्नेः । अ॒हम् । दे॒व॒य॒ज्ययेति॑ देव - य॒ज्यया᳚ । अ॒न्ना॒द इत्य॑न्न - अ॒दः । भू॒या॒स॒म् । \textbf{  6} \newline
                  \newline
                                \textbf{ TS 1.6.2.4} \newline
                  दब्धिः॑ । अ॒सि॒ । अद॑ब्धः । भू॒या॒स॒म् । अ॒मुम् । द॒भे॒य॒म् । अ॒ग्नीषोम॑यो॒रित्य॒ग्नी - सोम॑योः । अ॒हम् । दे॒व॒य॒ज्ययेति॑ देव- य॒ज्यया᳚ । वृ॒त्र॒हेति॑ वृत्र - हा । भू॒या॒स॒म् । इ॒न्द्रा॒ग्नि॒योरिती᳚न्द्र - अ॒ग्नि॒योः । अ॒हम् । दे॒व॒य॒ज्ययेति॑ देव - य॒ज्यया᳚ । इ॒न्द्रि॒या॒वी । अ॒न्ना॒द इत्य॑न्न - अ॒दः । भू॒या॒स॒म् । इन्द्र॑स्य । अ॒हम् । दे॒व॒य॒ज्ययेति॑ देव - य॒ज्यया᳚ । इ॒न्द्रि॒या॒वी । भू॒या॒स॒म् । म॒हे॒न्द्रस्येति॑ महा - इ॒न्द्रस्य॑ । अ॒हम् । दे॒व॒य॒ज्ययेति॑ देव - य॒ज्यया᳚ । जे॒मान᳚म् । म॒हि॒मान᳚म् । ग॒मे॒य॒म् । अ॒ग्नेः । स्वि॒ष्ट॒कृत॒ इति॑ स्विष्ट - कृतः॑ । अ॒हम् । दे॒व॒य॒ज्ययेति॑ देव - य॒ज्यया᳚ । आयु॑ष्मान् । य॒ज्ञेन॑ । प्र॒ति॒ष्ठामिति॑ प्रति - स्थाम् । ग॒मे॒य॒म् ॥ \textbf{  7 } \newline
                  \newline
                      (रिष्या᳚थ्-सपत्न॒क्षय॑ण्य-न्ना॒दो भू॑यासꣳ॒॒-षट्त्रिꣳ॑शच्च)  \textbf{(A2)} \newline \newline
                                \textbf{ TS 1.6.3.1} \newline
                  अ॒ग्निः । मा॒ । दुरि॑ष्टा॒दिति॒ दुः - इ॒ष्टा॒त् । पा॒तु॒ । स॒वि॒ता । अ॒घशꣳ॑सा॒दित्य॒घ-शꣳ॒॒सा॒त् । यः । मे॒ । अन्ति॑ । दू॒रे । अ॒रा॒ती॒यति॑ । तम् । ए॒तेन॑ । जे॒ष॒म् । सुरू॑पवर्.षवर्ण॒ इति॒ सुरू॑प - व॒र्॒.ष॒व॒र्णे॒ । एति॑ । इ॒हि॒ । इ॒मान् । भ॒द्रान् । दुर्यान्॑ । अ॒भि । एति॑ । इ॒हि॒ । माम् । अनु॑व्र॒तेत्यनु॑-व्र॒ता॒ । नीति॑ । उ॒ । शी॒र्.॒षाणि॑ । मृ॒ढ्व॒म् । इडे᳚ । एति॑ । इ॒हि॒ । अदि॑ते । एति॑ । इ॒हि॒ । सर॑स्वति । एति॑ । इ॒हि॒ । रन्तिः॑ । अ॒सि॒ । रम॑तिः । अ॒सि॒ । सू॒नरी᳚ । अ॒सि॒ । जुष्टे᳚ । जुष्टि᳚म् । ते॒ । अ॒शी॒य॒ । उप॑हूत॒ इत्युप॑ - हू॒ते॒ । उ॒प॒ह॒वमित्यु॑प - ह॒वम् । \textbf{  8} \newline
                  \newline
                                \textbf{ TS 1.6.3.2} \newline
                  ते॒ । अ॒शी॒य॒ । सा । मे॒ । स॒त्या । आ॒शीरित्या᳚ - शीः । अ॒स्य । य॒ज्ञ्स्य॑ । भू॒या॒त् । अरे॑डता । मन॑सा । तत् । श॒के॒य॒म् । य॒ज्ञ्ः । दिव᳚म् । रो॒ह॒तु॒ । य॒ज्ञ्ः । दिव᳚म् । ग॒च्छ॒तु॒ । यः । दे॒व॒यान॒ इति॑ देव-यानः॑ । पन्थाः᳚ । तेन॑ । य॒ज्ञ्ः । दे॒वान् । अपीति॑ । ए॒तु॒ । अ॒स्मासु॑ । इन्द्रः॑ । इ॒न्द्रि॒यम् । द॒धा॒तु॒ । अ॒स्मान् । रायः॑ । उ॒त । य॒ज्ञाः । स॒च॒न्ता॒म् । अ॒स्मासु॑ । स॒न्तु॒ । आ॒शिष॒ इत्या᳚ - शिषः॑ । सा । नः॒ । प्रि॒या । सु॒प्रतू᳚र्ति॒रिति॑ सु - प्रतू᳚र्तिः । म॒घोनी᳚ । जुष्टिः॑ । अ॒सि॒ । जु॒षस्व॑ । नः॒ । जुष्टा᳚ । नः॒ । \textbf{  9} \newline
                  \newline
                                \textbf{ TS 1.6.3.3} \newline
                  अ॒सि॒ । जुष्टि᳚म् । ते॒ । ग॒मे॒य॒म् । मनः॑ । ज्योतिः॑ । जु॒ष॒ता॒म् । आज्य᳚म् । विच्छि॑न्न॒मिति॒ वि - छि॒न्न॒म् । य॒ज्ञ्म् । समिति॑ । इ॒मम् । द॒धा॒तु॒ । बृह॒स्पतिः॑ । त॒नु॒ता॒म् । इ॒मम् । नः॒ । विश्वे᳚ । दे॒वाः । इ॒ह । मा॒द॒य॒न्ता॒म् । ब्रद्ध्न॑ । पिन्व॑स्व । दद॑तः । मे॒ । मा । क्षा॒यि॒ । कु॒र्व॒तः । मे॒ । मा । उपेति॑ । द॒स॒त् । प्र॒जाप॑ते॒रिति॑ प्र॒जा - प॒तेः॒ । भा॒गः । अ॒सि॒ । ऊर्ज॑स्वान् । पय॑स्वान् । प्रा॒णा॒पा॒नाविति॑ प्राण-अ॒पा॒नौ । मे॒ । पा॒हि॒ । स॒मा॒न॒व्या॒नाविति॑ समान - व्या॒नौ । मे॒ । पा॒हि॒ । उ॒दा॒न॒व्या॒नावित्यु॑दान - व्या॒नौ । मे॒ । पा॒हि॒ । अक्षि॑तः । अ॒सि॒ । अक्षि॑त्यै । त्वा॒ ( ) । मा । मे॒ । क्षे॒ष्ठाः॒ । अ॒मुत्र॑ । अ॒मुष्मिन्न्॑ । लो॒के ॥ \textbf{  10} \newline
                  \newline
                      (उ॒प॒ह॒वं-जुष्टा॑न-स्त्वा॒ षट् च॑)  \textbf{(A3)} \newline \newline
                                \textbf{ TS 1.6.4.1} \newline
                  ब॒र्॒.हिषः॑ । अ॒हम् । दे॒व॒य॒ज्ययेति॑ देव - य॒ज्यया᳚ । प्र॒जावा॒निति॑ प्र॒जा - वा॒न् । भू॒या॒स॒म् । नरा॒शꣳस॑स्य । अ॒हम् । दे॒व॒य॒ज्ययेति॑ देव - य॒ज्यया᳚ । प॒शु॒मानिति॑ पशु - मान् । भू॒या॒स॒म् । अ॒ग्नेः । स्वि॒ष्ट॒कृत॒ इति॑ स्विष्ट - कृतः॑ । अ॒हम् । दे॒व॒य॒ज्ययेति॑ देव - य॒ज्यया᳚ । आयु॑ष्मान् । य॒ज्ञेन॑ । प्र॒ति॒ष्ठामिति॑ प्रति - स्थाम् । ग॒मे॒य॒म् । अ॒ग्नेः । अ॒हम् । उज्जि॑ति॒मित्युत् - जि॒ति॒म् । अनु॑ । उदिति॑ । जे॒ष॒म् । सोम॑स्य । अ॒हम् । उज्जि॑ति॒मित्युत् - जि॒ति॒म् । अनु॑ । उदिति॑ । जे॒ष॒म् । अ॒ग्नेः । अ॒हम् । उज्जि॑ति॒मित्युत्-जि॒ति॒म् । अनु॑ । उदिति॑ । जे॒ष॒म् । अ॒ग्नीषोम॑यो॒रित्य॒ग्नी - सोम॑योः । अ॒हम् । उज्जि॑ति॒मित्युत् - जि॒ति॒म् । अनु॑ । उदिति॑ । जे॒ष॒म् । इ॒न्द्रा॒ग्नि॒योरिती᳚न्द्र - अ॒ग्नि॒योः । अ॒हम् । उज्जि॑ति॒मित्युत्-जि॒ति॒म् । अनु॑ । उदिति॑ । जे॒ष॒म् । इन्द्र॑स्य । अ॒हम् । \textbf{  11} \newline
                  \newline
                                \textbf{ TS 1.6.4.2} \newline
                  उज्जि॑ति॒मित्युत् - जि॒ति॒म् । अनु॑ । उदिति॑ । जे॒ष॒म् । म॒हे॒न्द्रस्येति॑ महा - इ॒न्द्रस्य॑ । अ॒हम् । उज्जि॑ति॒मित्युत्-जि॒ति॒म् । अनु॑ । उदिति॑ । जे॒ष॒म् । अ॒ग्नेः । स्वि॒ष्ट॒कृत॒ इति॑ स्विष्ट - कृतः॑ । अ॒हम् । उज्जि॑ति॒मित्युत् - जि॒ति॒म् । अनु॑ । उदिति॑ । जे॒ष॒म् । वाज॑स्य । मा॒ । प्र॒स॒वेनेति॑ प्र - स॒वेन॑ । उ॒द्ग्रा॒भेणेत्यु॑त् - ग्रा॒भेण॑ । उदिति॑ । अ॒ग्र॒भी॒त् ॥ अथ॑ । स॒पत्नान्॑ । इन्द्रः॑ । मे॒ । नि॒ग्रा॒भेणेति॑ नि - ग्रा॒भेण॑ । अध॑रान् । अ॒कः॒ ॥ उ॒द्ग्रा॒भमित्यु॑त् - ग्रा॒भम् । च॒ । नि॒ग्रा॒भमिति॑ नि - ग्रा॒भम् । च॒ । ब्रह्म॑ । दे॒वाः । अ॒वी॒वृ॒ध॒न्न् ॥ अथ॑ । स॒पत्नान्॑ । इ॒न्द्रा॒ग्नी इती᳚न्द्र - अ॒ग्नी । मे॒ । वि॒षू॒चीनान्॑ । वीति॑ । अ॒स्य॒ता॒म् ॥ एति॑ । इ॒माः । अ॒ग्म॒न्न् । आ॒शिष॒ इत्या᳚ - शिषः॑ । दोह॑कामा॒ इति॒ दोह॑ - का॒माः॒ । इन्द्र॑वन्त॒ इतीन्द्र॑- व॒न्तः॒ । \textbf{  12} \newline
                  \newline
                                \textbf{ TS 1.6.4.3} \newline
                  व॒ना॒म॒हे॒ । धु॒क्षी॒महि॑ । प्र॒जामिति॑ प्र-जाम् । इष᳚म् ॥ रोहि॑तेन । त्वा॒ । अ॒ग्निः । दे॒वता᳚म् । ग॒म॒य॒तु॒ । हरि॑भ्या॒मिति॒ हरि॑ - भ्या॒म् । त्वा॒ । इन्द्रः॑ । दे॒वता᳚म् । ग॒म॒य॒तु॒ । एत॑शेन । त्वा॒ । सूर्यः॑ । दे॒वता᳚म् । ग॒म॒य॒तु॒ । वीति॑ । ते॒ । मु॒ञ्चा॒मि॒ । र॒श॒नाः । वीति॑ । र॒श्मीन् । वीति॑ । योक्त्रा᳚ । यानि॑ । प॒रि॒चर्त॑ना॒नीति॑ परि - चर्त॑नानि । ध॒त्तात् । अ॒स्मासु॑ । द्रवि॑णम् । यत् । च॒ । भ॒द्रम् । प्रेति॑ । नः॒ । ब्रू॒ता॒त् । भा॒ग॒धानिति॑ भाग - धान् । दे॒वता॑सु ॥ विष्णोः᳚ । श॒म्ॅयोरिति॑ शं - योः । अ॒हम् । दे॒व॒य॒ज्ययेति॑ देव-य॒ज्यया᳚ । य॒ज्ञेन॑ । प्र॒ति॒ष्ठामिति॑ प्रति - स्थाम् । ग॒मे॒य॒म् । सोम॑स्य । अ॒हम् । दे॒व॒य॒ज्ययेति॑ देव - य॒ज्यया᳚ । \textbf{  13} \newline
                  \newline
                                \textbf{ TS 1.6.4.4} \newline
                  सु॒रेता॒ इति॑ सु - रेताः᳚ । रेतः॑ । धि॒षी॒य॒ । त्वष्टुः॑ । अ॒हम् । दे॒व॒य॒ज्ययेति॑ देव-य॒ज्यया᳚ । प॒शू॒नाम् । रू॒पम् । पु॒षे॒य॒म् । दे॒वाना᳚म् । पत्नीः᳚ । अ॒ग्निः । गृ॒हप॑ति॒रिति॑ गृ॒ह - प॒तिः॒ । य॒ज्ञ्स्य॑ । मि॒थु॒नम् । तयोः᳚ । अ॒हम् । दे॒व॒य॒ज्ययेति॑ देव - य॒ज्यया᳚ । मि॒थु॒नेन॑ । प्रेति॑ । भू॒या॒स॒म् । वे॒दः । अ॒सि॒ । वित्तिः॑ । अ॒सि॒ । वि॒देय॑ । कर्म॑ । अ॒सि॒ । क॒रुण᳚म् । अ॒सि॒ । क्रि॒यास᳚म् । स॒निः । अ॒सि॒ । स॒नि॒ता । अ॒सि॒ । स॒नेय᳚म् । घृ॒तव॑न्त॒मिति॑ घृ॒त-व॒न्त॒म् । कु॒ला॒यिन᳚म् । रा॒यः । पोष᳚म् । स॒ह॒स्रिण᳚म् । वे॒दः । द॒दा॒तु॒ । वा॒जिन᳚म् ॥ \textbf{  14} \newline
                  \newline
                      (इन्द्र॑स्या॒ह-मिन्द्र॑वन्तः॒-सोम॑स्या॒हं दे॑वय॒ज्यया॒-चतु॑श्चत्वारिꣳशच्च)  \textbf{(A4)} \newline \newline
                                \textbf{ TS 1.6.5.1} \newline
                  एति॑ । प्या॒य॒ता॒म् । ध्रु॒वा । घृ॒तेन॑ । य॒ज्ञ्ं ॅय॑ज्ञ्॒मिति॑ य॒ज्ञ्म् - य॒ज्ञ्॒म् । प्रतीति॑ । दे॒व॒यद्भ्य॒ इति॑ देव॒यत् - भ्यः॒ ॥ सू॒र्यायाः᳚ । ऊधः॑ । अदि॑त्याः । उ॒पस्थ॒ इत्यु॒प - स्थे॒ । उ॒रुधा॒रेत्यु॒रु - धा॒रा॒ । पृ॒थि॒वी । य॒ज्ञे । अ॒स्मिन्न् ॥ प्र॒जाप॑ते॒रिति॑ प्र॒जा-प॒तेः॒ । वि॒भानिति॑ वि-भान् । नाम॑ । लो॒कः । तस्मिन्न्॑ । त्वा॒ । द॒धा॒मि॒ । स॒ह । यज॑मानेन । सत् । अ॒सि॒ । सत् । मे॒ । भू॒याः॒ । सर्व᳚म् । अ॒सि॒ । सर्व᳚म् । मे॒ । भू॒याः॒ । पू॒र्णम् । अ॒सि॒ । पू॒र्णम् । मे॒ । भू॒याः॒ । अक्षि॑तम् । अ॒सि॒ । मा । मे॒ । क्ष॒ष्ठाः॒ । प्राच्या᳚म् । दि॒शि । दे॒वाः । ऋ॒त्विजः॑ । मा॒र्ज॒य॒न्ता॒म् । दक्षि॑णायाम् । \textbf{  15} \newline
                  \newline
                                \textbf{ TS 1.6.5.2} \newline
                  दि॒शि । मासाः᳚ । पि॒तरः॑ । मा॒र्ज॒य॒न्ता॒म् । प्र॒तीच्या᳚म् । दि॒शि । गृ॒हाः । प॒शवः॑ । मा॒र्ज॒य॒न्ता॒म् । उदी᳚च्याम् । दि॒शि । आपः॑ । ओष॑धयः । वन॒स्पत॑यः । मा॒र्ज॒य॒न्ता॒म् । ऊ॒द्‌र्ध्वाया᳚म् । दि॒शि । य॒ज्ञ्ः । सम्ॅव॒थ्स॒र इति॑ सं - व॒थ्स॒रः । य॒ज्ञ्प॑ति॒रिति॑ य॒ज्ञ् - प॒तिः॒ । मा॒र्ज॒य॒न्ता॒म् । विष्णोः᳚ । क्रमः॑ । अ॒सि॒ । अ॒भि॒मा॒ति॒हेत्य॑भिमाति - हा । गा॒य॒त्रेण॑ । छन्द॑सा । पृ॒थि॒वीम् । अनु॑ । वीति॑ । क्र॒मे॒ । निर्भ॑क्त॒ इति॒ निः - भ॒क्तः॒ । सः । यम् । द्वि॒ष्मः । विष्णोः᳚ । क्रमः॑ । अ॒सि॒ । अ॒भि॒श॒स्ति॒हेत्य॑भिशस्ति - हा । त्रैष्टु॑भेन । छन्द॑सा । अ॒न्तरि॑क्षम् । अनु॑ । वीति॑ । क्र॒मे॒ । निर्भ॑क्त॒ इति॒ निः-भ॒क्तः॒ । सः । यम् । द्वि॒ष्मः । विष्णोः᳚ ( ) । क्रमः॑ । अ॒सि॒ । अ॒रा॒ती॒य॒तः । ह॒न्ता । जाग॑तेन । छन्द॑सा । दिव᳚म् । अनु॑ । वीति॑ । क्र॒मे॒ । निर्भ॑क्त॒ इति॒ निः - भ॒क्तः॒ । सः । यम् । द्वि॒ष्मः । विष्णाः᳚ । क्रमः॑ । अ॒सि॒ । श॒त्रू॒य॒त इति॑ शत्रु - य॒तः । ह॒न्ता । आनु॑ष्टुभे॒नेत्यानु॑ - स्थु॒भे॒न॒ । छन्द॑सा । दिशः॑ । अनु॑ । वीति॑ । क्र॒मे॒ । निर्भ॑क्त॒ इति॒ निः - भ॒क्तः॒ । सः । यम् । द्वि॒ष्मः ॥ \textbf{  16} \newline
                  \newline
                      (दक्षि॑णाया - म॒न्तरि॑क्ष॒मनु॒ वि क्र॑मे॒ निर्भ॑क्तः॒ स यं द्वि॒ष्मो विष्णो॒- रेका॒न्नत्रिꣳ॒॒शच्च॑)  \textbf{(A5)} \newline \newline
                                \textbf{ TS 1.6.6.1} \newline
                  अग॑न्म । सुवः॑ । सुवः॑ । अ॒ग॒न्म॒ । स॒दृंश॒ इति॑ सं-दृशः॑ । ते॒ । मा । छि॒थ्सि॒ । यत् । ते॒ । तपः॑ । तस्मै᳚ । ते॒ । मा । एति॑ । वृ॒क्षि॒ । सु॒भूरिति॑ सु - भूः । अ॒सि॒ । श्रेष्ठः॑ । र॒श्मी॒नाम् । आ॒यु॒द्‌र्धा इत्या॑युः - धाः । अ॒सि॒ । आयुः॑ । मे॒ । धे॒हि॒ । व॒र्चो॒धा इति॑ वर्चः -धाः । अ॒सि॒ । वर्चः॑ । मयि॑ । ध॒हि॒ । इ॒दम् । अ॒हम् । अ॒मुम् । भ्रातृ॑व्यम् । आ॒भ्यः । दि॒ग्भ्य इति॑ दिक् - भ्यः । अ॒स्यै । दि॒वः । अ॒स्मात् । अ॒न्तरि॑क्षात् । अ॒स्यै । पृ॒थि॒व्याः । अ॒स्मात् । अ॒न्नाद्या॒दित्य॑न्न -अद्या᳚त् । निरिति॑ । भ॒जा॒मि॒ । निर्भ॑क्त॒ इति॒ निः - भ॒क्तः॒ । सः । यम् । द्वि॒ष्मः ॥ \textbf{  17} \newline
                  \newline
                                \textbf{ TS 1.6.6.2} \newline
                  समिति॑ । ज्योति॑षा । अ॒भू॒व॒म् । ऐ॒न्द्रीम् । आ॒वृत॒मित्या᳚ - वृत᳚म् । अ॒न्वाव॑र्त॒ इत्य॑नु - आव॑र्ते । समिति॑ । अ॒हम् । प्र॒जयेति॑ प्र-जया᳚ । समिति॑ । मया᳚ । प्र॒जेति॑ प्र - जा । समिति॑ । अ॒हम् । रा॒यः । पोषे॑ण । समिति॑ । मया᳚ । रा॒यः । पोषः॑ । समि॑द्ध॒ इति॒ सम् - इ॒द्धः॒ । अ॒ग्ने॒ । मे॒ । दी॒दि॒हि॒ । स॒मे॒द्धेति॑ सम् - ए॒द्धा । ते॒ । अ॒ग्ने॒ । दी॒द्या॒स॒म् । वसु॑मा॒निति॒ वसु॑ - मा॒न् । य॒ज्ञ्ः । वसी॑यान् । भू॒या॒स॒म् । अग्ने᳚ । आयूꣳ॑षि । प॒व॒से॒ । एति॑ । सु॒व॒ । ऊर्ज᳚म् । इष᳚म् । च॒ । नः॒ ॥ आ॒रे । बा॒ध॒स्व॒ । दु॒च्छुना᳚म् ॥ अग्ने᳚ । पव॑स्व । स्वपा॒ इति॑ सु-अपाः᳚ । अ॒स्मे इति॑ । वर्चः॑ । सु॒वीर्य॒मिति॑ सु-वीर्य᳚म् ॥ \textbf{  18} \newline
                  \newline
                                \textbf{ TS 1.6.6.3} \newline
                  दध॑त् । पोष᳚म् । र॒यिम् । मयि॑ ॥ अग्ने᳚ । गृ॒ह॒प॒त॒ इति॑ गृह - प॒ते॒ । सु॒गृ॒ह॒प॒तिरिति॑ सु - गृ॒ह॒प॒तिः । अ॒हम् । त्वया᳚ । गृ॒हप॑ति॒नेति॑ गृ॒ह - प॒ति॒ना॒ । भू॒या॒स॒म् । सु॒गृ॒ह॒प॒तिरिति॑ सु – गृ॒ह॒प॒तिः  । मया᳚ । त्वम् । गृ॒हप॑ति॒नेति॑ गृ॒ह - प॒ति॒ना॒ । भू॒याः॒ । श॒तम् । हिमाः᳚ । ताम् । आ॒शिष॒मित्या᳚ - शिष᳚म् । एति॑ । शा॒से॒ । तन्त॑वे । ज्योति॑ष्मतीम् । ताम् । आ॒शिष॒मित्या᳚ - शिष᳚म् । एति॑ । शा॒से॒ । अ॒मुष्मै᳚ । ज्योति॑ष्मतीम् । कः । त्वा॒ । यु॒न॒क्ति॒ । सः । त्वा॒ । वीति॑ । मु॒ञ्च॒तु॒ । अग्ने᳚ । व्र॒त॒प॒त॒ इति॑ व्रत - प॒ते॒ । व्र॒तम् । अ॒चा॒रि॒ष॒म् । तत् । अ॒श॒क॒म् । तत् । मे॒ । अ॒रा॒धि॒ । य॒ज्ञ्ः । ब॒भू॒व॒ । सः । एति॑ । \textbf{  19} \newline
                  \newline
                                \textbf{ TS 1.6.6.4} \newline
                  ब॒भू॒व॒ । सः । प्रेति॑ । ज॒ज्ञे॒ । सः । वा॒वृ॒धे॒ ॥ सः । दे॒वाना᳚म् । अधि॑पति॒रित्यधि॑ - प॒तिः॒ । ब॒भू॒व॒ । सः । अ॒स्मान् । अधि॑पती॒नित्यधि॑ - प॒ती॒न् । क॒रो॒तु॒ । व॒यम् । स्या॒म॒ । पत॑यः । र॒यी॒णाम् ॥ गोमा॒निति॒ गो - मा॒न् । अ॒ग्ने॒ । अवि॑मा॒नित्यवि॑ - मा॒न् । अ॒श्वी । य॒ज्ञ्ः । नृ॒वथ्स॒खेति॑ नृ॒वत् - स॒खा॒ । सद᳚म् । इत् । अ॒प्र॒मृ॒ष्य इत्य॑प्र - मृ॒ष्यः ॥ इडा॑वा॒नितीडा᳚ - वा॒न् । ए॒षः । अ॒सु॒र॒ । प्र॒जावा॒निति॑ प्र॒जा - वा॒न् । दी॒र्घः । र॒यिः । पृ॒थु॒बु॒द्ध्न इति॑ पृथु - बु॒द्ध्नः । स॒भावा॒निति॑ स॒भा - वा॒न् ॥ \textbf{  20} \newline
                  \newline
                      (द्वि॒ष्मः-सु॒वीर्यꣳ॒॒-स आ-पञ्च॑त्रिꣳशच्च)  \textbf{(A6)} \newline \newline
                                \textbf{ TS 1.6.7.1} \newline
                  यथा᳚ । वै । स॒मृ॒त॒सो॒मा इति॑ समृत - सो॒माः । ए॒वम् । वै । ए॒ते । स॒मृ॒त॒य॒ज्ञा इति॑ समृत - य॒ज्ञाः । यत् । द॒र्॒.श॒पू॒र्ण॒मा॒साविति॑ दर्.श - पू॒र्ण॒मा॒सौ । कस्य॑ । वा॒ । अह॑ । दे॒वाः । य॒ज्ञ्म् । आ॒गच्छ॒न्तीत्या᳚ - गच्छ॑न्ति । कस्य॑ । वा॒ । न । ब॒हू॒नाम् । यज॑मानानाम् । यः । वै । दे॒वताः᳚ । पूर्वः॑ । प॒रि॒गृ॒ह्णातीति॑ परि - गृ॒ह्णाति॑ । सः । ए॒नाः॒ । श्वः । भू॒ते । य॒ज॒ते॒ । ए॒तत् । वै । दे॒वाना᳚म् । आ॒यत॑न॒मित्या᳚ - यत॑नम् । यत् । आ॒ह॒व॒नीय॒ इत्या᳚ - ह॒व॒नीयः॑ । अ॒न्त॒रा । अ॒ग्नी इति॑ । प॒शू॒नाम् । गार्.ह॑पत्य॒ इति॒ गार्.ह॑ - प॒त्यः॒ । म॒नु॒ष्या॑णाम् । अ॒न्वा॒हा॒र्य॒पच॑न॒ इत्य॑न्वाहार्य - पच॑नः । पि॒तृ॒णाम् । अ॒ग्निम् । गृ॒ह्णा॒ति॒ । स्वे । ए॒व । आ॒यत॑न॒ इत्या᳚ - यत॑ने । दे॒वताः᳚ । परीति॑ । \textbf{  21} \newline
                  \newline
                                \textbf{ TS 1.6.7.2} \newline
                  गृ॒ह्णा॒ति॒ । ताः । श्वः । भू॒ते । य॒ज॒ते॒ । व्र॒तेन॑ । वै । मेद्ध्यः॑ । अ॒ग्निः । व्र॒तप॑ति॒रिति॑ व्र॒त - प॒तिः॒ । ब्रा॒ह्म॒णः । व्र॒त॒भृदिति॑ व्रत - भृत् । व्र॒तम् । उ॒पै॒ष्यन्नित्यु॑प - ए॒ष्यन्न् । ब्रू॒या॒त् । अग्ने᳚ । व्र॒त॒प॒त॒ इति॑ व्रत - प॒ते॒ । व्र॒तम् । च॒रि॒ष्या॒मि॒ । इति॑ । अ॒ग्निः । वै । दे॒वाना᳚म् । व्र॒तप॑ति॒रिति॑ व्र॒त - प॒तिः॒ । तस्मै᳚ । ए॒व । प्र॒ति॒प्रोच्येति॑ प्रति - प्रोच्य॑ । व्र॒तम् । एति॑ । ल॒भ॒ते॒ । ब॒र्॒.हिषा᳚ । पू॒र्णमा॑स॒ इति॑ पू॒र्ण - मा॒से॒ । व्र॒तम् । उपेति॑ । ए॒ति॒ । व॒थ्सैः । अ॒मा॒वा॒स्या॑या॒मित्य॑मा - वा॒स्या॑याम् । ए॒तत् । हि । ए॒तयोः᳚ । आ॒यत॑न॒मित्या᳚ - यत॑नम् । उ॒प॒स्तीर्य॒ इत्यु॑प - स्तीर्यः॑ । पूर्वः॑ । च॒ । अ॒ग्निः । अप॑रः । च॒ । इति॑ । आ॒हुः॒ । म॒नु॒ष्याः᳚ । \textbf{  22} \newline
                  \newline
                                \textbf{ TS 1.6.7.3} \newline
                  इत् । नु । वा । उप॑स्तीर्ण॒मियुप॑ - स्ती॒र्ण॒म् । इ॒च्छन्ति॑ । किम् । उ॒ । दे॒वाः । येषा᳚म् । नवा॑वसान॒मिति॒ नव॑ - अ॒व॒सा॒न॒म् । उपेति॑ । अ॒स्मि॒न्न् । श्वः । य॒क्ष्यमा॑णे । दे॒वताः᳚ । व॒स॒न्ति॒ । यः । ए॒वम् । वि॒द्वान् । अ॒ग्निम् । उ॒प॒स्तृ॒णातीत्यु॑प - स्तृ॒णाति॑ । यज॑मानेन । ग्रा॒म्याः । च॒ । प॒शवः॑ । अ॒व॒रुद्ध्या॒ इत्य॑व-रुद्ध्याः᳚ । आ॒र॒ण्याः । च॒ । इति॑ । आ॒हुः॒ । यत् । ग्रा॒म्यान् । उ॒प॒वस॒तीत्यु॑प - वस॑ति । तेन॑ । ग्रा॒म्यान् । अवेति॑ । रु॒न्धे॒ । यत् । आ॒र॒ण्यस्य॑ । अ॒श्नाति॑ । तेन॑ । आ॒र॒ण्यान् । यत् । अना᳚श्वान् । उ॒प॒वसे॒दित्यु॑प - वसे᳚त् । पि॒तृ॒दे॒व॒त्य॑ इति॑ पितृ - दे॒व॒त्यः॑ । स्या॒त् । आ॒र॒ण्यस्य॑ । अ॒श्ना॒ति॒ । इ॒न्द्रि॒यम् । \textbf{  23} \newline
                  \newline
                                \textbf{ TS 1.6.7.4} \newline
                  वै । आ॒र॒ण्यम् । इ॒न्द्रि॒यम् । ए॒व । आ॒त्मन्न् । ध॒त्ते॒ । यत् । अना᳚श्वान् । उ॒प॒वसे॒दित्यु॑प - वसे᳚त् । क्षोधु॑कः । स्या॒त् । यत् । अ॒श्नी॒यात् । रु॒द्रः । अ॒स्य॒ । प॒शून् । अ॒भीति॑ । म॒न्ये॒त॒ । अ॒पः । अ॒श्ना॒ति॒ । तत् । न । इ॒व॒ । अ॒शि॒तम् । न । इ॒व॒ । अन॑शितम् । न । क्षोधु॑कः । भव॑ति । न । अ॒स्य॒ । रु॒द्रः । प॒शून् । अ॒भीति॑ । म॒न्य॒ते॒ । वज्रः॑ । वै । य॒ज्ञ्ः । क्षुत् । खलु॑ । वै । म॒नु॒ष्य॑स्य । भ्रातृ॑व्यः । यत् । अना᳚श्वान् । उ॒प॒वस॒तीत्यु॑प-वस॑ति । वज्रे॑ण । ए॒व । सा॒क्षादिति॑ स-अ॒क्षात् ( ) । क्षुध᳚म् । भ्रातृ॑व्यम् । ह॒न्ति॒ ॥ \textbf{  24} \newline
                  \newline
                      (परि॑-मनु॒ष्या॑-इन्द्रि॒यꣳ-सा॒क्षात्-त्रीणि॑ च)  \textbf{(A7)} \newline \newline
                                \textbf{ TS 1.6.8.1} \newline
                  यः । वै । श्र॒द्धामिति॑ श्रत् - धाम् । अना॑र॒भ्येत्यना᳚ - र॒भ्य॒ । य॒ज्ञेन॑ । यज॑ते । न । अ॒स्य॒ । इ॒ष्टाय॑ । श्रत् । द॒ध॒ते॒ । अ॒पः । प्रेति॑ । न॒य॒ति॒ । श्र॒द्धेति॑ श्रत् - धा । वै । आपः॑ । श्र॒द्धामिति॑ श्रत् - धाम् । ए॒व । आ॒रभ्येत्या᳚ - रभ्य॑ । य॒ज्ञेन॑ । य॒ज॒ते॒ । उ॒भये᳚ । अ॒स्य॒ । दे॒व॒म॒नु॒ष्या इति॑ देव - म॒नु॒ष्याः । इ॒ष्टाय॑ । श्रत् । द॒ध॒ते॒ । तत् । आ॒हुः॒ । अतीति॑ । वै । ए॒ताः । वर्त्र᳚म् । ने॒द॒न्ति॒ । अतीति॑ । वाच᳚म् । मनः॑ । वाव । ए॒ताः । न । अतीति॑ । ने॒द॒न्ति॒ । इति॑ । मन॑सा । प्रेति॑ । न॒य॒ति॒ । इ॒यम् । वै । मनः॑ । \textbf{  25} \newline
                  \newline
                                \textbf{ TS 1.6.8.2} \newline
                  अ॒नया᳚ । ए॒व । ए॒नाः॒ । प्रेति॑ । न॒य॒ति॒ । अस्क॑न्नहवि॒रित्यस्क॑न्न - ह॒विः॒ । भ॒व॒ति॒ । यः । ए॒वम् । वेद॑ । य॒ज्ञा॒यु॒धानीति॑ यज्ञ् - आ॒यु॒धानि॑ । समिति॑ । भ॒र॒ति॒ । य॒ज्ञ्ः । वै । य॒ज्ञा॒यु॒धानीति॑ यज्ञ् - आ॒यु॒धानि॑ । य॒ज्ञ्म् । ए॒व । तत् । समिति॑ । भ॒र॒ति॒ । यत् । एक॑मेक॒मित्येक᳚म् - ए॒क॒म् । स॒भंरे॒दिति॑ सं-भरे᳚त् । पि॒तृ॒दे॒व॒त्या॑नीति॑ पितृ - दे॒व॒त्या॑नि । स्युः॒ । यत् । स॒ह । सर्वा॑णि । मा॒नु॒षाणि॑ । द्वेद्वे॒ इति॒ द्वे - द्वे॒ । समिति॑ । भ॒र॒ति॒ । या॒ज्या॒नु॒वा॒क्य॑यो॒रिति॑ याज्या - अ॒नु॒वा॒क्य॑योः । ए॒व । रू॒पम् । क॒रो॒ति॒ । अथो॒ इति॑ । मि॒थु॒नम् । ए॒व । यः । वै । दश॑ । य॒ज्ञा॒यु॒धानीति॑ यज्ञ् - आ॒यु॒धानि॑ । वेद॑ । मु॒ख॒तः । अ॒स्य॒ । य॒ज्ञ्ः । क॒ल्प॒ते॒ । स्फ्यः । \textbf{  26} \newline
                  \newline
                                \textbf{ TS 1.6.8.3} \newline
                  च॒ । क॒पाला॑नि । च॒ । अ॒ग्नि॒हो॒त्र॒हव॒णीत्य॑ग्निहोत्र - हव॑नी । च॒ । शूर्प᳚म् । च॒ । कृ॒ष्णा॒जि॒नमिति॑ कृष्ण - अ॒जि॒नम् । च॒ । शम्या᳚ । च॒ । उ॒लूख॑लम् । च॒ । मुस॑लम् । च॒ । दृ॒षत् । च॒ । उप॑ला । च॒ । ए॒तानि॑ । वै । दश॑ । य॒ज्ञा॒यु॒धानीति॑ यज्ञ् - आ॒यु॒धानि॑ । यः । ए॒वम् । वेद॑ । मु॒ख॒तः । अ॒स्य॒ । य॒ज्ञ्ः । क॒ल्प॒ते॒ । यः । वै । दे॒वेभ्यः॑ । प्र॒ति॒प्रोच्येति॑ प्रति - प्रोच्य॑ । य॒ज्ञेन॑ । यज॑ते । जु॒षन्ते᳚ । अ॒स्य॒ । दे॒वाः । ह॒व्यम् । ह॒विः । नि॒रु॒प्यमा॑ण॒मिति॑ निः - उ॒प्यमा॑नम् । अ॒भीति॑ । म॒न्त्र॒ये॒त॒ । अ॒ग्निम् । होता॑रम् । इ॒ह । तम् । हु॒वे॒ । इति॑ । \textbf{  27} \newline
                  \newline
                                \textbf{ TS 1.6.8.4} \newline
                  दे॒वेभ्यः॑ । ए॒व । प्र॒ति॒प्रोच्येति॑ प्रति - प्रोच्य॑ । य॒ज्ञेन॑ । य॒ज॒ते॒ । जु॒षन्ते᳚ । अ॒स्य॒ । दे॒वाः । ह॒व्यम् । ए॒षः । वै । य॒ज्ञ्स्य॑ । ग्रहः॑ । गृ॒ही॒त्वा । ए॒व । य॒ज्ञेन॑ । य॒ज॒ते॒ । तत् । उ॒दि॒त्वा । वाच᳚म् । य॒च्छ॒ति॒ । य॒ज्ञ्स्य॑ । धृत्यै᳚ । अथो॒ इति॑ । मन॑सा । वै । प्र॒जाप॑ति॒रिति॑ प्र॒जा - प॒तिः॒ । य॒ज्ञ्म् । अ॒त॒नु॒त॒ । मन॑सा । ए॒व । तत् । य॒ज्ञ्म् । त॒नु॒ते॒ । रक्ष॑साम् । अन॑न्ववचारा॒येत्यन॑नु-अ॒व॒चा॒रा॒य॒ । यः । वै । य॒ज्ञ्म् । योगे᳚ । आग॑त॒ इत्या - ग॒ते॒ । यु॒नक्ति॑ । यु॒ङ्क्ते । यु॒ञ्जा॒नेषु॑ । कः । त्वा॒ । यु॒न॒क्ति॒ । सः । त्वा॒ । यु॒न॒क्तु॒ ( ) । इति॑ । आ॒ह॒ । प्र॒जाप॑ति॒रिति॑ प्र॒जा - प॒तिः॒ । वै । कः । प्र॒जाप॑ति॒नेति॑ प्र॒जा - प॒ति॒ना॒ । ए॒व । ए॒न॒म् । यु॒न॒क्ति॒ । यु॒ङ्क्ते । यु॒ञ्जा॒नेषु॑ ॥ \textbf{  28} \newline
                  \newline
                      (वैम॒नः-स्फ्य-इति॑-युन॒क्त्वे-का॑दश च)  \textbf{(A8)} \newline \newline
                                \textbf{ TS 1.6.9.1} \newline
                  प्र॒जाप॑ति॒रिति॑ प्र॒जा - प॒तिः॒ । य॒ज्ञान् । अ॒सृ॒ज॒त॒ । अ॒ग्नि॒हो॒त्रमित्य॑ग्नि - हो॒त्रम् । च॒ । अ॒ग्नि॒ष्टो॒ममित्य॑ग्नि - स्तो॒मम् । च॒ । पौ॒र्ण॒मा॒सीमिति॑ पौर्ण - मा॒सीम् । च॒ । उ॒क्थ्य᳚म् । च॒ । अ॒मा॒वा॒स्या॑मित्य॑मा-वा॒स्या᳚म् । च॒ । अ॒ति॒रा॒त्रमित्य॑ति - रा॒त्रम् । च॒ । तान् । उदिति॑ । अ॒मि॒मी॒त॒ । याव॑त् । अ॒ग्नि॒हो॒त्रमित्य॑ग्नि - हो॒त्रम् । आसी᳚त् । तावान्॑ । अ॒ग्नि॒ष्टो॒म इत्य॑ग्नि - स्तो॒मः । याव॑ती । पौ॒र्ण॒मा॒सीति॑ पौर्ण - मा॒सी । तावान्॑ । उ॒क्थ्यः॑ । याव॑ती । अ॒मा॒वा॒स्येत्य॑मा - वा॒स्या᳚ । तावान्॑ । अ॒ति॒रा॒त्र इत्य॑ति-रा॒त्रः । यः । ए॒वम् । वि॒द्वान् । अ॒ग्नि॒हो॒त्रमित्य॑ग्नि - हो॒त्रम् । जु॒होति॑ । याव॑त् । अ॒ग्नि॒ष्टो॒मेनेत्य॑ग्नि - स्तो॒मेन॑ । उ॒पा॒प्नोतीत्यु॑प - आ॒प्नोति॑ । ताव॑त् । उपेति॑ । आ॒प्नो॒ति॒ । यः । ए॒वम् । वि॒द्वान् । पौ॒र्ण॒मा॒सीमिति॑ पौर्ण - मा॒सीम् । यज॑ते । याव॑त् । उ॒क्थ्ये॑न । उ॒पा॒प्नोतीत्यु॑प - आ॒प्नोति॑ । \textbf{  29} \newline
                  \newline
                                \textbf{ TS 1.6.9.2} \newline
                  ताव॑त् । उपेति॑ । आ॒प्नो॒ति॒ । यः । ए॒वम् । वि॒द्वान् । अ॒मा॒वा॒स्या॑मित्य॑मा - वा॒स्या᳚म् । यज॑ते । याव॑त् । अ॒ति॒रा॒त्रेणेत्य॑ति - रा॒त्रेण॑ । उ॒पा॒प्नोतीत्यु॑प - आ॒प्नोति॑ । ताव॑त् । उपेति॑ । आ॒प्नो॒ति॒ । प॒र॒मे॒ष्ठिनः॑ । वै । ए॒षः । य॒ज्ञ्ः । अग्रे᳚ । आ॒सी॒त् । तेन॑ । सः । प॒र॒माम् । काष्ठा᳚म् । अ॒ग॒च्छ॒त् । तेन॑ । प्र॒जाप॑ति॒मिति॑ प्र॒जा - प॒ति॒म् । नि॒रवा॑सायय॒दिति॑ निः - अवा॑साययत् । तेन॑ । प्र॒जाप॑ति॒रिति॑ प्र॒जा - प॒तिः॒ । प॒र॒माम् । काष्ठा᳚म् । अ॒ग॒च्छ॒त् । तेन॑ । इन्द्र᳚म् । नि॒रवा॑सायय॒दिति॑ निः-अवा॑साययत् । तेन॑ । इन्द्रः॑ । प॒र॒माम् । काष्ठा᳚म् । अ॒ग॒च्छ॒त् । तेन॑ । अ॒ग्नीषोमा॒वित्य॒ग्नी - सोमौ᳚ । नि॒रवा॑सायय॒दिति॑ निः - अवा॑साययत् । तेन॑ । अ॒ग्नीषोमा॒वित्य॒ग्नी - सोमौ᳚ । प॒र॒माम् । काष्ठा᳚म् । अ॒ग॒च्छ॒ता॒म् । यः । \textbf{  30} \newline
                  \newline
                                \textbf{ TS 1.6.9.3} \newline
                  ए॒वम् । वि॒द्वान् । द॒र्॒.श॒पू॒र्ण॒मा॒साविति॑ दर्.श - पू॒र्ण॒मा॒सौ । यज॑ते । प॒र॒माम् । ए॒व । काष्ठा᳚म् । ग॒च्छ॒ति॒ । यः । वै । प्रजा॑ते॒नेति॒ प्र - जा॒ते॒न॒ । य॒ज्ञेन॑ । यज॑ते । प्रेति॑ । प्र॒जयेति॑ प्र - जया᳚ । प॒शुभि॒रिति॑ प॒शु - भिः॒ । मि॒थु॒नैः । जा॒य॒ते॒ । द्वाद॑श । मासाः᳚ । सं॒ॅव॒थ्स॒र इति॑ सं - व॒थ्स॒रः । द्वाद॑श । द्व॒न्द्वानीति॑ द्वं - द्वानि॑ । द॒र्॒.श॒पू॒र्ण॒मा॒सयो॒रिति॑ दर्.श - पू॒र्ण॒मा॒सयोः᳚ । तानि॑ । स॒पांद्या॒नीति॑ सं- पाद्या॑नि । इति॑ । आ॒हुः॒ । व॒थ्सम् । च॒ । उ॒पा॒व॒सृ॒जतीत्यु॑प - अ॒व॒सृ॒जति॑ । उ॒खाम् । च॒ । अधीति॑ । श्र॒य॒ति॒ । अवेति॑ । च॒ । हन्ति॑ । दृ॒षदौ᳚ । च॒ । स॒माह॒न्तीति॑ सं - आह॑न्ति । अधीति॑ । च॒ । वप॑ते । क॒पाला॑नि । च॒ । उपेति॑ । द॒धा॒ति॒ । पु॒रो॒डाश᳚म् । च॒ । \textbf{  31} \newline
                  \newline
                                \textbf{ TS 1.6.9.4} \newline
                  अ॒धि॒श्रय॒तीत्य॑धि - श्रय॑ति । आज्य᳚म् । च॒ । स्त॒बं॒य॒जुरिति॑ स्तंब - य॒जुः । च॒ । हर॑ति । अ॒भीति॑ । च॒ । गृ॒ह्णा॒ति॒ । वेदि᳚म् । च॒ । प॒रि॒गृ॒ह्णातीति॑ परि - गृ॒ह्णाति॑ । पत्नी᳚म् । च॒ । समिति॑ । न॒ह्य॒ति॒ । प्रोक्ष॑णी॒रिति॑ प्र - उक्ष॑णीः । च॒ । आ॒सा॒दय॒तीत्या᳚ - सा॒दय॑ति । आज्य᳚म् । च॒ । ए॒तानि॑ । वै । द्वाद॑श । द्व॒न्द्वानीति॑ द्वं - द्वानि॑ । द॒र्॒.श॒पू॒र्ण॒मा॒सया॒रिति॑ दर्.श - पू॒र्ण॒मा॒सयोः᳚ । तानि॑ । यः । ए॒वम् । स॒पांद्येति॑ सं - पाद्य॑ । यज॑ते । प्रजा॑ते॒नेति॒ प्र - जा॒ते॒न॒ । ए॒व । य॒ज्ञेन॑ । य॒ज॒ते॒ । प्रेति॑ । प्र॒जयेति॑ प्र - जया᳚ । प॒शुभि॒रिति॑ प॒शु - भिः॒ । मि॒थु॒नैः । जा॒य॒ते॒ ॥ \textbf{  32} \newline
                  \newline
                      (उ॒क्थ्ये॑नोपा॒प्नोत्य॑-गच्छतां॒ ॅयः - पु॑रो॒डाशं॑-चत्वारिꣳ॒॒शच्च॑)  \textbf{(A9)} \newline \newline
                                \textbf{ TS 1.6.10.1} \newline
                  ध्रु॒वः । अ॒सि॒ । ध्रु॒वः । अ॒हम् । स॒जा॒तेष्विति॑ स - जा॒तेषु॑ । भू॒या॒स॒म् । इति॑ । आ॒ह॒ । ध्रु॒वान् । ए॒व । ए॒ना॒न् । कु॒रु॒ते॒ । उ॒ग्रः । अ॒सि॒ । उ॒ग्रः । अ॒हम् । स॒जा॒तेष्विति॑ स-जा॒तेषु॑ । भू॒या॒स॒म् । इति॑ । आ॒ह॒ । अप्र॑तिवादिन॒ इत्यप्र॑ति - वा॒दि॒नः॒ । ए॒व । ए॒ना॒न् । कु॒रु॒ते॒ । अ॒भि॒भूरित्य॑भि - भूः । अ॒सि॒ । अ॒भि॒भूरित्य॑भि - भूः । अ॒हम् । स॒जा॒तेष्विति॑ स - जा॒तेषु॑ । भू॒या॒स॒म् । इति॑ । आ॒ह॒ । यः । ए॒व । ए॒न॒म् । प्र॒त्यु॒त्पिपी॑त॒ इति॑ प्रति-उ॒त्पिपी॑ते । तम् । उपेति॑ । अ॒स्य॒ते॒ । यु॒नज्मि॑ । त्वा॒ । ब्रह्म॑णा । दैव्ये॑न । इति॑ । आ॒ह॒ । ए॒षः । वै । अ॒ग्नेः । योगः॑ । तेन॑ । \textbf{  33} \newline
                  \newline
                                \textbf{ TS 1.6.10.2} \newline
                  ए॒व । ए॒न॒म् । यु॒न॒क्ति॒ । य॒ज्ञ्स्य॑ । वै । समृ॑द्धे॒नेति॒ सं - ऋ॒द्धे॒न॒ । दे॒वाः । सु॒व॒र्गमिति॑ सुवः - गम् । लो॒कम् । आ॒य॒न्न् । य॒ज्ञ्स्य॑ । व्यृ॑द्धे॒नेति॒ वि - ऋ॒द्धे॒न॒ । असु॑रान् । परेति॑ । अ॒भा॒व॒य॒न्न् । यत् । मे॒ । अ॒ग्ने॒ । अ॒स्य । य॒ज्ञ्स्य॑ । रिष्या᳚त् । इति॑ । आ॒ह॒ । य॒ज्ञ्स्य॑ । ए॒व । तत् । समृ॑द्धे॒नेति॒ सं - ऋ॒द्धे॒न॒ । यज॑मानः । सु॒व॒र्गमिति॑ सुवः - गम् । लो॒कम् । ए॒ति॒ । य॒ज्ञ्स्य॑ । व्यृ॑द्धे॒नेति॒ वि - ऋ॒द्धे॒न॒ । भ्रातृ॑व्यान् । परेति॑ । भा॒व॒य॒ति॒ । अ॒ग्नि॒हो॒त्रमित्य॑ग्नि - हो॒त्रम् । ए॒ताभिः॑ । व्याहृ॑तीभि॒रिति॒ व्याहृ॑ति - भिः॒ । उपेति॑ । सा॒द॒ये॒त् । य॒ज्ञ्॒मु॒खमिति॑ यज्ञ् - म॒खम् । वै । अ॒ग्नि॒हो॒त्रमित्य॑ग्नि - हो॒त्रम् । ब्रह्म॑ । ए॒ताः । व्याहृ॑तय॒ इति॑ वि - आहृ॑तयः । य॒ज्ञ्॒मु॒ख इति॑ यज्ञ् - मु॒खे । ए॒व । ब्रह्म॑ । \textbf{  34} \newline
                  \newline
                                \textbf{ TS 1.6.10.3} \newline
                  कु॒रु॒ते॒ । सं॒ॅव॒थ्स॒र इति॑ सं - व॒थ्स॒रे । प॒र्याग॑त॒ इति॑ परि - आग॑ते । ए॒ताभिः॑ । ए॒व । उपेति॑ । सा॒द॒ये॒त् । ब्रह्म॑णा । ए॒व । उ॒भ॒यतः॑ । सं॒ॅव॒थ्स॒रमिति॑ सं - व॒थ्स॒रम् । परीति॑ । गृ॒ह्णा॒ति॒ । द॒र्॒.श॒पू॒र्ण॒मा॒साविति॑ दर्.श - पू॒र्ण॒मा॒सौ । चा॒तु॒र्मा॒स्यानीति॑ चातुः - मा॒स्यानि॑ । आ॒लभ॑मान॒ इत्या᳚ - लभ॑मानः । ए॒ताभिः॑ । व्याहृ॑तीभि॒रिति॒ व्याहृ॑ति - भिः॒ । ह॒वीꣳषि॑ । एति॑ । सा॒द॒ये॒त् । य॒ज्ञ्॒मु॒खमिति॑ यज्ञ् - मु॒खम् । वै । द॒र्॒.श॒पू॒र्ण॒मा॒साविति॑ दर्.श - पू॒र्ण॒मा॒सौ । चा॒तु॒र्मा॒स्यानीति॑ चातुः - मा॒स्यानि॑ । ब्रह्म॑ । ए॒ताः । व्याहृ॑तय॒ इति॑ वि - आहृ॑तयः । य॒ज्ञ्॒मु॒ख इति॑ यज्ञ् - मु॒खे । ए॒व । ब्रह्म॑ । कु॒रु॒ते॒ । सं॒ॅव॒थ्स॒र इति॑ सं - व॒थ्स॒रे । प॒र्याग॑त॒ इति॑ परि - आग॑ते । ए॒ताभिः॑ । ए॒व । एति॑ । सा॒द॒ये॒त् । ब्रह्म॑णा । ए॒व । उ॒भ॒यतः॑ । सं॒ॅव॒थ्स॒रमिति॑ सं - व॒थ्स॒रम् । परीति॑ । गृ॒ह्णा॒ति॒ । यत् । वै । य॒ज्ञ्स्य॑ । साम्ना᳚ । क्रि॒यते᳚ । रा॒ष्ट्रम् । \textbf{  35} \newline
                  \newline
                                \textbf{ TS 1.6.10.4} \newline
                  य॒ज्ञ्स्य॑ । आ॒शीरित्या᳚ - शीः । ग॒च्छ॒ति॒ । यत् । ऋ॒चा । विश᳚म् । य॒ज्ञ्स्य॑ । आ॒शीरित्या᳚ - शीः । ग॒च्छ॒ति॒ । अथ॑ । ब्रा॒ह्म॒णः । अ॒ना॒शीर्के॑ण । य॒ज्ञेन॑ । य॒ज॒ते॒ । सा॒मि॒धे॒नीरिति॑ साम् - इ॒धे॒नीः । अ॒नु॒व॒क्ष्यन्नित्य॑नु - व॒क्ष्यन्न् । ए॒ताः । व्याहृ॑ती॒रिति॑ वि - आहृ॑तीः । पु॒रस्ता᳚त् । द॒द्ध्या॒त् । ब्रह्म॑ । ए॒व । प्र॒ति॒पद॒मिति॑ प्रति-पद᳚म् । कु॒रु॒ते॒ । तथा᳚ । ब्रा॒ह्म॒णः । साशी᳚र्के॒णेति॒ स -आ॒शी॒र्के॒ण॒ । य॒ज्ञेन॑ । य॒ज॒ते॒ । यम् । का॒मये॑त । यज॑मानम् । भ्रातृ॑व्यम् । अ॒स्य॒ । य॒ज्ञ्स्य॑ । आ॒शीरित्या᳚ - शीः । ग॒च्छे॒त् । इति॑ । तस्य॑ । ए॒ताः । व्याहृ॑ती॒रिति॑ वि - आहृ॑तीः । पु॒रो॒नु॒वा॒क्या॑या॒मिति॑ पुरः - अ॒नु॒वा॒क्या॑याम् । द॒द्ध्या॒त् । भ्रा॒तृ॒व्य॒दे॒व॒त्येति॑ भ्रातृव्य - दे॒व॒त्या᳚ । वै । पु॒रो॒नु॒वा॒क्येति॑ पुरः - अ॒नु॒वा॒क्या᳚ । भ्रातृ॑व्यम् । ए॒व । अ॒स्य॒ । य॒ज्ञ्स्य॑ । \textbf{  36} \newline
                  \newline
                                \textbf{ TS 1.6.10.5} \newline
                  आ॒शीरित्या᳚ - शीः । ग॒च्छ॒ति॒ । यान् । का॒मये॑त । यज॑मानान् । स॒माव॑ती । ए॒ना॒न् । य॒ज्ञ्स्य॑ । आ॒शीरित्या᳚ - शीः । ग॒च्छे॒त् । इति॑ । तेषा᳚म् । ए॒ताः । व्याहृ॑ती॒रिति॑ वि - आहृ॑तीः । पु॒रो॒नु॒वा॒क्या॑या॒ इति॑ पुरः-अ॒नु॒वा॒क्या॑याः । अ॒द्‌र्ध॒र्च इत्य॒॑द्‌र्ध-ऋ॒चे । एका᳚म् । द॒द्ध्या॒त् । या॒ज्या॑यै । पु॒रस्ता᳚त् । एका᳚म् । या॒ज्या॑याः । अ॒द्‌र्ध॒र्च इत्य॑द्‌र्ध - ऋ॒चे । एका᳚म् । तथा᳚ । ए॒ना॒न् । स॒माव॑ती । य॒ज्ञ्स्य॑ । आ॒शीरित्या᳚ - शीः । ग॒च्छ॒ति॒ । यथा᳚ । वै । प॒र्जन्यः॑ । सुवृ॑ष्ट॒मिति॒ सु - वृ॒ष्ट॒म् । वर्.ष॑ति । ए॒वम् । य॒ज्ञ्ः । यज॑मानाय । व॒र्॒.ष॒ति॒ । स्थल॑या । उ॒द॒कम् । प॒रि॒गृ॒ह्णन्तीति॑ परि - गृ॒ह्णन्ति॑ । आ॒शिषेत्या᳚ - शिषा᳚ । य॒ज्ञ्म् । यज॑मानः । परीति॑ । गृ॒ह्णा॒ति॒ । मनः॑ । अ॒सि॒ । प्रा॒जा॒प॒त्यमिति॑ प्राजा - प॒त्यम् । \textbf{  37} \newline
                  \newline
                                \textbf{ TS 1.6.10.6} \newline
                  मन॑सा । मा॒ । भू॒तेन॑ । एति॑ । वि॒श॒ । इति॑ । आ॒ह॒ । मनः॑ । वै । प्रा॒जा॒प॒त्यमिति॑ प्राजा - प॒त्यम् । प्रा॒जा॒प॒त्य इति॑ प्राजा - प॒त्यः । य॒ज्ञ्ः । मनः॑ । ए॒व । य॒ज्ञ्म् । आ॒त्मन्न् । ध॒त्ते॒ । वाक् । अ॒सि॒ । ऐ॒न्द्री । स॒प॒त्न॒क्षय॒णीति॑ सपत्न - क्षय॑णी । वा॒चा । मा॒ । इ॒न्द्रि॒येण॑ । एति॑ । वि॒श॒ । इति॑ । आ॒ह॒ । ऐ॒न्द्री । वै । वाक् । वाच᳚म् । ए॒व । ऐ॒न्द्रीम् । आ॒त्मन्न् । ध॒त्ते॒ ॥ \textbf{  38} \newline
                  \newline
                      (तेनै॒-व ब्रह्म॑- रा॒ष्ट्रमे॒-वास्य॑ य॒ज्ञ्स्य॑-प्राजाप॒त्यꣳ-षट्त्रिꣳ॑शच्च)  \textbf{(A10)} \newline \newline
                                \textbf{ TS 1.6.11.1} \newline
                  यः । वै । स॒प्त॒द॒शमिति॑ सप्त - द॒शम् । प्र॒जाप॑ति॒मिति॑ प्र॒जा - प॒ति॒म् । य॒ज्ञ्म् । अ॒न्वाय॑त्त॒मित्य॑नु - आय॑त्तम् । वेद॑ । प्रतीति॑ । य॒ज्ञेन॑ । ति॒ष्ठ॒ति॒ । न । य॒ज्ञात् । भ्रꣳ॒॒श॒ते॒ । एति॑ । श्रा॒व॒य॒ । इति॑ । चतु॑रक्षर॒मिति॒ चतुः॑ - अ॒क्ष॒र॒म् । अस्तु॑ । श्रौष॑ट् । इति॑ । चतु॑रक्षर॒मिति॒ चतुः॑ - अ॒क्ष॒र॒म् । यज॑ । इति॑ । द्व्य॑क्षर॒मिति॒ द्वि - अ॒क्ष॒र॒म् । ये । यजा॑महे । इति॑ । पञ्चा᳚क्षर॒मिति॒ पञ्च॑ - अ॒क्ष॒र॒म् । द्व्य॒क्ष॒र इति॑ द्वि - अ॒क्ष॒रः । व॒ष॒ट्का॒र इति॑ वषट् - का॒रः । ए॒षः । वै । स॒प्त॒द॒श इति॑ सप्त - द॒शः । प्र॒जाप॑ति॒रिति॑ प्र॒जा - प॒तिः॒ । य॒ज्ञ्म् । अ॒न्वाय॑त्त॒ इत्य॑नु-आय॑त्तः । यः । ए॒वम् । वेद॑ । प्रतीति॑ । य॒ज्ञेन॑ । ति॒ष्ठ॒ति॒ । न । य॒ज्ञात् । भ्रꣳ॒॒श॒ते॒ । यः । वै । य॒ज्ञ्स्य॑ । प्राय॑ण॒मिति॑ प्र - अय॑नम् । प्र॒ति॒ष्ठामिति॑ प्रति - स्थाम् । \textbf{  39} \newline
                  \newline
                                \textbf{ TS 1.6.11.2} \newline
                  उ॒दय॑न॒मित्यु॑त् - अय॑नम् । वेद॑ । प्रति॑ष्ठिते॒नेति॒ प्रति॑ - स्थि॒ते॒न॒ । अरि॑ष्टेन । य॒ज्ञेन॑ । सꣳ॒॒स्थामिति॑ सं - स्थाम् । ग॒च्छ॒ति॒ । एति॑ । श्रा॒व॒य॒ । अस्तु॑ । श्रौष॑ट् । यज॑ । ये । यजा॑महे । व॒ष॒ट्का॒र इति॑ वषट् - का॒रः । ए॒तत् । वै । य॒ज्ञ्स्य॑ । प्राय॑ण॒मिति॑ प्र - अय॑नम् । ए॒षा । प्र॒ति॒ष्ठेति॑ प्रति - स्था । ए॒तत् । उ॒दय॑न॒मित्यु॑त् - अय॑नम् । यः । ए॒वम् । वेद॑ । प्रति॑ष्ठिते॒नेति॒ प्रति॑-स्थि॒ते॒न॒ । अरि॑ष्टेन । य॒ज्ञेन॑ । सꣳ॒॒स्थामिति॑ सं - स्थाम् । ग॒च्छ॒ति॒ । यः । वै । सू॒नृता॑यै । दोह᳚म् । वेद॑ । दु॒हे । ए॒व । ए॒ना॒म् । य॒ज्ञ्ः । वै । सू॒नृता᳚ । एति॑ । श्रा॒व॒य॒ । इति॑ । एति॑ । ए॒व । ए॒ना॒म् । अ॒ह्व॒त् । अस्तु॑ । \textbf{  40} \newline
                  \newline
                                \textbf{ TS 1.6.11.3} \newline
                  श्रौष॑ट् । इति॑ । उ॒पावा᳚स्रा॒गित्यु॑प-अवा᳚स्राक् । यज॑ । इति॑ । उदिति॑ । अ॒नै॒षी॒त् । ये । यजा॑महे । इति॑ । उपेति॑ । अ॒स॒द॒त् । व॒ष॒ट्का॒रेणेति॑ वषट्-का॒रेण॑ । दो॒ग्धि॒ । ए॒षः । वै । सू॒नृता॑यै । दोहः॑ । यः । ए॒वम् । वेद॑ । दु॒हे । ए॒व । ए॒ना॒म् । दे॒वाः । वै । स॒त्रम् । आ॒स॒त॒ । तेषा᳚म् । दिशः॑ । अ॒द॒स्य॒न्न् । ते । ए॒ताम् । आ॒र्द्राम् । प॒ङ्क्तिम् । अ॒प॒श्य॒न्न् । एति॑ । श्रा॒व॒य॒ । इति॑ । पु॒रो॒वा॒तमिति॑ पुरः - वा॒तम् । अ॒ज॒न॒य॒न्न् । अस्तु॑ । श्रौष॑ट् । इति॑ । अ॒भ्रम् । समिति॑ । अ॒प्ला॒व॒य॒न्न् । यज॑ । इति॑ । वि॒द्युत॒मिति॑ वि - द्युत᳚म् । \textbf{  41} \newline
                  \newline
                                \textbf{ TS 1.6.11.4} \newline
                  अ॒ज॒नय॒न्न् । ये । यजा॑महे । इति॑ । प्रेति॑ । अ॒व॒र्॒.ष॒य॒न्न् । अ॒भीति॑ । अ॒स्त॒न॒य॒न्न् । व॒ष॒ट्का॒रेणेति॑ वषट् - का॒रेण॑ । ततः॑ । वै । तेभ्यः॑ । दिशः॑ । प्रेति॑ । अ॒प्या॒य॒न्त॒ । यः । ए॒वम् । वेद॑ । प्रेति॑ । अ॒स्मै॒ । दिशः॑ । प्या॒य॒न्ते॒ । प्र॒जाप॑ति॒मिति॑ प्र॒जा - प॒ति॒म् । त्वो॒वेदेति॑ त्वः - वेद॑ । प्र॒जाप॑ति॒रिति॑ प्र॒जा - प॒तिः॒ । त्वं॒ॅवे॒देति॑  त्वं - वे॒द॒ । यम् । प्र॒जाप॑ति॒रिति॑ प्र॒जा-प॒तिः॒ । वेद॑ । सः । पुण्यः॑ । भ॒व॒ति॒ । ए॒षः । वै । छ॒न्द॒स्यः॑ । प्र॒जाप॑ति॒रिति॑ प्र॒जा - प॒तिः॒ । एति॑ । श्रा॒व॒य॒ । अस्तु॑ । श्रौष॑ट् । यज॑ । ये । यजा॑महे । व॒ष॒ट्का॒र इति॑ वषट् - का॒रः । यः । ए॒वम् । वेद॑ । पुण्यः॑ । भ॒व॒ति॒ । व॒स॒न्तम् । \textbf{  42} \newline
                  \newline
                                \textbf{ TS 1.6.11.5} \newline
                  ऋ॒तू॒नाम् । प्री॒णा॒मि॒ । इति॑ । आ॒ह॒ । ऋ॒तवः॑ । वै । प्र॒या॒जा इति॑ प्र - या॒जाः । ऋ॒तून् । ए॒व । प्री॒णा॒ति॒ । ते । अ॒स्मै॒ । प्री॒ताः । य॒था॒पू॒र्वमिति॑ यथा - पू॒र्वम् । क॒ल्प॒न्ते॒ । कल्प॑न्ते । अ॒स्मै॒ । ऋ॒तवः॑ । यः । ए॒वम् । वेद॑ । अ॒ग्नीषोम॑यो॒रित्य॒ग्नी - सोम॑योः । अ॒हम् । दे॒व॒य॒ज्ययेति॑ देव-य॒ज्यया᳚ । चक्षु॑ष्मान् । भू॒या॒स॒म् । इति॑ । आ॒ह॒ । अ॒ग्नीषोमा᳚भ्या॒मित्य॒ग्नी - सोमा᳚भ्याम् । वै । य॒ज्ञ्ः । चक्षु॑ष्मान् । ताभ्या᳚म् । ए॒व । चक्षुः॑ । आ॒त्मन्न् । ध॒त्ते॒ । अ॒ग्नेः । अ॒हम् । दे॒व॒य॒ज्ययेति॑ देव - य॒ज्यया᳚ । अ॒न्ना॒द इत्य॑न्न - अ॒दः । भू॒या॒स॒म् । इति॑ । आ॒ह॒ । अ॒ग्निः । वै । दे॒वाना᳚म् । अ॒न्ना॒द इत्य॑न्न - अ॒दः । तेन॑ । ए॒व । \textbf{  43} \newline
                  \newline
                                \textbf{ TS 1.6.11.6} \newline
                  अ॒न्नाद्य॒मित्य॑न्न - अद्य᳚म् । आ॒त्मन्न् । ध॒त्ते॒ । दब्धिः॑ । अ॒सि॒ । अद॑ब्धः । भू॒या॒स॒म् । अ॒मुम् । द॒भे॒य॒म् । इति॑ । आ॒ह॒ । ए॒तया᳚ । वै । दब्ध्या᳚ । दे॒वाः । असु॑रान् । अ॒द॒भ्नु॒व॒॒न्न् । तया᳚ । ए॒व । भ्रातृ॑व्यम् । द॒भ्नो॒ति॒ । अ॒ग्नीषोम॑यो॒रित्य॒ग्नी - सोम॑योः । अ॒हम् । दे॒व॒य॒ज्ययेति॑ देव - य॒ज्यया᳚ । वृ॒त्र॒हेति॑ वृत्र - हा । भू॒या॒स॒म् । इति॑ । आ॒ह॒ । अ॒ग्नीषोमा᳚भ्या॒मित्य॒ग्नी - सोमा᳚भ्याम् । वै । इन्द्रः॑ । वृ॒त्रम् । अ॒ह॒न्न् । ताभ्या᳚म् । ए॒व । भ्रातृ॑व्यम् । स्तृ॒णु॒ते॒ । इ॒न्द्रा॒ग्नि॒योरिती᳚न्द्र - अ॒ग्नि॒योः । अ॒हम् । दे॒व॒य॒ज्ययेति॑ देव - य॒ज्यया᳚ । इ॒न्द्रि॒या॒वी । अ॒न्ना॒द इत्य॑न्न - अ॒दः । भू॒या॒स॒म् । इति॑ । आ॒ह॒ । इ॒न्द्रि॒या॒वी । ए॒व । अ॒न्ना॒द इत्य॑न्न - अ॒दः । भ॒व॒ति॒ । इन्द्र॑स्य । \textbf{  44} \newline
                  \newline
                                \textbf{ TS 1.6.11.7} \newline
                  अ॒हम् । दे॒व॒य॒ज्ययेति॑ देव-य॒ज्यया᳚ । इ॒न्द्रि॒या॒वी । भू॒या॒स॒म् । इति॑ । आ॒ह॒ । इ॒न्द्रि॒या॒वी । ए॒व । भ॒व॒ति॒ । म॒हे॒न्द्रस्येति॑ महा - इ॒न्द्रस्य॑ । अ॒हम् । दे॒व॒य॒ज्ययेति॑ देव - य॒ज्यया᳚ । जे॒मान᳚म् । म॒हि॒मान᳚म् । ग॒मे॒य॒म् । इति॑ । आ॒ह॒ । जे॒मान᳚म् । ए॒व । म॒हि॒मान᳚म् । ग॒च्छ॒ति॒ । अ॒ग्नेः । स्वि॒ष्ट॒कृत॒ इति॑ स्विष्ट - कृतः॑ । अ॒हम् । दे॒व॒य॒ज्ययेति॑ देव - य॒ज्यया᳚ । आयु॑ष्मान् । य॒ज्ञेन॑ । प्र॒ति॒ष्ठामिति॑ प्रति- स्थाम् । ग॒मे॒य॒म् । इति॑ । आ॒ह॒ । आयुः॑ । ए॒व । आ॒त्मन्न् । ध॒त्ते॒ । प्रतीति॑ । य॒ज्ञेन॑ । ति॒ष्ठ॒ति॒ ॥ \textbf{  45} \newline
                  \newline
                      तेनै॒वे-न्द्र॑स्या॒-ऽष्टात्रिꣳ॑शच्च)  \textbf{(A11)} \newline \newline
                                \textbf{ TS 1.6.12.1} \newline
                  इन्द्र᳚म् । वः॒ । वि॒श्वतः॑ । परीति॑ । हवा॑महे । जने᳚भ्यः ॥ अ॒स्माक᳚म् । अ॒स्तु॒ । केव॑लः ॥ इन्द्र᳚म् । नरः॑ । ने॒मधि॒तेति॑ ने॒म - धि॒ता॒ । ह॒व॒न्ते॒ । यत् । पार्याः᳚ । यु॒नज॑ते । धियः॑ । ताः ॥ शूरः॑ । नृषा॒तेति॒ नृ - सा॒ता॒ । शव॑सः । च॒का॒नः । एति॑ । गोम॒तीति॒ गो - म॒ति॒ । व्र॒जे । भ॒ज॒ । त्वम् । नः॒ ॥ इ॒न्द्रि॒याणि॑ । श॒त॒क्र॒तो॒ इति॑ शत - क्र॒तो॒ । या । ते॒ । जने॑षु । प॒ञ्चस्विति॑॑ प॒ञ्च - सु॒ ॥ इन्द्र॑ । तानि॑ । ते॒ । एति॑ । वृ॒णे॒ ॥ अन्विति॑ । ते॒ । दा॒यि॒ । म॒हे । इ॒न्द्रि॒याय॑ । स॒त्रा । ते॒ । विश्व᳚म् । अन्विति॑ । वृ॒त्र॒हत्य॒ इति॑ वृत्र - हत्ये᳚ ॥ अन्विति॑ । \textbf{  46} \newline
                  \newline
                                \textbf{ TS 1.6.12.2} \newline
                  क्ष॒त्रम् । अन्विति॑ । सहः॑ । य॒ज॒त्र॒ । इन्द्र॑ । दे॒वेभिः॑ । अन्विति॑ । ते॒ । नृ॒षह्य॒ इति॑ नृ-सह्ये᳚ ॥ एति॑ । यस्मिन्न्॑ । स॒प्त । वा॒स॒वाः । तिष्ठ॑न्ति । स्वा॒रुह॒ इति॑ स्व - रुहः॑ । य॒था॒ ॥ ऋषिः॑ । ह॒ । दी॒र्घ॒श्रुत्त॑म॒ इति॑ दीर्घ॒श्रुत् - त॒मः॒ । इन्द्र॑स्य । घ॒र्मः । अति॑थिः ॥ आ॒मासु॑ । प॒क्वम् । ऐर॑यः । एति॑ । सूर्य᳚म् । रो॒ह॒यः॒ । दि॒वि ॥ घ॒र्मम् । न । सामन्न्॑ । त॒प॒त॒ । सु॒वृ॒क्तिभि॒रिति॑ सुवृ॒क्ति - भिः॒ । जुष्ट᳚म् । गिर्व॑णसे । गिरः॑ ॥ इन्द्र᳚म् । इत् । गा॒थिनः॑ । बृ॒हत् । इन्द्र᳚म् । अ॒र्केभिः॑ । अ॒र्किणः॑ ॥ इन्द्र᳚म् । वाणीः᳚ । अ॒नू॒ष॒त॒ ॥ गाय॑न्ति । त्वा॒ । गा॒य॒त्रिणः॑ । \textbf{  47} \newline
                  \newline
                                \textbf{ TS 1.6.12.3} \newline
                  अर्च॑न्ति । अ॒र्कम् । अ॒र्किणः॑ ॥ ब्र॒ह्माणः॑ । त्वा॒ । श॒त॒क्र॒त॒विति॑ शत - क्र॒तो॒ । उदिति॑ । वꣳ॒॒शम् । इ॒व॒ । ये॒मि॒रे॒ ॥ अꣳ॒॒हो॒मुच॒ इत्यꣳ॑हः - मुचे᳚ । प्रेति॑ । भ॒रे॒म॒ । म॒नी॒षाम् । ओ॒षि॒ष्ठ॒दाव्.न्न॒ इत्यो॑षिष्ठ - दाव्.न्ने᳚ । सु॒म॒तिमिति॑ सु - म॒तिम् । गृ॒णा॒नाः ॥ इ॒दम् । इ॒न्द्र॒ । प्रतीति॑ । ह॒व्यम् । गृ॒भा॒य॒ । स॒त्याः । स॒न्तु॒ । यज॑मानस्य । कामाः᳚ ॥ वि॒वेष॑ । यत् । मा॒ । धि॒षणा᳚ । ज॒जान॑ । स्तवै᳚ । पु॒रा । पार्या᳚त् । इन्द्र᳚म् । अह्नः॑ ॥ अꣳह॑सः । यत्र॑ । पी॒पर॑त् । यथा᳚ । नः॒ । ना॒वा । इ॒व॒ । यान्त᳚म् । उ॒भये᳚ । ह॒व॒न्ते॒ ॥ प्रेति॑ । स॒म्राज॒मिति॑ सं - राज᳚म् । प्र॒थ॒मम् । अ॒द्ध्व॒राणा᳚म् । \textbf{  48} \newline
                  \newline
                                \textbf{ TS 1.6.12.4} \newline
                  अꣳ॒॒हो॒मुच॒मित्यꣳ॑हः - मुच᳚म् । वृ॒ष॒भम् । य॒ज्ञिया॑नाम् ॥ अ॒पाम् । नपा॑तम् । अ॒श्वि॒ना॒ । हय॑न्तम् । अ॒स्मिन्न् । न॒रः॒ । इ॒न्द्रि॒यम् । ध॒त्त॒म् । ओजः॑ ॥ वीति॑ । नः॒ । इ॒न्द्र॒ । मृधः॑ । ज॒हि॒ । नी॒चा । य॒च्छ॒ । पृ॒त॒न्य॒तः ॥ अ॒ध॒स्प॒दमित्य॑धः - प॒दम् । तम् । ई॒म् । कृ॒धि॒ । यः । अ॒स्मान् । अ॒भि॒दास॒तीत्य॑भि - दास॑ति ॥ इन्द्र॑ । क्ष॒त्रम् । अ॒भीति॑ । वा॒मम् । ओजः॑ । अजा॑यथाः । वृ॒ष॒भ॒ । च॒र्॒.ष॒णी॒नाम् ॥ अपेति॑ । अ॒नु॒दः॒ । जन᳚म् । अ॒मि॒त्र॒यन्त॒मित्य॑मित्र - यन्त᳚म् । उ॒रुम् । दे॒वेभ्यः॑ । अ॒कृ॒णोः॒ । उ॒ । लो॒कम् ॥ मृ॒गः । न । भी॒मः । कु॒च॒रः । गि॒रि॒ष्ठा इति॑ गिरि - स्थाः । प॒रा॒वत॒ इति॑ परा - वतः॑ । \textbf{  49} \newline
                  \newline
                                \textbf{ TS 1.6.12.5} \newline
                  एति॑ । ज॒गा॒म॒ । पर॑स्याः ॥ सृ॒कम् । सꣳ॒॒शायेति॑ सं-शाय॑ । प॒विम् । इ॒न्द्र॒ । ति॒ग्मम् । वीति॑ । शत्रून्॑ । ता॒ढि॒ । वीति॑ । मृधः॑ । नु॒द॒स्व॒ ॥ वीति॑ । शत्रून्॑ । वीति॑ । मृधः॑ । नु॒द॒ । वीति॑ । वृ॒त्रस्य॑ । हनू॒ इति॑ । रु॒ज॒ ॥ वीति॑ । म॒न्युम् । इ॒न्द्र॒ । भा॒मि॒तः । अ॒मित्र॑स्य । अ॒भि॒दास॑त॒ इत्य॑भि - दास॑तः ॥ त्रा॒तार᳚म् । इन्द्र᳚म् । अ॒वि॒तार᳚म् । इन्द्र᳚म् । हवे॑हव॒ इति॒ हवे᳚ - ह॒वे॒ । सु॒हव॒मिति॑ सु - हव᳚म् । शूर᳚म् । इन्द्र᳚म् ॥ हु॒वे । नु । श॒क्रम् । पु॒रु॒हू॒तमिति॑ पुरु - हू॒तम् । इन्द्र᳚म् । स्व॒स्ति । नः॒ । म॒घवेति॑ म॒घ - वा॒ । धा॒तु॒ । इन्द्रः॑ ॥ मा । ते॒ । अ॒स्याम् । \textbf{  50} \newline
                  \newline
                                \textbf{ TS 1.6.12.6} \newline
                  स॒ह॒सा॒व॒न्निति॑ सहसा - व॒न्न् । परि॑ष्टौ । अ॒घाय॑ । भू॒म॒ । ह॒रि॒व॒ इति॑ हरि - वः॒ । प॒रा॒दा इति॑ परा - दै ॥ त्राय॑स्व । नः॒ । अ॒वृ॒केभिः॑ । वरू॑थैः । तव॑ । प्रि॒यासः॑ । सू॒रिषु॑ । स्या॒म॒ ॥ अन॑वः । ते॒ । रथ᳚म् । अश्वा॑य । त॒क्ष॒न्न् । त्वष्टा᳚ । वज्र᳚म् । पु॒रु॒हू॒तेति॑ पुरु - हू॒त॒ । द्यु॒मन्त॒मिति॑ द्यु - मन्त᳚म् ॥ ब्र॒ह्माणः॑ । इन्द्र᳚म् । म॒हय॑न्तः । अ॒र्कैः । अव॑र्धयन्न् । अह॑ये । हन्त॒वै । उ॒ ॥ वृष्णे᳚ । यत् । ते॒ । वृष॑णः । अ॒र्कम् । अर्चान्॑ । इन्द्र॑ । ग्रावा॑णः । अदि॑तिः । स॒जोषा॒ इति॑ स - जोषाः᳚ ॥ अ॒न॒श्वासः॑ । ये । प॒वयः॑ । अ॒र॒थाः । इन्द्रे॑षिता॒ इतीन्द्र॑ - इ॒षि॒ताः॒ । अ॒भ्यव॑र्त॒न्तेत्य॑भि - अव॑र्तन्त । दस्यून्॑ ॥ \textbf{  51} \newline
                  \newline
                      (वृ॒त्र॒हत्येऽनु॑-गाय॒त्रिणो᳚-ऽद्ध॒राणां᳚-परा॒वतो॒-ऽस्या-म॒ष्टाच॑त्वारिꣳशच्च)  \textbf{(A12)} \newline \newline
\textbf{praSna korvai with starting padams of 1 to 12 anuvAkams :-} \newline
(सन्त्वा॑ सिञ्चामि-ध्रु॒वो᳚-ऽस्य॒ग्निर्मा॑-ब॒र्॒.हिषो॒ऽह-मा प्या॑यता॒-मग॑न्म॒-यथा॒ वै-यो वै श्र॒द्धां- प्र॒जाप॑ति॒र् यज्ञान्-ध्रु॒वो॑ऽसीत्या॑ह॒-यो वै स॑प्तद॒श-मिन्द्रं॑ ॅवो॒-द्वाद॑श । ) \newline

\textbf{korvai with starting padams of1, 11, 21 series of pa~jcAtis :-} \newline
सन्त्वा॑-ब॒र्॒.हिषो॒ऽहं-ॅयथा॒ वा-ए॒वं ॅवि॒द्वा-ञ्छ्रौष॑ट्थ्-साहसाव॒-न्नेक॑पञ्चा॒शत् । \newline

\textbf{first and last padam of sixth praSnam:-} \newline
सन्त्वा॑-सिञ्चामि॒ दस्यून्॑ । \newline 


॥ हरिः॑ ॐ ॥॥ कृष्ण यजुर्वेदीय तैत्तिरीय संहितायां 
प्रथमकाण्डे षष्ठः प्रश्नः समाप्तः ॥ \newline
\pagebreak
\pagebreak
        


\end{document}
