\documentclass[17pt]{extarticle}
\usepackage{babel}
\usepackage{fontspec}
\usepackage{polyglossia}
\usepackage{extsizes}



\setmainlanguage{sanskrit}
\setotherlanguages{english} %% or other languages
\setlength{\parindent}{0pt}
\pagestyle{myheadings}
\newfontfamily\devanagarifont[Script=Devanagari]{AdishilaVedic}


\newcommand{\VAR}[1]{}
\newcommand{\BLOCK}[1]{}




\begin{document}
\begin{titlepage}
    \begin{center}
 
\begin{sanskrit}
    { \Large
    ॐ नमः परमात्मने, श्री महागणपतये नमः, 
श्री गुरुभ्यो नमः । ह॒रिः॒ ॐ ॥ 
    }
    \\
    \vspace{2.5cm}
    \mbox{ \Huge
    1.7     प्रथमकाण्डे सप्तमः प्रश्नः - (याजमान ब्राह्मणं)   }
\end{sanskrit}
\end{center}

\end{titlepage}
\tableofcontents

ॐ नमः परमात्मने, श्री महागणपतये नमः, 
श्री गुरुभ्यो नमः । ह॒रिः॒ ॐ ॥ \newline
1.7     प्रथमकाण्डे सप्तमः प्रश्नः - (याजमान ब्राह्मणं) \newline

\addcontentsline{toc}{section}{ 1.7     प्रथमकाण्डे सप्तमः प्रश्नः - (याजमान ब्राह्मणं)}
\markright{ 1.7     प्रथमकाण्डे सप्तमः प्रश्नः - (याजमान ब्राह्मणं) \hfill https://www.vedavms.in \hfill}
\section*{ 1.7     प्रथमकाण्डे सप्तमः प्रश्नः - (याजमान ब्राह्मणं) }
                                \textbf{ TS 1.7.1.1} \newline
                  पा॒क॒य॒ज्ञ्मिति॑ पाक - य॒ज्ञ्म् । वै । अन्विति॑ । आहि॑ताग्ने॒रित्याहि॑त - अ॒ग्नेः॒ । प॒शवः॑ । उपेति॑ । ति॒ष्ठ॒न्ते॒ । इडा᳚ । खलु॑ । वै । पा॒क॒य॒ज्ञ् इति॑ पाक - य॒ज्ञ्ः । सा । ए॒षा । अ॒न्त॒रा । प्र॒या॒जा॒नू॒या॒जानिति॑ प्रयाज - अ॒नू॒या॒जान् । यज॑मानस्य । लो॒के । अव॑हि॒तेयव॑ - हि॒ता॒ । ताम् । आ॒ह्रि॒यमा॑णा॒मित्या᳚ - ह्रि॒यमा॑णाम् । अ॒भीति॑ । म॒न्त्र॒ये॒त॒ । सुरू॑पवर्.षवर्ण॒ इति॒ सुरू॑प -व॒र्॒.ष॒व॒र्णे॒ । एति॑ । इ॒हि॒ । इति॑ । प॒शवः॑ । वै । इडा᳚ । प॒शून् । ए॒व । उपेति॑ । ह्व॒य॒ते॒ । य॒ज्ञ्म् । वै । दे॒वाः । अदु॑ह्रन्न् । य॒ज्ञ्ः । असु॑रान् । अ॒दु॒ह॒त् । ते । असु॑राः । य॒ज्ञ्दु॑ग्धा॒ इति॑ य॒ज्ञ् - दु॒ग्धाः॒ । परेति॑ । अ॒भ॒व॒न्न् । यः । वै । य॒ज्ञ्स्य॑ । दोह᳚म् । वि॒द्वान् । \textbf{  1} \newline
                  \newline
                                \textbf{ TS 1.7.1.2} \newline
                  यज॑ते । अपीति॑ । अ॒न्यम् । यज॑मानम् । दु॒हे॒ । सा । मे॒ । स॒त्या । आ॒शीरित्या᳚-शीः । अ॒स्य । य॒ज्ञ्स्य॑ । भू॒या॒त् । इति॑ । आ॒ह॒ । ए॒षः । वै । य॒ज्ञ्स्य॑ । दोहः॑ । तेन॑ । ए॒व । ए॒न॒म् । दु॒हे॒ । प्रत्ता᳚ । वै । गौः । दु॒हे॒ । प्रत्ता᳚ । इडा᳚ । यज॑मानाय । दु॒हे॒ । ए॒ते । वै । इडा॑यै । स्तनाः᳚ । इडा᳚ । उप॑हू॒तेत्युप॑ - हू॒ता॒ । इति॑ । वा॒युः । व॒थ्सः । यर्.हि॑ । होता᳚ । इडा᳚म् । उ॒प॒ह्वय॒तेत्यु॑प - ह्वये॑त । तर्.हि॑ । यज॑मानः । होता॑रम् । ईक्ष॑माणः । वा॒युम् । मन॑सा । ध्या॒ये॒त् । \textbf{  2} \newline
                  \newline
                                \textbf{ TS 1.7.1.3} \newline
                  मा॒त्रे । व॒थ्सम् । उ॒पाव॑सृज॒तीत्यु॑प - अव॑सृजति । सर्वे॑ण । वै । य॒ज्ञेन॑ । दे॒वाः । सु॒व॒र्गमिति॑ सुवः - गम् । लो॒कम् । आ॒य॒न्न् । पा॒क॒य॒ज्ञेनेति॑ पाक-य॒ज्ञेन॑ । मनुः॑ । अ॒श्रा॒म्य॒त् । सा । इडा᳚ । मनु᳚म् । उ॒पाव॑र्त॒तेत्यु॑प - आव॑र्तत । ताम् । दे॒वा॒सु॒रा इति॑ देव - अ॒सु॒राः । वीति॑ । अ॒ह्व॒य॒न्त॒ । प्र॒तीची᳚म् । दे॒वाः । परा॑चीम् । असु॑राः । सा । दे॒वान् । उ॒पाव॑र्त॒तेत्यु॑प - आव॑र्तत । प॒शवः॑ । वै । तत् । दे॒वान् । अ॒वृ॒ण॒त॒ । प॒शवः॑ । असु॑रान् । अ॒ज॒हुः॒ । यम् । का॒मये॑त । अ॒प॒शुः । स्या॒त् । इति॑ । परा॑चीम् । तस्य॑ । इडा᳚म् । उपेति॑ । ह्व॒ये॒त॒ । अ॒प॒शुः । ए॒व । भ॒व॒ति॒ । यम् । \textbf{  3} \newline
                  \newline
                                \textbf{ TS 1.7.1.4} \newline
                  का॒मये॑त । प॒शु॒मानिति॑ पशु - मान् । स्या॒त् । इति॑ । प्र॒तीची᳚म् । तस्य॑ । इडा᳚म् । उपेति॑ । ह्व॒ये॒त॒ । प॒शु॒मानिति॑ पशु - मान् । ए॒व । भ॒व॒ति॒ । ब्र॒ह्म॒वा॒दिन॒ इति॑ ब्रह्म - वा॒दिनः॑ । व॒द॒न्ति॒ । सः । तु । वै । इडा᳚म् । उपेति॑ । ह्व॒ये॒त॒ । यः । इडा᳚म् । उ॒प॒हूयेत्यु॑प - हूय॑ । आ॒त्मान᳚म् । इडा॑याम् । उ॒प॒ह्वये॒तेत्यु॑प - ह्वये॑त । इति॑ । सा । नः॒ । प्रि॒या । सु॒प्रतू᳚र्ति॒रिति॑ सु - प्रतू᳚र्तिः । म॒घोनी᳚ । इति॑ । आ॒ह॒ । इडा᳚म् । ए॒व । उ॒प॒हूयेत्यु॑प - हूय॑ । आ॒त्मान᳚म् । इडा॑याम् । उपेति॑ । ह्व॒य॒ते॒ । व्य॑स्त॒मिति॒ वि-अ॒स्त॒म् । इ॒व॒ । वै । ए॒तत् । य॒ज्ञ्स्य॑ । यत् । इडा᳚ । सा॒मि । प्रा॒श्नन्तीति॑ प्र - अ॒श्नन्ति॑ । \textbf{  4} \newline
                  \newline
                                \textbf{ TS 1.7.1.5} \newline
                  सा॒मि । मा॒र्ज॒य॒न्ते॒ । ए॒तत् । प्रतीति॑ । वै । असु॑राणाम् । य॒ज्ञ्ः । वीति॑ । अ॒च्छि॒द्य॒त॒ । ब्रह्म॑णा । दे॒वाः । समिति॑ । अ॒द॒धुः॒ । बृह॒स्पतिः॑ । त॒नु॒ता॒म् । इ॒मम् । नः॒ । इति॑ । आ॒ह॒ । ब्रह्म॑ । वै । दे॒वाना᳚म् । बृह॒स्पतिः॑ । ब्रह्म॑णा । ए॒व । य॒ज्ञ्म् । समिति॑ । द॒धा॒ति॒ । विच्छि॑न्न॒मिति॒ वि - छि॒न्न॒म् । य॒ज्ञ्म् । समिति॑ । इ॒मम् । द॒धा॒तु॒ । इति॑ । आ॒ह॒ । संत॑त्या॒ इति॒ सं - त॒त्यै॒ । विश्वे᳚ । दे॒वाः । इ॒ह । मा॒द॒य॒न्ता॒म् । इति॑ । आ॒ह॒ । स॒तंत्येति॑ सं - तत्य॑ । ए॒व । य॒ज्ञ्म् । दे॒वेभ्यः॑ । अन्विति॑ । दि॒श॒ति॒ । याम् । वै । \textbf{  5} \newline
                  \newline
                                \textbf{ TS 1.7.1.6} \newline
                  य॒ज्ञे । दक्षि॑णाम् । ददा॑ति । ताम् । अ॒स्य॒ । प॒शवः॑ । अनु॑ । समिति॑ । क्रा॒म॒न्ति॒ । सः । ए॒षः । ई॒जा॒नः । अ॒प॒शुः । भावु॑कः । यज॑मानेन । खलु॑ । वै । तत् । का॒र्य᳚म् । इति॑ । आ॒हुः॒ । यथा᳚ । दे॒व॒त्रेति॑ देव - त्रा । द॒त्तम् । कु॒र्वी॒त । आ॒त्मन्न् । प॒शून् । र॒मये॑त । इति॑ । ब्रद्ध्न॑ । पिन्व॑स्व । इति॑ । आ॒ह॒ । य॒ज्ञ्ः । वै । ब्र॒द्ध्नः । य॒ज्ञ्म् । ए॒व । तत् । म॒ह॒य॒ति॒ । अथो॒ इति॑ । दे॒व॒त्रेति॑ देव - त्रा । ए॒व । द॒त्तम् । कु॒रु॒ते॒ । आ॒त्मन्न् । प॒शून् । र॒म॒य॒ते॒ । दद॑तः । मे॒ ( ) । मा । क्षा॒यि॒ । इति॑ । आ॒ह॒ । अक्षि॑तिम् । ए॒व । उपेति॑ । ए॒ति॒ । कु॒र्व॒तः । मे॒ । मा । उपेति॑ । द॒स॒त् । इति॑ । आ॒ह॒ । भू॒मान᳚म् । ए॒व । उपेति॑ । ए॒ति॒ ॥ \textbf{  6 } \newline
                  \newline
                      (वि॒द्वान्-ध्या॑येद्-भवति॒ यं-प्रा॒श्ञन्ति॒-यां ॅवै-म॒-एका॒न्नविꣳ॑श॒तिश्च॑ )  \textbf{(A1)} \newline \newline
                                \textbf{ TS 1.7.2.1} \newline
                  सꣳश्र॑वा॒ इति॒ सं - श्र॒वाः॒ । ह॒ । सौ॒व॒र्च॒न॒सः । तुमि॑ञ्जम् । औपो॑दिति॒मित्यौप॑ - उ॒दि॒ति॒म् । उ॒वा॒च॒ । यत् । स॒त्रिणा᳚म् । होता᳚ । अभूः᳚ । काम् । इडा᳚म् । उपेति॑ । अ॒ह्व॒थाः॒ । इति॑ । ताम् । उपेति॑ । अ॒ह्वे॒ । इति॑ । ह॒ । उ॒वा॒च॒ । या । प्रा॒णेनेति॑ प्र - अ॒नेन॑ । दे॒वान् । दा॒धार॑ । व्या॒नेनेति॑ वि - अ॒नेन॑ । म॒नु॒ष्यान्॑ । अ॒पा॒नेनेत्य॑प-अ॒नेन॑ । पि॒तॄन् । इति॑ । छि॒नत्ति॑ । सा । न । छि॒न॒त्ती(3) । इति॑ । छि॒नत्ति॑ । इति॑ । ह॒ । उ॒वा॒च॒ । शरी॑रम् । वै । अ॒स्यै॒ । तत् । उपेति॑ । अ॒ह्व॒थाः॒ । इति॑ । ह॒ । उ॒वा॒च॒ । गौः । वै । \textbf{  7} \newline
                  \newline
                                \textbf{ TS 1.7.2.2} \newline
                  अ॒स्यै॒ । शरी॑रम् । गाम् । वाव । तौ । तत् । परीति॑ । अ॒व॒द॒ता॒म् । या । य॒ज्ञे । दी॒यते᳚ । सा । प्रा॒णेनेति॑ प्र - अ॒नेन॑ । दे॒वान् । दा॒धा॒र॒ । यया᳚ । म॒नु॒ष्याः᳚ । जीव॑न्ति । सा । व्या॒नेनेति॑ वि - अ॒नेन॑ । म॒नु॒ष्यान्॑ । याम् । पि॒तृभ्य॒ इति॑ पि॒तृ - भ्यः॒ । घ्नन्ति॑ । सा । अ॒पा॒नेनेत्य॑प - अ॒नेन॑ । पि॒तॄन् । यः । ए॒वम् । वेद॑ । प॒शु॒मानिति॑ पशु-मान् । भ॒व॒ति॒ । अथ॑ । वै । ताम् । उपेति॑ । अ॒ह्वे॒ । इति॑ । ह॒ । उ॒वा॒च॒ । या । प्र॒जा इति॑ प्र - जाः । प्र॒भव॑न्ती॒रिति॑ प्र - भव॑न्तीः । प्रतीति॑ । आ॒भव॒तीत्या᳚ - भव॑ति । इति॑ । अन्न᳚म् । वै । अ॒स्यै॒ । तत् । \textbf{  8} \newline
                  \newline
                                \textbf{ TS 1.7.2.3} \newline
                  उपेति॑ । अ॒ह्व॒थाः॒ । इति॑ । ह॒ । उ॒वा॒च॒ । ओष॑धयः । वै । अ॒स्याः॒ । अन्न᳚म् । ओष॑धयः । वै । प्र॒जा इति॑ प्र - जाः । प्र॒भव॑न्ती॒रिति॑ प्र - भव॑न्तीः । प्रति॑ । एति॑ । भ॒व॒न्ति॒ । यः । ए॒वम् । वेद॑ । अ॒न्ना॒द इत्य॑न्न - अ॒दः । भ॒व॒ति॒ । अथ॑ । वै । ताम् । उपेति॑ । अ॒ह्वे॒ । इति॑ । ह॒ । उ॒वा॒च॒ । या । प्र॒जा इति॑ प्र - जाः । प॒रा॒भव॑न्ती॒रिति॑ परा - भव॑न्तीः । अ॒नु॒गृ॒ह्णातीत्य॑नु - गृ॒ह्णाति॑ । प्रतीति॑ । आ॒भव॑न्ती॒रित्या᳚ - भव॑न्तीः । गृ॒ह्णाति॑ । इति॑ । प्र॒ति॒ष्ठामिति॑ प्रति - स्थाम् । वै । अ॒स्यै॒ । तत् । उपेति॑ । अ॒ह्व॒थाः॒ । इति॑ । ह॒ । उ॒वा॒च॒ । इ॒यम् । वै । अ॒स्यै॒ । प्र॒ति॒ष्ठेति॑ प्रति - स्था । \textbf{  9} \newline
                  \newline
                                \textbf{ TS 1.7.2.4} \newline
                  इ॒यम् । वै । प्र॒जा इति॑ प्र - जाः । प॒रा॒भव॑न्ती॒रिति॑ परा - भव॑न्तीः । अन्विति॑॑ । गृ॒ह्णा॒ति॒ । प्रतीति॑ । आ॒भव॑न्ती॒रित्या᳚ - भव॑न्तीः । गृ॒ह्णा॒ति॒ । यः । ए॒वम् । वेद॑ । प्रतीति॑ । ए॒व । ति॒ष्ठ॒ति॒ । अथ॑ । वै । ताम् । उपेति॑ । अ॒ह्वे॒ । इति॑ । ह॒ । उ॒वा॒च॒ । यस्यै᳚ । नि॒क्रम॑ण॒ इति॑ नि - क्रम॑णे । घृ॒तम् । प्र॒जा इति॑ प्र - जाः । स॒जींव॑न्ती॒रिति॑ सं - जीव॑न्तीः । पिब॑न्ति । इति॑ । छि॒नत्ति॑ । सा । न । छि॒न॒त्ती(3) । इति॑ । न । छि॒न॒त्ति॒ । इति॑ । ह॒ । उ॒वा॒च॒ । प्रेति॑ । तु । ज॒न॒य॒ति॒ । इति॑ । ए॒षः । वै । इडा᳚म् । उपेति॑ । अ॒ह्व॒थाः॒ । इति॑ ( ) । ह॒ । उ॒वा॒च॒ । वृष्टिः॑ । वै । इडा᳚ । वृष्‌ट्यै᳚ । वै । नि॒क्रम॑ण॒ इति॑ नि - क्रम॑णे । घृ॒तम् । प्र॒जा इति॑ प्र - जाः । स॒जींव॑न्ती॒रिति॑ सं - जीव॑न्तीः । पि॒ब॒न्ति॒ । यः । ए॒वम् । वेद॑ । प्रेति॑ । ए॒व । जा॒य॒ते॒ । अ॒न्ना॒द इत्य॑न्न - अ॒दः । भ॒व॒ति॒ ॥ \textbf{  10} \newline
                  \newline
                      (गौर्वा-अ॑स्यै॒ तत्-प्र॑ति॒ष्ठा-ऽह्व॑था॒ इति॑-विꣳश॒तिश्च॑)  \textbf{(A2)} \newline \newline
                                \textbf{ TS 1.7.3.1} \newline
                  प॒रोक्ष॒मिति॑ परः - अक्ष᳚म् । वै । अ॒न्ये । दे॒वाः । इ॒ज्यन्ते᳚ । प्र॒त्यक्ष॒मिति॑ प्रति - अक्ष᳚म् । अ॒न्ये । यत् । यज॑ते । ये । ए॒व । दे॒वाः । प॒रोक्ष॒मिति॑ परः - अक्ष᳚म् । इ॒ज्यन्ते᳚ । तान् । ए॒व । तत् । य॒ज॒ति॒ । यत् । अ॒न्वा॒हा॒र्य॑मित्य॑नु - आ॒हा॒र्य᳚म् । आ॒हर॒तीत्या᳚-हर॑ति । ए॒ते । वै । दे॒वाः । प्र॒त्यक्ष॒मिति॑ प्रति - अक्ष᳚म् । यत् । ब्रा॒ह्म॒णाः । तान् । ए॒व । तेन॑ । प्री॒णा॒ति॒ । अथो॒ इति॑ । दक्षि॑णा । ए॒व । अ॒स्य॒ । ए॒षा । अथो॒ इति॑ । य॒ज्ञ्स्य॑ । ए॒व । छि॒द्रम् । अपीति॑ । द॒धा॒ति॒ । यत् । वै । य॒ज्ञ्स्य॑ । क्रू॒रम् । यत् । विलि॑ष्ट॒मिति॒ वि - लि॒ष्ट॒म् । तत् । अ॒न्वा॒हा॒र्ये॑णेत्य॑नु - आ॒हा॒र्ये॑ण । \textbf{  11} \newline
                  \newline
                                \textbf{ TS 1.7.3.2} \newline
                  अ॒न्वाह॑र॒तीत्य॑नु - आह॑रति । तत् । अ॒न्वा॒हा॒र्य॑स्येत्य॑नु-आ॒हा॒र्य॑स्य । अ॒न्वा॒हा॒र्य॒त्वमित्य॑न्वाहार्य - त्वम् । दे॒व॒दू॒ता इति॑ देव - दू॒ताः । वै । ए॒ते । यत् । ऋ॒त्विजः॑ । यत् । अ॒न्वा॒हा॒र्य॑मित्य॑नु - आ॒हा॒र्य᳚म् । आ॒हर॒तीत्या᳚ - हर॑ति । दे॒व॒दू॒तानिति॑ देव - दू॒तान् । ए॒व । प्री॒णा॒ति॒ । प्र॒जाप॑ति॒रिति॑ प्र॒जा - प॒तिः॒ । दे॒वेभ्यः॑ । य॒ज्ञान् । व्यादि॑श॒दिति॑ वि - आदि॑शत् । सः । रि॒रि॒चा॒नः । अ॒म॒न्य॒त॒ । सः । ए॒तम् । अ॒न्वा॒हा॒र्य॑मित्य॑नु - आ॒हा॒र्य᳚म् । अभ॑क्तम् । अ॒प॒श्य॒त् । तम् । आ॒त्मन्न् । अ॒ध॒त्त॒ । सः । वै । ए॒षः । प्रा॒जा॒प॒त्य इति॑ प्राजा-प॒त्यः । यत् । अ॒न्वा॒हा॒र्य॑ इत्य॑नु - आ॒हा॒र्यः॑ । यस्य॑ । ए॒वम् । वि॒दुषः॑ । अ॒न्वा॒हा॒र्य॑ इत्य॑नु - आ॒हा॒र्यः॑ । आ॒ह्रि॒यत॒ इत्या᳚ - ह्रि॒यते᳚ । सा॒क्षादिति॑ स - अ॒क्षात् । ए॒व । प्र॒जाप॑ति॒मिति॑ प्र॒जा - प॒ति॒म् । ऋ॒द्ध्नो॒ति॒ । अप॑रिमित॒ इत्यप॑रि - मि॒तः॒ । नि॒रुप्य॒ इति॑ निः-उप्यः॑ । अप॑रिमित॒ इत्यप॑रि - मि॒तः॒ । प्र॒जाप॑ति॒रिति॑ प्र॒जा - प॒तिः॒ । प्र॒जाप॑ते॒रिति॑ प्र॒जा - प॒तेः॒ । \textbf{  12} \newline
                  \newline
                                \textbf{ TS 1.7.3.3} \newline
                  आप्त्यै᳚ । दे॒वाः । वै । यत् । य॒ज्ञे । अकु॑र्वत । तत् । असु॑राः । अ॒कु॒र्व॒त॒ । ते । दे॒वाः । ए॒तम् । प्रा॒जा॒प॒त्यमिति॑ प्राजा - प॒त्यम् । अ॒न्वा॒हा॒र्य॑मित्य॑नु - आ॒हा॒र्य᳚म् । अ॒प॒श्य॒न्न् । तम् । अ॒न्वाह॑र॒न्तेत्य॑नु - आह॑रन्त । ततः॑ । दे॒वाः । अभ॑वन्न् । परेति॑ । असु॑राः । यस्य॑ । ए॒वम् । वि॒दुषः॑ । अ॒न्वा॒हा॒र्य॑ इत्य॑नु - आ॒हा॒र्यः॑ । आ॒ह्रि॒यत॒ इत्या᳚ - ह्रि॒यते᳚ । भव॑ति । आ॒त्मना᳚ । परेति॑ । अ॒स्य॒ । भ्रातृ॑व्यः । भ॒व॒ति॒ । य॒ज्ञेन॑ । वै । इ॒ष्टी । प॒क्वेन॑ । पू॒र्ती । यस्य॑ । ए॒वम् । वि॒दुषः॑ । अ॒न्वा॒हा॒र्य॑ इत्य॑नु - आ॒हा॒र्यः॑ । आ॒ह्रि॒यत॒ इत्या᳚ - ह्रि॒यते᳚ । सः । तु । ए॒व । इ॒ष्टा॒पू॒र्तीती᳚ष्ट - पू॒र्ती । प्र॒जाप॑ते॒रिति॑ प्र॒जा - प॒तेः॒ । भा॒गः । अ॒सि॒ । \textbf{  13} \newline
                  \newline
                                \textbf{ TS 1.7.3.4} \newline
                  इति॑ । आ॒ह॒ । प्र॒जाप॑ति॒मिति॑ प्र॒जा - प॒ति॒म् । ए॒व । भा॒ग॒धेये॒नेति॑ भाग - धेये॑न । समिति॑ । अ॒द्‌र्ध॒य॒ति॒ । ऊर्ज॑स्वान् । पय॑स्वान् । इति॑ । आ॒ह॒ । ऊर्ज᳚म् । ए॒व । अ॒स्मि॒न्न् । पयः॑ । द॒धा॒ति॒ । प्रा॒णा॒पा॒नाविति॑ प्राण -अ॒पा॒नौ । मे॒ । पा॒हि॒ । स॒मा॒न॒व्या॒नाविति॑ समान - व्या॒नौ । मे॒ । पा॒हि॒ । इति॑ । आ॒ह॒ । आ॒शिष॒मित्या᳚ - शिष᳚म् । ए॒व । ए॒ताम् । एति॑ । शा॒स्ते॒ । अक्षि॑तः । अ॒सि॒ । अक्षि॑त्यै । त्वा॒ । मा । मे॒ । क्षे॒ष्ठाः॒ । अ॒मुत्र॑ । अ॒मुष्मिन्न्॑ । लो॒के । इति॑ । आ॒ह॒ । क्षीय॑ते । वै । अ॒मुष्मिन्न्॑ । लो॒के । अन्न᳚म् । इ॒तः प्र॑दान॒मिती॒तः - प्र॒दा॒न॒म् । हि । अ॒मुष्मिन्न्॑ । लो॒के ( ) । प्र॒जा इति॑ प्र - जाः । उ॒प॒जीव॒न्तीत्यु॑प - जीव॑न्ति । यत् । ए॒वम् । अ॒भि॒मृ॒शतीत्य॑भि - मृ॒शति॑ । अक्षि॑तिम् । ए॒व । ए॒न॒त् । ग॒म॒य॒ति॒ । न । अ॒स्य॒ । अ॒मुष्मिन्न्॑ । लो॒के । अन्न᳚म् । क्षी॒य॒ते॒ ॥ \textbf{  14} \newline
                  \newline
                      (अ॒न्वा॒हा॒र्ये॑ण-प्र॒जाप॑ते-रसि॒-ह्य॑मुष्मि॑न् ॅलो॒के-पञ्च॑दश च )  \textbf{(A3)} \newline \newline
                                \textbf{ TS 1.7.4.1} \newline
                  ब॒र्॒.हिषः॑ । अ॒हम् । दे॒व॒य॒ज्ययेति॑ देव - य॒ज्यया᳚ । प्र॒जावा॒निति॑ प्र॒जा -वा॒न् । भू॒या॒स॒म् । इति॑ । आ॒ह॒ । ब॒र॒.हिषा᳚ । वै । प्र॒जाप॑ति॒रिति॑ प्र॒जा - प॒तिः॒ । प्र॒जा इति॑ प्र - जाः । अ॒सृ॒ज॒त॒ । तेन॑ । ए॒व । प्र॒जा इति॑ प्र - जाः । सृ॒ज॒ते॒ । नरा॒शꣳस॑स्य । अ॒हम् । दे॒व॒य॒ज्ययेति॑ देव - य॒ज्यया᳚ । प॒शु॒मानिति॑ पशु-मान् । भू॒या॒स॒म् । इति॑ । आ॒ह॒ । नरा॒शꣳसे॑न । वै । प्र॒जाप॑ति॒रिति॑ प्र॒जा - प॒तिः॒ । प॒शून् । अ॒सृ॒ज॒त॒ । तेन॑ । ए॒व । प॒शून् । सृ॒ज॒ते॒ । अ॒ग्नेः । स्वि॒ष्ट॒कृत॒ इति॑ स्विष्ट - कृतः॑ । अ॒हम् । दे॒व॒य॒ज्ययेति॑ देव - य॒ज्यया᳚ । आयु॑ष्मान् । य॒ज्ञेन॑ । प्र॒ति॒ष्ठामिति॑ प्रति-स्थाम् । ग॒मे॒य॒म् । इति॑ । आ॒ह॒ । आयुः॑ । ए॒व । आ॒त्मन्न् । ध॒त्ते॒ । प्रतीति॑ । य॒ज्ञेन॑ । ति॒ष्ठ॒ति॒ । द॒र॒.श॒पू॒र्ण॒मा॒सयो॒रिति॑ दर्.श - पू॒र्ण॒मा॒सयोः᳚ । \textbf{  15} \newline
                  \newline
                                \textbf{ TS 1.7.4.2} \newline
                  वै । दे॒वाः । उज्जि॑ति॒मित्युत् - जि॒ति॒म् । अनु॑ । उदिति॑ । अ॒ज॒य॒न्न् । द॒र्॒.श॒पू॒र्ण॒मा॒साभ्या॒मिति॑ दर्.श-पू॒र्ण॒मा॒साभ्या᳚म् । असु॑रान् । अपेति॑ । अ॒नु॒द॒न्त॒ । अ॒ग्नेः । अ॒हम् । उज्जि॑ति॒मित्युत् - जि॒ति॒म् । अनु॑ । उदिति॑ । जे॒ष॒म् । इति॑ । आ॒ह॒ । द॒र्॒.श॒पू॒र्ण॒मा॒सयो॒रिति॑ दर्.श - पू॒र्ण॒मा॒सयोः᳚ । ए॒व । दे॒वता॑नाम् । यज॑मानः । उज्जि॑ति॒मित्युत् - जि॒ति॒म् । अनु॑ । उदिति॑ । ज॒य॒ति॒ । द॒र॒.श॒पू॒र्ण॒मा॒साभ्या॒मिति॑ दर्.श-पू॒र्ण॒मा॒साभ्या᳚म् । भ्रातृ॑व्यान् । अपेति॑ । नु॒द॒ते॒ । वाज॑वतीभ्या॒मिति॒ वाज॑ - व॒ती॒भ्या॒म् । वीति॑ । ऊ॒ह॒ति॒ । अन्न᳚म् । वै । वाजः॑ । अन्न᳚म् । ए॒व । अवेति॑ । रु॒न्धे॒ । द्वाभ्या᳚म् । प्रति॑ष्ठित्या॒ इति॒ प्रति॑ - स्थि॒त्यै॒ । यः । वै । य॒ज्ञ्स्य॑ । द्वौ । दोहौ᳚ । वि॒द्वान् । यज॑ते । उ॒भ॒यतः॑ । \textbf{  16} \newline
                  \newline
                                \textbf{ TS 1.7.4.3} \newline
                  ए॒व । य॒ज्ञ्म् । दु॒हे॒ । पु॒रस्ता᳚त् । च॒ । उ॒परि॑ष्टात् । च॒ । ए॒षः । वै । अ॒न्यः । य॒ज्ञ्स्य॑ । दोहः॑ । इडा॑याम् । अ॒न्यः । यर्.हि॑ । होता᳚ । यज॑मानस्य । नाम॑ । गृ॒ह्णी॒यात् । तर्.हि॑ । ब्रू॒या॒त् । एति॑ । इ॒माः । अ॒ग्म॒न्न् । आ॒शिष॒ इत्या᳚ - शिषः॑ । दोह॑कामा॒ इति॒ दोह॑ - का॒माः॒ । इति॑ । सꣳस्तु॑ता॒ इति॒ सं - स्तु॒ताः॒ । ए॒व । दे॒वताः᳚ । दु॒हे॒ । अथो॒ इति॑ । उ॒भ॒यतः॑ । ए॒व । य॒ज्ञ्म् । दु॒हे॒ । पु॒रस्ता᳚त् । च॒ । उ॒परि॑ष्ठत् । च॒ । रोहि॑तेन । त्वा॒ । अ॒ग्निः । दे॒वता᳚म् । ग॒म॒य॒तु॒ । इति॑ । आ॒ह॒ । ए॒ते । वै । दे॒वा॒श्वा इति॑ देव - अ॒श्वाः । \textbf{  17} \newline
                  \newline
                                \textbf{ TS 1.7.4.4} \newline
                  यज॑मानः । प्र॒स्त॒र इति॑ प्र - स्त॒रः । यत् । ए॒तैः । प्र॒स्त॒रमिति॑ प्र - स्त॒रम् । प्र॒हर॒तीति॑ प्र - हर॑ति । दे॒वा॒श्वैरिति॑ देव - अ॒श्वैः । ए॒व । यज॑मानम् । सु॒व॒र्गमिति॑ सुवः - गम् । लो॒कम् । ग॒म॒य॒ति॒ । वीति॑ । ते॒ । मु॒ञ्चा॒मि॒ । र॒श॒नाः । वीति॑ । र॒श्मीन् । इति॑ । आ॒ह॒ । ए॒षः । वै । अ॒ग्नेः । वि॒मो॒क इति॑ वि - मो॒कः । तेन॑ । ए॒व । ए॒न॒म् । वीति॑ । मु॒ञ्च॒ति॒ । विष्णोः᳚ । शं॒ॅयोरिति॑ शं - योः । अ॒हम् । दे॒व॒य॒ज्ययेति॑ देव - य॒ज्यया᳚ । य॒ज्ञेन॑ । प्र॒ति॒ष्ठामिति॑ प्रति - स्थाम् । ग॒मे॒य॒म् । इति॑ । आ॒ह॒ । य॒ज्ञ्ः । वै । विष्णुः॑ । य॒ज्ञे । ए॒व । अ॒न्त॒तः । प्रतीति॑ । ति॒ष्ठ॒ति॒ । सोम॑स्य । अ॒हम् । दे॒व॒य॒ज्ययेति॑ देव - य॒ज्यया᳚ । सु॒रेता॒ इति॑ सु - रेताः᳚ । \textbf{  18} \newline
                  \newline
                                \textbf{ TS 1.7.4.5} \newline
                  रेतः॑ । धि॒षी॒य॒ । इति॑ । आ॒ह॒ । सोमः॑ । वै । रे॒तो॒धा इति॑ रेतः-धाः । तेन॑ । ए॒व । रेतः॑ । आ॒त्मन्न् । ध॒त्ते॒ । त्वष्टुः॑ । अ॒हम् । दे॒व॒य॒ज्ययेति॑ देव - य॒ज्यया᳚ । प॒शू॒नाम् । रू॒पम् । पु॒षे॒य॒म् । इति॑ । आ॒ह॒ । त्वष्टा᳚ । वै । प॒शू॒नाम् । मि॒थु॒नाना᳚म् । रू॒प॒कृदिति॑ रूप - कृत् । तेन॑ । ए॒व । प॒शू॒नाम् । रू॒पम् । आ॒त्मन्न् । ध॒त्ते॒ । दे॒वाना᳚म् । पत्नीः᳚ । अ॒ग्निः । गृ॒हप॑ति॒रिति॑ गृ॒ह - प॒तिः॒ । य॒ज्ञ्स्य॑ । मि॒थु॒नम् । तयोः᳚ । अ॒हम् । दे॒व॒य॒ज्ययेति॑ देव - य॒ज्यया᳚ । मि॒थु॒नेन॑ । प्रेति॑ । भू॒या॒स॒म् । इति॑ । आ॒ह॒ । ए॒तस्मा᳚त् । वै । मि॒थु॒नात् । प्र॒जाप॑ति॒रिति॑ प्र॒जा - प॒तिः॒ । मि॒थु॒नेन॑ । \textbf{  19} \newline
                  \newline
                                \textbf{ TS 1.7.4.6} \newline
                  प्रेति॑ । अ॒जा॒य॒त॒ । तस्मा᳚त् । ए॒व । यज॑मानः । मि॒थु॒नेन॑ । प्रेति॑ । जा॒य॒ते॒ । वे॒दः । अ॒सि॒ । वित्तिः॑ । अ॒सि॒ । वि॒देय॑ । इति॑ । आ॒ह॒ । वे॒देन॑ । वै । दे॒वाः । असु॑राणाम् । वि॒त्तम् । वेद्य᳚म् । अ॒वि॒न्द॒न्त॒ । तत् । वे॒दस्य॑ । वे॒द॒त्वमिति॑ वेद - त्वम् । यद्य॒दिति॒ यत् - य॒त् । भ्रातृ॑व्यस्य । अ॒भि॒द्ध्याये॒दित्य॑भि-ध्याये᳚त् । तस्य॑ । नाम॑ । गृ॒ह्णी॒या॒त् । तत् । ए॒व । अ॒स्य॒ । सर्व᳚म् । वृ॒ङ्क्ते॒ । घृ॒तव॑न्त॒मिति॑ घृ॒त - व॒न्त॒म् । कु॒ला॒यिन᳚म् । रा॒यः । पोष᳚म् । स॒ह॒स्रिण᳚म् । वे॒दः । द॒दा॒तु॒ । वा॒जिन᳚म् । इति॑ । आ॒ह॒ । प्रेति॑ । स॒हस्र᳚म् । प॒शून् । आ॒प्नो॒ति॒ ( ) । एति॑ । अ॒स्य॒ । प्र॒जाया॒मिति॑ प्र - जाया᳚म् । वा॒जी । जा॒य॒ते॒ । यः । ए॒वम् । वेद॑ ॥ \textbf{  20} \newline
                  \newline
                       (द॒र्.॒श॒पू॒र्ण॒मासयो॑-रुभ॒यतो॑-देवा॒श्वाः-सु॒रेताः᳚-प्र॒जाप॑तिर् मिथु॒नेना᳚-प्नोत्य॒-ष्टौ च॑)  \textbf{(A4)} \newline \newline
                                \textbf{ TS 1.7.5.1} \newline
                  ध्रु॒वाम् । वै । रिच्य॑मानाम् । य॒ज्ञ्ः । अन्विति॑ । रि॒च्य॒ते॒ । य॒ज्ञ्म् । यज॑मानः । यज॑मानम् । प्र॒जा इति॑ प्र - जाः । ध्रु॒वाम् । आ॒प्याय॑माना॒मित्या᳚ - प्याय॑मानाम् । य॒ज्ञ्ः । अनु॑ । एति॑ । प्या॒य॒ते॒ । य॒ज्ञ्म् । यज॑मानः । यज॑मानम् । प्र॒जा इति॑ प्र - जाः । एति॑ । प्या॒य॒ता॒म् । ध्रु॒वा । घृ॒तेन॑ । इति॑ । आ॒ह॒ । ध्रु॒वाम् । ए॒व । एति॑ । प्या॒य॒य॒ति॒ । ताम् । आ॒प्याय॑माना॒मित्या᳚ - प्याय॑मानाम् । य॒ज्ञ्ः । अनु॑ । एति॑ । प्या॒य॒ते॒ । य॒ज्ञ्म् । यज॑मानः । यज॑मानम् । प्र॒जा इति॑ प्र - जाः । प्र॒जाप॑ते॒रिति॑ प्र॒जा - प॒तेः॒ । वि॒भानिति॑ वि - भान् । नाम॑ । लो॒कः । तस्मिन्न्॑ । त्वा॒ । द॒धा॒मि॒ । स॒ह । यज॑मानेन । इति॑ । \textbf{  21} \newline
                  \newline
                                \textbf{ TS 1.7.5.2} \newline
                  आ॒ह॒ । अ॒यम् । वै । प्र॒जाप॑ते॒रिति॑ प्र॒जा - प॒तेः॒ । वि॒भानिति॑ वि - भान् । नाम॑ । लो॒कः । तस्मिन्न्॑ । ए॒व । ए॒न॒म् । द॒धा॒ति॒ । स॒ह । यज॑मानेन । रिच्य॑ते । इ॒व॒ । वै । ए॒तत् । यत् । यज॑ते । यत् । य॒ज॒मा॒न॒भा॒गमिति॑ यजमान - भा॒गम् । प्रा॒श्नातीति॑ प्र - अ॒श्नाति॑ । आ॒त्मान᳚म् । ए॒व । प्री॒णा॒ति॒ । ए॒तावान्॑ । वै । य॒ज्ञ्ः । यावान्॑ । य॒ज॒मा॒न॒भा॒ग इति॑ यजमान - भा॒गः । य॒ज्ञ्ः । यज॑मानः । यत् । य॒ज॒मा॒न॒भा॒गमिति॑ यजमान - भा॒गम् । प्रा॒श्नातीति॑ प्र - अ॒श्नाति॑ । य॒ज्ञे । ए॒व । य॒ज्ञ्म् । प्रतीति॑ । स्था॒प॒य॒ति॒ । ए॒तत् । वै । सू॒यव॑स॒मिति॑ सु - यव॑सम् । सोद॑क॒मिति॒ स - उ॒द॒क॒म् । यत् । ब॒र्॒.हिः । च॒ । आपः॑ । च॒ । ए॒तत् । \textbf{  22} \newline
                  \newline
                                \textbf{ TS 1.7.5.3} \newline
                  यज॑मानस्य । आ॒यत॑न॒मित्या᳚ - यत॑नम् । यत् । वेदिः॑ । यत् । पू॒र्ण॒पा॒त्रमिति॑ पूर्ण - पा॒त्रम् । अ॒न्त॒र्वे॒दीत्य॑न्तः - वे॒दि । नि॒नय॒तीति॑ नि - नय॑ति । स्वे । ए॒व । आ॒यत॑न॒ इत्या᳚ - यत॑ने । सू॒यव॑स॒मिति॑ सु - यव॑सम् । सोद॑क॒मिति॒ स - उ॒द॒क॒म् । कु॒रु॒ते॒ । सत् । अ॒सि॒ । सत् । मे॒ । भू॒याः॒ । इति॑ । आ॒ह॒ । आपः॑ । वै । य॒ज्ञ्ः । आपः॑ । अ॒मृत᳚म् । य॒ज्ञ्म् । ए॒व । अ॒मृत᳚म् । आ॒त्मन्न् । ध॒त्ते॒ । सर्वा॑णि । वै । भू॒तानि॑ । व्र॒तम् । उ॒प॒यन्त॒मित्यु॑प - यन्त᳚म् । अनु॑ । उपेति॑ । य॒न्ति॒ । प्राच्या᳚म् । दि॒शि । दे॒वाः । ऋ॒त्विजः॑ । मा॒र्ज॒य॒न्ता॒म् । इति॑ । आ॒ह॒ । ए॒षः । वै । द॒र्॒.श॒पू॒र्ण॒मा॒सयो॒रिति॑ दर्.श - पू॒र्ण॒मा॒सयोः᳚ । अ॒व॒भृ॒थ इत्य॑व - भृ॒थः । \textbf{  23} \newline
                  \newline
                                \textbf{ TS 1.7.5.4} \newline
                  यानि॑ । ए॒व । ए॒न॒म् । भू॒तानि॑ । व्र॒तम् । उ॒प॒यन्त॒मित्यु॑प - यन्त᳚म् । अ॒नू॒प॒यन्तीत्य॑नु - उ॒प॒यन्ति॑ । तैः । ए॒व । स॒ह । अ॒व॒भृ॒थमित्य॑व - भृ॒थम् । अवेति॑ । ए॒ति॒ । विष्णु॑मुखा॒ इति॒ विष्णु॑ - मु॒खाः॒ । वै । दे॒वाः । छन्दो॑भि॒रिति॒ छन्दः॑ - भिः॒ । इ॒मान् । लो॒कान् । अ॒न॒प॒ज॒य्यमित्य॑नप-ज॒य्यम् । अ॒भीति॑ । अ॒ज॒य॒न्न् । यत् । वि॒ष्णु॒क्र॒मानिति॑ विष्णु - क्र॒मान् । क्रम॑ते । विष्णुः॑ । ए॒व । भू॒त्वा । यज॑मानः । छन्दो॑भि॒रिति॒ छन्दः॑ - भिः॒ । इ॒मान् । लो॒कान् । अ॒न॒प॒ज॒य्यमित्य॑नप - ज॒य्यम् । अ॒भीति॑ । ज॒य॒ति॒ । विष्णोः᳚ । क्रमः॑ । अ॒सि॒ । अ॒भि॒मा॒ति॒हेत्य॑भिमाति - हा । इति॑ । आ॒ह॒ । गा॒य॒त्री । वै । पृ॒थि॒वी । त्रैष्टु॑भम् । अ॒न्तरि॑क्षम् । जाग॑ती । द्यौः । आनु॑ष्टुभी॒रित्यानु॑ - स्थु॒भीः॒ । दिशः॑ ( ) । छन्दो॑भि॒रिति॒ छन्दः॑ - भिः॒ । ए॒व । इ॒मान् । लो॒कान् । य॒था॒पू॒र्वमिति॑ यथा - पू॒र्वम् । अ॒भीति॑ । ज॒य॒ति॒ ॥ \textbf{  24} \newline
                  \newline
                      (यज॑माने॒नेति॑-चै॒ तद॑-वभृ॒थो-दिशः॑-स॒प्त च॑)  \textbf{(A5)} \newline \newline
                                \textbf{ TS 1.7.6.1} \newline
                  अग॑न्म । सुवः॑ । सुवः॑ । अ॒ग॒न्म॒ । इति॑ । आ॒ह॒ । सु॒व॒र्गमिति॑ सुवः - गम् । ए॒व । लो॒कम् । ए॒ति॒ । स॒दृंश॒ इति॑ सं -दृशः॑ । ते॒ । मा । छि॒थ्सि॒ । यत् । ते॒ । तपः॑ । तस्मै᳚ । ते॒ । मा । एति॑ । वृ॒क्षि॒ । इति॑ । आ॒ह॒ । य॒था॒य॒जुरिति॑ यथा - य॒जुः । ए॒व । ए॒तत् । सु॒भूरिति॑ सु- भूः । अ॒सि॒ । श्रेष्ठः॑ । र॒श्मी॒नाम् । आ॒यु॒द्‌र्धा इत्या॑युः - धाः । अ॒सि॒ । आयुः॑ । मे॒ । धे॒हि॒ । इति॑ । आ॒ह॒ । आ॒शिष॒मित्या᳚-शिष᳚म् । ए॒व । ए॒ताम् । एति॑ । शा॒स्ते॒ । प्रेति॑ । वै । ए॒षः । अ॒स्मात् । लो॒कात् । च्य॒व॒ते॒ । यः । \textbf{  25} \newline
                  \newline
                                \textbf{ TS 1.7.6.2} \newline
                  वि॒ष्णु॒क्र॒मानिति॑ विष्णु - क्र॒मान् । क्रम॑ते । सु॒व॒र्गायेति॑ सुवः-गाय॑ । हि । लो॒काय॑ । वि॒ष्णु॒क्र॒मा इति॑ विष्णु - क्र॒माः । क्र॒म्यन्ते᳚ । ब्र॒ह्म॒वा॒दिन॒ इति॑ ब्रह्म - वा॒दिनः॑ । व॒द॒न्ति॒ । सः । तु । वै । वि॒ष्णु॒क्र॒मानिति॑ विष्णु - क्र॒मान् । क्र॒मे॒त॒ । यः । इ॒मान् । लो॒कान् । भ्रातृ॑व्यस्य । स॒म्ॅविद्येति॑ सं - विद्य॑ । पुनः॑ । इ॒मम् । लो॒कम् । प्र॒त्य॒व॒रोहे॒दिति॑ प्रति - अ॒व॒रोहे᳚त् । इति॑ । ए॒षः । वै । अ॒स्य । लो॒कस्य॑ । प्र॒त्य॒व॒रो॒ह इति॑ प्रति - अ॒व॒रो॒हः । यत् । आह॑ । इ॒दम् । अ॒हम् । अ॒मुम् । भ्रातृ॑व्यम् । आ॒भ्यः । दि॒ग्भ्य इति॑ दिक् - भ्यः । अ॒स्यै । दि॒वः । इति॑ । इ॒मान् । ए॒व । लो॒कान् । भ्रातृ॑व्यस्य । स॒म्ॅविद्येति॑ सं - विद्य॑ । पुनः॑ । इ॒मम् । लो॒कम् । प्र॒त्यव॑रोह॒तीति॑ प्रति - अव॑रोहति । समिति॑ । \textbf{  26} \newline
                  \newline
                                \textbf{ TS 1.7.6.3} \newline
                  ज्योति॑षा । अ॒भू॒व॒म् । इति॑ । आ॒ह॒ । अ॒स्मिन्न् । ए॒व । लो॒के । प्रतीति॑ । ति॒ष्ठ॒ति॒ । ऐ॒न्द्रीम् । आ॒वृत॒मित्या᳚ - वृत᳚म् । अ॒न्वाव॑र्त॒ इत्य॑नु - आव॑र्ते । इति॑ । आ॒ह॒ । अ॒सौ । वै । आ॒दि॒त्यः । इन्द्रः॑ । तस्य॑ । ए॒व । आ॒वृत॒मित्या᳚ - वृत᳚म् । अन्विति॑ । प॒र्याव॑र्तत॒ इति॑ परि - आव॑र्तते । द॒क्षि॒णा । प॒र्याव॑र्तत॒ इति॑ परि - आव॑र्तते । स्वम् । ए॒व । वी॒र्य᳚म् । अन्विति॑ । प॒र्याव॑र्तत॒ इति॑ परि - आव॑र्तते । तस्मा᳚त् । दक्षि॑णः । अद्‌र्धः॑ । आ॒त्मनः॑ । वी॒र्या॑वत्तर॒ इति॑ वी॒र्या॑वत् - त॒रः॒ । अथो॒ इति॑ । आ॒दि॒त्यस्य॑ । ए॒व । आ॒वृत॒मित्या᳚ - वृत᳚म् । अन्विति॑ । प॒र्याव॑र्तत॒ इति॑ परि - आव॑र्तते । समिति॑ । अ॒हम् । प्र॒जयेति॑ प्र - जया᳚ । समिति॑ । मया᳚ । प्र॒जेति॑ प्र - जा । इति॑ । आ॒ह॒ । आ॒शिष॒मित्या᳚ - शिष᳚म् । \textbf{  27} \newline
                  \newline
                                \textbf{ TS 1.7.6.4} \newline
                  ए॒व । ए॒ताम् । एति॑ । शा॒स्ते॒ । समि॑द्ध॒ इति॒ सं - इ॒द्धः॒ । अ॒ग्ने॒ । मे॒ । दी॒दि॒हि॒ । स॒मे॒द्धेति॑ सं - ए॒द्धा । ते॒ । अ॒ग्ने॒ । दी॒द्या॒स॒म् । इति॑ । आ॒ह॒ । य॒था॒य॒जुरिति॑ यथा - य॒जुः । ए॒व । ए॒तत् । वसु॑मा॒निति॒ वसु॑ - मा॒न् । य॒ज्ञ्ः । वसी॑यान् । भू॒या॒स॒म् । इति॑ । आ॒ह॒ । आ॒शिष॒मित्या᳚-शिष᳚म् । ए॒व । ए॒ताम् । एति॑ । शा॒स्ते॒ । ब॒हु । वै । गार्.ह॑पत्य॒स्येति॒ गार्.ह॑ - प॒त्य॒स्य॒ । अन्ते᳚ । मि॒श्रम् । इ॒व॒ । च॒र्य॒ते॒ । आ॒ग्नि॒पा॒व॒मा॒नीभ्या॒मित्या᳚ग्नि - पा॒व॒मा॒नीभ्या᳚म् । गार्.ह॑पत्य॒मिति॒ गार्.ह॑ - प॒त्य॒म् । उपेति॑ । ति॒ष्ठ॒ते॒ । पु॒नाति॑ । ए॒व । अ॒ग्निम् । पु॒नी॒ते । आ॒त्मान᳚म् । द्वाभ्या᳚म् । प्रति॑ष्ठित्या॒ इति॒ प्रति॑ - स्थि॒त्यै॒ । अग्ने᳚ । गृ॒ह॒प॒त॒ इति॑ गृह - प॒ते॒ । इति॑ । आ॒ह॒ । \textbf{  28} \newline
                  \newline
                                \textbf{ TS 1.7.6.5} \newline
                  य॒था॒य॒जुरिति॑ यथा - य॒जुः । ए॒व । ए॒तत् । श॒तम् । हिमाः᳚ । इति॑ । आ॒ह॒ । श॒तम् । त्वा॒ । हे॒म॒न्तान् । इ॒न्धि॒षी॒य॒ । इति॑ । वाव । ए॒तत् । आ॒ह॒ । पु॒त्रस्य॑ । नाम॑ । गृ॒ह्णा॒ति॒ । अ॒न्ना॒दमित्य॑न्न - अ॒दम् । ए॒व । ए॒न॒म् । क॒रो॒ति॒ । ताम् । आ॒शिष॒मित्या᳚ - शिष᳚म् । एति॑ । शा॒से॒ । तन्त॑वे । ज्योति॑ष्मतीम् । इति॑ । ब्रू॒या॒त् । यस्य॑ । पु॒त्रः । अजा॑तः । स्यात् । ते॒ज॒स्वी । ए॒व । अ॒स्य॒ । ब्र॒ह्म॒व॒र्च॒सीति॑ ब्रह्म - व॒र्च॒सी । पु॒त्रः । जा॒य॒ते॒ । ताम् । आ॒शिष॒मित्या᳚ - शिष᳚म् । एति॑ । शा॒से॒ । अ॒मुष्मै᳚ । ज्योति॑ष्मतीम् । इति॑ । ब्रू॒या॒त् । यस्य॑ । पु॒त्रः । \textbf{  29} \newline
                  \newline
                                \textbf{ TS 1.7.6.6} \newline
                  जा॒तः । स्यात् । तेजः॑ । ए॒व । अ॒स्मि॒न्न् । ब्र॒ह्म॒व॒र्च॒समिति॑ ब्रह्म - व॒र्च॒सम् । द॒धा॒ति॒ । यः । वै । य॒ज्ञ्म् । प्र॒युज्येति॑ प्र-युज्य॑ । न । वि॒मु॒ञ्चतीति॑ वि - मु॒ञ्चति॑ । अ॒प्र॒ति॒ष्ठा॒न इत्य॑प्रति - स्था॒नः । वै । सः । भ॒व॒ति॒ । कः । त्वा॒ । यु॒न॒क्ति॒ । सः । त्वा॒ । वीति॑ । मु॒ञ्च॒तु॒ । इति॑ । आ॒ह॒ । प्र॒जाप॑ति॒रिति॑ प्र॒जा - प॒तिः॒ । वै । कः । प्र॒जाप॑ति॒नेति॑ प्र॒जा - प॒ति॒ना॒ । ए॒व । ए॒न॒म् । यु॒नक्ति॑ । प्र॒जाप॑ति॒नेति॑ प्र॒जा - प॒ति॒ना॒ । वीति॑ । मु॒ञ्च॒ति॒ । प्रति॑ष्ठित्या॒ इति॒ प्रति॑ - स्थि॒त्यै॒ । ई॒श्व॒रम् । वै । व्र॒तम् । अवि॑सृष्ट॒मित्यवि॑ - सृ॒ष्ट॒म् । प्र॒दह॒ इति॑ प्र - दहः॑ । अग्ने᳚ । व्र॒त॒प॒त॒ इति॑ व्रत - प॒ते॒ । व्र॒तम् । अ॒चा॒रि॒ष॒म् । इति॑ । आ॒ह॒ । व्र॒तम् । ए॒व । \textbf{  30} \newline
                  \newline
                                \textbf{ TS 1.7.6.7} \newline
                  वीति॑ । सृ॒ज॒ते॒ । शान्त्यै᳚ । अप्र॑दाहा॒येत्यप्र॑ - दा॒हा॒य॒ । पराङ्॑ । वाव । य॒ज्ञ्ः । ए॒ति॒ । न । नीति॑ । व॒र्त॒ते॒ । पुनः॑ । यः । वै । य॒ज्ञ्स्य॑ । पु॒न॒रा॒ल॒भंमिति॑ पुनः - आ॒ल॒भंम् । वि॒द्वान् । यज॑ते । तम् । अ॒भि । नीति॑ । व॒र्त॒ते॒ । य॒ज्ञ्ः । ब॒भू॒व॒ । सः । एति॑ । ब॒भू॒व॒ । इति॑ । आ॒ह॒ । ए॒षः । वै । य॒ज्ञ्स्य॑ । पु॒न॒रा॒ल॒भं इति॑ पुनः-आ॒ल॒भंः । तेन॑ । ए॒व । ए॒न॒म् । पुनः॑ । एति॑ । ल॒भ॒ते॒ । अन॑वरु॒द्धेत्यन॑व - रु॒द्धा॒ । वै । ए॒तस्य॑ । वि॒राडिति॑ वि - राट् । यः । आहि॑ताग्नि॒रित्याहि॑त-अ॒ग्निः॒ । सन्न् । अ॒स॒भः । प॒शवः॑ । खलु॑ । वै ( ) । ब्रा॒ह्म॒णस्य॑ । स॒भा । इ॒ष्ट्वा । प्राङ् । उ॒त्क्रम्येत्यु॑त् - क्रम्य॑ । ब्रू॒या॒त् । गोमा॒निति॒ गो - मा॒न् । अ॒ग्ने॒ । अवि॑मा॒नित्यवि॑ - मा॒न् । अ॒श्वी । य॒ज्ञ्ः । इति॑ । अवेति॑ । स॒भाम् । रु॒न्धे । प्रेति॑ । स॒हस्र᳚म् । प॒शून् । आ॒प्नो॒ति॒ । एति॑ । अ॒स्य॒ । प्र॒जाया॒मिति॑ प्र - जाया᳚म् । वा॒जी । जा॒य॒ते॒ ॥ \textbf{  31} \newline
                  \newline
                      चतु॑र्विꣳशतिश्च)  \textbf{(A6)} \newline \newline
                                \textbf{ TS 1.7.7.1} \newline
                  देव॑ । स॒वि॒तः॒ । प्रेति॑ । सु॒व॒ । य॒ज्ञ्म् । प्रेति॑ । सु॒व॒ । य॒ज्ञ्प॑ति॒मिति॑ य॒ज्ञ् - प॒ति॒म् । भगा॑य । दि॒व्यः । ग॒न्ध॒र्वः ॥ के॒त॒पूरिति॑ केत - पूः । केत᳚म् । नः॒ । पु॒ना॒तु॒ । वा॒चः । पतिः॑ । वाच᳚म् । अ॒द्य । स्व॒दा॒ति॒ । नः॒ ॥ इन्द्र॑स्य । वज्रः॑ । अ॒सि॒ । वार्त्र॑घ्न॒ इति॒ वार्त्र॑ - घ्नः॒ । त्वया᳚ । अ॒यम् । वृ॒त्रम् । व॒द्ध्या॒त् ॥ वाज॑स्य । नु । प्र॒स॒व इति॑ प्र - स॒वे । मा॒तर᳚म् । म॒हीम् । अदि॑तिम् । नाम॑ । वच॑सा । क॒रा॒म॒हे॒ ॥ यस्या᳚म् । इ॒दम् । विश्व᳚म् । भुव॑नम् । आ॒वि॒वेशेत्या᳚ - वि॒वेश॑ । तस्या᳚म् । नः॒ । दे॒वः । स॒वि॒ता । धर्म॑ । सा॒वि॒ष॒त् ॥ अ॒फ्स्वित्य॑प् - सु । \textbf{  32} \newline
                  \newline
                                \textbf{ TS 1.7.7.2} \newline
                  अ॒न्तः । अ॒मृत᳚म् । अ॒फ्स्वित्य॑प् - सु । भे॒ष॒जम् । अ॒पाम् । उ॒त । प्रश॑स्ति॒ष्विति॒ प्र - श॒स्ति॒षु॒ । अश्वाः᳚ । भ॒व॒थ॒ । वा॒जि॒नः॒ ॥ वा॒युः । वा॒ । त्वा॒ । मनुः॑ । वा॒ । त्वा॒ । ग॒न्ध॒र्वाः । स॒प्तविꣳ॑शति॒रिति॑ स॒प्त - विꣳ॒॒श॒तिः॒ ॥ ते । अग्रे᳚ । अश्व᳚म् । आ॒यु॒ञ्ज॒न्न् । ते । अ॒स्मि॒न्न् । ज॒वम् । एति॑ । अ॒द॒धुः॒ ॥ अपा᳚म् । न॒पा॒त् । आ॒शु॒हे॒म॒न्नित्या॑शु - हे॒म॒न् । यः । ऊ॒र्मिः । क॒कुद्मा॒निति॑ क॒कुत् - मा॒न् । प्रतू᳚र्ति॒रिति॒ प्र - तू॒र्तिः॒ । वा॒ज॒सात॑म॒ इति॑ वाज - सात॑मः । तेन॑ । अ॒यम् । वाज᳚म् । से॒त् ॥ विष्णोः᳚ । क्रमः॑ । अ॒सि॒ । विष्णोः᳚ । क्रा॒न्तम् । अ॒सि॒ । विष्णोः᳚ । विक्रा᳚न्त॒मिति॒ वि - क्रा॒न्त॒म् । अ॒सि॒ । अ॒ङ्कौ । न्य॒ङ्काविति॑ नि - अ॒ङ्कौ ( ) । अ॒भितः॑ । रथ᳚म् । यौ । ध्वा॒न्तम् । वा॒ता॒ग्रमिति॑ वात - अ॒ग्रम् । अन्विति॑ । स॒चंर॑न्ता॒विति॑ सं - चर॑न्तौ । दू॒रेहे॑ति॒रिति॑ दू॒रे - हे॒तिः॒ । इ॒न्द्रि॒यावा॒निती᳚न्द्रि॒य - वा॒न् । प॒त॒त्री । ते । नः॒ । अ॒ग्नयः॑ । पप्र॑यः । पा॒र॒य॒न्तु॒ ॥ \textbf{  33} \newline
                  \newline
                      (अ॒फ्सु-न्य॒ङ्कौ-पञ्च॑दश च)  \textbf{(A7)} \newline \newline
                                \textbf{ TS 1.7.8.1} \newline
                  दे॒वस्य॑ । अ॒हम् । स॒वि॒तुः । प्र॒स॒व इति॑ प्र - स॒वे । बृह॒स्पति॑ना । वा॒ज॒जितेति॑ वाज - जिता᳚ । वाज᳚म् । जे॒ष॒म् । दे॒वस्य॑ । अ॒हम् । स॒वि॒तुः । प्र॒स॒व इति॑ प्र - स॒वे । बृह॒स्पति॑ना । वा॒ज॒जितेति॑ वाज - जिता᳚ । वर्.षि॑ष्ठम् । नाक᳚म् । रु॒हे॒य॒म् । इन्द्रा॑य । वाच᳚म् । व॒द॒त॒ । इन्द्र᳚म् । वाज᳚म् । जा॒प॒य॒त॒ । इन्द्रः॑ । वाज᳚म् । अ॒ज॒यि॒त् ॥ अश्वा॑ज॒नीत्यश्व॑ - अ॒ज॒नि॒ । वा॒जि॒नि॒ । वाजे॑षु । वा॒जि॒नी॒व॒तीति॑ वाजिनी - व॒ति॒ । अश्वान्॑ । स॒मथ्स्विति॑ स॒मत्-स॒ । वा॒ज॒य॒ ॥ अर्वा᳚ । अ॒सि॒ । सप्तिः॑ । अ॒सि॒ । वा॒जी । अ॒सि॒ । वाजि॑नः । वाज᳚म् । धा॒व॒त॒ । म॒रुता᳚म् । प्र॒स॒व इति॑ प्र - स॒वे । ज॒य॒त॒ । वीति॑ । योज॑ना । मि॒मी॒द्ध्व॒म् । अद्ध्व॑नः । स्क॒भ्नी॒त॒ । \textbf{  34} \newline
                  \newline
                                \textbf{ TS 1.7.8.2} \newline
                  काष्ठा᳚म् । ग॒च्छ॒त॒ । वाजे॑वाज॒ इति॒ वाजे᳚-वा॒जे॒ । अ॒व॒त॒ । वा॒जि॒नः॒ । नः॒ । धने॑षु । वि॒प्राः॒ । अ॒मृ॒ताः॒ । ऋ॒त॒ज्ञा॒ इत्यृ॑त - ज्ञाः॒ ॥ अ॒स्य । मद्ध्वः॑ । पि॒ब॒त॒ । मा॒दय॑द्ध्वम् । तृ॒प्ताः । या॒त॒ । प॒थिभि॒रिति॑ प॒थि - भिः॒ । दे॒व॒यान॒रिति॑ देव - यानैः᳚ ॥ ते । नः॒ । अर्व॑न्तः । ह॒व॒न॒श्रुत॒ इति॑ हवन - श्रुतः॑ । हव᳚म् । विश्वे᳚ । शृ॒ण्व॒न्तु॒ । वा॒जिनः॑ ॥ मि॒तद्र॑व॒ इति॑ मि॒त - द्र॒वः॒ । स॒ह॒स्र॒सा इति॑ सहस्र-साः । मे॒धसा॒तेति॑ मे॒ध-सा॒ता॒ । स॒नि॒ष्यवः॑ ॥ म॒हः । ये । रत्न᳚म् । स॒मि॒थेष्विति॑ सं - इ॒थेषु॑ । ज॒भ्रि॒रे । शम् । नः॒ । भ॒व॒न्तु॒ । वा॒जिनः॑ । हवे॑षु ॥ दे॒वता॒तेति॑ दे॒व - ता॒ता॒ । मि॒तद्र॑व॒ इति॑ मि॒त - द्र॒वः॒ । स्व॒र्का इति॑ सू - अ॒र्काः ॥ ज॒भंय॑न्तः । अहि᳚म् । वृक᳚म् । रक्षाꣳ॑सि । सने॑मि । अ॒स्मत् । यु॒य॒व॒न्न् । \textbf{  35} \newline
                  \newline
                                \textbf{ TS 1.7.8.3} \newline
                  अमी॑वाः ॥ ए॒षः । स्यः । वा॒जी । क्षि॒प॒णिम् । तु॒र॒ण्य॒ति॒ । ग्री॒वाया᳚म् । ब॒द्धः । अ॒पि॒क॒क्ष इत्य॑पि - क॒क्षे । आ॒सनि॑ ॥ क्रतु᳚म् । द॒धि॒क्रा इति॑ दधि - क्राः । अन्विति॑ । स॒न्तवी᳚त्व॒दिति॑ सं - तवी᳚त्वत् । प॒थाम् । अङ्काꣳ॑सि । अन्विति॑ । आ॒पनी॑फण॒दित्या᳚ - पनी॑फणत् ॥ उ॒त । स्म॒ । अ॒स्य॒ । द्रव॑तः । तु॒र॒ण्य॒तः । प॒र्णम् । न । वेः । अन्विति॑ । वा॒ति॒ । प्र॒ग॒द्‌र्धिन॒ इति॑ प्र - ग॒द्‌र्धिनः॑ ॥ श्ये॒नस्य॑ । इ॒व॒ । ध्रज॑तः । अ॒ङ्क॒सम् । परीति॑ । द॒धि॒क्राव्.ण्ण॒ इति॑ दधि-क्राव्.ण्णः॑ । स॒ह । ऊ॒र्जा । तरि॑त्रतः ॥ एति॑ । मा॒ । वाज॑स्य । प्र॒स॒व इति॑ प्र - स॒वः । ज॒ग॒म्या॒त् । एति॑ । द्यावा॑पृथि॒वी इति॒ द्यावा᳚ - पृ॒थि॒वी । वि॒श्वश॑भूं॒ इति॑ वि॒श्व - श॒भूं॒ ॥ एति॑ । मा॒ । ग॒न्ता॒म् । पि॒तरा᳚ । \textbf{  36} \newline
                  \newline
                                \textbf{ TS 1.7.8.4} \newline
                  मा॒तरा᳚ । च॒ । एति॑ । मा॒ । सोमः॑ । अ॒मृ॒त॒त्वायेत्य॑मृत - त्वाय॑ । ग॒म्या॒त् ॥ वाजि॑नः । वा॒ज॒जि॒त॒ इति॑ वाज - जि॒तः॒ । वाज᳚म् । स॒रि॒ष्यन्तः॑ । वाज᳚म् । जे॒ष्यन्तः॑ । बृह॒स्पतेः᳚ । भा॒गम् । अवेति॑ । जि॒घ्र॒त॒ । वाजि॑नः । वा॒ज॒जि॒त॒ इति॑ वाज - जि॒तः॒ । वाज᳚म् । स॒सृ॒वाꣳसः॑ । वाज᳚म् । जि॒गि॒वाꣳसः॑ । बृह॒स्पतेः᳚ । भा॒गे । नीति॑ । मृ॒ढ्व॒म् । इ॒यम् । वः॒ । सा । स॒त्या । स॒धेंति॑ सं - धा । अ॒भू॒त् । याम् । इन्द्रे॑ण । स॒मध॑द्ध्व॒मिति॑ सं - अध॑द्ध्वम् । अजी॑जिपत । व॒न॒स्प॒त॒यः॒ । इन्द्र᳚म् । वाज᳚म् । वीति॑ । मु॒च्य॒द्ध्व॒म् ॥ \textbf{  37} \newline
                  \newline
                      (स्क॒भ्नी॒त॒-यु॒य॒व॒न्-पि॒तरा॒-द्विच॑त्वारिꣳशच्च)  \textbf{(A8)} \newline \newline
                                \textbf{ TS 1.7.9.1} \newline
                  क्ष॒त्रस्य॑ । उल्ब᳚म् । अ॒सि॒ । क्ष॒त्रस्य॑ । योनिः॑ । अ॒सि॒ । जाये᳚ । एति॑ । इ॒हि॒ । सुवः॑ । रोहा॑व । रोहा॑व । हि । सुवः॑ । अ॒हम् । नौ॒ । उ॒भयोः᳚ । सुवः॑ । रो॒क्ष्या॒मि॒ । वाजः॑ । च॒ । प्र॒स॒व इति॑ प्र - स॒वः । च॒ । अ॒पि॒ज इत्य॑पि- जः । च॒ । क्रतुः॑ । च॒ । सुवः॑ । च॒ । मू॒द्‌र्धा । च॒ । व्यश्नि॑य॒ इति॑ वि - अश्नि॑यः । च॒ । आ॒न्त्या॒य॒नः । च॒ । अन्त्यः॑ । च॒ । भौ॒व॒नः । च॒ । भुव॑नः । च॒ । अधि॑पति॒रित्यधि॑ - प॒तिः॒ । च॒ ॥ आयुः॑ । य॒ज्ञेन॑ । क॒ल्प॒ता॒म् । प्रा॒ण इति॑ प्र - अ॒नः । य॒ज्ञेन॑ । क॒ल्प॒ता॒म् । अ॒पा॒न इत्य॑प - अ॒नः । \textbf{  38} \newline
                  \newline
                                \textbf{ TS 1.7.9.2} \newline
                  य॒ज्ञेन॑ । क॒ल्प॒ता॒म् । व्या॒न इति॑ वि - अ॒नः । य॒ज्ञेन॑ । क॒ल्प॒ता॒म् । चक्षुः॑ । य॒ज्ञेन॑ । क॒ल्प॒ता॒म् । श्रोत्र᳚म् । य॒ज्ञेन॑ । क॒ल्प॒ता॒म् । मनः॑ । य॒ज्ञेन॑ । क॒ल्प॒ता॒म् । वाक् । य॒ज्ञेन॑ । क॒ल्प॒ता॒म् । आ॒त्मा । य॒ज्ञेन॑ । क॒ल्प॒ता॒म् । य॒ज्ञ्ः । य॒ज्ञेन॑ । क॒ल्प॒ता॒म् । सुवः॑ । दे॒वान् । अ॒ग॒न्म॒ । अ॒मृता᳚ । अ॒भू॒म॒ । प्र॒जाप॑ते॒रिति॑ प्र॒जा-प॒तेः॒ । प्र॒जा इति॑ प्र - जाः । अ॒भू॒म॒ । समिति॑ । अ॒हम् । प्र॒जयेति॑ प्र - जया᳚ । समिति॑ । मया᳚ । प्र॒जेति॑ प्र-जा । समिति॑ । अ॒हम् । रा॒यः । पोषे॑ण । समिति॑ । मया᳚ । रा॒यः । पोषः॑ । अन्ना॑य । त्वा॒ । अ॒न्नाद्या॒येत्य॑न्न - अद्या॑य । त्वा॒ । वाजा॑य ( ) । त्वा॒ । वा॒ज॒जि॒त्याया॒ इति॑ वाज - जि॒त्यायै᳚ । त्वा॒ । अ॒मृत᳚म् । अ॒सि॒ । पुष्टिः॑ । अ॒सि॒ । प्र॒जन॑न॒मिति॑ प्र - जन॑नम् । अ॒सि॒ ॥ \textbf{  39 } \newline
                  \newline
                      (अ॒पा॒नो-वाजा॑य॒-नव॑ च)  \textbf{(A9)} \newline \newline
                                \textbf{ TS 1.7.10.1} \newline
                  वाज॑स्य । इ॒मम् । प्र॒स॒व इति॑ प्र - स॒वः । सु॒षु॒वे॒ । अग्रे᳚ । सोम᳚म् । राजा॑नम् । ओष॑धीषु । अ॒फ्स्वित्य॑प् - सु ॥ ताः । अ॒स्मभ्य॒मित्य॒स्म - भ्य॒म् । मधु॑मती॒रिति॒ मधु॑ - म॒तीः॒ । भ॒व॒न्तु॒ । व॒यम् । रा॒ष्ट्रे । जा॒ग्रि॒या॒म॒ । पु॒रोहि॑ता॒ इति॑ पु॒रः - हि॒ताः॒ ॥ वाज॑स्य । इ॒दम् । प्र॒स॒व इति॑ प्र - स॒वः । एति॑ । ब॒भू॒व॒ । इ॒मा । च॒ । विश्वा᳚ । भुव॑नानि । स॒र्वतः॑ ॥ सः । वि॒राज॒मिति॑ वि - राज᳚म् । परीति॑ । ए॒ति॒ । प्र॒जा॒नन्निति॑ प्र - जा॒नन्न् । प्र॒जामिति॑ प्र - जाम् । पुष्टि᳚म् । व॒द्‌र्धय॑मानः । अ॒स्मे इति॑ ॥ वाज॑स्य । इ॒माम् । प्र॒स॒व इति॑ प्र-स॒वः । शि॒श्रि॒ये॒ । दिव᳚म् । इ॒मा । च॒ । विश्वा᳚ । भुव॑नानि । स॒म्राडिति॑ सं - राट् ॥ अदि॑थ्सन्तम् । दा॒प॒य॒तु॒ । प्र॒जा॒नन्निति॑ प्र - जा॒नन्न् । र॒यिम् । \textbf{  40} \newline
                  \newline
                                \textbf{ TS 1.7.10.2} \newline
                  च॒ । नः॒ । सर्व॑वीरा॒मिति॒ सर्व॑ - वी॒रा॒म् । नीति॑ । य॒च्छ॒तु॒ ॥ अग्ने᳚ । अच्छ॑ । व॒द॒ । इ॒ह । नः॒ । प्रतीति॑ । नः॒ । सु॒मना॒ इति॑ सु - मनाः᳚ । भ॒व॒ ॥ प्रेति॑ । नः॒ । य॒च्छ॒ । भु॒वः॒ । प॒ते॒ । ध॒न॒दा इति॑ धन - दाः । अ॒सि॒ । नः॒ । त्वम् ॥ प्रेति॑ । नः॒ । य॒च्छ॒तु॒ । अ॒र्य॒मा । प्रेति॑ । भगः॑ । प्रेति॑ । बृह॒स्पतिः॑ ॥ प्रेति॑ । दे॒वाः । प्रेति॑ । उ॒त । सू॒नृता᳚ । प्रेति॑ । वाक् । दे॒वी । द॒दा॒तु॒ । नः॒ ॥ अ॒र्य॒मण᳚म् । बृह॒स्पति᳚म् । इन्द्र᳚म् । दाना॑य । चो॒द॒य॒ ॥ वाच᳚म् । विष्णु᳚म् । सर॑स्वतीम् । स॒वि॒तार᳚म् । \textbf{  41} \newline
                  \newline
                                \textbf{ TS 1.7.10.3} \newline
                  च॒ । वा॒जिन᳚म् ॥ सोम᳚म् । राजा॑नम् । वरु॑णम् । अ॒ग्निम् । अ॒न्वार॑भामह॒ इत्य॑नु - आर॑भामहे ॥ आ॒दि॒त्यान् । विष्णु᳚म् । सूर्य᳚म् । ब्र॒ह्माण᳚म् । च॒ । बृह॒स्पति᳚म् ॥ दे॒वस्य॑ । त्वा॒ । स॒वि॒तुः । प्र॒स॒व इति॑ प्र - स॒वे । अ॒श्विनोः᳚ । बा॒हुभ्या॒मिति॑ बा॒हु - भ्या॒म् । पू॒ष्णः । हस्ता᳚भ्याम् । सर॑स्वत्यै । वा॒चः । य॒न्तुः । य॒न्त्रेण॑ । अ॒ग्नेः । त्वा॒ । साम्रा᳚ज्ये॒नेति॒ सां - रा॒ज्ये॒न॒ । अ॒भीति॑ । सि॒ञ्चा॒मि॒ । इन्द्र॑स्य । बृह॒स्पतेः᳚ । त्वा॒ । साम्रा᳚ज्ये॒नेति॒ सां - रा॒ज्ये॒न॒ । अ॒भीति॑ । सि॒ञ्चा॒मि॒ ॥ \textbf{  42} \newline
                  \newline
                      (र॒यिꣳ-स॑वि॒तारꣳ॒॒-षट्त्रिꣳ॑शच्च)  \textbf{(A10)} \newline \newline
                                \textbf{ TS 1.7.11.1} \newline
                  अ॒ग्निः । एका᳚क्षरे॒णेत्येक॑ -अ॒क्ष॒रे॒ण॒ । वाच᳚म् । उदिति॑ । अ॒ज॒य॒त् । अ॒श्विनौ᳚ । द्व्य॑क्षरे॒णेति॒ द्वि - अ॒क्ष॒रे॒ण॒ । प्रा॒णा॒पा॒नाविति॑ प्राण- अ॒पा॒नौ । उदिति॑ । अ॒ज॒य॒ता॒म् । विष्णुः॑ । त्र्य॑क्षरे॒णेति॒ त्रि - अ॒क्ष॒रे॒ण॒ । त्रीन् । लो॒कान् । उदिति॑ । अ॒ज॒य॒त् । सोमः॑ । चतु॑रक्षरे॒णेति॒ चतुः॑ - अ॒क्ष॒रे॒ण॒ । चतु॑ष्पद॒ इति॒ चतुः॑ - प॒दः॒ । प॒शून् । उदिति॑ । अ॒ज॒य॒त् । पू॒षा । पञ्चा᳚क्षरे॒णेति॒ पञ्च॑ - अ॒क्ष॒रे॒ण॒ । प॒ङ्क्तिम् । उदिति॑ । अ॒ज॒य॒त् । धा॒ता । षड॑क्षरे॒णेति॒ षट् - अ॒क्ष॒रे॒ण॒ । षट् । ऋ॒तून् । उदिति॑ । अ॒ज॒य॒त् । म॒रुतः॑ । स॒प्ताक्ष॑रे॒णेति॑ स॒प्त - अ॒क्ष॒रे॒ण॒ । स॒प्तप॑दा॒मिति॑ स॒प्त - प॒दा॒म् । शक्व॑रीम् । उदिति॑ । अ॒ज॒य॒न्न् । बृह॒स्पतिः॑ । अ॒ष्टाक्ष॑रे॒णेत्य॒ष्टा - अ॒क्ष॒रे॒ण॒ । गा॒य॒त्रीम् । उदिति॑ । अ॒ज॒य॒त् । मि॒त्रः । नवा᳚क्षरे॒णेति॒ नव॑ - अ॒क्ष॒रे॒ण॒ । त्रि॒वृत॒मिति॑ त्रि - वृत᳚म् । स्तोम᳚म् । उदिति॑ । अ॒ज॒य॒त् । \textbf{  43} \newline
                  \newline
                                \textbf{ TS 1.7.11.2} \newline
                  वरु॑णः । दशा᳚क्षरे॒णेति॒ दश॑ - अ॒क्ष॒रे॒ण॒ । वि॒राज॒मिति॑ वि - राज᳚म् । उदिति॑ । अ॒ज॒य॒त् । इन्द्रः॑ । एका॑दशाक्षरे॒णेत्येका॑दश - अ॒क्ष॒रे॒ण॒ । त्रि॒ष्टुभ᳚म् । उदिति॑ । अ॒ज॒य॒त् । विश्वे᳚ । दे॒वाः । द्वाद॑शाक्षरे॒णेति॒ द्वाद॑श - अ॒क्ष॒रे॒ण॒ । जग॑तीम् । उदिति॑ । अ॒ज॒य॒न्न् । वस॑वः । त्रयो॑दशाक्षरे॒णेति॒ त्रयो॑दश - अ॒क्ष॒रे॒ण॒ । त्र॒यो॒द॒शमिति॑ त्रयः - द॒शम् । स्तोम᳚म् । उदिति॑ । अ॒ज॒य॒न्न् । रु॒द्राः । चतु॑र्दशाक्षरे॒णेति॒ चतु॑र्दश - अ॒क्ष॒रे॒ण॒ । च॒तु॒र्द॒शमिति॑ चतुः-द॒शम् । स्तोम᳚म् । उदिति॑ । अ॒ज॒य॒न्न् । आ॒दि॒त्याः । पञ्च॑दशाक्षरे॒णेति॒ पञ्च॑दश - अ॒क्ष॒रे॒ण॒ । प॒ञ्च॒द॒शमिति॑ पञ्च - द॒शम् । स्तोम᳚म् । उदिति॑ । अ॒ज॒य॒॒न्न् । अदि॑तिः । षोड॑शाक्षरे॒णेति॒ षोड॑श - अ॒क्ष॒रे॒ण॒ । षो॒ड॒शम् । स्तोम᳚म् । उदिति॑ । अ॒ज॒य॒त् । प्र॒जाप॑ति॒रिति॑ प्र॒जा - प॒तिः॒ । स॒प्तद॑शाक्षरे॒णेति॑ स॒प्तद॑श-अ॒क्ष॒रे॒ण॒ । स॒प्त॒द॒शमिति॑ सप्त - द॒शम् । स्तोम᳚म् । उदिति॑ । अ॒ज॒य॒त् ॥ \textbf{  44} \newline
                  \newline
                      (त्रि॒वृतꣳ॒॒ स्तोम॒मुद॑जय॒थ्-षट्च॑त्वारिꣳशच्च)  \textbf{(A11)} \newline \newline
                                \textbf{ TS 1.7.12.1} \newline
                  उ॒प॒या॒मगृ॑हीत॒ इत्यु॑पया॒म - गृ॒ही॒तः॒ । अ॒सि॒ । नृ॒षद॒मिति॑ नृ-सद᳚म् । त्वा॒ । द्रु॒षद॒मिति॑ द्रु - सद᳚म् । भु॒व॒न॒सद॒मिति॑ भुवन - सद᳚म् । इन्द्रा॑य । जुष्ट᳚म् । गृ॒ह्णा॒मि॒ । ए॒षः । ते॒ । योनिः॑ । इन्द्रा॑य । त्वा॒ । उ॒प॒या॒मगृ॑हीत॒ इत्यु॑पया॒म - गृ॒ही॒तः॒ । अ॒सि॒ । अ॒फ्सु॒षद॒मित्य॑फ्सु - सद᳚म् । त्वा॒ । घृ॒त॒सद॒मिति॑ घृत - सद᳚म् । व्यो॒म॒सद॒मिति॑ व्योम - सद᳚म् । इन्द्रा॑य । जुष्ट᳚म् । गृ॒ह्णा॒मि॒ । ए॒षः । ते॒ । योनिः॑ । इन्द्रा॑य । त्वा॒ । उ॒प॒या॒मगृ॑हीत॒ इत्यु॑पया॒म - गृ॒ही॒तः॒ । अ॒सि॒ । पृ॒थि॒वि॒षद॒मिति॑ पृथिवि - सद᳚म् । त्वा॒ । अ॒न्त॒रि॒क्ष॒सद॒मित्य॑न्तरिक्ष - सद᳚म् । ना॒क॒सद॒मिति॑ नाक - सद᳚म् । इन्द्रा॑य । जुष्ट᳚म् । गृ॒ह्णा॒मि॒ । ए॒षः । ते॒ । योनिः॑ । इन्द्रा॑य । त्वा॒ ॥ ये । ग्रहाः᳚ । प॒ञ्च॒ज॒नीना॒ इति॑ पञ्च - ज॒नीनाः᳚ । येषा᳚म् । ति॒स्रः । प॒र॒म॒जा इति॑ परम - जाः ॥ दैव्यः॑ । कोशः॑ । \textbf{  45} \newline
                  \newline
                                \textbf{ TS 1.7.12.2} \newline
                  समु॑ब्जित॒ इति॒ सं - उ॒ब्जि॒तः॒ ॥ तेषा᳚म् । विशि॑प्रियाणा॒मिति॒ वि - शि॒प्रि॒या॒णा॒म् । इष᳚म् । ऊर्ज᳚म् । समिति॑ । अ॒ग्र॒भी॒म् । ए॒षः । ते॒ । योनिः॑ । इन्द्रा॑य । त्वा॒ ॥ अ॒पाम् । रस᳚म् । उद्व॑यस॒मित्युत् - व॒य॒स॒म् । सूर्य॑रश्मि॒मिति॒ सूर्य॑ - र॒श्मि॒म् । स॒माभृ॑त॒मिति॑ सं - आभृ॑तम् ॥ अ॒पाम् । रस॑स्य । यः । रसः॑ । तम् । वः॒ । गृ॒ह्णा॒मी॒ । उ॒त्त॒ममित्यु॑त् - त॒मम् । ए॒षः । ते॒ । योनिः॑ । इन्द्रा॑य । त्वा॒ ॥ अ॒या । वि॒ष्ठा इति॑ वि-स्थाः । ज॒नयन्न्॑ । कर्व॑राणि । सः । हि । घृणिः॑ । उ॒रुः । वरा॑य । गा॒तुः ॥ सः । प्रति॑ । उदिति॑ । ऐ॒त् । ध॒रुणः॑ । मद्ध्वः॑ । अग्र᳚म् । स्वाया᳚म् । यत् । त॒नुवां᳚ ( ) । त॒नूम् । ऐर॑यत ॥ उ॒प॒या॒मगृ॑हीत॒ इत्यु॑पया॒म - गृ॒ही॒तः॒ । अ॒सि॒ । प्र॒जाप॑तय॒ इति॑ प्र॒जा - प॒त॒ये॒ । त्वा॒ । जुष्ट᳚म् । गृ॒ह्णा॒मि॒ । ए॒षः । ते॒ । योनिः॑ । प्र॒जाप॑तय॒ इति॑ प्र॒जा - प॒त॒ये॒ । त्वा॒ ॥ \textbf{  46 } \newline
                  \newline
                      (कोश॑-स्त॒नुवां॒-त्रयो॑दश च)  \textbf{(A12)} \newline \newline
                                \textbf{ TS 1.7.13.1} \newline
                  अन्विति॑ । अह॑ । मासाः᳚ । अन्विति॑ । इत् । वना॑नि । अन्विति॑ । ओष॑धीः । अन्विति॑ । पर्व॑तासः ॥ अन्विति॑ । इन्द्र᳚म् । रोद॑सी॒ इति॑ । वा॒व॒शा॒ने इति॑ । अन्विति॑ । आपः॑ । अ॒जि॒ह॒त॒ । जाय॑मानम् ॥ अन्विति॑ । ते॒ । दा॒यि॒ । म॒हे । इ॒न्द्रि॒याय॑ । स॒त्रा । ते॒ । विश्व᳚म् । अन्विति॑ । वृ॒त्र॒हत्य॒ इति॑ वृत्र - हत्ये᳚ ॥ अन्विति॑ । क्ष॒त्रम् । अन्विति॑ । सहः॑ । य॒ज॒त्र॒ । इन्द्र॑ । दे॒वेभिः॑ । अन्विति॑ । ते॒ । नृ॒षह्य॒ इति॑ नृ - सह्ये᳚ ॥ इ॒न्द्रा॒णीम् । आ॒सु । नारि॑षु । सु॒पत्नी॒मिति॑ सु-पत्नी᳚म् । अ॒हम् । अ॒श्र॒व॒म् ॥ न । हि । अ॒स्याः॒ । अ॒प॒रम् । च॒न । ज॒रसा᳚ । \textbf{  47} \newline
                  \newline
                                \textbf{ TS 1.7.13.2} \newline
                  मर॑ते । पतिः॑ ॥ न । अ॒हम् । इ॒न्द्रा॒णि॒ । रा॒र॒ण॒ । सख्युः॑ । वृ॒षाक॑पे॒रिति॑ वृ॒षा - क॒पेः॒ । ऋ॒ते ॥ यस्य॑ । इ॒दम् । अप्य᳚म् । ह॒विः । प्रि॒यम् । दे॒वेषु॑ । गच्छ॑ति ॥ यः । जा॒तः । ए॒व । प्र॒थ॒मः । मन॑स्वान् । दे॒वः । दे॒वान् । क्रतु॑ना । प॒र्यभू॑ष॒दिति॑ परि - अभू॑षत् ॥ यस्य॑ । शुष्मा᳚त् । रोद॑सी॒ इति॑ । अभ्य॑सेताम् । नृ॒णंस्य॑ । म॒ह्ना । सः । ज॒ना॒सः॒ । इन्द्रः॑ ॥ ऐति॑ । ते॒ । म॒हः । इ॒न्द्र॒ । ऊ॒ती । उ॒ग्र॒ । सम॑न्यव॒ इति स - म॒न्य॒वः॒ । यत् । स॒मर॒न्तेति॑ सं - अर॑न्त । सेनाः᳚ ॥ पता॑ति । दि॒द्युत् । नर्य॑स्य । बा॒हु॒वोः । मा । ते॒ । \textbf{  48} \newline
                  \newline
                                \textbf{ TS 1.7.13.3} \newline
                  मनः॑ । वि॒ष्व॒द्रिय॒गिति॑ विष्व - द्रिय॑क् । वीति॑ । चा॒री॒त् ॥ मा । नः॒ । म॒र्द्धीः॒ । एति॑ । भ॒र॒ । द॒द्धि । तत् । नः॒ । प्रेति॑ । दा॒शुषे᳚ । दात॑वे । भूरि॑ । यत् । ते॒ ॥ नव्ये᳚ । दे॒ष्णे । श॒स्ते । अ॒स्मिन्न् । ते॒ । उ॒क्थे । प्रेति॑ । ब्र॒वा॒म॒ । व॒यम् । इ॒न्द्र॒ । स्तु॒वन्तः॑ ॥ एति॑ । तु । भ॒र॒ । माकिः॑ । ए॒तत् । परीति॑ । स्था॒त् । वि॒द्म । हि । त्वा॒ । वसु॑पति॒मिति॒ वसु॑ - प॒ति॒म् । वसू॑नाम् ॥ इन्द्र॑ । यत् । ते॒ । माहि॑नम् । दत्र᳚म् । अस्ति॑ । अ॒स्मभ्य॒मित्य॒स्म - भ्य॒म् । तत् । ह॒र्य॒श्वेति॑ हरि - अ॒श्व॒ । \textbf{  49} \newline
                  \newline
                                \textbf{ TS 1.7.13.4} \newline
                  प्रेति॑ । य॒न्धि॒ ॥ प्र॒दा॒तार॒मिति॑ प्र - दा॒तार᳚म् । ह॒वा॒म॒हे॒ । इन्द्र᳚म् । एति॑ । ह॒विषा᳚ । व॒यम् ॥ उ॒भा । हि । हस्ता᳚ । वसु॑ना । पृ॒णस्व॑ । आ । प्रेति॑ । य॒च्छ॒ । दक्षि॑णात् । एति॑ । उ॒त । स॒व्यात् ॥ प्र॒दा॒तेति॑ प्र - दा॒ता । व॒ज्री । वृ॒ष॒भः । तु॒रा॒षाट् । शु॒ष्मी । राजा᳚ । वृ॒त्र॒हेति॑ वृत्र - हा । सो॒म॒पावेति॑ सोम - पावा᳚ ॥ अ॒स्मिन्न् । य॒ज्ञे । ब॒र्॒.हिषि॑ । एति॑ । नि॒षद्येति॑ नि - सद्य॑ । अथ॑ । भ॒व॒ । यज॑मानाय । शम् । योः ॥ इन्द्रः॑ । सु॒त्रामेति॑ सु - त्रामा᳚ । स्ववा॒निति॒ स्व-वा॒न् । अवो॑भि॒रित्यवः॑ - भिः॒ । सु॒मृ॒डी॒क इति॑ सु - मृ॒डी॒कः । भ॒व॒तु॒ । वि॒श्ववे॑दा॒ इति॑ वि॒श्व-वे॒दाः॒ ॥ बाध॑ताम् । द्वेषः॑ । अभ॑यम् । कृ॒णो॒तु॒ । सु॒वीर्य॒स्येति॑ सु - वीर्य॑स्य । \textbf{  50} \newline
                  \newline
                                \textbf{ TS 1.7.13.5} \newline
                  पत॑यः । स्या॒म॒ ॥ तस्य॑ । व॒यम् । सु॒म॒ताविति॑ सु-म॒तौ । य॒ज्ञिय॑स्य । अपीति॑ । भ॒द्रे । सौ॒म॒न॒से । स्या॒म॒ ॥ सः । सु॒त्रामेति॑ सु - त्रामा᳚ । स्ववा॒निति॒ स्व-वा॒न् । इन्द्रः॑ । अ॒स्मे इति॑ । आ॒रात् । चि॒त् । द्वेषः॑ । स॒नु॒तः । यु॒यो॒तु॒ ॥ रे॒वतीः᳚ । नः॒ । स॒ध॒माद॒ इति॑ सध - मादः॑ । इन्द्रे᳚ । स॒न्तु॒ । तु॒विवा॑जा॒ इति॑ तु॒वि - वा॒जाः॒ ॥ क्षु॒मन्तः॑ । याभिः॑ । मदे॑म ॥ प्रो इति॑ । स्विति॑ । अ॒स्मै॒ । पु॒रो॒र॒थमिति॑ पुरः - र॒थम् । इन्द्रा॑य । शू॒षम् । अ॒र्च॒त॒ ॥ अ॒भीके᳚ । चि॒त् । उ॒ । लो॒क॒कृदिति॑ लोक - कृत् । स॒ङ्गे । स॒मथ्स्विति॑ स॒मत् - सु॒ । वृ॒त्र॒हेति॑ वृत्र - हा ॥ अ॒स्माक᳚म् । बो॒धि॒ । चो॒दि॒ता । नभ॑न्ताम् । अ॒न्य॒केषा᳚म् ॥ ज्या॒काः । अधीति॑ ( ) । धन्व॒स्विति॒ धन्व॑ - सु॒ ॥ \textbf{  51} \newline
                  \newline
                      (ज॒रसा॒-मा ते॑-हर्यश्व-सु॒वीर्य॒स्या-द्ध्ये-कं॑ च )  \textbf{(A13)} \newline \newline
\textbf{praSna korvai with starting padams of 1 to 13 anuvAkams :-} \newline
(पा॒क॒य॒ज्ञ्ꣳ-सꣳश्र॑वाः-प॒रोक्षं॑-ब॒र्॒.हिषो॒ऽहं -ध्रु॒वा-मग॒न्मेत्या॑ह॒ -देव॑ सवितर्-दे॒वस्या॒हं-क्ष॒त्रस्योलꣳं॒-ॅवाज॑स्ये॒म-म॒ग्निरेका᳚क्षरेणो -पया॒मगृ॑हीतो॒ऽ-स्यन्वह॒ मासा॒-स्त्रयो॑दश ।) \newline

\textbf{korvai with starting padams of1, 11, 21 series of pa~jcAtis :-} \newline
(पा॒क॒य॒ज्ञ्ं-प॒रोक्षं॑-ध्रु॒वां-ॅवि सृ॑जते-च नः॒ सर्व॑वीरां॒ - पत॑यः स्यो॒-मैक॑पञ्चा॒शत् । ) \newline

\textbf{first and last padam of Seventh praSnam :-} \newline
(पा॒क॒य॒ज्ञ्ं-धन्व॑सु ।) \newline 


॥ हरि॑ ॐ ॥
॥ कृष्ण यजुर्वेदीय तैत्तिरीय संहितायां प्रथमकाण्डे सप्तमः प्रश्नः समाप्तः ॥ \newline
\pagebreak
\pagebreak
        


\end{document}
