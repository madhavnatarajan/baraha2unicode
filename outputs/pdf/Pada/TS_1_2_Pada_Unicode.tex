\documentclass[17pt]{extarticle}
\usepackage{babel}
\usepackage{fontspec}
\usepackage{polyglossia}
\usepackage{extsizes}



\setmainlanguage{sanskrit}
\setotherlanguages{english} %% or other languages
\setlength{\parindent}{0pt}
\pagestyle{myheadings}
\newfontfamily\devanagarifont[Script=Devanagari]{AdishilaVedic}


\newcommand{\VAR}[1]{}
\newcommand{\BLOCK}[1]{}




\begin{document}
\begin{titlepage}
    \begin{center}
 
\begin{sanskrit}
    { \Large
    ॐ नमः परमात्मने, श्री महागणपतये नमः, 
श्री गुरुभ्यो नमः । ह॒रिः॒ ॐ ॥ 
    }
    \\
    \vspace{2.5cm}
    \mbox{ \Huge
    1.2     प्रथमकाण्डे द्वितीयः प्रश्नः - (अग्निष्टोमे क्रयः)   }
\end{sanskrit}
\end{center}

\end{titlepage}
\tableofcontents

ॐ नमः परमात्मने, श्री महागणपतये नमः, 
श्री गुरुभ्यो नमः । ह॒रिः॒ ॐ ॥ \newline
1.2     प्रथमकाण्डे द्वितीयः प्रश्नः - (अग्निष्टोमे क्रयः) \newline

\addcontentsline{toc}{section}{ 1.2     प्रथमकाण्डे द्वितीयः प्रश्नः - (अग्निष्टोमे क्रयः)}
\markright{ 1.2     प्रथमकाण्डे द्वितीयः प्रश्नः - (अग्निष्टोमे क्रयः) \hfill https://www.vedavms.in \hfill}
\section*{ 1.2     प्रथमकाण्डे द्वितीयः प्रश्नः - (अग्निष्टोमे क्रयः) }
                                \textbf{ TS 1.2.1.1} \newline
                  आपः॑ । उ॒न्द॒न्तु॒ । जी॒वसे᳚ । दी॒र्घा॒यु॒त्वायेति॑ दीर्घायु - त्वाय॑ । वर्च॑से । ओष॑धे । त्राय॑स्व । ए॒न॒म् । स्वधि॑त॒ इति॒ स्व-धि॒ते॒ । मा । ए॒न॒म् । हिꣳ॒॒सीः॒ । दे॒व॒श्रूरिति॑ देव - श्रूः । ए॒तानि॑ । प्रेति॑ । व॒पे॒ । स्व॒स्ति । उत्त॑रा॒णीत्युत् - त॒रा॒णि॒ । अ॒शी॒य॒ । आपः॑ । अ॒स्मान् । मा॒तरः॑ । शु॒न्ध॒न्तु॒ । घृ॒तेन॑ । नः॒ । घृ॒त॒पुव॒ इति॑ घृत-पुवः॑ । पु॒न॒न्तु॒ । विश्व᳚म् । अ॒स्मत् । प्रेति॑ । व॒ह॒न्तु॒ । रि॒प्रम् । उदिति॑ । आ॒भ्यः॒ । शुचिः॑ । एति॑ । पू॒तः । ए॒मि॒ । सोम॑स्य । त॒नूः । अ॒सि॒ । त॒नुव᳚म् । मे॒ । पा॒हि॒ । म॒ही॒नाम् । पयः॑ । अ॒सि॒ । व॒र्चो॒धा इति वर्चः - धाः । अ॒सि॒ । वर्चः॑ । \textbf{  1} \newline
                  \newline
                                \textbf{ TS 1.2.1.2} \newline
                  मयि॑ । धे॒हि॒ । वृ॒त्रस्य॑ । क॒नीनि॑का । अ॒सि॒ । च॒क्षु॒ष्पा इति॑ चक्षुः - पाः । अ॒सि॒ । चक्षुः॑ । मे॒ । पा॒हि॒ । चि॒त्पति॒रिति॑ चित् - पतिः॑ । त्वा॒ । पु॒ना॒तु॒ । वा॒क्पति॒रिति॑ वाक् - पतिः॑ । त्वा॒ । पु॒ना॒तु॒ । दे॒वः । त्वा॒ । स॒वि॒ता । पु॒ना॒तु॒ । अच्छि॑द्रेण । प॒वित्रे॑ण । वसोः᳚ । सूर्य॑स्य । र॒श्मिभि॒रिति॑ र॒श्मि - भिः॒ । तस्य॑ । ते॒ । प॒वि॒त्र॒प॒त॒ इति॑ पवित्र -प॒ते॒ । प॒वित्रे॑ण । यस्मै᳚ । कम् । पु॒ने । तत् । श॒के॒य॒म् । एति॑ । वः॒ । दे॒वा॒सः॒ । ई॒म॒हे॒ । सत्य॑धर्माण॒ इति॒ सत्य॑ - ध॒र्मा॒णः॒ । अ॒द्ध्व॒रे । यत् । वः॒ । दे॒वा॒सः॒ । आ॒गु॒र इत्या᳚ - गु॒रे । यज्ञि॑यासः । हवा॑महे । इन्द्रा᳚ग्नी॒ इतीन्द्र॑ - अ॒ग्नी॒ । द्यावा॑ पृथिवी॒ इति॒ द्यावा᳚ - पृ॒थि॒वी॒ । आपः॑ । ओ॒ष॒धीः॒ ( ) । त्वम् । दी॒क्षाणा᳚म् । अधि॑पति॒रित्यधि॑ - प॒तिः॒ । अ॒सि॒ । इ॒ह । मा॒ । सन्त᳚म् । पा॒हि॒ ॥ \textbf{  2 } \newline
                  \newline
                      (वर्च॑ - ओषधी- र॒ष्टौ च॑ )  \textbf{(A1)} \newline \newline
                                \textbf{ TS 1.2.2.1} \newline
                  आकू᳚त्या॒ इत्या - कू॒त्यै॒ । प्र॒युज॒ इति॑ प्र - युजे᳚ । अ॒ग्नये᳚ । स्वाहा᳚ । मे॒धायै᳚ । मन॑से । अ॒ग्नये᳚ । स्वाहा᳚ । दी॒क्षायै᳚ । तप॑से । अ॒ग्नये᳚ । स्वाहा᳚ । सर॑स्वत्यै । पू॒ष्णे । अ॒ग्नये᳚ । स्वाहा᳚ । आपः॑ । दे॒वीः॒ । बृ॒ह॒तीः॒ । वि॒श्व॒शं॒भु॒व॒ इति॑ विश्व - शं॒भु॒वः॒ । द्यावा॑पृथि॒वी इति॒ द्यावा᳚ - पृ॒थि॒वी । उ॒रु । अ॒न्तरि॑क्षम् । बृह॒स्पतिः॑ । नः॒ । ह॒विषा᳚ । वृ॒धा॒तु॒ । स्वाहा᳚ । विश्वे᳚ । दे॒वस्य॑ । ने॒तुः । मर्तः॑ । वृ॒णी॒त॒ । स॒ख्यम् । विश्वे᳚ । रा॒यः । इ॒षु॒द्ध्य॒सि॒ । द्यु॒म्नम् । वृ॒णी॒त॒ । पु॒ष्यसे᳚ । स्वाहा᳚ । ऋ॒ख्सा॒मयो॒रित्यृ॑क् -सा॒मयोः᳚ । शिल्पे॒ इति॑ । स्थः॒ । ते इति॑ । वा॒म् । एति॑ । र॒भे॒ । ते इति॑ । मा॒ । \textbf{  3} \newline
                  \newline
                                \textbf{ TS 1.2.2.2} \newline
                  पा॒त॒म् । एति॑ । अ॒स्य । य॒ज्ञ्स्य॑ । उ॒दृच॒ इत्यु॑त् - ऋचः॑ । इ॒माम् । धिय᳚म् । शिक्ष॑माणस्य । दे॒व॒ । क्रतु᳚म् । दक्ष᳚म् । व॒रु॒ण॒ । समिति॑ । शि॒शा॒धि॒ । यया᳚ । अतीति॑ । विश्वा᳚ । दु॒रि॒तेति॑ दुः - इ॒ता । तरे॑म । सु॒तर्मा॑ण॒मिति॑ सु -तर्मा॑णम् । अधीति॑ । नाव᳚म् । रु॒हे॒म॒ । ऊर्क् । अ॒सि॒ । आ॒ङ्गि॒र॒सी । ऊर्ण॑म्रदा॒ इत्यूर्ण॑ - म्र॒दाः॒ । ऊर्ज᳚म् । मे॒ । य॒च्छ॒ । पा॒हि । मा॒  । मा । मा॒ । हिꣳ॒॒सीः॒ । विष्णोः᳚ । शर्म॑ । अ॒सि॒ । शर्म॑ । यज॑मानस्य । शर्म॑ । मे॒ । य॒च्छ॒ । नक्ष॑त्राणाम् । मा॒ । अ॒ती॒का॒शात् । पा॒हि॒ । इन्द्र॑स्य । योनिः॑ । अ॒सि॒ । \textbf{  4} \newline
                  \newline
                                \textbf{ TS 1.2.2.3} \newline
                  मा । मा॒ । हिꣳ॒॒सीः॒ । कृ॒ष्यै । त्वा॒ । सु॒स॒स्याया॒ इति॑ सु - स॒स्यायै᳚ । सु॒पि॒प्प॒लाभ्य॒ इति॑ सु - पि॒प्प॒लाभ्यः॑ । त्वा॒ । ओष॑धीभ्य॒ इत्योष॑धि - भ्यः॒ । सू॒प॒स्था इति॑ सु - उ॒प॒स्थाः । दे॒वः । वन॒स्पतिः॑ । ऊ॒र्द्ध्वः । मा॒ । पा॒हि॒ । एति॑ । उ॒दृच॒ इत्यु॑त् - ऋचः॑ । स्वाहा᳚ । य॒ज्ञ्म् । मन॑सा । स्वाहा᳚ । द्यावा॑पृथि॒वीभ्या॒मिति॒ द्यावा᳚ - पृ॒थि॒वीभ्या᳚म् । स्वाहा᳚ । उ॒रोः । अ॒न्तरि॑क्षात् । स्वाहा᳚ । य॒ज्ञ्म् । वाता᳚त् । एति॑ । र॒भे॒ ॥ \textbf{  5 } \newline
                  \newline
                      ( मा॒ - योनि॑रसि - त्रिꣳ॒॒शच्च॑ )  \textbf{(A2)} \newline \newline
                                \textbf{ TS 1.2.3.1} \newline
                  दैवी᳚म् । धिय᳚म् । म॒ना॒म॒हे॒ । सु॒मृ॒डी॒कामिति॑ सु - मृ॒डी॒काम् । अ॒भिष्ट॑ये । व॒र्चो॒धामिति॑ वर्चः - धाम् । य॒ज्ञ्वा॑हस॒मिति॑ य॒ज्ञ् - वा॒ह॒स॒म् । सु॒पा॒रेति॑ सु - पा॒रा । नः॒ । अ॒स॒त् । वशे᳚ ॥ ये । दे॒वाः । मनो॑जाता॒ इति॒ मनः॑ - जा॒ताः॒ । म॒नो॒युज॒ इति॑ मनः - युजः॑ । सु॒दक्षा॒ इति॑ सु - दक्षाः᳚ । दक्ष॑पितार॒ इति॒ दक्ष॑- पि॒ता॒रः॒ । ते । नः॒ । पा॒न्तु॒ । ते । नः॒ । अ॒व॒न्तु॒ । तेभ्यः॑ । नमः॑ । तेभ्यः॑ । स्वाहा᳚ । अग्ने᳚ । त्वम् । स्विति॑ । जा॒गृ॒हि॒ । व॒यम् । स्विति॑ । म॒न्दि॒षी॒म॒हि॒ । गो॒पा॒य । नः॒ । स्व॒स्तये᳚ । प्र॒बुध॒ इति॑ प्र - बुधे᳚ । नः॒ । पुनः॑ । द॒दः॒ ॥ त्वम् । अ॒ग्ने॒ । व्र॒त॒पा इति॑ व्रत - पाः । अ॒सि॒ । दे॒वः । एति॑ । मर्त्ये॑षु । आ ॥ त्वम् । \textbf{  6} \newline
                  \newline
                                \textbf{ TS 1.2.3.2} \newline
                  य॒ज्ञेषु॑ । ईड्‍यः॑ ॥ विश्वे᳚ । दे॒वाः । अ॒भीति॑ । माम् । एति॑ । अ॒व॒वृ॒त्र॒न्न् । पू॒षा । स॒न्या । सोमः॑ । राध॑सा । दे॒वः । स॒वि॒ता । वसोः᳚ । व॒सु॒दावेति॑ वसु - दावा᳚ । रास्व॑ । इय॑त् । सो॒म॒ । एति॑ । भूयः॑ । भ॒र॒ । मा । पृ॒णन्न् । पू॒र्त्या । वीति॑ । रा॒धि॒ । मा । अ॒हम् । आयु॑षा । च॒न्द्रम् । अ॒सि॒ । मम॑ । भोगा॑य । भ॒व॒ । वस्त्र᳚म् । अ॒सि॒ । मम॑ । भोगा॑य । भ॒व॒ । उ॒स्रा । अ॒सि॒ । मम॑ । भोगा॑य । भ॒व॒ । हयः॑ । अ॒सि॒ । मम॑ । भोगा॑य । भ॒व॒ । \textbf{  7} \newline
                  \newline
                                \textbf{ TS 1.2.3.3} \newline
                  छागः॑ । अ॒सि॒ । मम॑ । भोगा॑य । भ॒व॒ । मे॒षः । अ॒सि॒ । मम॑ । भोगा॑य । भ॒व॒ । वा॒यवे᳚ । त्वा॒ । वरु॑णाय । त्वा॒ । निर्.ऋ॑त्या॒ इति॒ निः - ऋ॒त्यै॒ । त्वा॒ । रु॒द्राय॑ । त्वा॒ । देवीः᳚ । आ॒पः॒ । अ॒पा॒म् । न॒पा॒त् । यः । ऊ॒र्मिः । ह॒वि॒ष्यः॑ । इ॒न्द्रि॒यावा॒निती᳚न्द्रि॒य - वा॒न् । म॒दिन्त॑मः । तम् । वः॒ । मा । अवेति॑ । क्र॒मि॒ष॒म् । अच्छि॑न्नम् । तन्तु᳚म् । पृ॒थि॒व्याः । अन्विति॑ । गे॒ष॒म् । भ॒द्रात् । अ॒भीति॑ । श्रेयः॑ । प्रेति॑ । इ॒हि॒ । बृह॒स्पतिः॑ । पु॒र॒ ए॒तेति॑ पुरः - ए॒ता । ते॒ । अ॒स्तु॒ । अथ॑ । ई॒म् । अवेति॑ । स्य॒ ( ) । वरे᳚ । एति॑ । पृ॒थि॒व्याः । आ॒रे । शत्रून्॑ । कृ॒णु॒हि॒ । सर्व॑वीर॒ इति॒ सर्व॑ - वी॒रः॒ । एति॑ । इ॒दम् । अ॒ग॒न्म॒ । दे॒व॒यज॑न॒मिति॑ देव - यज॑नम् । पृ॒थि॒व्याः । विश्वे᳚ । दे॒वाः । यत् । अजु॑षन्त । पूर्वे᳚ । ऋ॒ख्सा॒माभ्या॒मित्यृ॑ख्सा॒म -भ्या॒म् । यजु॑षा । सं॒तर॑न्त॒ इति॑ सं - तर॑न्तः । रा॒यः । पोषे॑ण । समिति॑ । इ॒षा । म॒दे॒म॒ ॥ \textbf{  8} \newline
                  \newline
                      ( आ त्वꣳ-हयो॑ऽसि॒ मम॒ भोगा॑य भव-स्य॒-पञ्च॑विꣳशतिश्च )  \textbf{(A3)} \newline \newline
                                \textbf{ TS 1.2.4.1} \newline
                  इ॒यम् । ते॒ । शु॒क्र॒ । त॒नूः । इ॒दम् । वर्चः॑ । तया᳚ । समिति॑ । भ॒व॒ । भ्राज᳚म् । ग॒च्छ॒ । जूः । अ॒सि॒ । धृ॒ता । मन॑सा । जुष्टा᳚ । विष्ण॑वे । तस्याः᳚ । ते॒ । स॒त्यस॑वस॒ इति॑ स॒त्य - स॒व॒सः॒ । प्र॒स॒वे इति॑ प्र -स॒वे । वा॒चः । य॒न्त्रम् । अ॒शी॒य॒ । स्वाहा᳚ । शु॒क्रम् । अ॒सि॒ । अ॒मृत᳚म् । अ॒सि॒ । वै॒श्व॒दे॒वमिति॑ वैश्व -दे॒वम् । ह॒विः । सूर्य॑स्य । चक्षुः॑ । एति॑ । अ॒रु॒ह॒म् । अ॒ग्नेः । अ॒क्ष्णः । क॒नीनि॑काम् । यत् । एत॑शेभिः । ईय॑से । भ्राज॑मानः । वि॒प॒श्चिता᳚ । चित् । अ॒सि॒ । म॒ना । अ॒सि॒ । धीः । अ॒सि॒ । दक्षि॑णा । \textbf{  9} \newline
                  \newline
                                \textbf{ TS 1.2.4.2} \newline
                  अ॒सि॒ । य॒ज्ञिया᳚ । अ॒सि॒ । क्ष॒त्रिया᳚ । अ॒सि॒ । अदि॑तिः । अ॒सि॒ । उ॒भ॒यतः॑ शी॒र्ष्णीत्यु॑भ॒यतः॑ - शी॒र्ष्णी॒ । सा । नः॒ । सुप्रा॒चीति॒ सु -प्रा॒ची॒ । सुप्र॑ती॒चीति॒ सु - प्र॒ती॒ची॒ । समिति॑ । भ॒व॒ । मि॒त्रः । त्वा॒ । प॒दि । ब॒द्ध्ना॒तु॒ । पू॒षा । अद्ध्व॑नः । पा॒तु॒ । इन्द्रा॑य । अद्ध्य॑क्षा॒येत्यधि॑ - अ॒क्षा॒य॒ । अन्विति॑ । त्वा॒ । मा॒ता । म॒न्य॒ता॒म् । अन्विति॑ । पि॒ता । अन्विति॑ । भ्राता᳚ । सग॑र्भ्य॒ इति॒ स - ग॒र्भ्यः॒ । अन्विति॑ । सखा᳚ । सयू॑थ्य॒ इति॒ स - यू॒थ्यः॒ । सा । दे॒वि॒ । दे॒वम् । अच्छ॑ । इ॒हि॒ । इन्द्रा॑य । सोम᳚म् । रु॒द्रः । त्वा॒ । एति॑ । व॒र्त॒य॒तु॒ । मि॒त्रस्य॑ । प॒था । स्व॒स्ति । सोम॑स॒खेति॒ सोम॑ -स॒खा॒ ( ) । पुनः॑ । एति॑ । इ॒हि॒ । स॒ह । र॒य्या ॥ \textbf{  10} \newline
                  \newline
                      ( दक्षि॑णा॒-सोम॑सखा॒-पञ्च॑ च )  \textbf{(A4)} \newline \newline
                                \textbf{ TS 1.2.5.1} \newline
                  वस्वी᳚ । अ॒सि॒ । रु॒द्रा । अ॒सि॒ । अदि॑तिः । अ॒सि॒ । आ॒दि॒त्या । अ॒सि॒ । शु॒क्रा । अ॒सि॒ । च॒न्द्रा । अ॒सि॒ । बृह॒स्पतिः॑ । त्वा॒ । सु॒म्ने । र॒ण्व॒तु॒ । रु॒द्रः । वसु॑भि॒रिति॒ वसु॑ -भिः॒ । एति॑ । चि॒के॒तु॒ । पृ॒थि॒व्याः । त्वा॒ । मू॒र्द्धन्न् । एति॑ । जि॒घ॒र्मि॒ । दे॒व॒यज॑न॒ इति॑ देव - यज॑ने । इडा॑याः । प॒दे । घृ॒तव॒तीति॑ घृ॒त - व॒ति॒ । स्वाहा᳚ । परि॑लिखित॒मिति॒ परि॑ - लि॒खि॒त॒म् । रक्षः॑ । परि॑लिखिता॒ इति॒ परि॑ - लि॒खि॒ताः॒ । अरा॑तयः । इ॒दम् । अ॒हम् । रक्ष॑सः । ग्री॒वाः । अपीति॑ । कृ॒न्ता॒मि॒ । यः । अ॒स्मान् । द्वेष्टि॑ । यम् । च॒ । व॒यम् । द्वि॒ष्मः । इ॒दम् । अ॒स्य॒ । ग्री॒वाः । \textbf{  11} \newline
                  \newline
                                \textbf{ TS 1.2.5.2} \newline
                  अपीति॑ । कृ॒न्ता॒मि॒ । अ॒स्मे इति॑ । रायः॑ । त्वे इति॑ । रायः॑ । तोते᳚ । रायः॑ । समिति॑ । दे॒वि॒ । दे॒व्या । उ॒र्वश्या᳚ । प॒श्य॒स्व॒ । त्वष्टी॑मती । ते॒ । स॒पे॒य॒ । सु॒रेता॒ इति॑ सु - रेताः᳚ । रेतः॑ । दधा॑ना । वी॒रम् । वि॒दे॒य॒ । तव॑ । सं॒दृशीति॑ सं - दृशि॑ । मा॒ । अ॒हम् । रा॒यः । पोषे॑ण । वीति॑ । यो॒ष॒म् ॥ \textbf{  12} \newline
                  \newline
                      ( अ॒स्य॒ ग्री॒वा-एका॒न्न त्रिꣳ॒॒शच्च॑ )  \textbf{(A5)} \newline \newline
                                \textbf{ TS 1.2.6.1} \newline
                  अꣳ॒॒शुना᳚ । ते॒ । अꣳ॒॒शुः । पृ॒च्य॒ता॒म् । परु॑षा । परुः॑ । ग॒न्धः । ते॒ । काम᳚म् । अ॒व॒तु॒ । मदा॑य । रसः॑ । अच्यु॑तः । अ॒मात्यः॑ । अ॒सि॒ । शु॒क्रः । ते॒ । ग्रहः॑ । अ॒भीति॑ । त्यम् । दे॒वम् । स॒वि॒तार᳚म् । ऊ॒ण्योः᳚ । क॒विक्र॑तु॒मिति॑ क॒वि - क्र॒तु॒म् । अर्चा॑मि । स॒त्यस॑वस॒मिति॑ स॒त्य -स॒व॒स॒म् । र॒त्न॒धामिति॑ रत्न - धाम् । अ॒भीति॑ । प्रि॒यम् । म॒तिम् । ऊ॒र्द्ध्वा । यस्य॑ । अ॒मतिः॑ । भाः । अदि॑द्युतत् । सवी॑मनि । हिर॑ण्यपाणि॒रिति॒ हिर॑ण्य - पा॒णिः॒ । अ॒मि॒मी॒त॒ । सु॒क्रतु॒रिति॑ सु - क्रतुः॑ । कृ॒पा । सुवः॑ ॥ प्र॒जाभ्य॒ इति॑ प्र - जाभ्यः॑ । त्वा॒ । प्रा॒णायेति॑ प्र - अ॒नाय॑ । त्वा॒ । व्या॒नायेति॑ वि -अ॒नाय॑ । त्वा॒ । प्र॒जा इति॑ प्र - जाः । त्वम् । अनु॑ ( ) । प्रेति॑ । अ॒नि॒हि॒ । प्र॒जा इति॑ प्र - जाः । त्वाम् । अनु॑ । प्रेति॑ । अ॒न॒न्तु॒ ॥ \textbf{  13} \newline
                  \newline
                       (अनु॑-स॒प्त च॑)  \textbf{(A6)} \newline \newline
                                \textbf{ TS 1.2.7.1} \newline
                  सोम᳚म् । ते॒ । क्री॒णा॒मि॒ । ऊर्ज॑स्वन्तम् । पय॑स्वन्तम् । वी॒र्या॑वन्त॒मिति॑ वी॒र्य॑ - व॒न्त॒म् । अ॒भि॒मा॒ति॒षाह॒मित्य॑भिमाति - साह᳚म् । शु॒क्रम् । ते॒ । शु॒क्रेण॑ । क्री॒णा॒मि॒ । च॒न्द्रम् । च॒न्द्रेण॑ । अ॒मृत᳚म् । अ॒मृते॑न । स॒म्यत् । ते॒ । गोः । अ॒स्मे इति॑ । च॒न्द्राणि॑ । तप॑सः । त॒नूः । अ॒सि॒ । प्र॒जाप॑ते॒रिति॑ प्र॒जा - प॒तेः॒ । वर्णः॑ । तस्याः᳚ । ते॒ । स॒ह॒स्र॒पो॒षमिति॑ सहस्र -पो॒षम् । पुष्य॑न्त्याः । च॒र॒मेण॑ । प॒शुना᳚ । क्री॒णा॒मि॒ । अ॒स्मे इति॑ । ते॒ । बन्धुः॑ । मयि॑ । ते॒ । रायः॑ । श्र॒य॒न्ता॒म् । अ॒स्मे इति॑ । ज्योतिः॑ । सो॒म॒वि॒क्र॒यिणीति॑ सोम - वि॒क्र॒यिणि॑ । तमः॑ । मि॒त्रः । नः॒ । एति॑ । इ॒हि॒ । सुमि॑त्रधा॒ इति॒ सुमि॑त्र - धाः॒ । इन्द्र॑स्य । ऊ॒रुम् ( ) । एति॑ । वि॒श॒ । दक्षि॑णम् । उ॒शन्न् । उ॒शन्त᳚म् । स्यो॒नः । स्यो॒नम् । स्वान॑ । भ्राज॑ । अङ्घा॑रे । बंभा॑रे । हस्त॑ । सुह॒स्तेति॒ सु - ह॒स्त॒ । कृशा॑न॒विति॒ कृश॑ - अ॒नो॒ । ए॒ते । वः॒ । सो॒म॒क्रय॑णा॒ इति॑ सोम - क्रय॑णाः । तान् । र॒क्ष॒द्ध्व॒म् । मा । वः॒ । द॒भ॒न्न् ॥ \textbf{  14 } \newline
                  \newline
                      (उ॒रुं-द्वाविꣳ॑शतिश्च)  \textbf{(A7)} \newline \newline
                                \textbf{ TS 1.2.8.1} \newline
                  उदिति॑ । आयु॑षा । स्वा॒युषेति॑ सु - आ॒युषा᳚ । उदिति॑ । ओष॑धीनाम् । रसे॑न । उदिति॑ । प॒र्जन्य॑स्य । शुष्मे॑ण । उदिति॑ । अ॒स्था॒म् । अ॒मृतान्॑ । अनु॑ ॥ उ॒रु । अ॒न्तरि॑क्षम् । अन्विति॑ । इ॒हि॒ । अदि॑त्याः । सदः॑ । अ॒सि॒ । अदि॑त्याः । सदः॑ । एति॑ । सी॒द॒ । अस्त॑भ्नात् । द्याम् । ऋ॒ष॒भः । अ॒न्तरि॑क्षम् । अमि॑मीत । व॒रि॒माण᳚म् । पृ॒थि॒व्याः । एति॑ । अ॒सी॒द॒त् । विश्वा᳚ । भुव॑नानि । सं॒राडिति सं - राट् । विश्वा᳚ । इत् । तानि॑ । वरु॑णस्य । व्र॒तानि॑ । वने॑षु । वीति॑ । अ॒न्तरि॑क्षम् । त॒ता॒न॒ । वाज᳚म् । अर्व॒थ्स्वित्यर्व॑त् - सु॒ । पयः॑ । अ॒घ्नि॒यासु॑ । हृ॒थ्स्विति॑ हृत् - सु । \textbf{  15} \newline
                  \newline
                                \textbf{ TS 1.2.8.2} \newline
                  क्रतु᳚म् । वरु॑णः । वि॒क्षु । अ॒ग्निम् । दि॒वि । सूर्य᳚म् । अ॒द॒धा॒त् । सोम᳚म् । अद्रौ᳚ । उदिति॑ । उ॒ । त्यम् । जा॒तवे॑दस॒मिति॑ जा॒त - वे॒द॒स॒म् । दे॒वम् । व॒ह॒न्ति॒ । के॒तवः॑ ॥ दृ॒शे । विश्वा॑य । सूर्य᳚म् ॥ उस्रौ᳚ । एति॑ । इ॒त॒म् । धू॒र्.॒षा॒हा॒विति॑ धूः - सा॒हौ॒ । अ॒न॒श्रू इति॑ । अवी॑रहणा॒वित्यवी॑र - ह॒नौ॒ । ब्र॒ह्म॒चोद॑ना॒विति॑ ब्रह्म - चोद॑नौ । वरु॑णस्य । स्कंभ॑नम् । अ॒सि॒ । वरु॑णस्य । स्कं॒भ॒सर्ज॑न॒मिति॑ स्कंभ -सर्ज॑नम् । अ॒सि॒ । प्रत्य॑स्त॒ इति॒ प्रति॑ - अ॒स्तः॒ । वरु॑णस्य । पाशः॑ ॥ \textbf{  16 } \newline
                  \newline
                      ( हृ॒थ्सु-पञ्च॑त्रिꣳशच्च )  \textbf{(A8)} \newline \newline
                                \textbf{ TS 1.2.9.1} \newline
                  प्रेति॑ । च्य॒व॒स्व॒ । भु॒वः॒ । प॒ते॒ । विश्वा॑नि । अ॒भीति॑ । धामा॑नि । मा । त्वा॒ । प॒रि॒प॒रीति॑ परि - प॒री । वि॒द॒त् । मा । त्वा॒ । प॒रि॒प॒न्थिन॒ इति॑ परि - प॒न्थिनः॑ । वि॒द॒न्न् । मा । त्वा॒ । वृकाः᳚ । अ॒घा॒यव॒ इत्य॑घ - यवः॑ । मा । ग॒न्ध॒र्वः । वि॒श्वाव॑सु॒रिति॑ वि॒श्व - व॒सुः॒ । एति॑ । द॒घ॒त् । श्ये॒नः । भू॒त्वा । परेति॑ । प॒त॒ । यज॑मानस्य । नः॒ । गृ॒हे । दे॒वैः । सꣳ॒॒स्कृ॒तम् । यज॑मानस्य । स्व॒स्त्यय॒नीति॑ स्वस्ति - अय॑नी । अ॒सि॒ । अपीति॑ । पन्था᳚म् । अ॒ग॒स्म॒हि॒ । स्व॒स्ति॒गामिति॑ स्वस्ति - गाम् । अ॒ने॒हस᳚म् । येन॑ । विश्वाः᳚ । परीति॑ । द्विषः॑ । वृ॒णक्ति॑ । वि॒न्दते᳚ । वसु॑ । नमः॑ । मि॒त्रस्य॑ ( ) । वरु॑णस्य । चक्ष॑से । म॒हः । दे॒वाय॑ । तत् । ऋ॒तम् । स॒प॒र्य॒त॒ । दू॒रे॒दृश॒ इति॑ दूरे - दृशे᳚ । दे॒वजा॑ता॒येति॑ दे॒व - जा॒ता॒य॒ । के॒तवे᳚ । दि॒वः । पु॒त्राय॑ । सूर्या॑य । शꣳ॒॒स॒त॒ । वरु॑णस्य । स्कंभ॑नम् । अ॒सि॒ । वरु॑णस्य । स्कं॒भ॒सर्ज॑न॒मिति॑ स्कंभ - सर्ज॑नम् । अ॒सि॒ । उन्मु॑क्त॒ इत्युत् - मु॒क्तः॒ । वरु॑णस्य । पाशः॑ ॥ \textbf{  17 } \newline
                  \newline
                      ( मि॒त्रस्य॒-त्रयो॑विꣳशतिश्च )  \textbf{(A9)} \newline \newline
                                \textbf{ TS 1.2.10.1} \newline
                  अ॒ग्नेः । आ॒ति॒थ्यम् । अ॒सि॒ । विष्ण॑वे । त्वा॒ । सोम॑स्य । आ॒ति॒थ्यम् । अ॒सि॒ । विष्ण॑वे । त्वा॒ । अति॑थेः । आ॒ति॒थ्यम् । अ॒सि॒ । विष्ण॑वे । त्वा॒ । अ॒ग्नये᳚ । त्वा॒ । रा॒य॒स्पो॒ष॒दाव्‌न्न॒ इति॑ रायस्पोष - दाव्‌न्ने᳚ । विष्ण॑वे । त्वा॒ । श्ये॒नाय॑ । त्वा॒ । सो॒म॒भृत॒ इति॑ सोम - भृते᳚ । विष्ण॑वे । त्वा॒ । या । ते॒ । धामा॑नि । ह॒विषा᳚ । यज॑न्ति । ता । ते॒ । विश्वा᳚ । प॒रि॒भूरिति॑ परि - भूः । अ॒स्तु॒ । य॒ज्ञ्म् । ग॒य॒स्फान॒ इति॑ गय - स्फानः॑ । प्र॒तर॑ण॒ इति॑ प्र - तर॑णः । सु॒वीर॒ इति॑ सु - वीरः॑ । अवी॑र॒हेत्यवी॑र - हा॒ । प्रेति॑ । च॒र ॒। सो॒म॒ । दुर्यान्॑ । अदि॑त्याः । सदः॑ । अ॒सि॒ । अदि॑त्याः । सदः॑ । एति॑ । \textbf{  18} \newline
                  \newline
                                \textbf{ TS 1.2.10.2} \newline
                  सी॒द॒ । वरु॑णः । अ॒सि॒ । धृ॒तव्र॑त॒ इति॑ धृ॒त - व्र॒तः॒ । वा॒रु॒णम् । अ॒सि॒ । शं॒योरिति॑ शं - योः । दे॒वाना᳚म् । स॒ख्यात् । मा । दे॒वाना᳚म् । अ॒पसः॑ । छि॒थ्स्म॒हि॒ । आप॑तय॒ इत्या - प॒त॒ये॒ । त्वा॒ ।  गृ॒ह्णा॒मि॒ । परि॑पतय॒ इति॒ परि॑ - प॒त॒ये॒ । त्वा॒ । गृ॒ह्णा॒मि॒ । तनू॒नप्त्र॒ इति॒ तनू᳚ - नप्त्रे᳚ । त्वा॒ । गृ॒ह्णा॒मि॒ । शा॒क्व॒राय॑ । त्वा॒  । गृ॒ह्णा॒मि॒  । शक्मन्न्॑ । ओजि॑ष्ठाय । त्वा॒ । गृ॒ह्णा॒मि॒ । अना॑धृष्ट॒मित्यना᳚ - धृ॒ष्ट॒म् । अ॒सि॒ । अ॒ना॒धृ॒ष्यमित्य॑ना - धृ॒ष्यम् । दे॒वाना᳚म् । ओजः॑ । अ॒भि॒श॒स्ति॒पा इत्य॑भिशस्ति - पाः । अ॒न॒भि॒श॒स्ते॒न्यमित्य॑नभि - श॒स्ते॒न्यम् । अन्विति॑ । मे॒ । दी॒क्षाम् । दी॒क्षाप॑ति॒रिति॑ दी॒क्षा - प॒तिः॒ । म॒न्य॒ता॒म् । अन्विति॑ । तपः॑ । तप॑स्पति॒रिति॒ तपः॑ - प॒तिः॒ । अञ्ज॑सा । स॒त्यम् । उपेति॑ । गे॒ष॒म् । सु॒वि॒ते । मा॒ ( ) । धाः॒ ॥ \textbf{  19 } \newline
                  \newline
                      ( आ-मै-कं॑ च )  \textbf{(A10)} \newline \newline
                                \textbf{ TS 1.2.11.1} \newline
                  अꣳ॒॒शुरꣳ॑शु॒रित्यꣳ॒॒शुः - अꣳ॒॒शुः॒ । ते॒ । दे॒व॒ । सो॒म॒ । एति॑ । प्या॒य॒ता॒म् । इन्द्रा॑य । ए॒क॒ध॒न॒विद॒ इत्ये॑कधन - विदे᳚ । एति॑ । तुभ्य᳚म् । इन्द्रः॑ । प्या॒य॒ता॒म् । एति॑ । त्वम् । इन्द्रा॑य । प्या॒य॒स्व॒ । एति॑ । प्या॒य॒य॒ । सखीन्॑ । स॒न्या । मे॒धया᳚ । स्व॒स्ति । ते॒ । दे॒व॒ । सो॒म॒ । सु॒त्याम् । अ॒शी॒य॒ । एष्टः॑ । रायः॑ । प्रेति॑ । इ॒षे । भगा॑य । ऋ॒तम् । ऋ॒त॒वा॒दिभ्य॒ इत्यृ॑तवा॒दि - भ्यः॒ । नमः॑ । दि॒वे । नमः॑ । पृ॒थि॒व्यै । अग्ने᳚ । व्र॒त॒प॒त॒ इति॑ व्रत - प॒ते॒ । त्वम् । व्र॒ताना᳚म् । व्र॒तप॑ति॒रिति॑ व्र॒त - प॒तिः॒ । अ॒सि॒ । या । मम॑ । त॒नूः । ए॒षा । सा । त्वयि॑ । \textbf{  20} \newline
                  \newline
                                \textbf{ TS 1.2.11.2} \newline
                  या । तव॑ । त॒नूः । इ॒यम् । सा । मयि॑ । स॒ह । नौ॒ । व्र॒त॒प॒त॒ इति॑ व्रत -प॒ते॒ । व्र॒तिनोः᳚ । व्र॒तानि॑ । या । ते॒ । अ॒ग्ने॒ । रुद्रि॑या । त॒नूः । तया᳚ । नः॒ । पा॒हि॒ । तस्याः᳚ । ते॒ । स्वाहा᳚ । या । ते॒ । अ॒ग्ने॒ । अ॒या॒श॒येत्य॑या - श॒या । र॒जा॒श॒येति॑ रजा - श॒या । ह॒रा॒श॒येति॑ हरा - श॒या । त॒नूः । वर्.षि॑ष्ठा । ग॒ह्व॒रे॒ष्ठेति॑ गह्वरे - स्था । उ॒ग्रम् । वचः॑ । अपेति॑ । अ॒व॒धी॒म् । त्वे॒षम् । वचः॑ । अपेति॑ । अ॒व॒धी॒म् । स्वाहा᳚ ॥ \textbf{  21 } \newline
                  \newline
                      ( त्वयि॑-चत्वारिꣳ॒॒शच्च॑ )  \textbf{(A11)} \newline \newline
                                \textbf{ TS 1.2.12.1} \newline
                  वि॒त्ताय॒नीति॑ वित्त - अय॑नी । मे॒ । अ॒सि॒ । ति॒क्ताय॒नीति॑ तिक्त - अय॑नी । मे॒ । अ॒सि॒ । अव॑तात् । मा॒ । ना॒थि॒तम् । अव॑तात् । मा॒ । व्य॒थि॒तम् । वि॒देः । अ॒ग्निः । नभः॑ । नाम॑ । अग्ने᳚ । अ॒ङ्गि॒रः॒ । यः । अ॒स्याम् । पृ॒थि॒व्याम् । असि॑ । आयु॑षा । नाम्ना᳚ । एति॑ । इ॒हि॒ । यत् । ते॒ । अना॑धृष्ट॒मित्यना᳚ - धृ॒ष्ट॒म् । नाम॑ । य॒ज्ञिय᳚म् । तेन॑ । त्वा॒ । एति॑ । द॒धे॒ । अग्ने᳚ । अ॒ङ्गि॒रः॒ । यः । द्वि॒तीय॑स्याम् । तृ॒तीय॑स्याम् । पृ॒थि॒व्याम् । असि॑ । आयु॑षा । नाम्ना᳚ । एति॑ । इ॒हि॒ । यत् । ते॒ । अना॑धृष्ट॒मित्यना᳚ - धृ॒ष्ट॒म् । नाम॑ । \textbf{  22} \newline
                  \newline
                                \textbf{ TS 1.2.12.2} \newline
                  य॒ज्ञिय᳚म् । तेन॑ । त्वा॒ । एति॑ । द॒धे॒ । सिꣳ॒॒हीः । अ॒सि॒ । म॒हि॒षीः । अ॒सि॒ । उ॒रु । प्र॒थ॒स्व॒ । उ॒रु । ते॒ । य॒ज्ञ्प॑ति॒रिति॑ य॒ज्ञ् - प॒तिः॒ । प्र॒थ॒ता॒म् । ध्रु॒वा । अ॒सि॒ । दे॒वेभ्यः॑ । शु॒न्ध॒स्व॒ । दे॒वेभ्यः॑ । शुं॒भ॒स्व॒ । इ॒न्द्र॒घो॒ष इती᳚न्द्र - घो॒षः । त्वा॒ । वसु॑भि॒रिति॒ वसु॑ - भिः॒ । पु॒रस्ता᳚त् । पा॒तु॒ । मनो॑जवा॒ इति॒ मनः॑ - ज॒वाः॒ । त्वा॒ । पि॒तृभि॒रिति॑ पि॒तृ - भिः॒ । द॒क्षि॒ण॒तः । पा॒तु॒ । प्रचे॑ता॒ इति॒ प्र - चे॒ताः॒ । त्वा॒ । रु॒द्रैः । प॒श्चात् । पा॒तु॒ । वि॒श्वक॒र्मेति॑ वि॒श्व - क॒र्मा॒ । त्वा॒ । आ॒दि॒त्यैः । उ॒त्त॒र॒त इत्यु॑त् - त॒र॒तः । पा॒तु॒ । सिꣳ॒॒हीः । अ॒सि॒ । स॒प॒त्न॒सा॒हीति॑ सपत्न - सा॒ही । स्वाहा᳚ । सिꣳ॒॒हीः । अ॒सि॒ । सु॒प्र॒जा॒वनि॒रिति॑ सुप्रजा - वनिः॑ । स्वाहा᳚ । सिꣳ॒॒हीः । \textbf{  23} \newline
                  \newline
                                \textbf{ TS 1.2.12.3} \newline
                  अ॒सि॒ । रा॒य॒स्पो॒ष॒वनि॒रिति॑ रायस्पोष - वनिः॑ । स्वाहा᳚ । सिꣳ॒॒हीः । अ॒सि॒ । आ॒दि॒त्य॒वनि॒रित्या॑दित्य - वनिः॑ । स्वाहा᳚ । सिꣳ॒॒हीः । अ॒सि॒ । एति॑ । व॒ह॒ । दे॒वान् । दे॒व॒य॒त इति॑ देव - य॒ते । यज॑मानाय । स्वाहा᳚ । भू॒तेभ्यः॑ । त्वा॒ । वि॒श्वायु॒रिति॑ वि॒श्व - आ॒युः॒ । अ॒सि॒ । पृ॒थि॒वीम् । दृꣳ॒॒ह॒ । ध्रु॒व॒क्षिदिति॑ ध्रुव - क्षित् । अ॒सि॒ । अ॒न्तरि॑क्षम् । दृꣳ॒॒ह॒ । अ॒च्यु॒त॒क्षिदित्य॑च्युत -क्षित् । अ॒सि॒ । दिव᳚म् । दृꣳ॒॒ह॒ । अ॒ग्नेः । भस्म॑ । अ॒सि॒ । अ॒ग्नेः । पुरी॑षम् । अ॒सि॒ ॥ \textbf{  24} \newline
                  \newline
                      (नाम॑-सुप्रजा॒वनिः॒ स्वाहा॑ सिꣳ॒॒ह॑007आ;ः-पञ्च॑त्रिꣳशच्च )  \textbf{(A12)} \newline \newline
                                \textbf{ TS 1.2.13.1} \newline
                  यु॒ञ्जते᳚ । मनः॑ । उ॒त । यु॒ञ्ज॒ते॒ । धियः॑ । विप्राः᳚ । विप्र॑स्य । बृ॒ह॒तः । वि॒प॒श्चितः॑ ॥ वीति॑ । होत्राः᳚ । द॒धे॒ । व॒यु॒ना॒विदिति॑ वयुन - वित् । एकः॑ । इत् । म॒ही । दे॒वस्य॑ । स॒वि॒तुः । परि॑ष्टुति॒रिति॒ परि॑ - स्तु॒तिः॒ ॥ सु॒वागिति॑ सु- वाक् । दे॒व॒ । दुर्यान्॑ । एति॑ । व॒द॒ । दे॒व॒श्रुता॒विति॑ देव - श्रुतौ᳚ । दे॒वेषु॑ । एति॑ । घो॒षे॒था॒म् । एति॑ । नः॒ । वी॒रः । जा॒य॒ता॒म् । क॒र्म॒ण्यः॑ । यम् । सर्वे᳚ । अ॒नु॒जीवा॒मेत्य॑नु - जीवा॑म । यः । ब॒हू॒नाम् । अस॑त् । व॒शी ॥ इ॒दम् । विष्णुः॑ । वीति॑ । च॒क्र॒मे॒ । त्रे॒धा । नीति॑ । द॒धे॒ । प॒दम् ॥ समू॑ढ॒मिति॒ सम् - ऊ॒ढ॒म् । अ॒स्य॒ । \textbf{  25} \newline
                  \newline
                                \textbf{ TS 1.2.13.2} \newline
                  पाꣳ॒॒सु॒रे । इरा॑वती॒ इतीरा᳚ - व॒ती॒ । धे॒नु॒मती॒ इति॑ धेनु - मती᳚ । हि । भू॒तम् । सू॒य॒व॒सिनी॒ इति॑ सु - य॒व॒सिनी᳚ । मन॑वे । य॒श॒स्ये॑ इति॑ ॥ वीति॑ । अ॒स्क॒भ्ना॒त् । रोद॑सी॒ इति॑ । विष्णुः॑ । ए॒ते इति॑ । दा॒धार॑ । पृ॒थि॒वीम् । अ॒भितः॑ । म॒यूखैः᳚ ॥ प्राची॒ इति॑ । प्रेति॑ । इ॒त॒म् । अ॒द्ध्व॒रम् । क॒ल्पय॑न्ती॒ इति॑ । ऊ॒र्द्ध्वम् । य॒ज्ञ्म् । न॒य॒त॒म् । मा । जी॒ह्व॒र॒त॒म् । अत्र॑ । र॒मे॒था॒म् । वर्.ष्मन्न्॑ । पृ॒थि॒व्याः । दि॒वः । वा॒ । वि॒ष्णो॒ । उ॒त । वा॒ । पृ॒थि॒व्याः । म॒हः । वा॒ । वि॒ष्णो॒ । उ॒त । वा॒ । अ॒न्तरि॑क्षात् । हस्तौ᳚ । पृ॒ण॒स्व॒ । ब॒हुभि॒रिति॑ ब॒हु - भिः॒ । व॒स॒व्यैः᳚ । आ* । प्रेति॑ । य॒च्छ॒ । \textbf{  26} \newline
                  \newline
                                \textbf{ TS 1.2.13.3} \newline
                  दक्षि॑णात् । एति॑ । उ॒त । स॒व्यात् ॥ विष्णोः᳚ । नुक᳚म् । वी॒र्या॑णि । प्रेति॑ । वो॒च॒म् । यः । पार्थि॑वानि । वि॒म॒म इति॑ वि - म॒मे । रजाꣳ॑सि । यः । अस्क॑भायत् । उत्त॑र॒मित्युत् - त॒र॒म् । स॒धस्थ॒मिति॑ स॒ध - स्थ॒म् । वि॒च॒क्र॒मा॒ण इति॑ वि - च॒क्र॒मा॒णः । त्रे॒धा । उ॒रु॒गा॒य इत्यु॑रु - गा॒यः । विष्णोः᳚ । र॒राट᳚म् । अ॒सि॒ । विष्णोः᳚ । पृ॒ष्ठम् । अ॒सि॒ । विष्णोः᳚ । श्नप्त्रे॒ इति॑ । स्थः॒ । विष्णोः᳚ । स्यूः । अ॒सि॒ । विष्णोः᳚ । ध्रु॒वम् । अ॒सि॒ । वै॒ष्ण॒वम् । अ॒सि॒ । विष्ण॑वे । त्वा॒ ॥ \textbf{  27} \newline
                  \newline
                      ( अ॒स्य॒-य॒च्छैका॒न्न च॑त्वारिꣳ॒॒शच्च॑ )  \textbf{(A13)} \newline \newline
                                \textbf{ TS 1.2.14.1} \newline
                  कृ॒णु॒ष्व । पाजः॑ । प्रसि॑ति॒मिति॒ प्र - सि॒ति॒म् । न । पृ॒थ्वीम् । या॒हि । राजा᳚ । इ॒व॒ । अम॑वा॒नित्यम॑ - वा॒न् । इभे॑न ॥ तृ॒ष्वीम् । अन्विति॑ । प्रसि॑ति॒मिति॒ प्र - सि॒ति॒म् । द्रू॒णा॒नः । अस्ता᳚ । अ॒सि॒ । विद्ध्य॑ । र॒क्षसः॑ । तपि॑ष्ठैः ॥ तव॑ । भ्र॒मासः॑ । आ॒शु॒या । प॒त॒न्ति॒ । अन्विति॑ । स्पृ॒श॒ । धृ॒ष॒ता । शोशु॑चानः ॥ तपूꣳ॑षि । अ॒ग्ने॒ । जु॒ह्वा᳚ । प॒त॒ङ्गान् । अस॑न्दित॒ इत्यसं᳚ - दि॒तः॒ । वीति॑ । सृ॒ज॒ । विष्व॑क् । उ॒ल्काः ॥ प्रतीति॑ । स्पशः॑ । वीति॑ । सृ॒ज॒ । तूर्णि॑तम॒ इति॒ तूर्णि॑ - त॒मः॒ । भव॑ । पा॒युः । वि॒शः । अ॒स्याः । अद॑ब्दः ॥ यः । नः॒ । दू॒रे । अ॒घशꣳ॑स॒ इत्य॒घ - शꣳ॒॒सः॒ । \textbf{  28} \newline
                  \newline
                                \textbf{ TS 1.2.14.2} \newline
                  यः । अन्ति॑ । अग्ने᳚ । माकिः॑ । ते॒ । व्यथिः॑ । एति॑ । द॒ध॒र्.षी॒त् ॥ उदिति॑ । अ॒ग्ने॒ । ति॒ष्ठ॒ । प्रईति॑ । एति॑ । त॒नु॒ष्व॒ । नीति॑ । अ॒मित्रान्॑ । ओ॒ष॒ता॒त् । ति॒ग्म॒हे॒त॒ इति॑ तिग्म - हे॒ते॒ ॥ यः । नः॒ । अरा॑तिम् । स॒मि॒धा॒नेति॑ सम् - इ॒धा॒न॒ । च॒क्रे । नी॒चा । तम् । ध॒क्षि॒ । अ॒त॒सम् । न । शुष्क᳚म् ॥ ऊ॒र्द्ध्वः । भ॒व॒ । प्रतीति॑ । वि॒द्ध्य॒ । अधीति॑ । अ॒स्मत् । आ॒विः । कृ॒णु॒ष्व॒ । दैव्या॑नी । अ॒ग्ने॒ ॥ अवेति॑ । स्थि॒रा । त॒नु॒हि॒ । या॒तु॒जूना᳚म् । जा॒मिम् । अजा॑मिम् । प्रेति॑ । मृ॒णी॒हि॒ । शत्रून्॑ ॥ सः । ते॒ । \textbf{  29} \newline
                  \newline
                                \textbf{ TS 1.2.14.3} \newline
                  जा॒ना॒ति॒ । सु॒म॒तिमिति॑ सु - म॒तिम् । य॒वि॒ष्ठ॒ । यः । ईव॑ते । ब्रह्म॑णे । गा॒तुम् । ऐर॑त् ॥ विश्वा॑नि । अ॒स्मै॒ । सु॒दिना॒नीति॑ सु - दिना॑नि । रा॒यः । द्यु॒म्नानि॑ । अ॒र्यः । वीति॑ । दुरः॑ । अ॒भीति॑ । द्यौ॒त् ॥ सः । इत् । अ॒ग्ने॒ । अ॒स्तु॒ । सु॒भग॒ इति॑ सु - भगः॑ । सु॒दानु॒रिति॑ सु - दानुः॑ । यः । त्वा॒ । नित्ये॑न । ह॒विषा᳚ । यः । उ॒क्थैः ॥ पिप्री॑षति । स्वे । आयु॑षि । दु॒रो॒ण इति॑ दुः - ओ॒ने । विश्वा᳚ । इत् । अ॒स्मै॒ । सु॒दिनेति॑ सु- दिना᳚ । सा । अस॑त् । इ॒ष्टिः ॥ अर्चा॑मि । ते॒ । सु॒म॒तिमिति॑ सु - म॒तिम् । घोषि॑ । अ॒र्वाक् । समिति॑ । ते॒ । वा॒वाता᳚ । ज॒र॒तां॒ । \textbf{  30} \newline
                  \newline
                                \textbf{ TS 1.2.14.4} \newline
                  इ॒यम् । गीः ॥ स्वश्वा॒ इति॑ सु - अश्वाः᳚ । त्वा॒ । सु॒रथा॒ इति॑ सु - रथाः᳚ । म॒र्ज॒ये॒म॒ । अ॒स्मे इति॑ । क्ष॒त्राणि॑ । धा॒र॒येः॒ । अन्विति॑ । द्यून् ॥ इ॒ह । त्वा॒ । भूरि॑ । एति॑ । च॒रे॒त् । उपेति॑ । त्‍मन्न् । दोषा॑वस्त॒रिति॒ दोषा᳚ - व॒स्तः॒ । दी॒दि॒वाꣳस᳚म् । अन्विति॑ । द्यून् ॥ क्रीड॑न्तः । त्वा॒ । सु॒मन॑स॒ इति॑ सु - मन॑सः । स॒पे॒म॒ । अ॒भीति॑ । द्यु॒म्ना । त॒स्थि॒वाꣳसः॑ । जना॑नाम् ॥ यः । त्वा॒ । स्वश्व॒ इति॑ सु - अश्वः॑ । सु॒हि॒र॒ण्य इति॑ सु - हि॒र॒ण्यः । अ॒ग्ने॒ । उ॒प॒यातीत्यु॑प - याति॑ । वसु॑म॒तेति॒ वसु॑ - म॒ता॒ । रथे॑न ॥ तस्य॑ । त्रा॒ता । भ॒व॒सि॒ । तस्य॑ । सखा᳚ । यः । ते॒ । आ॒ति॒थ्यम् । आ॒नु॒षक् । जुजो॑षत् ॥ म॒हः । रु॒जा॒मि॒ । \textbf{  31} \newline
                  \newline
                                \textbf{ TS 1.2.14.5} \newline
                  ब॒न्धुता᳚ । वचो॑भि॒रिति॒ वचः॑ - भिः॒ । तत् । मा॒ । पि॒तुः । गोत॑मात् । अन्विति॑ । इ॒या॒य॒ ॥ त्वम् । नः॒ । अ॒स्य । वच॑सः । चि॒कि॒द्धि॒ । होतः॑ । य॒वि॒ष्ठ॒ । सु॒क्र॒तो॒ इति॑ सु - क्र॒तो॒ । दमू॑नाः ॥ अस्व॑प्नज॒ इत्यस्व॑प्न - जः॒ । त॒रण॑यः । सु॒शेवा॒ इति॑ सु-शेवाः᳚ । अत॑न्द्रासः । अ॒वृ॒काः । अश्र॑मिष्ठाः ॥ ते । पा॒यवः॑ । स॒द्ध्रिय॑ञ्चः । नि॒षद्येति॑ नि - सद्य॑ । अग्ने᳚ । तव॑ । नः॒ । पा॒न्तु॒ । अ॒मू॒र॒ ॥ ये । पा॒यवः॑ । मा॒म॒ते॒यम् । ते॒ । अ॒ग्ने॒ । पश्य॑न्तः । अ॒न्धम् । दु॒रि॒तादिति॑ दुः - इ॒तात् । अर॑क्षन्न् ॥ र॒रक्ष॑ । तान् । सु॒कृत॒ इति॑ सु - कृतः॑ । वि॒श्ववे॑दा॒ इति॑ वि॒श्व - वे॒दाः॒ । दिफ्स॑न्तः । इत् । रि॒पवः॑ । न । ह॒ । \textbf{  32} \newline
                  \newline
                                \textbf{ TS 1.2.14.6} \newline
                  दे॒भुः॒ ॥ त्वया᳚ । व॒यम् । स॒ध॒न्य॑ इति॑ सध - न्यः॑ । त्वोताः᳚ । तव॑ । प्रणी॒तीति॒ प्र - नी॒ती॒ । अ॒श्या॒म॒ । वाजान्॑ ॥ उ॒भा । शꣳसा᳚ । सू॒द॒य॒ । स॒त्य॒ता॒त॒ इति॑ सत्य - ता॒ते॒ । अ॒नु॒ष्ठु॒या । कृ॒णु॒हि॒ । अ॒ह्र॒या॒ण॒ ॥ अ॒या । ते॒ । अ॒ग्ने॒ । स॒मिधेति॑ सं - इधा᳚ । वि॒धे॒म॒ । प्रतीति॑ । स्तोम᳚म् । श॒स्यमा॑नम् । गृ॒भा॒य॒ ॥ दह॑ । अ॒शसः॑ । र॒क्षसः॑ । पा॒हि । अ॒स्मान् । द्रु॒हः । नि॒दः । मि॒त्र॒म॒ह॒ इति॑ मित्र - म॒हः॒ । अ॒व॒द्यात् ॥ र॒क्षो॒हण॒मिति॑ रक्षः - हन᳚म् । वा॒जिन᳚म् । एति॑ । जि॒घ॒र्मि॒ । मि॒त्रम् । प्रथि॑ष्ठम् । उपेति॑ । या॒मि॒ । शर्म॑ ॥ शिशा॑नः । अ॒ग्निः । क्रतु॑भि॒रिति॒ क्रतु॑ - भिः॒ । समि॑द्ध॒ इति॒ सं - इ॒द्धः॒ । सः । नः॒ । दिवा᳚ । \textbf{  33} \newline
                  \newline
                                \textbf{ TS 1.2.14.7} \newline
                  सः । रि॒षः । पा॒तु॒ । नक्त᳚म् ॥ वीति॑ । ज्योति॑षा । बृ॒ह॒ता । भा॒ति॒ । अ॒ग्निः । आ॒विः । विश्वा॑नि । कृ॒णु॒ते॒ । म॒हि॒त्वेति॑ महि - त्वा ॥ प्रेति॑ । अदे॑वीः । मा॒याः । स॒ह॒ते॒ । दु॒रेवा॒ इति॑ दुः - एवाः᳚ । शिशी॑ते । शृङ्गे॒ इति॑ । रक्ष॑से । वि॒निक्ष॒ इति॑ वि - निक्षे᳚ ॥ उ॒त । स्वा॒नासः॑ । दि॒वि । स॒न्तु॒ । अ॒ग्नेः । ति॒ग्मायु॑धा॒ इति॑ ति॒ग्म - आ॒यु॒धाः॒ । रक्ष॑से । हन्त॒वै । उ॒ ॥ मदे᳚ । चि॒त् । अ॒स्य॒ । प्रेति॑ । रु॒ज॒न्ति॒ । भामाः᳚ । न । व॒र॒न्ते॒ । प॒रि॒बाध॒ इति॑ परि - बाधः॑ । अदे॑वीः ॥ \textbf{  34} \newline
                  \newline
                      (अ॒घशꣳ॑सः॒-स ते॑-जरताꣳ-रुजामि-ह॒ -दिवै - क॑चत्वारिꣳशच्च)  \textbf{(A14)} \newline \newline
\textbf{praSna korvai with starting padams of 1 to 14 anuvAkams :-} \newline
(आप॑ उन्द॒न्त्वा-कू᳚त्यै॒-दैवी॑-मि॒यन्ते॒-वस्व्य॑स्य॒-ꣳ॒शुना॑ते॒-सोम॑न्त॒-उदायु॑षा॒ प्र च्य॑वस्वा॒- ग्नेरा॑ति॒थ्य -मꣳ॒॒शुरꣳ॑ शुर्-वि॒त्ताय॑नी मेऽसि -यु॒ञ्चते॑-कृणु॒ष्व पाज॒-श्चतु॑र्दश ।) \newline

\textbf{korvai with starting padams of1, 11, 21 series of pa~jcAtis :-} \newline
(आपो॒-वस्व्य॑सि॒ या तवे॒-यङ्गी-श्चतु॑स्त्रिꣳशत् ।) \newline

\textbf{first and last padam of second praSnam :-} \newline
(आप॑ उन्द॒न्-त्वदे॑वीः ।) \newline 


॥ हरिः॑ ॐ ॥
॥ कृष्ण यजुर्वेदीय तैत्तिरीय संहितायां प्रथमकाण्डे द्वितीयः प्रश्नः समाप्तः ॥ \newline
\pagebreak
\pagebreak
        


\end{document}
