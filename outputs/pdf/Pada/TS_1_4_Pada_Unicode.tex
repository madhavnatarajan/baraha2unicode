\documentclass[17pt]{extarticle}
\usepackage{babel}
\usepackage{fontspec}
\usepackage{polyglossia}
\usepackage{extsizes}



\setmainlanguage{sanskrit}
\setotherlanguages{english} %% or other languages
\setlength{\parindent}{0pt}
\pagestyle{myheadings}
\newfontfamily\devanagarifont[Script=Devanagari]{AdishilaVedic}


\newcommand{\VAR}[1]{}
\newcommand{\BLOCK}[1]{}




\begin{document}
\begin{titlepage}
    \begin{center}
 
\begin{sanskrit}
    { \Large
    ॐ नमः परमात्मने, श्री महागणपतये नमः, 
श्री गुरुभ्यो नमः । ह॒रिः॒ ॐ ॥ 
    }
    \\
    \vspace{2.5cm}
    \mbox{ \Huge
    1.4     प्रथमकाण्डे चतुर्त्थः प्रश्नः-(सुत्यादिने कर्तव्या ग्रहाः)   }
\end{sanskrit}
\end{center}

\end{titlepage}
\tableofcontents

ॐ नमः परमात्मने, श्री महागणपतये नमः, 
श्री गुरुभ्यो नमः । ह॒रिः॒ ॐ ॥ \newline
1.4     प्रथमकाण्डे चतुर्त्थः प्रश्नः-(सुत्यादिने कर्तव्या ग्रहाः) \newline

\addcontentsline{toc}{section}{ 1.4     प्रथमकाण्डे चतुर्त्थः प्रश्नः-(सुत्यादिने कर्तव्या ग्रहाः)}
\markright{ 1.4     प्रथमकाण्डे चतुर्त्थः प्रश्नः-(सुत्यादिने कर्तव्या ग्रहाः) \hfill https://www.vedavms.in \hfill}
\section*{ 1.4     प्रथमकाण्डे चतुर्त्थः प्रश्नः-(सुत्यादिने कर्तव्या ग्रहाः) }
                                \textbf{ TS 1.4.1.1} \newline
                  एति॑ । द॒दे॒ । ग्रावा᳚ । अ॒सि॒ । अ॒द्ध्व॒र॒कृदित्य॑द्ध्वर - कृत् । दे॒वेभ्यः॑ । ग॒भीं॒रम् । इ॒मम् । अ॒द्ध्व॒रम् । कृ॒धि॒ । उ॒त्त॒मेनेत्यु॑त् - त॒मेन॑ । प॒विना᳚ । इन्द्रा॑य । सोम᳚म् । सुषु॑त॒मिति॒ सु - सु॒त॒म् । मधु॑मन्त॒मिति॒ मधु॑ - म॒न्त॒म् । पय॑स्वन्तम् । वृ॒ष्टि॒वनि॒मिति॑ वृष्टि - वनि᳚म् । इन्द्रा॑य । त्वा॒ । वृ॒त्र॒घ्न इति॑ वृत्र - घ्ने । इन्द्रा॑य । त्वा॒ । वृ॒त्र॒तुर॒ इति॑ वृत्र - तुरे᳚ । इन्द्रा॑य । त्वा॒ । अ॒भि॒मा॒ति॒घ्न इत्य॑भिमाति - घ्ने । इन्द्रा॑य । त्वा॒ । आ॒दि॒त्यव॑त॒ इत्या॑दि॒त्य - व॒ते॒ । इन्द्रा॑य । त्वा॒ । वि॒श्वदे᳚व्यावत॒ इति॑ वि॒श्वदे᳚व्य - व॒ते॒ । श्वा॒त्राः । स्थ॒ । वृ॒त्र॒तुर॒ इति॑ वृत्र - तुरः॑ । राधो॑गूर्ता॒ इति॒ राधः॑ - गू॒र्ताः॒ । अ॒मृत॑स्य । पत्नीः᳚ । ताः । दे॒वीः॒ । दे॒व॒त्रेति॑ देव - त्रा । इ॒मम् । य॒ज्ञ्म् । ध॒त्त॒ । उप॑हूता॒ इत्युप॑ - हू॒ताः॒ । सोम॑स्य । पि॒ब॒त॒ । उप॑हूत॒ इत्युप॑-हू॒तः॒ । यु॒ष्माक᳚म् । \textbf{  1} \newline
                  \newline
                                \textbf{ TS 1.4.1.2} \newline
                  सोमः॑ । पि॒ब॒तु॒ । यत् । ते॒ । सो॒म॒ । दि॒वि । ज्योतिः॑ । यत् । पृ॒थि॒व्याम् । यत् । उ॒रौ । अ॒न्तरि॑क्षे । तेन॑ । अ॒स्मै । यज॑मानाय । उ॒रु । रा॒या । कृ॒धि॒ । अधीति॑ । दा॒त्रे । वो॒चः॒ । धिष॑णे॒ इति॑ । वी॒डू इति॑ । स॒ती इति॑ । वी॒ड॒ये॒था॒म् । ऊर्ज᳚म् । द॒धा॒था॒म् । ऊर्ज᳚म् । मे॒ । ध॒त्त॒म् । मा । वा॒म् । हिꣳ॒॒॒सि॒ष॒म् । मा । मा॒ । हिꣳ॒॒सि॒ष्ट॒म् । प्राक् । अपा᳚क् । उद॑क् । अ॒ध॒राक् । ताः । त्वा॒ । दिशः॑ । एति॑ । धा॒व॒न्तु॒ । अबं॑ । नीति॑ । स्व॒र॒ ॥ यत् । ते॒ ( ) । सो॒म॒ । अदा᳚भ्यम् । नाम॑ । जागृ॑वि । तस्मै᳚ । ते॒ । सो॒म॒ । सोमा॑य । स्वाहा᳚ ॥ \textbf{  2 } \newline
                  \newline
                      (यु॒ष्माकꣳ॑ - स्वर॒ यत्ते॒ -नव॑ च )  \textbf{(A1)} \newline \newline
                                \textbf{ TS 1.4.2.1} \newline
                  वा॒चः । पत॑ये । प॒व॒स्व॒ । वा॒जि॒न्न् । वृषा᳚ । वृष्णः॑ । अꣳ॒॒शुभ्या॒मित्यꣳ॒॒शु - भ्या॒म् । गभ॑स्तिपूत॒ इति॒ गभ॑स्ति - पू॒तः॒ । दे॒वः । दे॒वाना᳚म् । प॒वित्र᳚म् । अ॒सि॒ । येषा᳚म् । भा॒गः । असि॑ । तेभ्यः॑ । त्वा॒ । स्वाङ्कृ॑तः । अ॒सि॒ । मधु॑मती॒रिति॒ मधु॑ - म॒तीः॒ । नः॒ । इषः॑ । कृ॒धि॒ । विश्वे᳚भ्यः । त्वा॒ । इ॒न्द्रि॒येभ्यः॑ । दि॒व्येभ्यः॑ । पार्त्थि॑वेभ्यः । मनः॑ । त्वा॒ । अ॒ष्टु॒ । उ॒रु । अ॒न्तरि॑क्षम् । अन्विति॑ । इ॒हि॒ । स्वाहा᳚ । त्वा॒ । सु॒भ॒व॒ इति॑ सु - भ॒वः॒ । सूर्या॑य । दे॒वेभ्यः॑ । त्वा॒ । म॒री॒चि॒पेभ्य॒ इति॑ मरीचि - पेभ्यः॑ । ए॒षः । ते॒ । योनिः॑ । प्रा॒णायेति॑ प्र - अ॒नाय॑ । त्वा॒ ॥ \textbf{  3 } \newline
                  \newline
                      (वा॒चः-स॒प्तच॑त्वारिꣳशत्)  \textbf{(A2)} \newline \newline
                                \textbf{ TS 1.4.3.1} \newline
                  उ॒प॒या॒मगृ॑हीत॒ इत्यु॑पया॒म - गृ॒ही॒तः॒ । अ॒सि॒ । अ॒न्तः । य॒च्छ॒ । म॒घ॒व॒न्निति॑ मघ - व॒न्न् । पा॒हि । सोम᳚म् । उ॒रु॒ष्य । रायः॑ । समिति॑ । इषः॑ । य॒ज॒स्व॒ । अ॒न्तः । ते॒ । द॒धा॒मि॒ । द्यावा॑पृथि॒वी इति॒ द्यावा᳚ - पृ॒थि॒वी । अ॒न्तः । उ॒रु । अ॒न्तरि॑क्षम् । स॒जोषा॒ इति॑ स - जोषाः᳚ । दे॒वैः । अव॑रैः । परैः᳚ । च॒ । अ॒न्त॒र्या॒म इत्य॑न्तः-या॒मे । म॒घ॒व॒न्निति॑ मघ - व॒न्न् । मा॒द॒य॒स्व॒ । स्वाङ्कृ॑तः । अ॒सि॒ । मधु॑मती॒रिति॒ मधु॑ - म॒तीः॒ । नः॒ । इषः॑ । कृ॒धि॒ । विश्वे᳚भ्यः । त्वा॒ । इ॒न्द्रि॒येभ्यः॑ । दि॒व्येभ्यः॑ । पार्त्थि॑वेभ्यः । मनः॑ । त्वा॒ । अ॒ष्टु॒ । उ॒रु । अ॒न्तरि॑क्षम् । अन्विति॑ । इ॒हि॒ । स्वाहा᳚ । त्वा॒ । सु॒भ॒व॒ इति॑ सु - भ॒वः॒ । सूर्या॑य । दे॒वेभ्यः॑( ) । त्वा॒ । म॒री॒चि॒पेभ्य॒ इति॑ मरीचि - पेभ्यः॑ । ए॒षः । ते॒ । योनिः॑ । अ॒पा॒नायेत्य॑प - अ॒नाय॑ । त्वा॒ ॥ \textbf{ } \newline
                  \newline
                      योनि॑रपा॒नाय॑ त्वा ॥ 4 (दे॒वेभ्यः॑-स॒प्त च॑)  \textbf{(A3)} \newline \newline
                                \textbf{ TS 1.4.4.1} \newline
                  एति॑ । वा॒यो॒ इति॑ । भू॒ष॒ । शु॒चि॒पा॒ इति॑ शुचि - पाः॒ । उपेति॑ । नः॒ । स॒हस्र᳚म् । ते॒ । नि॒युत॒ इति॑ नि - युतः॑ । वि॒श्व॒वा॒रेति॑ विश्व - वा॒र॒ ॥ उपो॒ इति॑ । ते॒ । अन्धः॑ । मद्य᳚म् । अ॒या॒मि॒ । यस्य॑ । दे॒व॒ । द॒धि॒षे । पू॒र्व॒पेय॒मिति॑ पूर्व - पेय᳚म् ॥ उ॒प॒या॒मगृ॑हीत॒ इत्यु॑पया॒म - गृ॒ही॒तः॒ । अ॒सि॒ । वा॒यवे᳚ । त्वा॒ । इन्द्र॑वायू॒ इतीन्द्र॑ - वा॒यू॒ । इ॒मे । सु॒ताः ॥ उपेति॑ । प्रयो॑भि॒रिति॒ प्रयः॑ - भिः॒ । एति॑ । ग॒त॒म् । इन्द॑वः । वा॒म् । उ॒शन्ति॑ । हि ॥ उ॒प॒या॒मगृ॑हीत॒ इत्यु॑पया॒म - गृ॒ही॒तः॒ । अ॒सि॒ । इ॒न्द्र॒वा॒युभ्या॒मिती᳚न्द्रवा॒यु - भ्या॒म् । त्वा॒ । ए॒षः । ते॒ । योनिः॑ । स॒जोषा᳚भ्या॒मिति॑ स - जोषा᳚भ्याम् । त्वा॒ ॥ \textbf{  5} \newline
                  \newline
                      (आ वा॑यो॒- त्रिच॑त्वारिꣳशत्)  \textbf{(A4)} \newline \newline
                                \textbf{ TS 1.4.5.1} \newline
                  अ॒यम् । वा॒म् । मि॒त्रा॒व॒रु॒णेति॑ मित्रा - व॒रु॒णा॒ । सु॒तः । सोमः॑ । ऋ॒ता॒वृ॒धेत्यृ॑त - वृ॒धा॒ ॥ मम॑ । इत् । इ॒ह । श्रु॒त॒म् । हव᳚म् ॥ उ॒प॒या॒मगृ॑हीत॒ इत्यु॑पया॒म - गृ॒हीतः॒ । अ॒सि॒ । मि॒त्रावरु॑णाभ्या॒मिति॑ मि॒त्रा - वरु॑णाभ्याम् । त्वा॒ । ए॒षः । ते॒ । योनिः॑ । ऋ॒ता॒युभ्या॒मित्यृ॑ता॒यु - भ्या॒म् । त्वा॒ ॥ \textbf{  6 } \newline
                  \newline
                      (अ॒यं ॅवां᳚ - ॅविꣳश॒तिः)  \textbf{(A5)} \newline \newline
                                \textbf{ TS 1.4.6.1} \newline
                  या । वा॒म् । कशा᳚ । मधु॑म॒तीति॒ मधु॑ - म॒ती॒ । अश्वि॑ना । सू॒नृता॑व॒तीति॑ सू॒नृता᳚ - व॒ती॒ ॥ तया᳚ । य॒ज्ञ्म् । मि॒मि॒क्ष॒त॒म् ॥ उ॒प॒या॒मगृ॑हीत॒ इत्यु॑पया॒म - गृ॒ही॒तः॒ । अ॒सि॒ । अ॒श्विभ्या॒मित्य॒श्वि - भ्या॒म् । त्वा॒ । ए॒षः । ते॒ । योनिः॑ । माद्ध्वी᳚भ्याम् । त्वा॒ ॥ \textbf{  7 } \newline
                  \newline
                      (या वा॑- म॒ष्टाद॑श)  \textbf{(A6)} \newline \newline
                                \textbf{ TS 1.4.7.1} \newline
                  प्रा॒त॒र्युजा॒विति॑ प्रातः - युजौ᳚ । वीति॑ । मु॒च्ये॒था॒म् । अश्वि॑नौ । एति॑ । इ॒ह । ग॒च्छ॒त॒म् ॥ अ॒स्य । सोम॑स्य । पी॒तये᳚ ॥ उ॒प॒या॒मगृ॑हीत॒ इत्यु॑पया॒म - गृ॒ह॒तः॒ । अ॒सि॒ । अ॒श्विभ्या॒मित्य॒श्वि - भ्या॒म् । त्वा॒ । ए॒षः । ते॒ । योनिः॑ । अ॒श्विभ्या॒मित्य॒श्वि - भ्या॒म् । त्वा॒ ॥ \textbf{  8} \newline
                  \newline
                      (प्रा॒त॒र्युजा॒वे-का॒न्नविꣳ॑श॒तिः)  \textbf{(A7)} \newline \newline
                                \textbf{ TS 1.4.8.1} \newline
                  अ॒यम् । वे॒नः । चो॒द॒य॒त्॒ । पृश्नि॑गर्भा॒ इति॒ पृश्नि॑ - ग॒र्भाः॒ । ज्योति॑र्जरायु॒रिति॒ ज्योतिः॑ - ज॒रा॒युः॒ । रज॑सः । वि॒मान॒ इति॑ वि - माने᳚ ॥ इ॒मम् । अ॒पाम् । स॒गं॒म इति॑ सं - ग॒मे । सूर्य॑स्य । शिशु᳚म् । न । विप्राः᳚ । म॒तिभि॒रिति॑ म॒ति - भिः॒ । रि॒ह॒न्ति॒ ॥ उ॒प॒या॒मगृ॑हीत॒ इत्यु॑पया॒म - गृही॒तः॒ । अ॒सि॒ । शण्डा॑य । त्वा॒ । ए॒षः । ते॒ । योनिः॑ । वी॒रता᳚म् । पा॒हि॒ ॥ \textbf{  9} \newline
                  \newline
                      (अ॒यं ॅवे॒नः- पञ्च॑विꣳशतिः)  \textbf{(A8)} \newline \newline
                                \textbf{ TS 1.4.9.1} \newline
                  तम् । प्र॒त्नथा᳚ । पू॒र्वथा᳚ । वि॒श्वथा᳚ । इ॒मथा᳚ । ज्ये॒ष्ठता॑ति॒मिति॑ ज्ये॒ष्ठ - ता॒ति॒म् । ब॒र्॒.हि॒षद॒मिति॑ बर्.हि - सद᳚म् । सु॒व॒र्विद॒मिति॑ सुवः - विद᳚म् । प्र॒ती॒ची॒नम् । वृ॒जन᳚म् । दो॒ह॒से॒ । गि॒रा । आ॒शुम् । जय॑न्तम् । अन्विति॑ । यासु॑ । वर्ध॑से ॥ उ॒प॒या॒मगृ॑हीत॒ इत्यु॑पया॒म - गृ॒ही॒तः॒ । अ॒सि॒ । मर्का॑य । त्वा॒ । ए॒षः । ते॒ । योनिः॑ । प्र॒जा इति॑ प्र - जाः । पा॒हि॒ ॥ \textbf{  10 } \newline
                  \newline
                      (तं प्र॒त्नया॒-षट्विꣳ॑शतिः )  \textbf{(A9)} \newline \newline
                                \textbf{ TS 1.4.10.1} \newline
                  ये । दे॒वाः॒ । दि॒वि । एका॑दश । स्थ । पृ॒थि॒व्याम् । अधीति॑ । एका॑दश । स्थ । अ॒फ्सु॒षद॒ इत्य॑फ्सु - सदः॑ । म॒हि॒ना । एका॑दश । स्थ । ते । दे॒वाः॒ । य॒ज्ञ्म् । इ॒मम् । जु॒ष॒द्ध्व॒म् । उ॒प॒या॒मगृ॑हीत॒ इत्यु॑पया॒म - गृ॒ही॒तः॒ । अ॒सि॒ । आ॒ग्र॒य॒णः । अ॒सि॒ । स्वा᳚ग्रयण॒ इति॒ सु - आ॒ग्र॒य॒णः॒ । जिन्व॑ । य॒ज्ञ्म् । जिन्व॑ । य॒ज्ञ्प॑ति॒मिति॑ य॒ज्ञ् - प॒ति॒म् । अ॒भीति॑ । सव॑ना । पा॒हि॒ । विष्णुः॑ । त्वाम् । पा॒तु॒ । विश᳚म् । त्वम् । पा॒हि॒ । इ॒न्द्रि॒येण॑ । ए॒षः । ते॒ । योनिः॑ । विश्वे᳚भ्यः । त्वा॒ । दे॒वेभ्यः॑ ॥ \textbf{  11 } \newline
                  \newline
                      ये दे॑वा॒-स्त्रिच॑त्वारिꣳशत्)  \textbf{(A10)} \newline \newline
                                \textbf{ TS 1.4.11.1} \newline
                  त्रिꣳ॒॒शत् । त्रयः॑ । च॒ । ग॒णिनः॑ । रु॒जन्तः॑ । दिव᳚म् । रु॒द्राः । पृ॒थि॒वीम् । च॒ । स॒च॒न्ते॒ ॥ ए॒का॒द॒शासः॑ । अ॒फ्सु॒षद॒ इत्य॑फ्सु-सदः॑ । सु॒तम् । सोम᳚म् । जु॒ष॒न्ता॒म् । सव॑नाय । विश्वे᳚ ॥ उ॒प॒या॒मगृ॑हीत॒ इत्यु॑पया॒म - गृ॒ही॒तः॒ । अ॒सि॒ । आ॒ग्र॒य॒णः । अ॒सि॒ । स्वा᳚ग्रयण॒ इति॒ सु - आ॒ग्र॒य॒णः॒ । जिन्व॑ । य॒ज्ञ्म् । जिन्व॑ । य॒ज्ञ्प॑ति॒मिति॑ य॒ज्ञ् - प॒ति॒म् । अ॒भीति॑ । सव॑ना । पा॒हि॒ । विष्णुः॑ । त्वाम् । पा॒तु॒ । विश᳚म् । त्वम् । पा॒हि॒ । इ॒न्द्रि॒येण॑ । ए॒षः । ते॒ । योनिः॑ । विश्वे᳚भ्यः । त्वा॒ । दे॒वेभ्यः॑ ॥ \textbf{  12} \newline
                  \newline
                      (त्रिꣳ॒॒शत्त्रयो॒-द्विच॑त्वारिꣳशत्)  \textbf{(A11)} \newline \newline
                                \textbf{ TS 1.4.12.1} \newline
                  उ॒प॒या॒मगृ॑हीत॒ इत्यु॑पया॒म - गृ॒ही॒तः॒ । अ॒सि॒ । इन्द्रा॑य । त्वा॒ । बृ॒हद्व॑त॒ इति॑ बृ॒हत् - व॒ते॒ । वय॑स्वते । उ॒क्था॒युव॒ इत्यु॑क्थ - युवे᳚ । यत् । ते॒ । इ॒न्द्र॒ । बृ॒हत् । वयः॑ । तस्मै᳚ । त्वा॒ । विष्ण॑वे । त्वा॒ । ए॒षः । ते॒ । योनिः॑ । इन्द्रा॑य । त्वा॒ । उ॒क्था॒युव॒ इत्यु॑क्थ - युवे᳚ ॥ \textbf{  13} \newline
                  \newline
                      (उ॒प॒या॒मगृ॑हीतो॒ऽसीन्द्रा॑य॒-द्वाविꣳ॑शतिः)  \textbf{(A12)} \newline \newline
                                \textbf{ TS 1.4.13.1} \newline
                  मू॒र्धान᳚म् । दि॒वः । अ॒र॒तिम् । पृ॒थि॒व्याः । वै॒श्वा॒न॒रम् । ऋ॒ताय॑ । जा॒तम् । अ॒ग्निम् ॥ क॒विम् । स॒म्राज॒मिति॑ सं - राज᳚म् । अति॑थिम् । जना॑नाम् । आ॒सन्न् । एति॑ । पात्र᳚म् । ज॒न॒य॒न्त॒ । दे॒वाः ॥ उ॒प॒या॒मगृ॑हीत॒ इत्यु॑पया॒म - गृ॒ही॒तः॒ । अ॒सि॒ । अ॒ग्नये᳚ । त्वा॒ । वै॒श्वा॒न॒राय॑ । ध्रु॒वः । अ॒सि॒ । ध्रु॒वक्षि॑ति॒रिति॑ ध्रु॒व-क्षि॒तिः॒ । ध्रु॒वाणा᳚म् । ध्रु॒वत॑म॒ इति॑ ध्रु॒व - त॒मः॒ । अच्यु॑तानाम् । अ॒च्यु॒त॒क्षित्त॑म॒ इत्य॑च्युत॒क्षित् - त॒मः॒ । ए॒षः । ते॒ । योनिः॑ । अ॒ग्नये᳚ । त्वा॒ । वै॒श्वा॒न॒राय॑ ॥ \textbf{  14} \newline
                  \newline
                      (मू॒र्द्धानं॒-पञ्च॑त्रिꣳशत्)  \textbf{(A13)} \newline \newline
                                \textbf{ TS 1.4.14.1} \newline
                  मधुः॑ । च॒ । माध॑वः । च॒ । शु॒क्रः । च॒ । शुचिः॑ । च॒ । नभः॑ । च॒ । न॒भ॒स्यः॑ । च॒ । इ॒षः । च॒ । ऊ॒र्जः । च॒ । सहः॑ । च॒ । स॒ह॒स्यः॑ । च॒ । तपः॑ । च॒ । त॒प॒स्यः॑ । च॒ । उ॒प॒या॒मगृ॑हीत॒ इत्यु॑पया॒म-गृ॒ही॒तः॒ । अ॒सि॒ । सꣳ॒॒सर्प॒ इति॑ सं - सर्पः॑ । अ॒सि॒ । अꣳ॒॒ह॒स्प॒त्यायेत्यꣳ॑हः - प॒त्याय॑ । त्वा॒ ॥ \textbf{  15 } \newline
                  \newline
                      (मधु॑श्च-त्रिꣳ॒॒शत्)  \textbf{(A14)} \newline \newline
                                \textbf{ TS 1.4.15.1} \newline
                  इन्द्रा᳚ग्नी॒ इतीन्द्र॑ - अ॒ग्नी॒ । एति॑ । ग॒त॒म् । सु॒तम् । गी॒र्भिः । नभः॑ । वरे᳚ण्यम् ॥ अ॒स्य । पा॒त॒म् । धि॒या । इ॒षि॒ता ॥ उ॒प॒या॒मगृ॑हीत॒ इत्यु॑पया॒म - गृ॒ही॒तः॒ । अ॒सि॒ । इ॒न्द्रा॒ग्निभ्या॒मिती᳚न्द्रा॒ग्नि - भ्या॒म् । त्वा॒ । ए॒षः । ते॒ । योनिः॑ । इ॒न्द्रा॒ग्निभ्या॒मिती᳚न्द्रा॒ग्नि - भ्या॒म् । त्वा॒ ॥ \textbf{  16} \newline
                  \newline
                       इन्द्रा᳚ग्नी॒ विꣳश॒तिः)  \textbf{(A15)} \newline \newline
                                \textbf{ TS 1.4.16.1} \newline
                  ओमा॑सः । च॒र्॒.ष॒णी॒धृ॒त॒ इति॑ चर्.षणि - धृ॒तः॒ । विश्वे᳚ । दे॒वा॒सः॒ । एति॑ । ग॒त॒ ॥ दा॒श्वाꣳसः॑ । दा॒शुषः॑ । सु॒तम् ॥ उ॒प॒या॒मगृ॑हीत॒ इत्यु॑पया॒म - गृ॒ही॒तः॒ । अ॒सि॒ । विश्वे᳚भ्यः । त्वा॒ । दे॒वेभ्यः॑ । ए॒षः । ते॒ । योनिः॑ । विश्वे᳚भ्यः । त्वा॒ । दे॒वेभ्यः॑ ॥ \textbf{  17} \newline
                  \newline
                      (ओमा॑सो विꣳश॒तिः)  \textbf{(A16)} \newline \newline
                                \textbf{ TS 1.4.17.1} \newline
                  म॒रुत्व॑न्तम् । वृ॒ष॒भम् । वा॒वृ॒धा॒नम् । अक॑वारि॒मित्यक॑वा - अ॒रि॒म् । दि॒व्यम् । शा॒सम् । इन्द्र᳚म् ॥ वि॒श्वा॒साह॒मिति॑ विश्व-साह᳚म् । अव॑से । नूत॑नाय । उ॒ग्रम् । स॒हो॒दामिति॑ सहः - दाम् । इ॒ह । तम् । हु॒वे॒म॒ ॥ उ॒प॒या॒मगृ॑हीत॒ इत्यु॑पया॒म - गृ॒ही॒तः॒ । अ॒सि॒ । इन्द्रा॑य । त्वा॒ । म॒रुत्व॑ते । ए॒षः । ते॒ । योनिः॑ । इन्द्रा॑य । त्वा॒ । म॒रुत्व॑ते ॥ \textbf{  18} \newline
                  \newline
                      (म॒रुत्व॑न्तꣳ॒॒-षट्विꣳ॑शतिः)  \textbf{(A17)} \newline \newline
                                \textbf{ TS 1.4.18.1} \newline
                  इन्द्र॑ । म॒रु॒त्वः॒ । इ॒ह । पा॒हि॒ । सोम᳚म् । यथा᳚ । शा॒र्या॒ते । अपि॑बः । सु॒तस्य॑ ॥ तव॑ । प्रणी॒तीति॒ प्र - नी॒ती॒ । तव॑ । शू॒र॒ । शर्मन्न्॑ । एति॑ । वि॒वा॒स॒न्ति॒ । क॒वयः॑ । सु॒य॒ज्ञा इति॑ सु - य॒ज्ञाः ॥ उ॒प॒या॒मगृ॑हीत॒ इत्यु॑पया॒म - गृ॒ही॒तः॒ । अ॒सि॒ । इन्द्रा॑य । त्वा॒ । म॒रुत्व॑ते । ए॒षः । ते॒ । योनिः॑ । इन्द्रा॑य । त्वा॒ । म॒रुत्व॑ते ॥ \textbf{  19 } \newline
                  \newline
                      (इन्द्रै॒का॒न्न त्रिꣳ॒॒शत्)  \textbf{(A18)} \newline \newline
                                \textbf{ TS 1.4.19.1} \newline
                  म॒रुत्वान्॑ । इ॒न्द्र॒ । वृ॒ष॒भः । रणा॑य । पिब॑ । सोम᳚म् । अ॒नु॒ष्व॒धमित्य॑नु - स्व॒धम् । मदा॑य ॥ एति॑ । सि॒ञ्च॒स्व॒ । ज॒ठरे᳚ । मद्ध्वः॑ । ऊ॒र्मिम् । त्वम् । राजा᳚ । अ॒सि॒ । प्र॒दिव॒ इति॑ प्र - दिवः॑ । सु॒ताना᳚म् ॥ उ॒प॒या॒मगृ॑हीत॒ इत्यु॑पया॒म - गृ॒ही॒तः॒ । अ॒सि॒ । इन्द्रा॑य । त्वा॒ । म॒रुत्व॑ते । ए॒षः । ते॒ । योनिः॑ । इन्द्रा॑य । त्वा॒ । म॒रुत्व॑ते ॥ \textbf{  20} \newline
                  \newline
                      (म॒रुत्वा॒नेका॒न्नत्रिꣳ॒॒शत्)  \textbf{(A19)} \newline \newline
                                \textbf{ TS 1.4.20.1} \newline
                  म॒हान् । इन्द्रः॑ । यः । ओज॑सा । प॒र्जन्यः॑ । वृ॒ष्टि॒मानिति॑ वृष्टि-मान् । इ॒व॒ ॥ स्तोमैः᳚ । व॒थ्सस्य॑ । वा॒वृ॒धे॒ ॥ उ॒प॒या॒मगृ॑हीत॒ इत्यु॑पया॒म - गृ॒ही॒तः॒ । अ॒सि॒ । म॒हे॒न्द्रायेति॑ महा - इ॒न्द्राय॑ । त्वा॒ । ए॒षः । ते॒ । योनिः॑ । म॒हे॒न्द्रायेति॑ महा - इ॒न्द्राय॑ । त्वा॒ ॥ 21(19) \textbf{  21} \newline
                  \newline
                      (म॒हानेका॒न्नविꣳ॑शतिः)  \textbf{(A20)} \newline \newline
                                \textbf{ TS 1.4.21.1} \newline
                  म॒हान् । इन्द्रः॑ । नृ॒वदिति॑ नृ - वत् । एति॑ । च॒र्॒.ष॒णि॒प्रा इति॑ चर्.षणि - प्राः । उ॒त । द्वि॒बर्.हा॒ इति॑ द्वि - बर्.हाः᳚ । अ॒मि॒नः । सहो॑भि॒रिति॒ सहः॑ - भिः॒ ॥ अ॒स्म॒द्रिय॒गित्य॑स्म - द्रिय॑क् । वा॒वृ॒धे॒ । वी॒र्या॑य । उ॒रुः । पृ॒थुः । सुकृ॑त॒ इति॒ सु - कृ॒तः॒ । क॒र्तृभि॒रिति॑ क॒र्तृ - भिः॒ । भू॒त् ॥ उ॒प॒या॒मगृ॑हीत॒ इत्यु॑पया॒म - गृ॒ही॒तः॒ । अ॒सि॒ । म॒हे॒न्द्रायेति॑ महा - इ॒न्द्राय॑ । त्वा॒ । ए॒षः । ते॒ । योनिः॑ । म॒हे॒न्द्रायेति॑ महा - इ॒न्द्राय॑ । त्वा॒ ॥ \textbf{  22} \newline
                  \newline
                      (म॒हान् नृ॒वथ् - षड्विꣳ॑शतिः)  \textbf{(A21)} \newline \newline
                                \textbf{ TS 1.4.22.1} \newline
                  क॒दा । च॒न । स्त॒रीः । अ॒सि॒ । न । इ॒न्द्र॒ । स॒श्च॒सि॒ । दा॒शुषे᳚ ॥ उपो॒पेत्युप॑ - उ॒प॒ । इत् । नु । म॒घ॒व॒न्निति॑ मघ - व॒न्न्॒ । भूयः॑ । इत् । नु । ते॒ । दान᳚म् । दे॒वस्य॑ । पृ॒च्य॒ते॒ ॥ उ॒प॒या॒मगृ॑हीत॒ इत्यु॑पया॒म - गृ॒ही॒तः॒ । अ॒सि॒ । आ॒दि॒त्येभ्यः॑ । त्वा॒ ॥ क॒दा । च॒न । प्रेति॑ । यु॒च्छ॒सि॒ । उ॒भे इति॑ । नीति॑ । पा॒सि॒ । जन्म॑नी॒ इति॑ ॥ तुरी॑य । आ॒दि॒त्य॒ । सव॑नम् । ते॒ । इ॒न्द्रि॒यम् । एति॑ । त॒स्थौ॒ । अ॒मृत᳚म् । दि॒वि ॥ य॒ज्ञ्ः । दे॒वाना᳚म् । प्रतीति॑ । ए॒ति॒ । सु॒म्नम् । आदि॑त्यासः । भव॑त । मृ॒ड॒यन्तः॑ ॥ एति॑ । वः॒ ( ) । अ॒र्वाची᳚ । सु॒म॒तिरिति॑ सु - म॒तिः । व॒वृ॒त्या॒त् । अꣳ॒॒होः । चि॒त् । या । व॒रि॒वो॒वित्त॒रेति॑ वरिवो॒वित् - त॒रा॒ । अस॑त् ॥ विव॑स्वः । आ॒दि॒त्य॒ । ए॒षः । ते॒ । सो॒म॒पी॒थ इति॑ सोम - पी॒थः । तेन॑ । म॒न्द॒स्व॒ । तेन॑ । तृ॒प्य॒ । तृ॒प्यास्म॑ । ते॒ । व॒यम् । त॒र्प॒यि॒तारः॑ । या । दि॒व्या । वृष्टिः॑ । तया᳚ । त्वा॒ । श्री॒णा॒मि॒ ॥ \textbf{  23 } \newline
                  \newline
                      (वः॒- स॒प्तविꣳ॑शतिश्च)  \textbf{(A22)} \newline \newline
                                \textbf{ TS 1.4.23.1} \newline
                  वा॒मम् । अ॒द्य । स॒वि॒तः॒ । वा॒मम् । उ॒ । श्वः । दि॒वेदि॑व॒ इति॑ दि॒वे - दि॒वे॒ । वा॒मम् । अ॒स्मभ्य॒मित्य॒स्म - भ्य॒म् । सा॒वीः॒ ॥ वा॒मस्य॑ । हि । क्षय॑स्य । दे॒व॒ । भूरेः᳚ । अ॒या । धि॒या । वा॒म॒भाज॒ इति॑ वाम - भाजः॑ । स्या॒म॒ ॥ उ॒प॒या॒मगृ॑हीत॒ इत्यु॑पया॒म - गृ॒ही॒तः॒ । अ॒सि॒ । दे॒वाय॑ । त्वा॒ । स॒वि॒त्रे ॥ \textbf{  24} \newline
                  \newline
                      (वा॒मं-चतु॑र्विꣳशतिः)  \textbf{(A23)} \newline \newline
                                \textbf{ TS 1.4.24.1} \newline
                  अद॑ब्धेभिः । स॒वि॒तः॒ । पा॒युभि॒रिति॑ पा॒यु - भिः॒ । त्वम् । शि॒वेभिः॑ । अ॒द्य । परीति॑ । पा॒हि॒ । नः॒ । गय᳚म् ॥ हिर॑ण्यजिह्व॒ इति॒ हिर॑ण्य - जि॒ह्वः॒ । सु॒वि॒ताय॑ । नव्य॑से । रक्ष॑ । माकिः॑ । नः॒ । अ॒घशꣳ॑स॒ इत्य॒घ - शꣳ॒॒सः॒ । ई॒श॒त॒ ॥ उ॒प॒या॒मगृ॑हीत॒ इत्यु॑पया॒म - गृ॒ही॒तः॒ । अ॒सि॒ । दे॒वाय॑ । त्वा॒ । स॒वि॒त्रे ॥ \textbf{  25 } \newline
                  \newline
                      (अद॑ब्धेभि॒-स्त्रियो॑विꣳशतिः)  \textbf{(A24)} \newline \newline
                                \textbf{ TS 1.4.25.1} \newline
                  हिर॑ण्यपाणि॒मिति॒ हिर॑ण्य - पा॒णि॒म् । ऊ॒तये᳚ । स॒वि॒तार᳚म् । उपेति॑ । ह्व॒ये॒ ॥ सः । चेत्ता᳚ । दे॒वता᳚ । प॒दम् ॥ उ॒प॒या॒मगृ॑हीत॒ इत्यु॑पया॒म - गृ॒ही॒तः॒ । अ॒सि॒ । दे॒वाय॑ । त्वा॒ । स॒वि॒त्रे ॥ \textbf{  26} \newline
                  \newline
                      (हिर॑ण्यपाणिं॒-चतु॑र्दश)  \textbf{(A25)} \newline \newline
                                \textbf{ TS 1.4.26.1} \newline
                  सु॒शर्मेति॑ सु - शर्मा᳚ । अ॒सि॒ । सु॒प्र॒ति॒ष्ठा॒न इति॑ सु - प्र॒ति॒ष्ठा॒नः । बृ॒हत् । उ॒क्षे । नमः॑ । ए॒षः । ते॒ । योनिः॑ । विश्वे᳚भ्यः । त्वा॒ । दे॒वेभ्यः॑ ॥ \textbf{  27} \newline
                  \newline
                      (सु॒शर्मा॒-द्वाद॑श)  \textbf{(A26)} \newline \newline
                                \textbf{ TS 1.4.27.1} \newline
                  बृह॒स्पति॑सुत॒स्येति॒ बृह॒स्पति॑ - सु॒त॒स्य॒ । ते॒ । इ॒न्दो॒ इति॑ । इ॒न्द्रि॒याव॑त॒ इती᳚न्द्रि॒य - व॒तः॒ । पत्नी॑वन्त॒मिति॒ पत्नी᳚ - व॒न्त॒म् । ग्रह᳚म् । गृ॒ह्णा॒मि॒ । अग्ना(3) इ । पत्नी॒वा(3) इति॒ पत्नी᳚ - वा(3)ः । स॒जूरिति॑ स - जूः । दे॒वेन॑ । त्वष्ट्रा᳚ । सोम᳚म् । पि॒ब॒ । स्वाहा᳚ ॥ \textbf{  28} \newline
                  \newline
                      (बृह॒स्पति॑सुतस्य॒-पञ्च॑दश)  \textbf{(A27)} \newline \newline
                                \textbf{ TS 1.4.28.1} \newline
                  हरिः॑ । अ॒सि॒ । हा॒रि॒यो॒ज॒न इति॑ हारि - यो॒ज॒नः । हर्योः᳚ । स्था॒ता । वज्र॑स्य । भ॒र्ता । पृश्नेः᳚ । प्रे॒ता । तस्य॑ । ते॒ । दे॒व॒ । सो॒म॒ । इ॒ष्टय॑जुष॒ इती॒ष्ट - य॒जु॒षः॒ । स्तु॒तस्तो॑म॒स्येति॑ स्तु॒त - स्तो॒म॒स्य॒ । श॒स्तोक्थ॒स्येति॑ श॒स्त - उ॒क्थ॒स्य॒ । हरि॑वन्त॒मिति॒ हरि॑ - व॒न्त॒म् । ग्रह᳚म् । गृ॒ह्णा॒मि॒ । ह॒रीः । स्थ॒ । हर्योः᳚ । धा॒नाः । स॒हसो॑मा॒ इति॑ स॒ह - सो॒माः॒ । इन्द्रा॑य । स्वाहा᳚ ॥ \textbf{  29 } \newline
                  \newline
                      (हरि॑रसि॒-षड्विꣳ॑शतिः)  \textbf{(A28)} \newline \newline
                                \textbf{ TS 1.4.29.1} \newline
                  अग्ने᳚ । आयूꣳ॑षि । प॒व॒से॒ । एति॑ । सु॒व॒ । ऊर्ज᳚म् । इष᳚म् । च॒ । नः॒ ॥ आ॒रे । बा॒ध॒स्व॒ । दु॒च्छुना᳚म् ॥ उ॒प॒या॒मगृ॑हीत॒ इत्यु॑पया॒म - गृ॒ही॒तः॒ । अ॒सि॒ । अ॒ग्नये᳚ । त्वा॒ । तेज॑स्वते । ए॒षः । ते॒ । योनिः॑ । अ॒ग्नये᳚ । त्वा॒ । तेज॑स्वते ॥ \textbf{  30 } \newline
                  \newline
                      (अग्न॒ आयूꣳ॑षि॒-त्रयो॑विꣳशतिः)  \textbf{(A29)} \newline \newline
                                \textbf{ TS 1.4.30.1} \newline
                  उ॒त्तिष्ठ॒न्नित्यु॑त् - तिष्ठन्न्॑ । ओज॑सा । स॒ह । पी॒त्वा । शिप्रे॒ इति॑ । अ॒वे॒प॒यः॒ ॥ सोम᳚म् । इ॒न्द्र॒ । च॒मू इति॑ । सु॒तम् ॥ उ॒प॒या॒मगृ॑हीत॒ इत्यु॑पया॒म - गृ॒ही॒तः॒ । अ॒सि॒ । इन्द्रा॑य । त्वा॒ । ओज॑स्वते । ए॒षः । ते॒ । योनिः॑ । इन्द्रा॑य । त्वा॒ । ओज॑स्वते ॥ 31(21) \textbf{  31} \newline
                  \newline
                      (उ॒त्तिष्ठ॒न्नेक॑विꣳशतिः)  \textbf{(A30)} \newline \newline
                                \textbf{ TS 1.4.31.1} \newline
                  त॒रणिः॑ । वि॒श्वद॑र्.शत॒ इति॑ वि॒श्व - द॒र्.॒श॒तः॒ । ज्यो॒ति॒ष्कृदिति॑ ज्योतिः - कृत् । अ॒सि॒ । सू॒र्य॒ ॥ विश्व᳚म् । एति॑ । भा॒सि॒ । रो॒च॒नम् ॥ उ॒प॒या॒मगृ॑हीत॒ इत्यु॑पया॒म - गृ॒ही॒तः॒ । अ॒सि॒ । सूर्या॑य । त्वा॒ । भ्राज॑स्वते । ए॒षः । ते॒ । योनिः॑ । सूर्या॑य । त्वा॒ । भ्राज॑स्वते ॥ \textbf{  32 } \newline
                  \newline
                      (त॒रणि॑र् विꣳश॒तिः)  \textbf{(A31)} \newline \newline
                                \textbf{ TS 1.4.32.1} \newline
                  एति॑ । प्या॒य॒स्व॒ । म॒दि॒न्त॒म॒ । सोम॑ । विश्वा॑भिः । ऊ॒तिभि॒रित्यू॒ति - भिः॒ ॥ भव॑ । नः॒ । स॒प्रथ॑स्तम॒ इति॑ स॒प्रथः॑ - त॒मः॒ ॥ \textbf{  33 } \newline
                  \newline
                      (आ प्या॑यस्व॒-नव॑)  \textbf{(A32)} \newline \newline
                                \textbf{ TS 1.4.33.1} \newline
                  ई॒युः । ते । ये । पूर्व॑तरा॒मिति॒ पूर्व॑ - त॒रा॒म् । अप॑श्यन्न् । व्यु॒च्छन्ती॒मिति॑ वि - उ॒च्छन्ती᳚म् । उ॒षस᳚म् । मर्त्या॑सः ॥ अ॒स्माभिः॑ । उ॒ । नु । प्र॒ति॒चक्ष्येति॑ प्रति - चक्ष्या᳚ । अ॒भू॒त् । ओ इति॑ । ते । य॒न्ति॒ । ये । अ॒प॒रीषु॑ । पश्यान्॑ ॥ \textbf{  34 } \newline
                  \newline
                      (ई॒यु-रेका॒न्नविꣳ॑शतिः)  \textbf{(A33)} \newline \newline
                                \textbf{ TS 1.4.34.1} \newline
                  ज्योति॑ष्मतीम् । त्वा॒ । सा॒द॒या॒मि॒ । ज्यो॒ति॒ष्कृत॒मिति॑ ज्योतिः-कृत᳚म् । त्वा॒ । सा॒द॒या॒मि॒ । ज्यो॒ति॒र्विद॒मिति॑ ज्योतिः - विद᳚म् । त्वा॒ । सा॒द॒या॒मि॒ । भास्व॑तीम् । त्वा॒ । सा॒द॒या॒मि॒ । ज्वल॑न्तीम् । त्वा॒ । सा॒द॒या॒मि॒ । म॒ल्म॒ला॒भव॑न्ती॒मिति॑ मल्मला - भव॑न्तीम् । त्वा॒ । सा॒द॒या॒मि॒ । दीप्य॑मानाम् । त्वा॒ । सा॒द॒या॒मि॒ । रोच॑मानाम् । त्वा॒ । सा॒द॒या॒मि॒ । अज॑स्राम् । त्वा॒ । सा॒द॒या॒मि॒ । बृ॒हज्ज्यो॑तिष॒मिति॑ बृ॒हत् - ज्यो॒ति॒ष॒म् । त्वा॒ । सा॒द॒या॒मि॒ । बो॒धय॑न्तीम् । त्वा॒ । सा॒द॒या॒मि॒ । जाग्र॑तीम् । त्वा॒ । सा॒द॒या॒मि॒ ॥ \textbf{  35 } \newline
                  \newline
                      (ज्योति॑ष्मतीꣳ॒॒-षट्त्रिꣳ॑शत्)  \textbf{(A34)} \newline \newline
                                \textbf{ TS 1.4.35.1} \newline
                  प्र॒या॒सायेति॑ प्र - या॒साय॑ । स्वाहा᳚ । आ॒या॒सायेत्या᳚ - या॒साय॑ । स्वाहा᳚ । वि॒या॒सायेति॑ वि - या॒साय॑ । स्वाहा᳚ । स॒म्ॅया॒सायेति॑ सं - या॒साय॑ । स्वाहा᳚ । उ॒द्या॒सायेत्यु॑त् - या॒साय॑ । स्वाहा᳚ । अ॒व॒या॒सायेत्य॑व - या॒साय॑ । स्वाहा᳚ । शु॒चे । स्वाहा᳚ । शोका॑य । स्वाहा᳚ । त॒प्य॒त्वै । स्वाहा᳚ । तप॑ते । स्वाहा᳚ । ब्र॒ह्म॒ह॒त्याया॒ इति॑ ब्रह्म - ह॒त्यायै᳚ । स्वाहा᳚ । सर्व॑स्मै । स्वाहा᳚ ॥ \textbf{  36 } \newline
                  \newline
                      (प्र॒या॒साय॒-चतु॑र्विꣳशतिः )  \textbf{(A35)} \newline \newline
                                \textbf{ TS 1.4.36.1} \newline
                  चि॒त्तम् । स॒तां॒नेनेति॑ सं - ता॒नेन॑ । भ॒वम् । य॒क्ना । रु॒द्रम् । तनि॑म्ना । प॒शु॒पति॒मिति॑ पशु - पति᳚म् । स्थू॒ल॒हृ॒द॒येनेति॑ स्थूल - हृ॒द॒येन॑ । अ॒ग्निम् । हृद॑येन । रु॒द्रम् । लोहि॑तेन । श॒र्वम् । मत॑स्नाभ्याम् । म॒हा॒दे॒वमिति॑ महा - दे॒वम् । अ॒न्तः पा᳚र्श्वे॒नेत्य॒न्तः - पा॒र्श्वे॒न॒ । ओ॒षि॒ष्ठ॒हन॒मित्यो॑षिष्ठ - हन᳚म् । शि॒ङ्गी॒नि॒को॒श्या᳚भ्या॒मिति॑ शिङ्गी - नि॒को॒श्या᳚भ्याम् ॥ \textbf{ } \newline
                  \newline
                      चि॒त्तꣳ स॑न्ता॒नेन॑ भ॒वं ॅय॒क्ना रु॒द्रं तनि॑म्ना पशु॒पतिꣳ॑ स्थूलहृद॒येना॒ग्निꣳ हृद॑येन रु॒द्रं ॅलोहि॑तेन श॒र्वं मत॑स्नाभ्यां महादे॒व-म॒न्तःपा᳚र्श्वेनौषिष्ठ॒हनꣳ॑ शिङ्गीनिको॒श्या᳚भ्यां ॥ 37 (चि॒त्त-म॒ष्टाद॑श)  \textbf{(A36)} \newline \newline
                                \textbf{ TS 1.4.37.1} \newline
                  एति॑ । ति॒ष्ठ॒ । वृ॒त्र॒ह॒न्निति॑ वृत्र - ह॒न्न् । रथ᳚म् । यु॒क्ता । ते॒ । ब्रह्म॑णा । हरी॒ इति॑ ॥ अ॒र्वा॒चीन᳚म् । स्विति॑ । ते॒ । मनः॑ । ग्रावा᳚ । कृ॒णो॒तु॒ । व॒ग्नुना᳚ ॥ उ॒प॒या॒मगृ॑हीत॒ इत्यु॑पया॒म - गृ॒ही॒तः॒ । अ॒सि॒ । इन्द्रा॑य । त्वा॒ । षो॒ड॒शिने᳚ । ए॒षः । ते॒ । योनिः॑ । इन्द्रा॑य । त्वा॒ । षो॒ड॒शिने᳚ ॥ \textbf{  38} \newline
                  \newline
                      (आ ति॑ष्ट॒-षट्विꣳ॑शतिः)  \textbf{(A37)} \newline \newline
                                \textbf{ TS 1.4.38.1} \newline
                  इन्द्र᳚म् । इत् । हरी॒ इति॑ । व॒ह॒तः॒ । अप्र॑तिधृष्टशवस॒मित्यप्र॑तिधृष्ट - श॒व॒स॒म् । ऋषी॑णाम् । च॒ । स्तु॒तीः । उपेति॑ । य॒ज्ञ्म् । च॒ । मानु॑षाणाम् ॥ उ॒प॒या॒मगृ॑हीत॒ इत्यु॑पया॒म - गृ॒ही॒तः॒ । अ॒सि॒ । इन्द्रा॑य । त्वा॒ । षो॒ड॒शिने᳚ । ए॒षः । ते॒ । योनिः॑ । इन्द्रा॑य । त्वा॒ । षो॒ड॒शिने᳚ ॥ \textbf{  39 } \newline
                  \newline
                      (इन्द्र॒मित्-त्रयो॑विꣳशतिः)  \textbf{(A38)} \newline \newline
                                \textbf{ TS 1.4.39.1} \newline
                  असा॑वि । सोमः॑ । इ॒न्द्र॒ । ते॒ । शवि॑ष्ठ । धृ॒ष्णो॒ । एति॑ । ग॒हि॒ ॥ एति॑ । त्वा॒ । पृ॒ण॒क्‌तु॒ । इ॒न्द्रि॒यम् । रजः॑ । सूर्य᳚म् । न । र॒श्मिभि॒रिति॑ र॒श्मि - भिः॒ ॥ उ॒प॒या॒मगृ॑हीत॒ इत्यु॑पया॒म - गृ॒ही॒तः॒ । अ॒सि॒ । इन्द्रा॑य । त्वा॒ । षो॒ड॒शिने᳚ । ए॒षः । ते॒ । योनिः॑ । इन्द्रा॑य । त्वा॒ । षो॒ड॒शिने᳚ ॥ \textbf{  40 } \newline
                  \newline
                      (असा॑वि-स॒प्तविꣳ॑शतिः)  \textbf{(A39)} \newline \newline
                                \textbf{ TS 1.4.40.1} \newline
                  सर्व॑स्य । प्र॒ति॒शीव॒रीति॑ प्रति - शीव॑री । भूमिः॑ । त्वा॒ । उ॒पस्थ॒ इत्यु॒प - स्थे॒ । एति॑ । अ॒धि॒त॒ ॥ स्यो॒ना । अ॒स्मै॒ । सु॒षदेति॑ सु - सदा᳚ । भ॒व॒ । यच्छ॑ । अ॒स्मै॒ । शर्म॑ । स॒प्रथा॒ इति॑ स-प्रथाः᳚ ॥ उ॒प॒या॒मगृ॑हीत॒ इत्यु॑पया॒म - गृ॒ही॒तः॒ । अ॒सि॒ । इन्द्रा॑य । त्वा॒ । षो॒ड॒शिने᳚ । ए॒षः । ते॒ । योनिः॑ । इन्द्रा॑य । त्वा॒ । षो॒ड॒शिने᳚ ॥ \textbf{  41} \newline
                  \newline
                      (सर्व॑स्य॒ षड्विꣳ॑शतिः)  \textbf{(A40)} \newline \newline
                                \textbf{ TS 1.4.41.1} \newline
                  म॒हान् । इन्द्रः॑ । वज्र॑बाहु॒रिति॒ वज्र॑ - बा॒हुः॒ । षो॒ड॒शी । शर्म॑ । य॒च्छ॒तु॒ ॥ स्व॒स्ति । नः॒ । म॒घवेति॑ म॒घ - वा॒ । क॒रो॒तु॒ । हन्तु॑ । पा॒प्मान᳚म् । यः । अ॒स्मान् । द्वेष्टि॑ ॥ उ॒प॒या॒मगृ॑हीत॒ इत्यु॑पया॒म - गृ॒ही॒तः॒ । अ॒सि॒ । इन्द्रा॑य । त्वा॒ । षो॒ड॒शिने᳚ । ए॒षः । ते॒ । योनिः॑ । इन्द्रा॑य । त्वा॒ । षो॒ड॒शिने᳚ ॥ \textbf{  42} \newline
                  \newline
                       (म॒हान्-षड्विꣳ॑शतिः)  \textbf{(A41)} \newline \newline
                                \textbf{ TS 1.4.42.1} \newline
                  स॒जोषा॒ इति॑ स - जोषाः᳚ । इ॒न्द्र॒ । सग॑ण॒ इति॒ स - ग॒णः॒ । म॒रुद्भि॒रिति॑ म॒रुत् - भिः॒ । सोम᳚म् । पि॒ब॒ । वृ॒त्र॒ह॒न्निति॑ वृत्र - ह॒न्न् । शू॒र॒ । वि॒द्वान् ॥ ज॒हि । शत्रून्॑ । अपेति॑ । मृधः॑ । नु॒द॒स्व॒ । अथ॑ । अभ॑यम् । कृ॒णु॒हि॒ । वि॒श्वतः॑ । नः॒ ॥ उ॒प॒या॒मगृ॑हीत॒ इत्यु॑पया॒म - गृ॒ही॒तः॒ । अ॒सि॒ । इन्द्रा॑य । त्वा॒ । षो॒ड॒शिने᳚ । ए॒षः । ते॒ । योनिः॑ । इन्द्रा॑य । त्वा॒ । षो॒ड॒शिने᳚ ॥ \textbf{  43} \newline
                  \newline
                      (स॒जोषाः᳚-त्रिꣳ॒॒शत्)  \textbf{(A42)} \newline \newline
                                \textbf{ TS 1.4.43.1} \newline
                  उदिति॑ । उ॒ । त्यम् । जा॒तवे॑दस॒मिति॑ जा॒त - वे॒द॒स॒म् । दे॒वम् । व॒ह॒न्ति॒ । के॒तवः॑ ॥ दृ॒शे । विश्वा॑य । सूर्य᳚म् ॥ चि॒त्रम् । दे॒वाना᳚म् । उदिति॑ । अ॒गा॒त् । अनी॑कम् । चक्षुः॑ । मि॒त्रस्य॑ । वरु॑णस्य । अ॒ग्नेः ॥ ऐति॑ । अ॒प्राः॒ । द्यावा॑पृथि॒वी इति॒ द्यावा᳚ - पृ॒थि॒वी । अ॒न्तरि॑क्षम् । सूर्यः॑ । आ॒त्मा । जग॑तः । त॒स्थुषः॑ । च॒ ॥ अग्ने᳚ । नय॑ । सु॒पथेति॑ सु - पथा᳚ । रा॒ये । अ॒स्मान् । विश्वा॑नि । दे॒व॒ । व॒युना॑नि । वि॒द्वान् ॥ यु॒यो॒धि । अ॒स्मत् । जु॒हु॒रा॒णम् । एनः॑ । भूयि॑ष्ठाम् । ते॒ । नम॑उक्ति॒मिति॒ नमः॑ - उ॒क्ति॒म् । वि॒धे॒म॒ ॥ दिव᳚म् । ग॒च्छ॒ । सुवः॑ । प॒त॒ । रू॒पेण॑ । \textbf{  44} \newline
                  \newline
                                \textbf{ TS 1.4.43.2} \newline
                  वः॒ । रू॒पम् । अ॒भि । एति॑ । ए॒मि॒ । वय॑सा । वयः॑ ॥ तु॒थः । वः॒ । वि॒श्ववे॑दा॒ इति॑ वि॒श्व - वे॒दाः॒ । वीति॑ । भ॒ज॒तु॒ । वर्.षि॑ष्ठे । अधीति॑ । नाके᳚ ॥ ए॒तत् । ते॒ । अ॒ग्ने॒ । राधः॑ । एति॑ । ए॒ति॒ । सोम॑च्युत॒मिति॒ सोम॑ - च्यु॒त॒म् । तत् । मि॒त्रस्य॑ । प॒था । न॒य॒ । ऋ॒तस्य॑ । प॒था । प्रेति॑ । इ॒त॒ । च॒न्द्रद॑क्षिणा॒ इति॑ च॒न्द्र - द॒क्षि॒णाः॒ । य॒ज्ञ्स्य॑ । प॒था । सु॒वि॒ता । नय॑न्तीः । ब्रा॒ह्म॒णम् । अ॒द्य । रा॒द्ध्या॒स॒म् । ऋषि᳚म् । आ॒र्.॒षे॒यम् । पि॒तृ॒मन्त॒मिति॑ पितृ - मन्त᳚म् । पै॒तृ॒म॒त्यमिति॑ पैतृ - म॒त्यम् । सु॒धातु॑दक्षिण॒मिति॑ सु॒धातु॑ - द॒क्षि॒ण॒म् । वीति॑ । सुवः॑ । पश्य॑ । वीति॑ । अ॒न्तरि॑क्षम् । यत॑स्व । स॒द॒स्यैः᳚ ( ) । अ॒स्मद्दा᳚त्रा॒ इत्य॒स्मत् - दा॒त्राः॒ । दे॒व॒त्रेति॑ देव - त्रा । ग॒च्छ॒त॒ । मधु॑मती॒रिति॒ मधु॑ - म॒तीः॒ । प्र॒दा॒तार॒मिति॑ प्र - दा॒तार᳚म् । एति॑ । वि॒श॒त॒ । अन॑वहा॒येत्यन॑व - हा॒य॒ । अ॒स्मान् । दे॒व॒याने॒नेति॑ देव - याने॑न । प॒था । इ॒त॒ । सु॒कृता॒मिति॑ सु - कृता᳚म् । लो॒के । सी॒द॒त॒ । तत् । नः॒ । सꣳ॒॒स्कृ॒तम् ॥ \textbf{  45 } \newline
                  \newline
                       (रू॒पेण॑-सद॒स्यै॑-र॒ष्टाद॑श च)  \textbf{(A43)} \newline \newline
                                \textbf{ TS 1.4.44.1} \newline
                  ध॒ता । रा॒तिः । स॒वि॒ता । इ॒दम् । जु॒ष॒न्ता॒म् । प्र॒जाप॑ति॒रिति॑ प्र॒जा - प॒तिः॒ । नि॒धि॒पति॒रिति॑ निधि - पतिः॑ । नः॒ । अ॒ग्निः ॥ त्वष्टा᳚ । विष्णुः॑ । प्र॒जयेति॑ प्र - जया᳚ । सꣳ॒॒र॒रा॒ण इति॑ सं-र॒रा॒णः । यज॑मानाय । द्रवि॑णम् । द॒धा॒तु॒ ॥ समिति॑ । इ॒न्द्र॒ । नः॒ । मन॑सा । ने॒षि॒ । गोभिः॑ । समिति॑ । सू॒रिभि॒रिति॑ सू॒रि - भिः॒ । म॒घ॒व॒न्निति॑ मघ - व॒न्न् । समिति॑ । स्व॒स्त्या ॥ समिति॑ । ब्रह्म॑णा । दे॒वकृ॑त॒मिति॑ दे॒व - कृ॒त॒म् । यत् । अस्ति॑ । समिति॑ । दे॒वाना᳚म् । सु॒म॒त्येति॑ सु - म॒त्या । य॒ज्ञिया॑नाम् ॥ समिति॑ । वर्च॑सा । पय॑सा । समिति॑ । त॒नूभिः॑ । अग॑न्महि । मन॑सा । समिति॑ । शि॒वेन॑ ॥ त्वष्टा᳚ । नः॒ । अत्र॑ । वरि॑वः । कृ॒णो॒तु॒ । \textbf{  46} \newline
                  \newline
                                \textbf{ TS 1.4.44.2} \newline
                  अन्विति॑ । मा॒र्ष्टु॒ । त॒नुवः॑ । यत् । विलि॑ष्ट॒मिति॒ वि - लि॒ष्ट॒म् ॥ यत् । अ॒द्य । त्वा॒ । प्र॒य॒तीति॑ प्र - य॒ति । य॒ज्ञे । अ॒स्मिन्न् । अग्ने᳚ । होता॑रम् । अवृ॑णीमहि । इ॒ह ॥ ऋध॑क् । अ॒या॒ट् । ऋध॑क् । उ॒त । अश॑मिष्ठाः । प्र॒जा॒नन्निति॑ प्र - जा॒नन्न् । य॒ज्ञ्म् । उपेति॑ । या॒हि॒ । वि॒द्वान् ॥ स्व॒गेति॑ स्व - गा । वः॒ । दे॒वाः॒ । सद॑नम् । अ॒क॒र्म॒ । ये । आ॒ज॒ग्मेत्या᳚ - ज॒ग्म । सव॑ना । इ॒दम् । जु॒षा॒णाः ॥ ज॒क्षि॒वाꣳसः॑ । प॒पि॒वाꣳसः॑ । च॒ । विश्वे᳚ । अ॒स्मे इति॑ । ध॒त्त॒ । व॒स॒वः॒ । वसू॑नि ॥ यान् । एति॑ । अव॑हः । उ॒श॒तः । दे॒व॒ । दे॒वान् । तान् । \textbf{  47} \newline
                  \newline
                                \textbf{ TS 1.4.44.3} \newline
                  प्रेति॑ । ई॒र॒य॒ । स्वे । अ॒ग्ने॒ । स॒धस्थ॒ इति॑ स॒ध - स्थे॒ ॥ वह॑मानाः । भर॑माणाः । ह॒वीꣳषि॑ । वसु᳚म् । घ॒र्मम् । दिव᳚म् । एति॑ । ति॒ष्ठ॒त॒ । अनु॑ ॥ यज्ञ्॑ । य॒ज्ञ्म् । ग॒च्छ॒ । य॒ज्ञ्प॑ति॒मिति॑ य॒ज्ञ् - प॒ति॒म् । ग॒च्छ॒ । स्वाम् । योनि᳚म् । ग॒च्छ॒ । स्वाहा᳚ । ए॒षः । ते॒ । य॒ज्ञ्ः । य॒ज्ञ्॒प॒त॒ इति॑ यज्ञ् - प॒ते॒ । स॒हसू᳚क्तवाक॒ इति॑ स॒हस᳚क्त - वा॒कः॒ । सु॒वीर॒ इति॑ सु - वीरः॑ । स्वाहा᳚ । देवाः᳚ । गा॒तु॒वि॒द॒ इति॑ गातु - वि॒दः॒ । गा॒तुम् । वि॒त्त्वा । गा॒तुम् । इ॒त॒ । मन॑सः । प॒ते॒ । इ॒मम् । नः॒ । दे॒व॒ । दे॒वेषु॑ । य॒ज्ञ्म् । स्वाहा᳚ । वा॒चि । स्वाहा᳚ । वाते᳚ । धाः॒ ॥(कृ॒णो॒तु॒ - तान॒ - ष्टाच॑त्वारिꣳशच्च )(आ44 ) \textbf{  48} \newline
                  \newline
                      (कृ॒णो॒तु॒-तान॒-ष्टाच॑त्वारिꣳशच्च )  \textbf{(A44)} \newline \newline
                                \textbf{ TS 1.4.45.1} \newline
                  उ॒रुम् । हि । राजा᳚ । वरु॑णः । च॒कार॑ । सूर्या॑य । पन्था᳚म् । अन्वे॑त॒वा इत्यनु॑ - ए॒त॒वै । उ॒ ॥ अ॒पदे᳚ । पादा᳚ । प्रति॑धातव॒ इति॒ प्रति॑-धा॒त॒वे॒ । अ॒कः॒ । उ॒त । अ॒प॒व॒क्तेत्य॑प - व॒क्ता । हृ॒द॒या॒विध॒ इति॑ हृदय - विधः॑ । चि॒त् ॥ श॒तम् । ते॒ । रा॒ज॒न्न् । भि॒षजः॑ । स॒हस्र᳚म् । उ॒र्वी । ग॒भीं॒रा । सु॒म॒तिरिति॑ सु - म॒तिः । ते॒ । अ॒स्तु॒ ॥ बाध॑स्व । द्वेषः॑ । निर्.ऋ॑ति॒मिति॒ निः - ऋ॒ति॒म् । प॒रा॒चैः । कृ॒तम् । चि॒त् । एनः॑ । प्रेति॑ । मु॒मु॒ग्धि॒ । अ॒स्मत् ॥ अ॒भिष्ठि॑त॒ इत्य॒भि - स्थि॒तः॒ । वरु॑णस्य । पाशः॑ । अ॒ग्नेः । अनी॑कम् । अ॒पः । एति॑ । वि॒वे॒श॒ ॥ अपा᳚म् । न॒पा॒त् । प्र॒ति॒रक्ष॒न्निति॑ प्रति - रक्षन्न्॑ । अ॒सु॒र्य᳚म् । दमे॑दम॒ इति॒ दमे᳚ - द॒मे॒ । \textbf{  49} \newline
                  \newline
                                \textbf{ TS 1.4.45.2} \newline
                  स॒मिध॒मिति॑ सं - इध᳚म् । य॒क्षि॒ । अ॒ग्ने॒ ॥ प्रतीति॑ । ते॒ । जि॒ह्वा । घृ॒तम् । उदिति॑ । च॒र॒ण्ये॒त् । स॒मु॒द्रे । ते॒ । हृद॑यम् । अ॒फ्स्वित्य॑प् - सु । अ॒न्तः ॥ समिति॑ । त्वा॒ । वि॒श॒न्तु॒ । ओष॑धीः । उ॒त । आपः॑ । य॒ज्ञ्स्य॑ । त्वा॒ । य॒ज्ञ्॒प॒त॒ इति॑ यज्ञ् - प॒ते॒ । ह॒विर्भि॒रिति॑ ह॒विः - भिः॒ ॥ सू॒क्त॒वा॒क इति॑ सूक्त - वा॒के । न॒मो॒वा॒क इति॑ नमः - वा॒के । वि॒धे॒म॒ । अव॑भृ॒थेत्यव॑ - भृ॒थ॒ । नि॒च॒ङ्कु॒णेति॑ नि - च॒ङ्कु॒ण॒ । नि॒चे॒रुरिति॑ नि - चे॒रुः । अ॒सि॒ । नि॒च॒ङ्कु॒णेति॑ नि - च॒ङ्कु॒ण॒ । अवेति॑ । दे॒वैः । दे॒वकृ॑त॒मिति॑ दे॒व - कृ॒त॒म् । एनः॑ । अ॒या॒ट् । अवेति॑ । मर्त्यैः᳚ । मर्त्य॑कृत॒मिति॒ मर्त्य॑ - कृ॒त॒म् । उ॒रोः । एति॑ । नः॒ । दे॒व॒ । रि॒षः । पा॒हि॒ । सु॒मि॒त्रा इति॑ सु - मि॒त्राः । नः॒ । आपः॑ । ओष॑धयः । \textbf{  50} \newline
                  \newline
                                \textbf{ TS 1.4.45.3} \newline
                  स॒न्तु॒ । दु॒र्मि॒त्रा इति॑ दुः-मि॒त्राः । तस्मै᳚ । भू॒या॒सुः॒ । यः । अ॒स्मान् । द्वेष्टि॑ । यम् । च॒ । व॒यम् । द्वि॒ष्मः । देवीः᳚ । आ॒पः॒ । ए॒षः । वः॒ । गर्भः॑ । तम् । वः॒ । सुप्री॑त॒मिति॒ सु - प्री॒त॒म् । सुभृ॑त॒मिति॒ सु - भृ॒त॒म् । अ॒क॒र्म॒ । दे॒वेषु॑ । नः॒ । सु॒कृत॒ इति॑ सु-कृतः॑ । ब्रू॒ता॒त् । प्रति॑युत॒ इति॒ प्रति॑ - यु॒तः॒ । वरु॑णस्य । पाशः॑ । प्रत्य॑स्त॒ इति॒ प्रति॑ - अ॒स्तः॒ । वरु॑णस्य । पाशः॑ । एधः॑ । अ॒सि॒ । ए॒धि॒षी॒महि॑ । स॒मिदिति॑ सम् - इत् । अ॒सि॒ । तेजः॑ । अ॒सि॒ । तेजः॑ । मयि॑ । धे॒ह॒ । अ॒पः । अन्विति॑ । अ॒चा॒रि॒ष॒म् । रसे॑न । समिति॑ । अ॒सृ॒क्ष्म॒हि॒ । पय॑स्वान् । अ॒ग्ने॒ । एति॑ ( ) । अ॒ग॒म॒म् । तम् । मा॒ । समिति॑ । सृ॒ज॒ । वर्च॑सा ॥ \textbf{  51} \newline
                  \newline
                      (दमे॑दम॒-ओष॑धय॒- आ-षट्च॑)  \textbf{(A45)} \newline \newline
                                \textbf{ TS 1.4.46.1} \newline
                  यः । त्वा॒ । हृ॒दा । की॒रिणा᳚ । मन्य॑मानः । अम॑र्त्यम् । मर्त्यः॑ । जोह॑वीमि ॥ जात॑वेद॒ इति॒ जात॑ - वे॒दः॒ । यशः॑ । अ॒स्मासु॑ । धे॒हि॒ । प्र॒जाभि॒रिति॑ प्र - जाभिः॑ । अ॒ग्ने॒ । अ॒मृ॒त॒त्वमित्य॑मृत - त्वम् । अ॒श्या॒म् ॥ यस्मै᳚ । त्वम् । सु॒कृत॒ इति॑ सु - कृते᳚ । जा॒त॒वे॒द॒ इति॑ जात-वे॒दः॒ । उ । लो॒कम् । अ॒ग्ने॒ । कृ॒णवः॑ । स्यो॒नम् ॥ अ॒श्विन᳚म् । सः । पु॒त्रिण᳚म् । वी॒रव॑न्त॒मिति॑ वी॒र - व॒न्त॒म् । गोम॑न्त॒मिति॒ गो - म॒न्त॒म् । र॒यिम् । न॒श॒ते॒ । स्व॒स्ति ॥ त्वे इति॑ । स्विति॑ । पु॒त्र॒ । श॒व॒सः॒ । अवृ॑त्रन्न् । काम॑कातय॒ इति॒ काम॑ - का॒त॒यः॒ ॥ न । त्वाम् । इ॒न्द्र॒ । अतीति॑ । रि॒च्य॒ते॒ ॥ उ॒क्थ‌उ॑क्थ॒ इत्यु॒क्थे - उ॒क्थे॒ । सोमः॑ । इन्द्र᳚म् । म॒मा॒द॒ । नी॒थेनी॑थ॒ इति॑ नी॒थे - नी॒थे॒ । म॒घवा॑न॒मिति॑ म॒घ - वा॒न॒म् । \textbf{  52} \newline
                  \newline
                                \textbf{ TS 1.4.46.2} \newline
                  सु॒तासः॑ ॥ यत् । ई॒म् । स॒बाध॒ इति॑ स - बाधः॑ । पि॒तर᳚म् । न । पु॒त्राः । स॒मा॒नद॑क्षा॒ इति॑ समा॒न-द॒क्षाः॒ । अव॑से । हव॑न्ते ॥ अग्ने᳚ । रसे॑न । तेज॑सा । जात॑वेद॒ इति॒ जात॑ - वे॒दः॒ । वीति॑ । रो॒च॒से॒ ॥ र॒क्षो॒हेति॑ रक्षः - हा । अ॒मी॒व॒चात॑न॒ इत्य॑मीव - चात॑नः ॥ अ॒पः । अन्विति॑ । अ॒चा॒रि॒ष॒म् । रसे॑न । समिति॑ । अ॒सृ॒क्ष्म॒हि॒ ॥ पय॑स्वान् । अ॒ग्ने॒ । एति॑ । अ॒ग॒म॒म् । तम् । मा॒ । समिति॑ । सृ॒ज॒ । वर्च॑सा ॥ वसुः॑ । वसु॑पति॒रिति॒ वसु॑ - प॒तिः॒ । हिक᳚म् । असि॑ । अ॒ग्ने॒ । वि॒भाव॑सु॒रिति॑ वि॒भा - व॒सुः॒ ॥ स्याम॑ । ते॒ । सु॒म॒ताविति॑ सु-म॒तौ । अपि॑ ॥ त्वाम् । अ॒ग्ने॒ । वसु॑पति॒मिति॒ वसु॑ - प॒ति॒म् । वसू॑नाम् । अ॒भि । प्रेति॑ । म॒न्दे॒ । \textbf{  53} \newline
                  \newline
                                \textbf{ TS 1.4.46.3} \newline
                  अ॒द्ध्व॒रेषु॑ । रा॒ज॒न्न् ॥ त्वया᳚ । वाज᳚म् । वा॒ज॒यन्त॒ इति॑ वाज - यन्तः॑ । ज॒ये॒म॒ । अ॒भीति॑ । स्या॒म॒ । पृ॒थ्सु॒तीः । मर्त्या॑नाम् ॥ त्वाम् । अ॒ग्ने॒ । वा॒ज॒सात॑म॒मिति॑ वाज - सात॑मम् । विप्राः᳚ । व॒र्ध॒न्ति॒ । सुष्टु॑त॒मिति॒ सु - स्तु॒त॒म् ॥ सः । नः॒ । रा॒स्व॒ । सु॒वीर्य॒मिति॑ सु - वीर्य᳚म् ॥ अ॒यम् । नः॒ । अ॒ग्निः । वरि॑वः । कृ॒णो॒तू॒ । अ॒यम् । मृधः॑ । पु॒रः । ए॒तु॒ । प्र॒भि॒न्दन्निति॑ प्र - भि॒न्दन्न् ॥ अ॒यम् । शत्रून्॑ । ज॒य॒तु॒ । जर्.हृ॑षाणः । अ॒यम् । वाज᳚म् । ज॒य॒तु॒ । वाज॑सात॒विति॒ वाज॑-सा॒तौ॒ ॥ अ॒ग्निना᳚ । अ॒ग्निः । समिति॑ । इ॒द्ध्य॒ते॒ । क॒विः । गृ॒हप॑ति॒रिति॑ गृ॒ह - प॒तिः॒ । युवा᳚ ॥ ह॒व्य॒वाडिति॑ हव्य - वाट् । जु॒ह्वा᳚स्य॒ इति॑ जु॒हु - आ॒स्यः॒ ॥ त्वम् । हि । अ॒ग्ने॒ ( ) । अ॒ग्निना᳚ । विप्रः॑ । विप्रे॑ण । सन्न् । स॒ता ॥ सखा᳚ । सख्या᳚ । स॒मि॒द्ध्यस॒ इति॑ सम् - इ॒द्ध्यसे᳚ ॥ उदिति॑ । अ॒ग्ने॒ । शुच॑यः । तव॑ । वीति॑ । ज्योति॑षा ॥ \textbf{  54} \newline
                  \newline
                      (म॒घवा॑नं-मन्दे॒-ह्य॑ग्ने॒-चतु॑र्दश च)  \textbf{(A46)} \newline \newline
\textbf{praSna korvai with starting padams of 1 to 46 anuvAkams :-} \newline
(आ द॑दे-वा॒चस्पत॑य-उपया॒मगृ॑हीतो॒ऽस्या वा॑यो -अ॒यं ॅवां॒ - ॅया वां᳚-प्रात॒र्युजा॑-व॒यं-तं -ॅये दे॑वा-स्त्रिꣳ॒॒श-दु॑पया॒मगृ॑हीतोऽसि-मू॒र्द्धानं॒-मधु॒श्चे-न्द्रा᳚ग्न॒॑007आ; ओमा॑सो-म॒रुत्व॑न्त॒-मिन्द्र॑ मरुत्वो-म॒रुत्वा᳚न्- म॒हान्-म॒हान्नु॒वत्-क॒दा-वा॒म-मद॑ब्धेभि॒र्॒. हिर॑ण्यपाणिꣳ-सु॒शर्मा॒-बृह॒स्पति॑ सुतस्य॒ - हरि॑र॒स्य-ग्न॑-उ॒त्तिष्ठ॑न्-त॒रणि॒- राप्या॑यस्वे॒-युष्टे ये-ज्योति॑ष्मतीं-प्रया॒साय॑-चि॒त्त-माति॒ष्ठे-न्द्र॒-मसा॑वि॒-सर्व॑स्य-म॒हान्थ्-स॒जोषा॒-उदु॒त्यं-धा॒तो-रुꣳ हि-य-स्त्वा॒ षट्च॑त्वारिꣳशत् ।) \newline

\textbf{korvai with starting padams of1, 11, 21 series of pa~jcAtis :-} \newline
(आ द॑दे॒-ये दे॑वा-म॒हा-नु॒त्तिष्ठ॒न्थ्-सर्व॑स्य-सन्तु दुर्मि॒त्रा-श्चतु॑ष्पञ्चा॒शत् ।) \newline

\textbf{first and last padam of fourth praSnam :-} \newline
(आ द॑दे॒-वि ज्योति॑षा ) \newline 

    \textbf{special korvai for this praSnam:-} \newline
    (वा॒च प्रा॒णाय॑ त्वा । उ॒प॒या॒मगृ॑हीतोऽस्यपा॒नाय॑ त्वा । आ वा॑यो वा॒यवे॑ स॒जोषा᳚भ्यां त्वा । अ॒यमृ॑ता॒युभ्यां᳚ त्वा । या वा॑म॒श्विभ्यां॒ माद्ध्वी᳚भ्यां त्वा । प्रा॒त॒र्युजा॑व॒श्विभ्या॑म॒श्विभ्यां᳚ त्वा । अ॒यꣳ शण्डा॑य वी॒रतां᳚ पाहि । तं मर्का॑य प्र॒जाः पा॑हि । ये दे॑वा स्त्रिꣳ॒॒शदा᳚ग्रय॒णो॑ऽसि॒ विश्वे᳚भ्यस्त्वा दे॒वेभ्यः॑ । उ॒प॒या॒मगृ॑हीतो॒-ऽसीन्द्रा॑य त्वोक्था॒युवे᳚ । मू॒र्द्धान॑म॒ग्नये᳚ त्वा वैश्वान॒राय॑ । मधु॑श्च सꣳ॒॒ सर्पो॑ऽसि । इन्द्रा᳚ग्नी इन्द्रा॒ग्निभ्यां᳚ त्वा । ओमा॑सो॒ विश्वे᳚भ्यस्त्वा दे॒वेभ्यः॑ । म॒रुत्वं॑ त॒न्त्रीणीन्द्रा॑य त्वा म॒रुत्व॑ते । म॒हान्द्वे म॑हे॒न्द्राय॑ त्वा । क॒दा च॒नाऽऽदि॒त्येभ्य॑स्त्वा । क॒दा च॒न स्त॒रीर् विव॑स्व आदित्य । इन्द्रꣳ॒॒ शुचि॑र॒पः । वा॒मन्त्रीणी॑ दे॒वाय॑ त्वा सवि॒त्रे । सु॒शर्मा॑ऽसि॒ विश्वे᳚भ्यस्त्वा दे॒वेभ्यः॑ । बृह॒स्पति॑-सुतस्य॒ त्वष्ट्रा॒ सोमं॑ पिब॒ स्वाहा᳚ । हरि॑रसि स॒हसो॑मा॒ इन्द्रा॑य॒ स्वाहा᳚ । अग्न॒ आयूꣳ॑ष्य॒ग्नये᳚ त्वा॒ तेज॑स्वते । उ॒त्तिष्ठ॒न्निन्द्रा॑य॒ त्वौज॑स्वते । त॒रणिः॒ सूर्या॑य त्वा॒ भ्राज॑स्वते । आ ति॑ष्ठाद्या॒ष्षटिन्द्रा॑य त्वा षोड॒शिने᳚ । उदु॒ त्यं चि॒त्रं । अग्ने॒ नय॒ दिवं॑ गच्छ । उ॒रूमायु॑ष्टे॒ यद्दे॑वा मुमुग्धि । अग्ना॑विष्णू सुक्रतू मुमुक्तं । परा॒ वै प॒ङ्क्त्यः॑ । दे॒वा वै ये दे॒वाः प॒ङ्क्त्यो᳚ । परा॒ वै स वाचं᳚ । भूमि॒र्व्य॑तृष्यन्न् । प्र॒जाप॑ति॒र् व्य॑क्षुद्ध्यन्न् । भूमि॑रादि॒या वै । अ॒ग्नि॒हो॒त्रमा॑दि॒त्यो वै । भूमि॒र् लेकः॒ सले॑कः सु॒लेकः॑ । विष्णो॒रुदु॑त्त॒मं । अन्न॑पते॒ पुन॑स्वाऽऽदि॒त्याः । उ॒रुꣳ सꣳ सृ॑ज॒ वर्च॑सा । यस्त्वा॒ सुष्टु॑तं । त्वम॑ग्ने यु॒क्ष्वा हि सु॑ष्टि॒तिं । त्वम॑ग्ने॒ विच॑र्षणे । यत्वा॒ वि रो॑चसे ।) \newline

॥ हरिः॑ ॐ ॥॥ कृष्ण यजुर्वेदीय तैत्तिरीय संहितायां प्रथमकाण्डे चतुर्त्थः प्रश्नः समाप्तः ॥ \newline
\pagebreak
1.4.1   Annexure for 1.4\\1.4.46.3 - उद॑ग्ने॒ शुच॑य॒स्तव॒ >1 \\\\उद॑ग्ने॒ शुच॑य॒स्तव॑ शु॒क्रा भ्राज॑न्त ईरते । \\तव॒ ज्योतीꣳ॑ष्य॒र्चयः॑ । (ट्श्1-3-14-8)\\1.4.46.3 - वि ज्योति॑षा >2 \\विज्योति॑षा बृह॒ता भा᳚त्य॒ग्निरा॒विर् विश्वा॑नि कृणुते महि॒त्वा । \\प्रादे॑वीर्मा॒याः स॑हते-दु॒रेवाः॒ शिशी॑ते॒ शृङ्गे॒ रक्ष॑से वि॒निक्षे᳚ । (ट्श्1-2-14-7)\\=========================\\
\pagebreak
        


\end{document}
