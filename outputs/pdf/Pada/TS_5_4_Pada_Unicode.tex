\documentclass[17pt]{extarticle}
\usepackage{babel}
\usepackage{fontspec}
\usepackage{polyglossia}
\usepackage{extsizes}



\setmainlanguage{sanskrit}
\setotherlanguages{english} %% or other languages
\setlength{\parindent}{0pt}
\pagestyle{myheadings}
\newfontfamily\devanagarifont[Script=Devanagari]{AdishilaVedic}


\newcommand{\VAR}[1]{}
\newcommand{\BLOCK}[1]{}




\begin{document}
\begin{titlepage}
    \begin{center}
 
\begin{sanskrit}
    { \Large
    ॐ नमः परमात्मने, श्री महागणपतये नमः, श्री गुरुभ्यो नमः ॥ ह॒रिः॒ ॐ 
    }
    \\
    \vspace{2.5cm}
    \mbox{ \Huge
    5.4      पञ्चमकाण्डे चतुर्थः प्रश्नः इष्टकात्रयाभिधानं   }
\end{sanskrit}
\end{center}

\end{titlepage}
\tableofcontents

ॐ नमः परमात्मने, श्री महागणपतये नमः, श्री गुरुभ्यो नमः
ह॒रिः॒ ॐ \newline
5.4      पञ्चमकाण्डे चतुर्थः प्रश्नः इष्टकात्रयाभिधानं \newline

\addcontentsline{toc}{section}{ 5.4      पञ्चमकाण्डे चतुर्थः प्रश्नः इष्टकात्रयाभिधानं}
\markright{ 5.4      पञ्चमकाण्डे चतुर्थः प्रश्नः इष्टकात्रयाभिधानं \hfill https://www.vedavms.in \hfill}
\section*{ 5.4      पञ्चमकाण्डे चतुर्थः प्रश्नः इष्टकात्रयाभिधानं }
                                \textbf{ TS 5.4.1.1} \newline
                  दे॒वा॒सु॒रा इति॑ देव - अ॒सु॒राः । संॅय॑त्ता॒ इति॒ सं-य॒त्ताः॒ । आ॒स॒न्न् । ते । न । वीति॑ । अ॒ज॒य॒न्त॒ । सः । ए॒ताः । इन्द्रः॑ । त॒नूः । अ॒प॒श्य॒त् । ताः । उपेति॑ । अ॒ध॒त्त॒ । ताभिः॑ । वै । सः । त॒नुव᳚म् । इ॒न्द्रि॒यम् । वी॒र्य᳚म् । आ॒त्मन्न् । अ॒ध॒त्त॒ । ततः॑ । दे॒वाः । अभ॑वन्न् । परेति॑ । असु॑राः । यत् । इ॒न्द्र॒त॒नूरिती᳚न्द्र - त॒नूः । उ॒प॒दधा॒तीत्यु॑प - दधा॑ति । त॒नुव᳚म् । ए॒व । ताभिः॑ । इ॒न्द्रि॒यम् । वी॒र्य᳚म् । यज॑मानः । आ॒त्मन्न् । ध॒त्ते॒ । अथो॒ इति॑ । सैन्द्र॒मिति॒ स - इ॒न्द्र॒म् । ए॒व । अ॒ग्निम् । सत॑नु॒मिति॒ स - त॒नु॒म् । चि॒नु॒ते॒ । भव॑ति । आ॒त्मना᳚ । परेति॑ । अ॒स्य॒ । भ्रातृ॑व्यः । \textbf{  1} \newline
                  \newline
                                \textbf{ TS 5.4.1.2} \newline
                  भ॒व॒ति॒ । य॒ज्ञ्ः । दे॒वेभ्यः॑ । अपेति॑ । अ॒क्रा॒म॒त् । तम् । अ॒व॒रुध॒मित्य॑व - रुध᳚म् । न । अ॒श॒क्नु॒व॒न्न् । ते । ए॒ताः । य॒ज्ञ्॒त॒नूरिति॑ यज्ञ् - त॒नूः । अ॒प॒श्य॒न्न् । ताः । उपेति॑ । अ॒द॒ध॒त॒ । ताभिः॑ । वै । ते । य॒ज्ञ्म् । अवेति॑ । अ॒रु॒न्ध॒त॒ । यत् । य॒ज्ञ्॒त॒नूरिति॑ यज्ञ् - त॒नूः । उ॒प॒दधा॒तीत्यु॑प - दधा॑ति । य॒ज्ञ्म् । ए॒व । ताभिः॑ । यज॑मानः । अवेति॑ । रु॒न्धे॒ । त्रय॑स्त्रिꣳशत॒मिति॒ त्रयः॑ - त्रिꣳ॒॒श॒त॒म् । उपेति॑ । द॒धा॒ति॒ । त्रय॑स्त्रिꣳश॒दिति॒ त्रयः॑ - त्रिꣳ॒॒श॒त् । वै । दे॒वताः᳚ । दे॒वताः᳚ । ए॒व । अवेति॑ । रु॒न्धे॒ । अथो॒ इति॑ । सात्मा॑न॒मिति॒ स - आ॒त्मा॒न॒म् । ए॒व । अ॒ग्निम् । सत॑नु॒मिति॒ स-त॒नु॒म् । चि॒नु॒ते॒ । सात्मेति॒ स - आ॒त्मा॒ । अ॒मुष्मिन्॑ । लो॒के । \textbf{  2} \newline
                  \newline
                                \textbf{ TS 5.4.1.3} \newline
                  भ॒व॒ति॒ । यः । ए॒वम् । वेद॑ । ज्योति॑ष्मतीः । उपेति॑ । द॒धा॒ति॒ । ज्योतिः॑ । ए॒व । अ॒स्मि॒न्न् । द॒धा॒ति॒ । ए॒ताभिः॑ । वै । अ॒ग्निः । चि॒तः । ज्व॒ल॒ति॒ । ताभिः॑ । ए॒व । ए॒न॒म् । समिति॑ । इ॒न्धे॒ । उ॒भयोः᳚ । अ॒स्मै॒ । लो॒कयोः᳚ । ज्योतिः॑ । भ॒व॒ति॒ । न॒क्ष॒त्रे॒ष्ट॒का इति॑ नक्षत्र - इ॒ष्ट॒काः । उपेति॑ । द॒धा॒ति॒ । ए॒तानि॑ । वै । दि॒वः । ज्योतीꣳ॑षि । तानि॑ । ए॒व । अवेति॑ । रु॒न्धे॒ । सु॒कृता॒मिति॑ सु - कृता᳚म् । वै । ए॒तानि॑ । ज्योतीꣳ॑षि । यत् । नक्ष॑त्राणि । तानि॑ । ए॒व । आ॒प्नो॒ति॒ । अथो॒ इति॑ । अ॒नू॒का॒शमित्य॑नु - का॒शम् । ए॒व । ए॒तानि॑ । \textbf{  3} \newline
                  \newline
                                \textbf{ TS 5.4.1.4} \newline
                  ज्योतीꣳ॑षि । कु॒रु॒ते॒ । सु॒व॒र्गस्येति॑ सुवः - गस्य॑ । लो॒कस्य॑ । अनु॑ख्यात्या॒ इत्यनु॑ - ख्या॒त्यै॒ । यत् । सꣳस्पृ॑ष्टा॒ इति॒ सं - स्पृ॒ष्टाः॒ । उ॒प॒द॒द्ध्यादित्यु॑प-द॒द्ध्यात् । वृष्ट्यै᳚ । लो॒कम् । अपीति॑ । द॒द्ध्या॒त् । अव॑र्.षुकः । प॒र्जन्यः॑ । स्या॒त् । असꣳ॑स्पृष्टा॒ इत्यसं᳚ - स्पृ॒ष्टाः॒ । उपेति॑ । द॒धा॒ति॒ । वृष्ट्यै᳚ । ए॒व । लो॒कम् । क॒रो॒ति॒ । वर्.षु॑कः । प॒र्जन्यः॑ । भ॒व॒ति॒ । पु॒रस्ता᳚त् । अ॒न्याः । प्र॒तीचीः᳚ । उपेति॑ । द॒धा॒ति॒ । प॒श्चात् । अ॒न्याः । प्राचीः᳚ । तस्मा᳚त् । प्रा॒चीना॑नि । च॒ । प्र॒ती॒चीना॑नि । च॒ । नक्ष॑त्राणि । एति॑ । व॒र्त॒न्ते॒ ॥ \textbf{  4} \newline
                  \newline
                      (भ्रातृ॑व्यो - लो॒क - ए॒वैतान्ये - क॑चत्वारिꣳशच्च)  \textbf{(A1)} \newline \newline
                                \textbf{ TS 5.4.2.1} \newline
                  ऋ॒त॒व्याः᳚ । उपेति॑ । द॒धा॒ति॒ । ऋ॒तू॒नाम् । क्लृप्त्यै᳚ । द्व॒द्वंमिति॑ द्वं-द्वम् । उपेति॑ । द॒धा॒ति॒ । तस्मा᳚त् । द्व॒द्वंमिति॑ द्वं - द्वम् । ऋ॒तवः॑ । अधृ॑ता । इ॒व॒ । वै । ए॒षा । यत् । म॒द्ध्य॒मा । चितिः॑ । अ॒न्तरि॑क्षम् । इ॒व॒ । वै । ए॒षा । द्व॒द्वंमिति॑ द्वं - द्वम् । अ॒न्यासु॑ । चिती॑षु । उपेति॑ । द॒धा॒ति॒ । चत॑स्रः । मद्ध्ये᳚ । धृत्यै᳚ । अ॒न्त॒श्श्लेष॑ण॒मित्य॑न्तः - श्लेष॑णम् । वै । ए॒ताः । चिती॑नाम् । यत् । ऋ॒त॒व्याः᳚ । यत् । ऋ॒त॒व्याः᳚ । उ॒प॒दधा॒तीत्यु॑प - दधा॑ति । चिती॑नाम् । विधृ॑त्या॒ इति॒ वि - धृ॒त्यै॒ । अव॑काम् । अनु॑ । उपेति॑ । द॒धा॒ति॒ । ए॒षा । वै । अ॒ग्नेः । योनिः॑ । सयो॑नि॒मिति॒ स - यो॒नि॒म् । \textbf{  5} \newline
                  \newline
                                \textbf{ TS 5.4.2.2} \newline
                  ए॒व । अ॒ग्निम् । चि॒नु॒ते॒ । उ॒वाच॑ । ह॒ । वि॒श्वामि॑त्र॒ इति॑ वि॒श्व-मि॒त्रः॒ । अद॑त् । इत् । सः । ब्रह्म॑णा । अन्न᳚म् । यस्य॑ । ए॒ताः । उ॒प॒धी॒यान्ता॒ इत्यु॑प - धी॒यान्तै᳚ । यः । उ॒ । च॒ । ए॒नाः॒ । ए॒वम् । वेद॑त् । इति॑ । सं॒ॅव॒थ्स॒र इति॑ सं - व॒थ्स॒रः । वै । ए॒तम् । प्र॒ति॒ष्ठाया॒ इति॑ प्रति - स्थायै᳚ । नु॒द॒ते॒ । यः । अ॒ग्निम् । चि॒त्वा । न । प्र॒ति॒तिष्ठ॒तीति॑ प्रति - तिष्ठ॑ति । पञ्च॑ । पूर्वाः᳚ । चित॑यः । भ॒व॒न्ति॒ । अथ॑ । ष॒ष्ठीम् । चिति᳚म् । चि॒नु॒ते॒ । षट् । वै । ऋ॒तवः॑ । सं॒ॅव॒थ्स॒र इति॑ सं-व॒थ्स॒रः । ऋ॒तुषु॑ । ए॒व । सं॒ॅव॒थ्स॒र इति॑ सं - व॒थ्स॒रे । प्रतीति॑ । ति॒ष्ठ॒ति॒ । ए॒ताः । वै । \textbf{  6} \newline
                  \newline
                                \textbf{ TS 5.4.2.3} \newline
                  अधि॑पत्नी॒रित्यधि॑ - प॒त्नीः॒ । नाम॑ । इष्ट॑काः । यस्य॑ । ए॒ताः । उ॒प॒धी॒यन्त॒ इत्यु॑प - धी॒यन्ते᳚ । अधि॑पति॒रित्यधि॑ - प॒तिः॒ । ए॒व । स॒मा॒नाना᳚म् । भ॒व॒ति॒ । यम् । द्वि॒ष्यात् । तम् । उ॒प॒दध॒दित्यु॑प-दध॑त् । ध्या॒ये॒त् । ए॒ताभ्यः॑ । ए॒व । ए॒न॒म् । दे॒वता᳚भ्यः । एति॑ । वृ॒श्च॒ति॒ । ता॒जक् । आर्ति᳚म् । एति॑ । ऋ॒च्छ॒ति॒ । अङ्गि॑रसः । सु॒व॒र्गमिति॑ सुवः - गम् । लो॒कम् । यन्तः॑ । या । य॒ज्ञ्स्य॑ । निष्कृ॑ति॒रिति॒ निः-कृ॒तिः॒ । आसी᳚त् । ताम् । ऋषि॑भ्य॒ इत्यृषि॑-भ्यः॒ । प्रतीति॑ । औ॒ह॒न्न् । तत् । हिर॑ण्यम् । अ॒भ॒व॒त् । यत् । हि॒र॒ण्य॒श॒ल्कैरिति॑ हिरण्य - श॒ल्कैः । प्रो॒क्षतीति॑ प्र - उ॒क्षति॑ । य॒ज्ञ्स्य॑ । निष्कृ॑त्या॒ इति॒ निः - कृ॒त्यै॒ । अथो॒ इति॑ । भे॒ष॒जम् । ए॒व । अ॒स्मै॒ । क॒रो॒ति॒ । \textbf{  7} \newline
                  \newline
                                \textbf{ TS 5.4.2.4} \newline
                  अथो॒ इति॑ । रू॒पेण॑ । ए॒व । ए॒न॒म् । समिति॑ । अ॒द्‌र्ध॒य॒ति॒ । अथो॒ इति॑ । हिर॑ण्यज्योति॒षेति॒ हिर॑ण्य - ज्यो॒ति॒षा॒ । ए॒व । सु॒व॒र्गमिति॑ सुवः - गम् । लो॒कम् । ए॒ति॒ । सा॒ह॒स्रव॒तेति॑ साह॒स्र - व॒ता॒ । प्रेति॑ । उ॒क्ष॒ति॒ । सा॒ह॒स्रः । प्र॒जाप॑ति॒रिति॑ प्र॒जा - प॒तिः॒ । प्र॒जाप॑ते॒रिति॑ प्र॒जा - प॒तेः॒ । आप्त्यै᳚ । इ॒माः । मे॒ । अ॒ग्ने॒ । इष्ट॑काः । धे॒नवः॑ । स॒न्तु॒ । इति॑ । आ॒ह॒ । धे॒नूः । ए॒व । ए॒नाः॒ । कु॒रु॒ते॒ । ताः । ए॒न॒म् । का॒म॒दुघा॒ इति॑ काम - दुघाः᳚ । अ॒मुत्र॑ । अ॒मुष्मिन्न्॑ । लो॒के । उपेति॑ । ति॒ष्ठ॒न्ते॒ ॥ \textbf{  8} \newline
                  \newline
                      (सयो॑नि - मे॒ता वै - क॑रो॒त्ये - का॒न्नच॑त्वारिꣳ॒॒शच्च॑)  \textbf{(A2)} \newline \newline
                                \textbf{ TS 5.4.3.1} \newline
                  रु॒द्रः । वै । ए॒षः । यत् । अ॒ग्निः । सः । ए॒तर्.हि॑ । जा॒तः । यर्.हि॑ । सर्वः॑ । चि॒तः । सः । यथा᳚ । व॒थ्सः । जा॒तः । स्तन᳚म् । प्रे॒फ्सतीति॑ प्र - ई॒फ्सति॑ । ए॒वम् । वै । ए॒षः । ए॒तर्.हि॑ । भा॒ग॒धेय॒मिति॑ भाग-धेय᳚म् । प्रेति॑ । ई॒फ्स॒ति॒ । तस्मै᳚ । यत् । आहु॑ति॒मित्या-हु॒ति॒म् । न । जु॒हु॒यात् । अ॒द्ध्व॒र्युम् । च॒ । यज॑मानम् । च॒ । ध्या॒ये॒त् । श॒त॒रु॒द्रीय॒मिति॑ शत - रु॒द्रीय᳚म् । जु॒हो॒ति॒ । भा॒ग॒धेये॒नेति॑ भाग - धेये॑न । ए॒व । ए॒न॒म् । श॒म॒य॒ति॒ । न । आर्ति᳚म् । एति॑ । ऋ॒च्छ॒ति॒ । अ॒द्ध्व॒र्युः । न । यज॑मानः । यत् । ग्रा॒म्याणा᳚म् । प॒शू॒नाम् । \textbf{  9} \newline
                  \newline
                                \textbf{ TS 5.4.3.2} \newline
                  पय॑सा । जु॒हु॒यात् । ग्रा॒म्यान् । प॒शून् । शु॒चा । अ॒र्प॒ये॒त् । यत् । आ॒र॒ण्याना᳚म् । आ॒र॒ण्यान् । ज॒र्ति॒ल॒य॒वा॒ग्वेति॑ जर्तिल - य॒वा॒ग्वा᳚ । वा॒ । जु॒हु॒यात् । ग॒वी॒धु॒क॒य॒वा॒ग्वेति॑ गवीधुक - य॒वा॒ग्वा᳚ । वा॒ । न । ग्रा॒म्यान् । प॒शून् । हि॒नस्ति॑ । न । आ॒र॒ण्यान् । अथो॒ इति॑ । खलु॑ । आ॒हुः॒ । अना॑हुति॒रित्यना᳚-हु॒तिः॒ । वै । ज॒र्तिलाः᳚ । च॒ । ग॒वीधु॑काः । च॒ । इति॑ । अ॒ज॒क्षी॒रेणेत्य॑ज - क्षी॒रेण॑ । जु॒हो॒ति॒ । आ॒ग्ने॒यी । वै । ए॒षा । यत् । अ॒जा । आहु॒त्येत्या - हु॒त्या॒ । ए॒व । जु॒हो॒ति॒ । न । ग्रा॒म्यान् । प॒शून् । हि॒नस्ति॑ । न । आ॒र॒ण्यान् । अङ्गि॑रसः । सु॒व॒र्गमिति॑ सुवः - गम् । लो॒कम् । यन्तः॑ । \textbf{  10} \newline
                  \newline
                                \textbf{ TS 5.4.3.3} \newline
                  अ॒जाया᳚म् । घ॒र्मम् । प्रेति॑ । अ॒सि॒ञ्च॒न्न् । सा । शोच॑न्ती । प॒र्णम् । परेति॑ । अ॒जि॒ही॒त॒ । सः । अ॒र्कः । अ॒भ॒व॒त् । तत् । अ॒र्कस्य॑ । अ॒र्क॒त्वमित्य॑र्क - त्वम् । अ॒र्क॒प॒र्णेनेत्य॑र्क - प॒र्णेन॑ । जु॒हो॒ति॒ । स॒यो॒नि॒त्वायेति॑ सयोनि - त्वाय॑ । उदङ्॑ । तिष्ठन्न्॑ । जु॒हो॒ति॒ । ए॒षा । वै । रु॒द्रस्य॑ । दिक् । स्वाया᳚म् । ए॒व । दि॒शि । रु॒द्रम् । नि॒रव॑दयत॒ इति॑ निः-अव॑दयते । च॒र॒माया᳚म् । इष्ट॑कायाम् । जु॒हो॒ति॒ । अ॒न्त॒तः । ए॒व । रु॒द्रम् । नि॒रव॑दयत॒ इति॑ निः - अव॑दयते । त्रे॒धा॒वि॒भ॒क्तमिति॑ त्रेधा - वि॒भ॒क्तम् । जु॒हो॒ति॒ । त्रयः॑ । इ॒मे । लो॒काः । इ॒मान् । ए॒व । लो॒कान् । स॒माव॑द्वीर्या॒निति॑ स॒माव॑त् - वी॒र्या॒न् । क॒रो॒ति॒ । इय॑ति । अग्रे᳚ । जु॒हो॒ति॒ । \textbf{  11} \newline
                  \newline
                                \textbf{ TS 5.4.3.4} \newline
                  अथ॑ । इय॑ति । अथ॑ । इय॑ति । त्रयः॑ । इ॒मे । लो॒काः । ए॒भ्यः । ए॒व । ए॒न॒म् । लो॒केभ्यः॑ । श॒म॒य॒ति॒ । ति॒स्रः । उत्त॑रा॒ इत्युत् - त॒राः॒ । आहु॑ती॒रित्या-हु॒तीः॒ । जु॒हो॒ति॒ । षट् । समिति॑ । प॒द्य॒न्ते॒ । षट् । वै । ऋ॒तवः॑ । ऋ॒तुभि॒रित्यृ॒तु - भिः॒ । ए॒व । ए॒न॒म् । श॒म॒य॒ति॒ । यत् । अ॒नु॒प॒रि॒क्राम॒मित्य॑नु - प॒रि॒क्राम᳚म् । जु॒हु॒यात् । अ॒न्त॒र॒व॒चा॒रिण॒मित्य॑न्तः - अ॒व॒चा॒रिण᳚म् । रु॒द्रम् । कु॒र्या॒त् । अथो॒ इति॑ । खलु॑ । आ॒हुः॒ । कस्या᳚म् । वा॒ । अह॑ । दि॒शि । रु॒द्रः । कस्या᳚म् । वा॒ । इति॑ । अ॒नु॒प॒रि॒क्राम॒मित्य॑नु - प॒रि॒क्राम᳚म् । ए॒व । हो॒त॒व्य᳚म् । अप॑रिवर्ग॒मित्यप॑रि - व॒र्ग॒म् । ए॒व । ए॒न॒म् । श॒म॒य॒ति॒ । \textbf{  12} \newline
                  \newline
                                \textbf{ TS 5.4.3.5} \newline
                  ए॒ताः । वै । दे॒वताः᳚ । सु॒व॒र्ग्या॑ इति॑ सुवः - ग्याः᳚ । याः । उ॒त्त॒मा इत्यु॑त्-त॒माः । ताः । यज॑मानम् । वा॒च॒य॒ति॒ । ताभिः॑ । ए॒व । ए॒न॒म् । सु॒व॒र्गमिति॑ सुवः-गम् । लो॒कम् । ग॒म॒य॒ति॒ । यम् । द्वि॒ष्यात् । तस्य॑ । सं॒च॒र इति॑ सं - च॒रे । प॒शू॒नाम् । नीति॑ । अ॒स्ये॒त् । यः । प्र॒थ॒मः । प॒शुः । अ॒भि॒तिष्ठ॒तीत्य॑भि - तिष्ठ॑ति । सः । आर्ति᳚म् । एति॑ । ऋ॒च्छ॒ति॒ ॥ \textbf{  13 } \newline
                  \newline
                      (प॒शू॒नां - ॅयन्तो - ऽग्ने॑ जुहो॒त्य - प॑रिवर्गमे॒वैनꣳ॑ शमयति - त्रिꣳ॒॒शच्च॑)  \textbf{(A3)} \newline \newline
                                \textbf{ TS 5.4.4.1} \newline
                  अश्मन्न्॑ । ऊर्ज᳚म् । इति॑ । परीति॑ । सि॒ञ्च॒ति॒ । मा॒र्जय॑ति । ए॒व । ए॒न॒म् । अथो॒ इति॑ । त॒र्पय॑ति । ए॒व । सः । ए॒न॒म् । तृ॒प्तः । अक्षु॑द्ध्यन्न् । अशो॑चन्न् । अ॒मुष्मिन्न्॑ । लो॒के । उपेति॑ । ति॒ष्ठ॒ते॒ । तृप्य॑ति । प्र॒जयेति॑ प्र - जया᳚ । प॒शुभि॒रिति॑ प॒शु - भिः॒ । यः । ए॒वम् । वेद॑ । ताम् । नः॒ । इष᳚म् । ऊर्ज᳚म् । ध॒त्त॒ । म॒रु॒तः॒ । सꣳ॒॒र॒रा॒णा इति॑ सं - र॒रा॒णाः । इति॑ । आ॒ह॒ । अन्न᳚म् । वै । ऊर्क् । अन्न᳚म् । म॒रुतः॑ । अन्न᳚म् । ए॒व । अवेति॑ । रु॒न्धे॒ । अश्मन्न्॑ । ते॒ । क्षुत् । अ॒मुम् । ते॒ । शुक् । \textbf{  14 } \newline
                  \newline
                                \textbf{ TS 5.4.4.2} \newline
                  ऋ॒च्छ॒तु॒ । यम् । द्वि॒ष्मः । इति॑ । आ॒ह॒ । यम् । ए॒व । द्वेष्टि॑ । तम् । अ॒स्य॒ । क्षु॒धा । च॒ । शु॒चा । च॒ । अ॒र्प॒य॒ति॒ । त्रिः । प॒रि॒षि॒ञ्चन्निति॑ परि - सि॒ञ्चन्न् । परीति॑ । ए॒ति॒ । त्रि॒वृदिति॑ त्रि-वृत् । वै । अ॒ग्निः । यावान्॑ । ए॒व । अ॒ग्निः । तस्य॑ । शुच᳚म् । श॒म॒य॒ति॒ । त्रिः । पुनः॑ । परीति॑ । ए॒ति॒ । षट् । समिति॑ । प॒द्य॒न्ते॒ । षट् । वै । ऋ॒तवः॑ । ऋ॒तुभि॒रित्यृ॒तु-भिः॒। ए॒व । अ॒स्य॒ । शुच᳚म् । श॒म॒य॒ति॒ । अ॒पाम् । वै । ए॒तत् । पुष्प᳚म् । यत् । वे॒त॒सः । अ॒पाम् । \textbf{  15} \newline
                  \newline
                                \textbf{ TS 5.4.4.3} \newline
                  शरः॑ । अव॑काः । वे॒त॒स॒शा॒खयेति॑ वेतस - शा॒खया᳚ । च॒ । अव॑काभिः । च॒ । वीति॑ । क॒र्.॒ष॒ति॒ । आपः॑ । वै । शा॒न्ताः । शा॒न्ताभिः॑ । ए॒व । अ॒स्य॒ । शुच᳚म् । श॒म॒य॒ति॒ । यः । वै । अ॒ग्निम् । चि॒तम् । प्र॒थ॒मः । प॒शुः । अ॒धि॒क्राम॒तीत्य॑धि-क्राम॑ति । ई॒श्व॒रः । वै । तम् । शु॒चा । प्र॒दह॒ इति॑ प्र - दहः॑ । म॒ण्डूके॑न । वीति॑ । क॒र्.॒ष॒ति॒ । ए॒षः । वै । प॒शू॒नाम् । अ॒नु॒प॒जी॒व॒नी॒य इत्य॑नुप -  जी॒व॒नी॒यः । न । वै । ए॒षः । ग्रा॒म्येषु॑ । प॒शुषु॑ । हि॒तः । न । आ॒र॒ण्येषु॑ । तम् । ए॒व । शु॒चा । अ॒र्प॒य॒ति॒ । अ॒ष्टा॒भिः । वीति॑ । क॒र्.॒ष॒ति॒ । \textbf{  16} \newline
                  \newline
                                \textbf{ TS 5.4.4.4} \newline
                  अ॒ष्टाक्ष॒रेत्य॒ष्टा-अ॒क्ष॒रा॒ । गा॒य॒त्री । गा॒य॒त्रः । अ॒ग्निः । यावान्॑ । ए॒व । अ॒ग्निः । तस्य॑ । शुच᳚म् । श॒म॒य॒ति॒ । पा॒व॒कव॑तीभि॒रिति॑ पाव॒क - व॒ती॒भिः॒ । अन्न᳚म् । वै । पा॒व॒कः । अन्ने॑न । ए॒व । अ॒स्य॒ । शुच᳚म् । श॒म॒य॒ति॒ । मृ॒त्युः । वै ।   ए॒षः । यत् । अ॒ग्निः । ब्रह्म॑णः । ए॒तत् । रू॒पम् । यत् । कृ॒ष्णा॒जि॒नमिति॑ कृष्ण - अ॒जि॒नम् । कार्ष्णी॒ इति॑ । उ॒पा॒नहौ᳚ । उपेति॑ । मु॒ञ्च॒ते॒ । ब्रह्म॑णा । ए॒व । मृ॒त्योः । अ॒न्तः । ध॒त्ते॒ । अ॒न्तः । मृ॒त्योः । ध॒त्ते॒ । अ॒न्तः । अ॒न्नाद्या॒दित्य॑न्न - अद्या᳚त् । इति॑ । आ॒हुः॒ । अ॒न्याम् । उ॒प॒मु॒ञ्चत॒ इत्यु॑प - मु॒ञ्चते᳚ । अ॒न्याम् । न । अ॒न्तः । \textbf{  17} \newline
                  \newline
                                \textbf{ TS 5.4.4.5} \newline
                  ए॒व । मृ॒त्योः । ध॒त्ते॒ । अवेति॑ । अ॒न्नाद्य॒मित्य॑न्न - अद्य᳚म् । रु॒न्धे॒ । नमः॑ । ते॒ । हर॑से । शो॒चिषे᳚ । इति॑ । आ॒ह॒ । न॒म॒स्कृत्येति॑ नमः - कृत्य॑ । हि । वसी॑याꣳसम् । उ॒प॒चर॒न्तीत्यु॑प - चर॑न्ति । अ॒न्यम् । ते॒ । अ॒स्मत् । त॒प॒न्तु॒ । हे॒तयः॑ । इति॑ । आ॒ह॒ । यम् । ए॒व । द्वेष्टि॑ । तम् । अ॒स्य॒ । शु॒चा । अ॒र्प॒य॒ति॒ । पा॒व॒कः । अ॒स्मभ्य॒मित्य॒स्म - भ्य॒म् । शि॒वः । भ॒व॒ । इति॑ । आ॒ह॒ । अन्न᳚म् । वै । पा॒व॒कः । अन्न᳚म् । ए॒व । अवेति॑ । रु॒न्धे॒ । द्वाभ्या᳚म् । अधीति॑ । क्रा॒म॒ति॒ । प्रति॑ष्ठित्या॒ इति॒ प्रति॑ - स्थि॒त्यै॒ । अ॒प॒स्य॑वतीभ्या॒मित्य॑प॒स्य॑ - व॒ती॒भ्या॒म् । शान्त्यै᳚ ॥ \textbf{  18 } \newline
                  \newline
                      (शु - ग्वे॑त॒सो॑ऽपा - म॑ष्टा॒भिर्वि क॑र्.षति॒ - नान्त - रेका॒न्न प॑ञ्चा॒शच्च॑)  \textbf{(A4)} \newline \newline
                                \textbf{ TS 5.4.5.1} \newline
                  नृ॒षद॒ इति॑ नृ - सदे᳚ । वट् । इति॑ । व्याघा॑रय॒तीति॑ वि-आघा॑रयति । प॒ङ्क्त्या । आहु॒त्येत्या-हु॒त्या॒ । य॒ज्ञ्॒मु॒खमिति॑ यज्ञ् - मु॒खम् । एति॑ । र॒भ॒ते॒ । अ॒क्ष्ण॒या । व्याघा॑रय॒तीति॑ वि - आघा॑रयति । तस्मा᳚त् । अ॒क्ष्ण॒या । प॒शवः॑ । अङ्गा॑नि । प्रेति॑ । ह॒र॒न्ति॒ । प्रति॑ष्ठित्या॒ इति॒ प्रति॑ - स्थि॒त्यै॒ । यत् । व॒ष॒ट्कु॒र्यादिति॑ वषट् - कु॒र्यात् । या॒तया॒मेति॑ या॒त - या॒मा॒ । अ॒स्य॒ । व॒ष॒ट्का॒र इति॑ वषट्-का॒रः । स्या॒त् । यत् । न । व॒ष॒ट्कु॒र्यादिति॑ वषट् - कु॒र्यात् । रक्षाꣳ॑सि । य॒ज्ञ्म् । ह॒न्युः॒ । वट् । इति॑ । आ॒ह॒ । प॒रोक्ष॒मिति॑ परः - अक्ष᳚म् । ए॒व । वष॑ट् । क॒रो॒ति॒ । न । अ॒स्य॒ । या॒तया॒मेति॑ या॒त - या॒मा॒ । व॒ष॒ट्का॒र इति॑ वषट् - का॒रः । भव॑ति । न । य॒ज्ञ्म् । रक्षाꣳ॑सि । घ्न॒न्ति॒ । हु॒ताद॒ इति॑ हुत - अदः॑ । वै । अ॒न्ये । दे॒वाः । \textbf{  19} \newline
                  \newline
                                \textbf{ TS 5.4.5.2} \newline
                  अ॒हु॒ताद॒ इत्य॑हुत - अदः॑ । अ॒न्ये । तान् । अ॒ग्नि॒चिदित्य॑ग्नि-चित् । ए॒व । उ॒भयान्॑ । प्री॒णा॒ति॒ । ये । दे॒वाः । दे॒वाना᳚म् । इति॑ । द॒द्ध्ना । म॒धु॒मि॒श्रेणेति॑ मधु - मि॒श्रेण॑ । अवेति॑ । उ॒क्ष॒ति॒ । हु॒ताद॒ इति॑ हुत - अदः॑ । च॒ । ए॒व । दे॒वान् । अ॒हु॒ताद॒ इत्य॑हुत - अदः॑ । च॒ । यज॑मानः । प्री॒णा॒ति॒ । ते । यज॑मानम् । प्री॒ण॒न्ति॒ । द॒द्ध्ना । ए॒व । हु॒ताद॒ इति॑ हुत - अदः॑ । प्री॒णाति॑ । मधु॑षा । अ॒हु॒ताद॒ इत्य॑हुत - अदः॑ । ग्रा॒म्यम् । वै । ए॒तत् । अन्न᳚म् । यत् । दधि॑ । आ॒र॒ण्यम् । मधु॑ । यत् । द॒द्ध्ना । म॒धु॒मि॒श्रेणेति॑ मधु - मि॒श्रेण॑ । अ॒वोक्ष॒तीत्य॑व - उक्ष॑ति । उ॒भय॑स्य । अव॑रुद्ध्या॒ इत्यव॑ - रु॒ध्यै॒ । ग्रु॒मु॒ष्टिना᳚ । अवेति॑ । उ॒क्ष॒ति॒ । प्रा॒जा॒प॒त्य इति॑ प्राजा-प॒त्यः । \textbf{  20} \newline
                  \newline
                                \textbf{ TS 5.4.5.3} \newline
                  वै । ग्रु॒मु॒ष्टिः । स॒यो॒नि॒त्वायेति॑ सयोनि - त्वाय॑ । द्वाभ्या᳚म् । प्रति॑ष्ठित्या॒ इति॒ प्रति॑ - स्थि॒त्यै॒ । अ॒नु॒प॒रि॒चार॒मित्य॑नु - प॒रि॒चार᳚म् । अवेति॑ । उ॒क्ष॒ति॒ । अप॑रिवर्ग॒मित्यप॑रि - व॒र्ग॒म् । ए॒व । ए॒ना॒न् । प्री॒णा॒ति॒ । वीति॑ । वै । ए॒षः । प्रा॒णैरिति॑ प्र - अ॒नैः । प्र॒जयेति॑ प्र - जया᳚ । प॒शुभि॒रिति॑ प॒शु - भिः॒ । ऋ॒द्ध्य॒ते॒ । यः । अ॒ग्निम् । चि॒न्वन्न् । अ॒धि॒क्राम॒तीत्य॑धि - क्राम॑ति । प्रा॒ण॒दा इति॑ प्राण - दाः । अ॒पा॒न॒दा इत्य॑पान - दाः । इति॑ । आ॒ह॒ । प्रा॒णानिति॑ प्र - अ॒नान् । ए॒व । आ॒त्मन् । ध॒त्ते॒ । व॒र्चो॒दा इति॑ वर्चः - दाः । व॒रि॒वो॒दा इति॑ वरिवः - दाः । इति॑ । आ॒ह॒ । प्र॒जेति॑ प्र - जा । वै । वर्चः॑ । प॒शवः॑ । वरि॑वः । प्र॒जामिति॑ प्र - जाम् । ए॒व । प॒शून् । आ॒त्मन्न् । ध॒त्ते॒ । इन्द्रः॑ । वृ॒त्रम् । अ॒ह॒न्न् । तम् । वृ॒त्रः । \textbf{  21} \newline
                  \newline
                                \textbf{ TS 5.4.5.4} \newline
                  ह॒तः । षो॒ड॒शभि॒रिति॑ षोड॒श - भिः॒ । भो॒गैः । अ॒सि॒ना॒त् । सः । ए॒ताम् । अ॒ग्नये᳚ । अनी॑कवत॒ इत्यनी॑क - व॒ते॒ । आहु॑ति॒मित्या - हु॒ति॒म् । अ॒प॒श्य॒त् । ताम् । अ॒जु॒हो॒त् । तस्य॑ । अ॒ग्निः । अनी॑कवा॒नित्यनी॑क - वा॒न् । स्वेन॑ । भा॒ग॒धेये॒नेति॑ भाग - धेये॑न । प्री॒तः । षो॒ड॒श॒धेति॑ षोडश-धा । वृ॒त्रस्य॑ । भो॒गान् । अपीति॑ । अ॒द॒ह॒त् । वै॒श्व॒क॒र्म॒णेनेति॑ वैश्व - क॒र्म॒णेन॑ । पा॒प्मनः॑ । निरिति॑ । अ॒मु॒च्य॒त॒ । यत् । अ॒ग्नये᳚ । अनी॑कवत॒ इत्यनी॑क - व॒ते॒ । आहु॑ति॒मित्या - हु॒ति॒म् । जु॒होति॑ । अ॒ग्निः । ए॒व । अ॒स्य॒ । आनी॑कवा॒नित्यनी॑क - वा॒न् । स्वेन॑ । भा॒ग॒धेये॒नेति॑ भाग - धेये॑न । प्री॒तः । पा॒प्मान᳚म् । अपीति॑ । द॒ह॒ति॒ । वै॒श्व॒क॒र्म॒णेनेति॑ वैश्व - क॒र्म॒णेन॑ । पा॒प्मनः॑ । निरिति॑ । मु॒च्य॒ते॒ । यम् । का॒मये॑त । चि॒रम् । पा॒प्मनः॑ । \textbf{  22} \newline
                  \newline
                                \textbf{ TS 5.4.5.5} \newline
                  निरिति॑ । मु॒च्ये॒त॒ । इति॑ । एकै॑क॒मित्येकं᳚-ए॒क॒म् । तस्य॑ । जु॒हु॒या॒त् । चि॒रम् । ए॒व । पा॒प्मनः॑ । निरिति॑ । मु॒च्य॒ते॒ । यम् । का॒मये॑त । ता॒जक् । पा॒प्मनः॑ । निरिति॑ । मु॒च्ये॒त॒ । इति॑ । सर्वा॑णि । तस्य॑ । अ॒नु॒द्रुत्येत्य॑नु-द्रुत्य॑ । जु॒हु॒या॒त् । ता॒जक् । ए॒व । पा॒प्मनः॑ । निरिति॑ । मु॒च्य॒ते॒ । अथो॒ इति॑ । खलु॑ । नाना᳚ । ए॒व । सू॒क्ताभ्या॒मिति॑ सु - उ॒क्ताभ्या᳚म् । जु॒हो॒ति॒ । नाना᳚ । ए॒व । सू॒क्तयो॒रिति॑ सु - उ॒क्तयोः᳚ । वी॒र्य᳚म् । द॒धा॒ति॒ । अथो॒ इति॑ । प्रति॑ष्ठित्या॒ इति॒ प्रति॑ - स्थि॒त्यै॒ ॥ \textbf{  23 } \newline
                  \newline
                      (दे॒वाः - प्रा॑जाप॒त्यो-वृ॒त्र - श्चि॒रं पा॒प्मन॑ - श्चत्वारिꣳ॒॒शच्च॑)  \textbf{(A5)} \newline \newline
                                \textbf{ TS 5.4.6.1} \newline
                  उदिति॑ । ए॒न॒म् । उ॒त्त॒रामित्यु॑त् - त॒राम् । न॒य॒ । इति॑ । स॒मिध॒ इति॑ सं - इधः॑ । एति॑ । द॒धा॒ति॒ । यथा᳚ । जन᳚म् । य॒ते । अ॒व॒सम् । क॒रोति॑ । ता॒दृक् । ए॒व । तत् । ति॒स्रः । एति॑ । द॒धा॒ति॒ । त्रि॒वृदिति॑ त्रि - वृत् । वै । अ॒ग्निः । यावान्॑ । ए॒व । अ॒ग्निः । तस्मै᳚ । भा॒ग॒धेय॒मिति॑ भाग - धेय᳚म् । क॒रो॒ति॒ । औदु॑बंरीः । भ॒व॒न्ति॒ । ऊर्क् । वै । उ॒दु॒बंरः॑ । ऊर्ज᳚म् । ए॒व । अ॒स्मै॒ । अपीति॑ । द॒धा॒ति॒ । उदिति॑ । उ॒ । त्वा॒ । विश्वे᳚ । दे॒वाः । इति॑ । आ॒ह॒ । प्रा॒णा इति॑ प्र - अ॒नाः । वै । विश्वे᳚ । दे॒वाः । प्रा॒णैरिति॑ प्र - अ॒नैः । \textbf{  24} \newline
                  \newline
                                \textbf{ TS 5.4.6.2} \newline
                  ए॒व । ए॒न॒म् । उदिति॑ । य॒च्छ॒ते॒ । अग्ने᳚ । भर॑न्तु । चित्ति॑भि॒रिति॒ चित्ति॑ - भिः॒ । इति॑ । आ॒ह॒ । यस्मै᳚ । ए॒व । ए॒न॒म् । चि॒त्ताय॑ । उ॒द्यच्छ॑त॒ इत्यु॑त्-यच्छ॑ते । तेन॑ । ए॒व । ए॒न॒म् । समिति॑ । अ॒र्द्ध॒य॒ति॒ । पञ्च॑ । दिशः॑ । दैवीः᳚ । य॒ज्ञ्म् । अ॒व॒न्तु॒ । दे॒वीः । इति॑ । आ॒ह॒ । दिशः॑ । हि । ए॒षः । अन्विति॑ । प्र॒च्यव॑त॒ इति॑ प्र - च्यव॑ते । अपेति॑ । अम॑तिम् । दु॒र्म॒तिमिति॑ दुः - म॒तिम् । बाध॑मानाः । इति॑ । आ॒ह॒ । रक्ष॑साम् । अप॑हत्या॒ इत्यप॑ - ह॒त्यै॒ । रा॒यः । पोषे᳚ । य॒ज्ञ्प॑ति॒मिति॑ य॒ज्ञ्-प॒ति॒म् । आ॒भज॑न्ती॒रित्या᳚ - भज॑न्तीः । इति॑ । आ॒ह॒ । प॒शवः॑ । वै । रा॒यः । पोषः॑ । \textbf{  25} \newline
                  \newline
                                \textbf{ TS 5.4.6.3} \newline
                  प॒शून् । ए॒व । अवेति॑ । रु॒न्धे॒ । ष॒ड्भिरिति॑ षट्-भिः । ह॒र॒ति॒ । षट् । वै । ऋ॒तवः॑ । ऋ॒तुभि॒रित्यृ॒तु - भिः॒ । ए॒व । ए॒न॒म् । ह॒र॒ति॒ । द्वे इति॑ । प॒रि॒गृह्य॑वती॒ इति॑ परि॒गृह्य॑ - व॒ती॒ । भ॒व॒तः॒ । रक्ष॑साम् । अप॑हत्या॒ इत्यप॑ - ह॒त्यै॒ । सूर्य॑रश्मि॒रिति॒ सूर्य॑ - र॒श्मिः॒ । हरि॑केश॒ इति॒ हरि॑-के॒शः॒ । पु॒रस्ता᳚त् । इति॑ । आ॒ह॒ । प्रसू᳚त्या॒ इति॒ प्र-सू॒त्यै॒ । ततः॑ । पा॒व॒काः । आ॒शिष॒ इत्या᳚ - शिषः॑ । नः॒ । जु॒ष॒न्ता॒म् । इति॑ । आ॒ह॒ । अन्न᳚म् । वै । पा॒व॒कः । अन्न᳚म् । ए॒व । अवेति॑ । रु॒न्धे॒ । दे॒वा॒सु॒रा इति॑ देव-अ॒सु॒राः । संॅय॑त्ता॒ इति॒ सं - य॒त्ताः॒ । आ॒स॒न्न् । ते । दे॒वाः । ए॒तत् । अप्र॑तिरथ॒मित्यप्र॑ति - र॒थ॒म् । अ॒प॒श्य॒न्न् । तेन॑ । वै । ते । अ॒प्र॒ति । \textbf{  26} \newline
                  \newline
                                \textbf{ TS 5.4.6.4} \newline
                  असु॑रान् । अ॒ज॒य॒न्न् । तत् । अप्र॑तिरथ॒स्येत्यप्र॑ति - र॒थ॒स्य॒ । अ॒प्र॒ति॒र॒थ॒त्वमित्य॑प्रतिरथ - त्वम् । यत् । अप्र॑तिरथ॒मित्यप्र॑ति-र॒थ॒म् । द्वि॒तीयः॑ । होता᳚ । अ॒न्वाहेत्य॑नु - आह॑ । अ॒प्र॒ति । ए॒व । तेन॑ । यज॑मानः । भ्रातृ॑व्यान् । ज॒य॒ति॒ । अथो॒ इति॑ । अन॑भिजित॒मित्यन॑भि - जि॒त॒म् । ए॒व । अ॒भीति॑ । ज॒य॒ति॒ । द॒श॒र्चमिति॑ दश - ऋ॒चम् । भ॒व॒ति॒ । दशा᳚क्ष॒रेति॒ दश॑ - अ॒क्ष॒रा॒ । वि॒राडिति॑ वि - राट् । वि॒राजेति॑ वि - राजा᳚ । इ॒मौ । लो॒कौ । विधृ॑ता॒विति॒ वि - धृ॒तौ॒ । अ॒नयोः᳚ । लो॒कयोः᳚ । विधृ॑त्या॒ इति॒ वि - धृ॒त्यै॒ । अथो॒ इति॑ । दशा᳚क्ष॒रेति॒ दश॑ - अ॒क्ष॒रा॒ । वि॒राडिति॑ वि - राट् । अन्न᳚म् । वि॒राडिति॑ वि - राट् । वि॒राजीति॑ वि - राजि॑ । ए॒व । अ॒न्नाद्य॒ इत्य॑न्न - अद्ये᳚ । प्रतीति॑ । ति॒ष्ठ॒ति॒ । अस॑त् । इ॒व॒ । वै । अ॒न्तरि॑क्षम् । अ॒न्तरि॑क्षम् । इ॒व॒ । आग्नी᳚ध्र॒मित्याग्नि॑ - इ॒ध्र॒म् । आग्नी᳚द्ध्र॒ इत्याग्नि॑ - इ॒द्ध्रे॒ । \textbf{  27} \newline
                  \newline
                                \textbf{ TS 5.4.6.5} \newline
                  अश्मा॑नम् । नीति॑ । द॒धा॒ति॒ । स॒त्त्वायेति॑ सत् - त्वाय॑ । द्वाभ्या᳚म् । प्रति॑ष्ठित्या॒ इति॒ प्रति॑ - स्थि॒त्यै॒ । वि॒मान॒ इति॑ वि - मानः॑ । ए॒षः । दि॒वः । मद्ध्ये᳚ । आ॒स्ते॒ । इति॑ । आ॒ह॒ । वीति॑ । ए॒व । ए॒तया᳚ । मि॒मी॒ते॒ । मद्ध्ये᳚ । दि॒वः । निहि॑त॒ इति॒ नि - हि॒तः॒ । पृश्निः॑ । अश्मा᳚ । इति॑ । आ॒ह॒ । अन्न᳚म् । वै । पृश्नि॑ । अन्न᳚म् । ए॒व । अवेति॑ । रु॒न्धे॒ । च॒त॒सृभि॒रिति॑ चत॒सृ-भिः॒ । एति॑ । पुच्छा᳚त् । ए॒ति॒ । च॒त्वारि॑ । छन्दाꣳ॑सि । छन्दो॑भि॒रिति॒ छन्दः॑ - भिः॒ । ए॒व । इन्द्र᳚म् । विश्वाः᳚ । अ॒वी॒वृ॒ध॒न्न् । इति॑ । आ॒ह॒ । वृद्धि᳚म् । ए॒व । उ॒पाव॑र्तत॒ इत्यु॑प-आव॑र्तते । वाजा॑नाम् । सत्प॑ति॒मिति॒ सत् - प॒ति॒म् । पति᳚म् । \textbf{  28} \newline
                  \newline
                                \textbf{ TS 5.4.6.6} \newline
                  इति॑ । आ॒ह॒ । अन्न᳚म् । वै । वाजः॑ । अन्न᳚म् । ए॒व । अवेति॑ । रु॒न्धे॒ । सु॒म्न॒हूरिति॑ सुम्न - हूः । य॒ज्ञ्ः । दे॒वान् । एति॑ । च॒ । व॒क्ष॒त् । इति॑ । आ॒ह॒ । प्र॒जेति॑ प्र - जा । वै । प॒शवः॑ । सु॒म्नम् । प्र॒जामिति॑ प्र - जाम् । ए॒व । प॒शून् । आ॒त्मन्न् । ध॒त्ते॒ । यक्ष॑त् । अ॒ग्निः । दे॒वः । दे॒वान् । एति॑ । च॒ । व॒क्ष॒त् । इति॑ । आ॒ह॒ । स्व॒गाकृ॑त्या॒ इति॑ स्व॒गा - कृ॒त्यै॒ । वाज॑स्य । मा॒ । प्र॒स॒वेनेति॑ प्र - स॒वेन॑ । उ॒द्ग्रा॒भेणेत्यू॑त्-ग्रा॒भेण॑ । उदिति॑ । अ॒ग्र॒भी॒त् । इति॑ । आ॒ह॒ । अ॒सौ । वै । आ॒दि॒त्यः । उ॒द्यन्नित्यु॑त् - यन्न् । उ॒द्ग्रा॒भ इत्यु॑त् - ग्रा॒भः । ए॒षः ( ) । नि॒म्रोच॒न्निति॑ नि - म्रोचन्न्॑ । नि॒ग्रा॒भ इति॑ नि - ग्रा॒भः । ब्रह्म॑णा । ए॒व । आ॒त्मान᳚म् । उ॒द्गृ॒ह्णातीत्यु॑त् - गृ॒ह्णाति॑ । ब्रह्म॑णा । भ्रातृ॑व्यम् । नीति॑ । गृ॒ह्णा॒ति॒ ॥ \textbf{  29 } \newline
                  \newline
                      (प्रा॒णैः - पोषो᳚ - प्र॒त्या - ग्नी᳚द्धे॒ - पति॑ - मे॒ष - दश॑ च)  \textbf{(A6)} \newline \newline
                                \textbf{ TS 5.4.7.1} \newline
                  प्राची᳚म् । अन्विति॑ । प्र॒दिश॒मिति॑ प्र-दिश᳚म् । प्रेति॑ । इ॒हि॒ । वि॒द्वान् । इति॑ । आ॒ह॒ । दे॒व॒लो॒कमिति॑ देव - लो॒कम् । ए॒व । ए॒तया᳚ । उ॒पाव॑र्तत॒ इत्यु॑प - आव॑र्तते । क्रम॑द्ध्वम् । अ॒ग्निना᳚ । नाक᳚म् । इति॑ । आ॒ह॒ । इ॒मान् । ए॒व । ए॒तया᳚ । लो॒कान् । क्र॒म॒ते॒ । पृ॒थि॒व्याः । अ॒हम् । उदिति॑ । अ॒न्तरि॑क्षम् । एति॑ । अ॒रु॒ह॒म् । इति॑ । आ॒ह॒ । इ॒मान् । ए॒व । ए॒तया᳚ । लो॒कान् । स॒मारो॑ह॒तीति॑ सं - आरो॑हति । सुवः॑ । यन्तः॑ । न । अपेति॑ । ई॒क्ष॒न्ते॒ । इति॑ । आ॒ह॒ । सु॒व॒र्गमिति॑ सुवः - गम् । ए॒व । ए॒तया᳚ । लो॒कम् । ए॒ति॒ । अग्ने᳚ । प्रेति॑ । इ॒हि॒ । \textbf{  30} \newline
                  \newline
                                \textbf{ TS 5.4.7.2} \newline
                  प्र॒थ॒मः । दे॒व॒य॒तामिति॑ देव - य॒ताम् । इति॑ । आ॒ह॒ । उ॒भये॑षु । ए॒व । ए॒तया᳚ । दे॒व॒म॒नु॒ष्येष्विति॑ देव - म॒नु॒ष्येषु॑ । चक्षुः॑ । द॒धा॒ति॒ । प॒ञ्चभि॒रिति॑ प॒ञ्च - भिः॒ । अधीति॑ । क्रा॒म॒ति॒ । पाङ्क्तः॑ । य॒ज्ञ्ः । यावान्॑ । ए॒व । य॒ज्ञ्ः । तेन॑ । स॒ह । सु॒व॒र्गमिति॑ सुवः - गम् । लो॒कम् । ए॒ति॒ । नक्तो॒षासा᳚ । इति॑ । पु॒रो॒नु॒वा॒क्या॑मिति॑ पुरः - अ॒नु॒वा॒क्या᳚म् । अन्विति॑ । आ॒ह॒ । प्रत्यै᳚ । अग्ने᳚ । स॒ह॒स्रा॒क्षेति॑ सहस्र - अ॒क्ष॒ । इति॑ । आ॒ह॒ । सा॒ह॒स्रः । प्र॒जाप॑ति॒रिति॑ प्र॒जा - प॒तिः॒ । प्र॒जाप॑ते॒रिति॑ प्र॒जा - प॒तेः॒ । आप्त्यै᳚ । तस्मै᳚ । ते॒ । वि॒धे॒म॒ । वाजा॑य । स्वाहा᳚ । इति॑ । आ॒ह॒ । अन्न᳚म् । वै । वाजः॑ । अन्न᳚म् । ए॒व । अवेति॑ । \textbf{  31} \newline
                  \newline
                                \textbf{ TS 5.4.7.3} \newline
                  रु॒न्धे॒ । द॒द्ध्नः । पू॒र्णाम् । औदु॑बंरीम् । स्व॒य॒मा॒तृ॒ण्णाया॒मिति॑ स्वयं - आ॒तृ॒ण्णाया᳚म् । जु॒हो॒ति॒ । ऊर्क् । वै । दधि॑ । ऊर्क् । उ॒दु॒बंरः॑ । अ॒सौ । स्व॒य॒मा॒तृ॒ण्णेति॑ स्वयं - आ॒तृ॒ण्णा । अ॒मुष्या᳚म् । ए॒व । ऊर्ज᳚म् । द॒धा॒ति॒ । तस्मा᳚त् । अ॒मुतः॑ । अ॒र्वाची᳚म् । ऊर्ज᳚म् । उपेति॑ । जी॒वा॒मः॒ । ति॒सृभि॒रिति॑ ति॒सृ - भिः॒। सा॒द॒य॒ति॒ । त्रि॒वृदिति॑ त्रि - वृत् । वै । अ॒ग्निः । यावान्॑ । ए॒व । अ॒ग्निः । तम् । प्र॒ति॒ष्ठामिति॑ प्रति-स्थाम् । ग॒म॒य॒ति॒ । प्रेद्ध॒ इति॒ प्र - इ॒द्धः॒ । अ॒ग्ने॒ । दी॒दि॒हि॒ । पु॒रः । नः॒ । इति॑ । औदु॑बंरीम् । एति॑ । द॒धा॒ति॒ । ए॒षा । वै । सू॒र्मी । कर्ण॑काव॒तीति॒ कर्ण॑क - व॒ति॒ । ए॒तया᳚ । ह॒ । स्म॒ । \textbf{  32} \newline
                  \newline
                                \textbf{ TS 5.4.7.4} \newline
                  वै । दे॒वाः । असु॑राणाम् । श॒त॒त॒र्.॒हानिति॑ शत-त॒र्॒.हान् । तृꣳ॒॒ह॒न्ति॒ । यत् । ए॒तया᳚ । स॒मिध॒मिति॑ सं - इध᳚म् । आ॒दधा॒तीत्या᳚ - दधा॑ति । वज्र᳚म् । ए॒व । ए॒तत् । श॒त॒घ्नीमिति॑ शत - घ्नीम् । यज॑मानः । भ्रातृ॑व्याय । प्रेति॑ । ह॒र॒ति॒ । स्तृत्यै᳚ । अच्छ॑बंट्कार॒मित्यछ॑बंट्-का॒र॒म् । वि॒धेम॑ । ते॒ । प॒र॒मे । जन्मन्न्॑ । अ॒ग्ने॒ । इति॑ । वैक॑ङ्कतीम् । एति॑ । द॒धा॒ति॒ । भाः । ए॒व । अवेति॑ । रु॒न्धे॒ । ताम् । स॒वि॒तुः । वरे᳚ण्यस्य । चि॒त्राम् । इति॑ । श॒मी॒मयी॒मिति॑ शमी - मयी᳚म् । शान्त्यै᳚ । अ॒ग्निः । वा॒ । ह॒ । वै । अ॒ग्नि॒चित॒मित्य॑ग्नि - चित᳚म् । दु॒हे । अ॒ग्नि॒चिदित्य॑ग्नि - चित् । वा॒ । अ॒ग्निम् । दु॒हे॒ । ताम् । \textbf{  33} \newline
                  \newline
                                \textbf{ TS 5.4.7.5} \newline
                  स॒वि॒तुः । वरे᳚ण्यस्य । चि॒त्राम् । इति॑ । आ॒ह॒ । ए॒षः । वै । अ॒ग्नेः । दोहः॑ । तम् । अ॒स्य॒ । कण्वः॑ । ए॒व । श्रा॒य॒सः । अ॒वे॒त् । तेन॑ । ह॒ । स्म॒ । ए॒न॒म् । सः । दु॒हे॒ । यत् । ए॒तया᳚ । स॒मिध॒मिति॑ सं - इध᳚म् । आ॒दधा॒तीत्या᳚ - दधा॑ति । अ॒ग्नि॒चिदित्य॑ग्नि - चित् । ए॒व । तत् । अ॒ग्निम् । दु॒हे॒ । स॒प्त । ते॒ । अ॒ग्ने॒ । स॒मिध॒ इति॑ सं - इधः॑ । स॒प्त । जि॒ह्वाः । इति॑ । आ॒ह॒ । स॒प्त । ए॒व । अ॒स्य॒ । साप्ता॑नि । प्री॒णा॒ति॒ । पू॒र्णया᳚ । जु॒हो॒ति॒ । पू॒र्णः । इ॒व॒ । हि । प्र॒जाप॑ति॒रिति॑ प्र॒जा - प॒तिः॒ । प्र॒जाप॑ते॒रिति॑ प्र॒जा - प॒तेः॒ । \textbf{  34} \newline
                  \newline
                                \textbf{ TS 5.4.7.6} \newline
                  आप्त्यै᳚ । न्यू॑न॒येति॒ नि-ऊ॒न॒या॒ । जु॒हो॒ति॒ । न्यू॑ना॒दिति॒ नि - ऊ॒ना॒त् । हि । प्र॒जाप॑ति॒रिति॑ प्र॒जा - प॒तिः॒ । प्र॒जा इति॑ प्र - जाः । असृ॑जत । प्र॒जाना॒मिति॑ प्र - जाना᳚म् । सृष्ट्यै᳚ । अ॒ग्निः । दे॒वेभ्यः॑ । निला॑यत । सः । दिशः॑ । अनु॑ । प्रेति॑ । अ॒वि॒श॒त् । जुह्व॑त् । मन॑सा । दिशः॑ । ध्या॒ये॒त् । दि॒ग्भ्य इति॑ दिक् - भ्यः । ए॒व । ए॒न॒म् । अवेति॑ । रु॒न्धे॒ । द॒द्ध्ना । पु॒रस्ता᳚त् । जु॒हो॒ति॒ । आज्ये॑न । उ॒परि॑ष्टात् । तेजः॑ । च॒ । ए॒व । अ॒स्मै॒ । इ॒न्द्रि॒यम् । च॒ । स॒मीची॒ इति॑ । द॒धा॒ति॒ । द्वाद॑शकपाल॒ इति॒ द्वाद॑श - क॒पा॒लः॒ । वै॒श्वा॒न॒रः । भ॒व॒ति॒ । द्वाद॑श । मासाः᳚ । सं॒ॅव॒थ्स॒र इति॑ सं-व॒थ्स॒रः । सं॒ॅव॒थ्स॒र इति॑ सं-व॒थ्स॒रः । अ॒ग्निः । वै॒श्वा॒न॒रः । सा॒क्षादिति॑ स - अ॒क्षात् । \textbf{  35} \newline
                  \newline
                                \textbf{ TS 5.4.7.7} \newline
                  ए॒व । वै॒श्वा॒न॒रम् । अवेति॑ । रु॒न्धे॒ । यत् । प्र॒या॒जा॒नू॒या॒जानिति॑ प्रयाज - अ॒नू॒या॒जान् । कु॒र्यात् । विक॑स्ति॒रिति॒ वि - क॒स्तिः॒ । सा । य॒ज्ञ्स्य॑ । द॒र्वि॒हो॒ममिति॑ दर्वि - हो॒मम् । क॒रो॒ति॒ । य॒ज्ञ्स्य॑ । प्रति॑ष्ठित्या॒ इति॒ प्रति॑ - स्थि॒त्यै॒ । रा॒ष्ट्रम् । वै । वै॒श्वा॒न॒रः । विट् । म॒रुतः॑ । वै॒श्वा॒न॒रम् । हु॒त्वा । मा॒रु॒ताम् । जु॒हो॒ति॒ । रा॒ष्ट्रे । ए॒व । विश᳚म् । अन्विति॑ । ब॒द्ध्ना॒ति॒ । उ॒च्चैः । वै॒श्वा॒न॒रस्य॑ । एति॑ । श्रा॒व॒य॒ति॒ । उ॒पाꣳ॒॒श्वित्यु॑प - अꣳ॒॒शु । मा॒रु॒तान् । जु॒हो॒ति॒ । तस्मा᳚त् । रा॒ष्ट्रम् । विश᳚म् । अतीति॑ । व॒द॒ति॒ । मा॒रु॒ताः । भ॒व॒न्ति॒ । म॒रुतः॑ । वै । दे॒वाना᳚म् । विशः॑ । दे॒व॒वि॒शेनेति॑ देव - वि॒शेन॑ । ए॒व । अ॒स्मै॒ । म॒नु॒ष्य॒वि॒शमिति॑ मनुष्य - वि॒शम् ( ) । अवेति॑ । रु॒न्धे॒ । स॒प्त । भ॒व॒न्ति॒ । स॒प्तग॑णा॒ इति॑ स॒प्त - ग॒णाः॒ । वै । म॒रुतः॑ । ग॒ण॒श इति॑ गण - शः । ए॒व । विश᳚म् । अवेति॑ । रु॒न्धे॒ । ग॒णेन॑ । ग॒णम् । अ॒नु॒द्रुत्येत्य॑नु - द्रुत्य॑ । जु॒हो॒ति॒ । विश᳚म् । ए॒व । अ॒स्मै॒ । अनु॑वर्त्मान॒मित्यनु॑ - व॒र्त्मा॒न॒म् । क॒रो॒ति॒ ॥ \textbf{  36 } \newline
                  \newline
                      (अग्ने॒ प्रेह्य - व॑ - स्म - दुहे॒ तां - प्र॒जाप॑तेः - सा॒क्षान् - म॑नुष्यवि॒श - मेक॑विꣳशतिश्च  \textbf{(A7)} \newline \newline
                                \textbf{ TS 5.4.8.1} \newline
                  वसोः᳚ । धारा᳚म् । जु॒हो॒ति॒ । वसोः᳚ । मे॒ । धारा᳚ । अ॒स॒त् । इति॑ । वै । ए॒षा । हू॒य॒ते॒ । घृ॒तस्य॑ । वै । ए॒न॒म् । ए॒षा । धारा᳚ । अ॒मुष्मिन्न्॑ । लो॒के । पिन्व॑माना । उपेति॑ । ति॒ष्ठ॒ते॒ । आज्ये॑न । जु॒हो॒ति॒ । तेजः॑ । वै । आज्य᳚म् । तेजः॑ । वसोः᳚ । धारा᳚ । तेज॑सा । ए॒व । अ॒स्मै॒ । तेजः॑ । अवेति॑ । रु॒न्धे॒ । अथो॒ इति॑ । कामाः᳚ । वै । वसोः᳚ । धारा᳚ । कामान्॑ । ए॒व । अवेति॑ । रु॒न्धे॒ । यम् । का॒मये॑त । प्रा॒णानिति॑ प्र - अ॒नान् । अ॒स्य॒ । अ॒न्नाद्य॒मित्य॑न्न - अद्य᳚म् । वीति॑ । \textbf{  37} \newline
                  \newline
                                \textbf{ TS 5.4.8.2} \newline
                  छि॒न्द्या॒म् । इति॑ । वि॒ग्राह॒मिति॑ वि - ग्राह᳚म् । तस्य॑ । जु॒हु॒या॒त् । प्रा॒णानिति॑ प्र - अ॒नान् । ए॒व । अ॒स्य॒ । अ॒न्नाद्य॒मित्य॑न्न - अद्य᳚म् । वीति॑ । छि॒न॒त्ति॒ । यम् । का॒मये॑त । प्रा॒णानिति॑ प्र-अ॒नान् । अ॒स्य॒ । अ॒न्नाद्य॒मित्य॑न्न - अद्य᳚म् । समिति॑ । त॒नु॒या॒म् । इति॑ । संत॑ता॒मिति॒ सं - त॒ता॒म् । तस्य॑ । जु॒हु॒या॒त् । प्रा॒णानिति॑ प्र - अ॒नान् । ए॒व । अ॒स्य॒ । अ॒न्नाद्य॒मित्य॑न्न - अद्य᳚म् । समिति॑ । त॒नो॒ति॒ । द्वाद॑श । द्वा॒द॒शानि॑ । जु॒हो॒ति॒ । द्वाद॑श । मासाः᳚ । सं॒ॅव॒थ्स॒र इति॑ सं-व॒थ्स॒रः । सं॒ॅव॒थ्स॒रेणेति॑ सं - व॒थ्स॒रेण॑ । ए॒व । अ॒स्मै॒ । अन्न᳚म् । अवेति॑ । रु॒न्धे॒ । अन्न᳚म् । च॒ । मे॒ । अक्षु॑त् । च॒ । मे॒ । इति॑ । आ॒ह॒ । ए॒तत् । वै । \textbf{  38} \newline
                  \newline
                                \textbf{ TS 5.4.8.3} \newline
                  अन्न॑स्य । रू॒पम् । रू॒पेण॑ । ए॒व । अन्न᳚म् । अवेति॑ । रु॒न्धे॒ । अ॒ग्निः । च॒ । मे॒ । आपः॑ । च॒ । मे॒ । इति॑ । आ॒ह॒ । ए॒षा । वै । अन्न॑स्य । योनिः॑ । सयो॒नीति॒ स - यो॒नि॒ । ए॒व । अन्न᳚म् । अवेति॑ । रु॒न्धे॒ । अ॒द्‌र्धे॒न्द्राणीत्य॑द्‌र्ध - इ॒न्द्राणि॑ । जु॒हो॒ति॒ । दे॒वताः᳚ । ए॒व । अवेति॑ । रु॒न्धे॒ । यत् । सर्वे॑षाम् । अ॒द्‌र्धम् । इन्द्रः॑ । प्रतीति॑ । तस्मा᳚त् । इन्द्रः॑ । दे॒वता॑नाम् । भू॒यि॒ष्ठ॒भाक्त॑म॒ इति॑ भूयिष्ठ॒भाक्-त॒मः॒ । इन्द्र᳚म् । उत्त॑र॒मित्युत् -त॒र॒म् । आ॒ह॒ । इ॒न्द्रि॒यम् । ए॒व । अ॒स्मि॒न्न् । उ॒परि॑ष्टात् । द॒धा॒ति॒ । य॒ज्ञा॒यु॒धानीति॑ यज्ञ्-आ॒यु॒धानि॑ । जु॒हो॒ति॒ । य॒ज्ञ्ः । \textbf{  39} \newline
                  \newline
                                \textbf{ TS 5.4.8.4} \newline
                  वै । य॒ज्ञा॒यु॒धानीति॑ यज्ञ्-आ॒यु॒धानि॑ । य॒ज्ञ्म् । ए॒व । अवेति॑ । रु॒न्धे॒ । अथो॒ इति॑ । ए॒तत् । वै । य॒ज्ञ्स्य॑ । रू॒पम् । रू॒पेण॑ । ए॒व । य॒ज्ञ्म् । अवेति॑ । रु॒न्धे॒ । अ॒व॒भृ॒थ इत्य॑व - भृ॒थः । च॒ । मे॒ । स्व॒गा॒का॒र इति॑ स्वगा - का॒रः । च॒ । मे॒ । इति॑ । आ॒ह॒ । स्व॒गाकृ॑त्या॒ इति॑ स्व॒गा - कृ॒त्यै॒ । अ॒ग्निः । च॒ । मे॒ । घ॒र्मः । च॒ । मे॒ । इति॑ । आ॒ह॒ । ए॒तत् । वै । ब्र॒ह्म॒व॒र्च॒सस्येति॑ ब्रह्म - व॒र्च॒सस्य॑ । रू॒पम् । रू॒पेण॑ । ए॒व । ब्र॒ह्म॒व॒र्च॒समिति॑ ब्रह्म - व॒र्च॒सम् । अवेति॑ । रु॒न्धे॒ । ऋक् । च॒ । मे॒ । साम॑ । च॒ । मे॒ । इति॑ । आ॒ह॒ । \textbf{  40} \newline
                  \newline
                                \textbf{ TS 5.4.8.5} \newline
                  ए॒तत् । वै । छन्द॑साम् । रू॒पम् । रू॒पेण॑ । ए॒व । छन्दाꣳ॑सि । अवेति॑ । रु॒न्धे॒ । गर्भाः᳚ । च॒ । मे॒ । व॒थ्साः । च॒ । मे॒ । इति॑ । आ॒ह॒ । ए॒तत् । वै । प॒शू॒नाम् । रू॒पम् । रू॒पेण॑ । ए॒व । प॒शून् । अवेति॑ । रु॒न्धे॒ । कल्पान्॑ । जु॒हो॒ति॒ । अक्लृ॑प्तस्य । क्लृप्त्यै᳚ । यु॒ग्म॒द॒यु॒जे इति॑ युग्मत् - अ॒यु॒जे । जु॒हो॒ति॒ । मि॒थु॒न॒त्वायेति॑ मिथुन - त्वाय॑ । उ॒त्त॒राव॑ती॒ इत्यु॑त्त॒रा-व॒ती॒ । भ॒व॒तः॒ । अ॒भिक्रा᳚न्त्या॒ इत्य॒भि - क्रा॒न्त्यै॒ । एका᳚ । च॒ । मे॒ । ति॒स्रः । च॒ । मे॒ । इति॑ । आ॒ह॒ । दे॒व॒छ॒न्द॒समिति॑ देव - छ॒न्द॒सम् । वै । एका᳚ । च॒ । ति॒स्रः । च॒ । \textbf{  41} \newline
                  \newline
                                \textbf{ TS 5.4.8.6} \newline
                  म॒नु॒ष्य॒छ॒न्द॒समिति॑ मनुष्य - छ॒न्द॒सम् । चत॑स्रः । च॒ । अ॒ष्टौ । च॒ । दे॒व॒छ॒न्द॒समिति॑ देव - छ॒न्द॒सम् । च॒ । ए॒व । म॒नु॒ष्य॒छ॒न्द॒समिति॑ मनुष्य - छ॒न्द॒सम् । च॒ । अवेति॑ । रु॒न्धे॒ । एति॑ । त्रय॑स्त्रिꣳशत॒ इति॒ त्रयः॑ - त्रिꣳ॒॒श॒तः॒ । जु॒हो॒ति॒ । त्रय॑स्त्रिꣳश॒दिति॒ त्रयः॑ - त्रिꣳ॒॒श॒त् । वै । दे॒वताः᳚ । दे॒वताः᳚ । ए॒व । अवेति॑ । रु॒न्धे॒ । एति॑ । अ॒ष्टाच॑त्वारिꣳशत॒ इत्य॒ष्टा - च॒त्वा॒रिꣳ॒॒श॒तः॒ । जु॒हो॒ति॒ । अ॒ष्टाच॑त्वारिꣳशदक्ष॒रेत्य॒ष्टाच॑त्वारिꣳशत्-अ॒क्ष॒रा॒ । जग॑ती । जाग॑ताः । प॒शवः॑ । जग॑त्या । ए॒व । अ॒स्मै॒ । प॒शून् । अवेति॑ । रु॒न्धे॒ । वाजः॑ । च॒ । प्र॒स॒व इति॑ प्र-स॒वः । च॒ । इति॑ । द्वा॒द॒शम् । जु॒हो॒ति॒ । द्वाद॑श । मासाः᳚ । सं॒ॅव॒थ्स॒र इति॑ सं - व॒थ्स॒रः । सं॒ॅव॒थ्स॒र इति॑ सं - व॒थ्स॒रे । ए॒व । प्रतीति॑ । ति॒ष्ठ॒ति॒ ॥ \textbf{  42} \newline
                  \newline
                      (वि - वै - य॒ज्ञ्ः-साम॑ च म॒ इत्या॑ह - च ति॒स्र - श्चैका॒न्न प॑ञ्चा॒शच्च॑)  \textbf{(A8)} \newline \newline
                                \textbf{ TS 5.4.9.1} \newline
                  अ॒ग्निः । दे॒वेभ्यः॑ । अपेति॑ । अ॒क्रा॒म॒त् । भा॒ग॒धेय॒मिति॑ भाग-धेय᳚म् । इ॒च्छमा॑नः । तम् । दे॒वाः । अ॒ब्रु॒व॒न्न् । उपेति॑ । नः॒ । एति॑ । व॒र्त॒स्व॒ । ह॒व्यम् । नः॒ । व॒ह॒ । इति॑ । सः । अ॒ब्र॒वी॒त् । वर᳚म् । वृ॒णै॒ । मह्य᳚म् । ए॒व । वा॒ज॒प्र॒स॒वीय॒मिति॑ वाज - प्र॒स॒वीय᳚म् । जु॒ह॒व॒न्न् । इति॑ । तस्मा᳚त् । अ॒ग्नये᳚ । वा॒ज॒प्र॒स॒वीय॒मिति॑ वाज - प्र॒स॒वीय᳚म् । जु॒ह॒ति॒ । यत् । वा॒ज॒प्र॒स॒वीय॒मिति॑ वाज - प्र॒स॒वीय᳚म् । जु॒होति॑ । अ॒ग्निम् । ए॒व । तत् । भा॒ग॒धेये॒नेति॑ भाग - धेये॑न । समिति॑ । अ॒द्‌र्ध॒य॒ति॒ । अथो॒ इति॑ । अ॒भि॒षे॒क इत्य॑भि - से॒कः । ए॒व । अ॒स्य॒ । सः । च॒तु॒र्द॒शभि॒रिति॑ चतुर्द॒श - भिः॒ । जु॒हो॒ति॒ । स॒प्त । ग्रा॒म्याः । ओष॑धयः । स॒प्त । \textbf{  43} \newline
                  \newline
                                \textbf{ TS 5.4.9.2} \newline
                  आ॒र॒ण्याः । उ॒भयी॑षाम् । अव॑रुद्ध्या॒ इत्यव॑ - रु॒द्ध्यै॒ । अन्न॑स्यान्न॒स्येत्यन्न॑स्य - अ॒न्न॒स्य॒ । जु॒हो॒ति॒ । अन्न॑स्यान्न॒स्येत्यन्न॑स्य - अ॒न्न॒स्य॒ । अव॑रुद्ध्या॒ इत्यव॑ - रु॒द्ध्यै॒ । औदु॑बंरेण । स्रु॒वेण॑ । जु॒हो॒ति॒ । ऊर्क् । वै । उ॒दु॒बंरः॑ । ऊर्क् । अन्न᳚म् । ऊ॒र्जा । ए॒व । अ॒स्मै॒ । ऊर्ज᳚म् । अन्न᳚म् । अवेति॑ । रु॒न्धे॒ । अ॒ग्निः । वै । दे॒वाना᳚म् । अ॒भिषि॑क्त॒ इत्य॒भि - सि॒क्तः॒ । अ॒ग्नि॒चिदित्य॑ग्नि - चित् । म॒नु॒ष्या॑णाम् । तस्मा᳚त् । अ॒ग्नि॒चिदित्य॑ग्नि - चित् । वर्.ष॑ति । न । धा॒वे॒त् । अव॑रुद्ध॒मित्यव॑ - रु॒द्ध॒म् । हि । अ॒स्य॒ । अन्न᳚म् । अन्न᳚म् । इ॒व॒ । खलु॑ । वै । व॒र्॒.षम् । यत् । धावे᳚त् । अ॒न्नाद्या॒दित्य॑न्न - अद्या᳚त् । धा॒वे॒त् । उ॒पाव॑र्ते॒तेत्य॑प-आव॑र्तेत । अ॒न्नाद्य॒मित्य॑न्न - अद्य᳚म् । ए॒व । अ॒भीति॑ । \textbf{  44} \newline
                  \newline
                                \textbf{ TS 5.4.9.3} \newline
                  उ॒पाव॑र्तत॒ इत्यु॑प - आव॑र्तते । नक्तो॒षासा᳚ । इति॑ । कृ॒ष्णायै᳚ । श्वे॒तव॑थ्साया॒ इति॑ श्वे॒त - व॒थ्सा॒यै॒ । पय॑सा । जु॒हो॒ति॒ । अह्ना᳚ । ए॒व । अ॒स्मै॒ । रात्रि᳚म् । प्रेति॑ । दा॒प॒य॒ति॒ । रात्रि॑या । अहः॑ । अ॒हो॒रा॒त्रे इत्य॑हः - रा॒त्रे । ए॒व । अ॒स्मै॒ । प्रत्ते॒ इति॑ । काम᳚म् । अ॒न्नाद्य॒मित्य॑न्न - अद्य᳚म् । दु॒हा॒ते॒ इति॑ । रा॒ष्ट्र॒भृत॒ इति॑ राष्ट्र - भृतः॑ । जु॒हो॒ति॒ । रा॒ष्ट्रम् । ए॒व । अवेति॑ । रु॒न्धे॒ । ष॒ड्भिरिति॑ षट् - भिः । जु॒हो॒ति॒ । षट् । वै । ऋ॒तवः॑ । ऋ॒तुषु॑ । ए॒व । प्रतीति॑ । ति॒ष्ठ॒ति॒ । भुव॑नस्य । प॒ते॒ । इति॑ । र॒थ॒मु॒ख इति॑ रथ - मु॒खे । पञ्च॑ । आहु॑ती॒रित्या - हु॒तीः॒ । जु॒हो॒ति॒ । वज्रः॑ । वै । रथः॑ । वज्रे॑ण । ए॒व । दिशः॑ । \textbf{  45} \newline
                  \newline
                                \textbf{ TS 5.4.9.4} \newline
                  अ॒भीति॑ । ज॒य॒ति॒ । अ॒ग्नि॒चित॒मित्य॑ग्नि - चित᳚म् । ह॒ । वै । अ॒मुष्मिन्न्॑ । लो॒के । वातः॑ । अ॒भीति॑ । प॒व॒ते॒ । वा॒त॒ना॒मानीति॑ वात - ना॒मानि॑ । जु॒हो॒ति॒ । अ॒भीति॑ । ए॒व । ए॒न॒म् । अ॒मुष्मिन्न्॑ । लो॒के । वातः॑ । प॒व॒ते॒ । त्रीणि॑ । जु॒हो॒ति॒ । त्रयः॑ । इ॒मे । लो॒काः । ए॒भ्यः । ए॒व । लो॒केभ्यः॑ । वात᳚म् । अवेति॑ । रु॒न्धे॒ । स॒मु॒द्रः । अ॒सि॒ । नभ॑स्वान् । इति॑ । आ॒ह॒ । ए॒तत् । वै । वात॑स्य । रू॒पम् । रू॒पेण॑ । ए॒व । वात᳚म् । अवेति॑ । रु॒न्धे॒ । अ॒ञ्ज॒लिना᳚ । जु॒हो॒ति॒ । न । हि । ए॒तेषा᳚म् । अ॒न्यथा᳚ ( ) । आहु॑ति॒रित्या - हु॒तिः॒ । अ॒व॒कल्प॑त॒ इत्य॑व - कल्प॑ते ॥ \textbf{  46} \newline
                  \newline
                      (ओष॑धयः स॒प्ता - भि - दिशो॒ - ऽन्यथा॒ - द्वे च॑)  \textbf{(A9)} \newline \newline
                                \textbf{ TS 5.4.10.1} \newline
                  सु॒व॒र्गायेति॑ सुवः - गाय॑ । वै । लो॒काय॑ । दे॒व॒र॒थ इति॑ देव - र॒थः । यु॒ज्य॒ते॒ । य॒त्रा॒कू॒तायेति॑ यत्र - आ॒कू॒ताय॑ । म॒नु॒ष्य॒र॒थ इति॑ मनुष्य - र॒थः । ए॒षः । खलु॑ । वै । दे॒व॒र॒थ इति॑ देव - र॒थः । यत् । अ॒ग्निः । अ॒ग्निम् । यु॒न॒ज्मि॒ । शव॑सा । घृ॒तेन॑ । इति॑ । आ॒ह॒ । यु॒नक्ति॑ । ए॒व । ए॒न॒म् । सः । ए॒न॒म् । यु॒क्तः । सु॒व॒र्गमिति॑ सुवः-गम् । लो॒कम् । अ॒भीति॑ । व॒ह॒ति॒ । यत् । सर्वा॑भिः । प॒ञ्चभि॒रिति॑ प॒ञ्च - भिः॒ । यु॒ञ्ज्यात् । यु॒क्तः । अ॒स्य॒ । अ॒ग्निः । प्रच्यु॑त॒ इति॒ प्र - च्यु॒तः॒ । स्या॒त् । अप्र॑तिष्ठिता॒ इत्यप्र॑ति - स्थि॒ताः॒ । आहु॑तय॒ इत्या - हु॒त॒यः॒ । स्युः । अप्र॑तिष्ठिता॒ इत्यप्र॑ति - स्थि॒ताः॒ । स्तोमाः᳚ । अप्र॑तिष्ठिता॒नीत्यप्र॑ति - स्थि॒ता॒नि॒ । उ॒क्थानि॑ । ति॒सृभि॒रिति॑ ति॒सृ - भिः॒ । प्रा॒त॒स्स॒व॒न इति॑ प्रातः - स॒व॒ने । अ॒भीति॑ । मृ॒श॒ति॒ । त्रि॒वृदिति॑ त्रि - वृत् । \textbf{  47} \newline
                  \newline
                                \textbf{ TS 5.4.10.2} \newline
                  वै । अ॒ग्निः । यावान्॑ । ए॒व । अ॒ग्निः । तम् । यु॒न॒क्ति॒ । यथा᳚ । अन॑सि । यु॒क्ते । आ॒धी॒यत॒ इत्या᳚ - धी॒यते᳚ । ए॒वम् । ए॒व । तत् । प्रतीति॑ । आहु॑तय॒ इत्या - हु॒त॒यः॒ । तिष्ठ॑न्ति । प्रतीति॑ । स्तोमाः᳚ । प्रतीति॑ । उ॒क्थानि॑ । य॒ज्ञा॒य॒ज्ञिय॑स्य । स्तो॒त्रे । द्वाभ्या᳚म् । अ॒भीति॑ । मृ॒श॒ति॒ । ए॒तावान्॑ । वै । य॒ज्ञ्ः । यावान्॑ । अ॒ग्नि॒ष्टो॒म इत्य॑ग्नि - स्तो॒मः । भू॒मा । तु । वै । अ॒स्य॒ । अतः॑ । ऊ॒द्‌र्ध्वः । क्रि॒य॒ते॒ । यावान्॑ । ए॒व । य॒ज्ञ्ः । तम् । अ॒न्त॒तः । अ॒न्वारो॑ह॒तीत्य॑नु - आरो॑हति । द्वाभ्या᳚म् । प्रति॑ष्ठित्या॒ इति॒ प्रति॑ - स्थि॒त्यै॒ । एक॑या । अप्र॑स्तुत॒मित्यप्र॑-स्तु॒त॒म् । भव॑ति । अथ॑ । \textbf{  48} \newline
                  \newline
                                \textbf{ TS 5.4.10.3} \newline
                  अ॒भीति॑ । मृ॒श॒ति॒ । उपेति॑ । ए॒न॒म् । उत्त॑र॒ इत्युत् - त॒रः॒ । य॒ज्ञ्ः । न॒म॒ति॒ । अथो॒ इति॑ । संत॑त्या॒ इति॒ सं - त॒त्यै॒ । प्रेति॑ । वै । ए॒षः । अ॒स्मात् । लो॒कात् । च्य॒व॒ते॒ । यः । अ॒ग्निम् । चि॒नु॒ते । न । वै । ए॒तस्य॑ । अ॒नि॒ष्ट॒के । आहु॑ति॒रित्या - हु॒तिः॒ । अवेति॑ । क॒ल्प॒ते॒ । याम् । वै । ए॒षः । अ॒नि॒ष्ट॒के । आहु॑ति॒मित्या - हु॒ति॒म् । जु॒होति॑ । स्रव॑ति । वै । सा । ताम् । स्रव॑न्तीम् । य॒ज्ञ्ः । अनु॑ । परेति॑ । भ॒व॒ति॒ । य॒ज्ञ्म् । यज॑मानः । यत् । पु॒न॒श्चि॒तिमिति॑ पुनः - चि॒तिम् । चि॒नु॒ते । आहु॑तीना॒मित्या-हु॒ती॒ना॒म् । प्रति॑ष्ठित्या॒ इति॒ प्रति॑-स्थि॒त्यै॒ । प्रतीति॑ । आहु॑तय॒ इत्या - हु॒त॒यः॒ । तिष्ठ॑न्ति । \textbf{  49} \newline
                  \newline
                                \textbf{ TS 5.4.10.4} \newline
                  न । य॒ज्ञ्ः । प॒रा॒भव॒तीति॑ परा - भव॑ति । न । यज॑मानः । अ॒ष्टौ । उपेति॑ । द॒धा॒ति॒ । अ॒ष्टाक्ष॒रेत्य॒ष्टा - अ॒क्ष॒रा॒ । गा॒य॒त्री । गा॒य॒त्रेण॑ । ए॒व । ए॒न॒म् । छन्द॑सा । चि॒नु॒ते॒ । यत् । एका॑दश । त्रैष्टु॑भेन । यत् । द्वाद॑श । जाग॑तेन । छन्दो॑भि॒रिति॒ छन्दः॑ - भिः॒ । ए॒व । ए॒न॒म् । चि॒नु॒ते॒ । न॒पा॒त्कः । वै । नाम॑ । ए॒षः । अ॒ग्निः । यत् । पु॒न॒श्चि॒तिरिति॑ पुनः - चि॒तिः । यः । ए॒वम् । वि॒द्वान् । पु॒न॒श्चि॒तिमिति॑ पुनः - चि॒तिम् । चि॒नु॒ते । एति॑ । तृ॒तीया᳚त् । पुरु॑षात् । अन्न᳚म् । अ॒त्ति॒ । यथा᳚ । वै । पु॒न॒रा॒धेय॒ इति॑ पुनः - आ॒धेयः॑ । ए॒वम् । पु॒न॒श्चि॒तिरिति॑ पुनः - चि॒तिः । यः । अ॒ग्न्या॒धेये॒नेत्य॑ग्नि-आ॒धेये॑न । न । \textbf{  50} \newline
                  \newline
                                \textbf{ TS 5.4.10.5} \newline
                  ऋ॒ध्नोति॑ । सः । पु॒न॒रा॒धेय॒मिति॑ पुनः-आ॒धेय᳚म् । एति॑ । ध॒त्ते॒ । यः । अ॒ग्निम् । चि॒त्वा । न । ऋ॒द्ध्नोति॑ । सः । पु॒न॒श्चि॒तिमिति॑ पुनः - चि॒तिम् । चि॒नु॒ते॒ । यत् । पु॒न॒श्चि॒तिमिति॑ पुनः - चि॒तिम् । चि॒नु॒ते । ऋद्ध्यै᳚ । अथो॒ इति॑ । खलु॑ । आ॒हुः॒ । न । चे॒त॒व्या᳚ । इति॑ । रु॒द्रः । वै । ए॒षः । यत् । अ॒ग्निः । यथा᳚ । व्या॒घ्रम् । सु॒प्तम् । बो॒धय॑ति । ता॒दृक् । ए॒व । तत् । अथो॒ इति॑ । खलु॑ । आ॒हुः॒ । चे॒त॒व्या᳚ । इति॑ । यथा᳚ । वसी॑याꣳसम् । भा॒ग॒धेये॒नेति॑ भाग - धेये॑न । बो॒धय॑ति । ता॒दृक् । ए॒व । तत् । मनुः॑ । अ॒ग्निम् । अ॒चि॒नु॒त॒ ( ) । तेन॑ । न । आ॒द्‌र्ध्नो॒त् । सः । ए॒ताम् । पु॒न॒श्चि॒तिमिति॑ पुनः - चि॒तिम् । अ॒प॒श्य॒त् । ताम् । अ॒चि॒नु॒त॒ । तया᳚ । वै । सः । आ॒द्‌र्ध्नो॒त् । यत् । पु॒न॒श्चि॒तिमिति॑ पुनः - चि॒तिम् । चि॒नु॒ते । ऋद्ध्यै᳚ ॥ \textbf{  51} \newline
                  \newline
                      (त्रि॒वृ-दथ॒-तिष्ठ॑-न्त्यग्न्या॒धेये॑न॒ ना-चि॑नुत-स॒प्तद॑श- च)  \textbf{(A10)} \newline \newline
                                \textbf{ TS 5.4.11.1} \newline
                  छ॒न्द॒श्चित॒मिति॑ छन्दः - चित᳚म् । चि॒न्वी॒त॒ । प॒शुका॑म॒ इति॑ प॒शु - का॒मः॒ । प॒शवः॑ । वै । छन्दाꣳ॑सि । प॒शु॒मानिति॑ पशु-मान् । ए॒व । भ॒व॒ति॒ । श्ये॒न॒चित॒मिति॑ श्येन - चित᳚म् । चि॒न्वी॒त॒ । सु॒व॒र्गका॑म॒ इति॑ सुव॒र्ग - का॒मः॒ । श्ये॒नः । वै । वय॑साम् । पति॑ष्ठः । श्ये॒नः । ए॒व । भू॒त्वा । सु॒व॒र्गमिति॑ सुवः - गम् । लो॒कम् । प॒त॒ति॒ । क॒ङ्क॒चित॒मिति॑ कङ्क - चित᳚म् । चि॒न्वी॒त॒ । यः । का॒मये॑त । शी॒र्.॒ष॒ण्वानिति॑ शीर्.षण् - वान् । अ॒मुष्मिन्न्॑ । लो॒के । स्या॒म् । इति॑ । शी॒र्.॒ष॒ण्वानिति॑ शीर्.षण् - वान् । ए॒व । अ॒मुष्मिन्न्॑ । लो॒के । भ॒व॒ति॒ । अ॒ल॒ज॒चित॒मित्य॑लज - चित᳚म् । चि॒न्वी॒त॒ । चतु॑स्सीत॒मिति॒ चतुः॑ - सी॒त॒म् । प्र॒ति॒ष्ठाका॑म॒ इति॑ प्रति॒ष्ठा-का॒मः॒ । चत॑स्रः । दिशः॑ । दि॒क्षु । ए॒व । प्रतीति॑ । ति॒ष्ठ॒ति॒ । प्र॒उ॒ग॒चित॒मिति॑ प्र‌उग - चित᳚म् । चि॒न्वी॒त॒ । भ्रातृ॑व्यवा॒निति॒ भ्रातृ॑व्य - वान् । प्रेति॑ । \textbf{  52} \newline
                  \newline
                                \textbf{ TS 5.4.11.2} \newline
                  ए॒व । भ्रातृ॑व्यान् । नु॒द॒ते॒ । उ॒भ॒यतः॑ प्र‌उग॒मित्यु॑भ॒यतः॑ - प्र॒उ॒ग॒म् । चि॒न्वी॒त॒ । यः । का॒मये॑त । प्रेति॑ । जा॒तान् । भ्रातृ॑व्यान् । नु॒देय॑ । प्रतीति॑ । ज॒नि॒ष्यमा॑णान् । इति॑ । प्रेति॑ । ए॒व । जा॒तान् । भ्रातृ॑व्यान् । नु॒दते᳚ । प्रतीति॑ । ज॒नि॒ष्यमा॑णान् । र॒थ॒च॒क्र॒चित॒मिति॑ रथचक्र - चित᳚म् । चि॒न्वी॒त॒ । भ्रातृ॑व्यवा॒निति॒ भ्रातृ॑व्य - वा॒न् । वज्रः॑ । वै । रथः॑ । वज्र᳚म् । ए॒व । भ्रातृ॑व्येभ्यः । प्रेति॑ । ह॒र॒ति॒ । द्रो॒ण॒चित॒मिति॑ द्रोण - चित᳚म् । चि॒न्वी॒त॒ । अन्न॑काम॒ इत्यन्न॑-का॒मः॒ । द्रोणे᳚ । वै । अन्न᳚म् । भ्रि॒य॒ते॒ । सयो॒नीति॒ स-यो॒नि॒ । ए॒व । अन्न᳚म् । अवेति॑ । रु॒न्धे॒ । स॒मू॒ह्य॑मिति॑ सं - ऊ॒ह्य᳚म् । चि॒न्वी॒त॒ । प॒शुका॑म॒ इति॑ प॒शु - का॒मः॒ । प॒शु॒मानिति॑ पशु -मान् । ए॒व । भ॒व॒ति॒ । \textbf{  53} \newline
                  \newline
                                \textbf{ TS 5.4.11.3} \newline
                  प॒रि॒चा॒य्य॑मिति॑ परि - चा॒य्य᳚म् । चि॒न्वी॒त॒ । ग्राम॑काम॒ इति॒ ग्राम॑ - का॒मः॒ । ग्रा॒मी । ए॒व । भ॒व॒ति॒ । श्म॒शा॒न॒चित॒मिति॑ श्मशान - चित᳚म् । चि॒न्वी॒त॒ । यः । का॒मये॑त । पि॒तृ॒लो॒क इति॑ पितृ - लो॒के । ऋ॒द्ध्नु॒या॒म् । इति॑ । पि॒तृ॒लो॒क इति॑ पितृ - लो॒के । ए॒व । ऋ॒द्ध्नो॒ति॒ । वि॒श्वा॒मि॒त्र॒ज॒म॒द॒ग्नी इति॑ विश्वामित्र - ज॒म॒द॒ग्नी । वसि॑ष्ठेन । अ॒स्प॒द्‌र्धे॒ता॒म् । सः । ए॒ताः । ज॒मद॑ग्निः । वि॒ह॒व्या॑ इति॑ वि - ह॒व्याः᳚ । अ॒प॒श्य॒त् । ताः । उपेति॑ । अ॒ध॒त्त॒ । ताभिः॑ । वै । सः । वसि॑ष्ठस्य । इ॒न्द्रि॒यम् । वी॒र्य᳚म् । अ॒वृ॒ङ्क्त॒ । यत् । वि॒ह॒व्या॑ इति॑ वि - ह॒व्याः᳚ । उ॒प॒दधा॒तीत्यु॑प - दधा॑ति । इ॒न्द्रि॒यम् । ए॒व । ताभिः॑ । वी॒र्य᳚म् । यज॑मानः । भ्रातृ॑व्यस्य । वृ॒ङ्क्ते॒ । होतुः॑ । धिष्णि॑ये । उपेति॑ । द॒धा॒ति॒ । य॒ज॒मा॒ना॒य॒त॒नमिति॑ यजमान - आ॒य॒त॒नम् । वै । \textbf{  54} \newline
                  \newline
                                \textbf{ TS 5.4.11.4} \newline
                  होता᳚ । स्वे । ए॒व । अ॒स्मै॒ । आ॒यत॑न॒ इत्या᳚ - यत॑ने । इ॒न्द्रि॒यम् । वी॒र्य᳚म् । अवेति॑ । रु॒न्धे॒ । द्वाद॑श । उपेति॑ । द॒धा॒ति॒ । द्वाद॑शाक्ष॒रेति॒ द्वाद॑श - अ॒क्ष॒रा॒ । जग॑ती । जाग॑ताः । प॒शवः॑ । जग॑त्या । ए॒व । अ॒स्मै॒ । प॒शून् । अवेति॑ । रु॒न्धे॒ । अ॒ष्टाव॑ष्टा॒वित्य॒ष्टौ-अ॒ष्टौ॒ । अ॒न्येषु॑ । धिष्णि॑येषु । उपेति॑ । द॒धा॒ति॒ । अ॒ष्टाश॑फा॒ इत्य॒ष्टा - श॒फाः॒ । प॒शवः॑ । प॒शून् । ए॒व । अवेति॑ । रु॒न्धे॒ । षट् । मा॒र्जा॒लीये᳚ । षट् । वै । ऋ॒तवः॑ । ऋ॒तवः॑ । खलु॑ । वै । दे॒वाः । पि॒तरः॑ । ऋ॒तून् । ए॒व । दे॒वान् । पि॒तॄन् । प्री॒णा॒ति॒ ॥ \textbf{  55} \newline
                  \newline
                      (प्र - भ॑वति - यजमानायत॒नं ॅवा - अ॒ष्टाच॑त्वारिꣳशच्च)  \textbf{(A11)} \newline \newline
                                \textbf{ TS 5.4.12.1} \newline
                  पव॑स्व । वाज॑सातय॒ इति॒ वाज॑ - सा॒त॒ये॒ । इति॑ । अ॒नु॒ष्टुगित्य॑नु - स्तुक् । प्र॒ति॒पदिति॑ प्रति - पत् । भ॒व॒ति॒ । ति॒स्रः । अ॒नु॒ष्टुभ॒ इत्य॑नु - स्तुभः॑ । चत॑स्रः । गा॒य॒त्रियः॑ । यत् । ति॒स्रः । अ॒नु॒ष्टुभ॒ इत्य॑नु - स्तुभः॑ । तस्मा᳚त् । अश्वः॑ । त्रि॒भिरिति॑ त्रि-भिः । तिष्ठन्न्॑ । ति॒ष्ठ॒ति॒ । यत् । चत॑स्रः । गा॒य॒त्रियः॑ । तस्मा᳚त् । सर्वान्॑ । च॒तुरः॑ । प॒दः । प्र॒ति॒दध॒दिति॑ प्रति - दध॑त् । पला॑यते । प॒र॒मा । वै । ए॒षा । छन्द॑साम् । यत् । अ॒नु॒ष्टुगित्य॑नु - स्तुक् । प॒र॒मः । च॒तु॒ष्टो॒म इति॑ चतुः - स्तो॒मः । स्तोमा॑नाम् । प॒र॒मः । त्रि॒रा॒त्र इति॑ त्रि-रा॒त्रः । य॒ज्ञाना᳚म् । प॒र॒मः । अश्वः॑ । प॒शू॒नाम् । प॒र॒मेण॑ । ए॒व । ए॒न॒म् । प॒र॒मता᳚म् । ग॒म॒य॒ति॒ । ए॒क॒विꣳ॒॒शमित्ये॑क - विꣳ॒॒शम् । अहः॑ । भ॒व॒ति॒ । \textbf{  56} \newline
                  \newline
                                \textbf{ TS 5.4.12.2} \newline
                  यस्मिन्न्॑ । अश्वः॑ । आ॒ल॒भ्यत॒ इत्या᳚ - ल॒भ्यते᳚ । द्वाद॑श । मासाः᳚ । पञ्च॑ । ऋ॒तवः॑ । त्रयः॑ । इ॒मे । लो॒काः । अ॒सौ । आ॒दि॒त्यः । ए॒क॒विꣳ॒॒श इत्ये॑क - विꣳ॒॒शः । ए॒षः । प्र॒जाप॑ति॒रिति॑ प्र॒जा-प॒तिः॒ । प्रा॒जा॒प॒त्य इति॑ प्राजा - प॒त्यः । अश्वः॑ । तम् । ए॒व । सा॒क्षादिति॑ स - अ॒क्षात् । ऋ॒द्ध्नो॒ति॒ । शक्व॑रयः । पृ॒ष्ठम् । भ॒व॒न्ति॒ । अ॒न्यद॑न्य॒दित्य॒न्यत् - अ॒न्य॒त् । छन्दः॑ । अ॒न्ये᳚ऽन्य॒ इत्य॒न्ये - अ॒न्ये॒ । वै । ए॒ते । प॒शवः॑ । एति॑ । ल॒भ्य॒न्ते॒ । उ॒त । इ॒व॒ । ग्रा॒म्याः । उ॒त । इ॒व॒ । आ॒र॒ण्याः । यत् । शक्व॑रयः । पृ॒ष्ठम् । भव॑न्ति । अश्व॑स्य । स॒र्व॒त्वायेति॑ सर्व - त्वाय॑ । पा॒र्थु॒र॒श्ममिति॑ पार्थु - र॒श्मम् । ब्र॒ह्म॒सा॒ममिति॑ ब्रह्म - सा॒मम् । भ॒व॒ति॒ । र॒श्मिना᳚ । वै । अश्वः॑ । \textbf{  57} \newline
                  \newline
                                \textbf{ TS 5.4.12.3} \newline
                  य॒तः । ई॒श्व॒रः । वै । अश्वः॑ । अय॑तः । अप्र॑तिष्ठित॒ इत्यप्र॑ति-स्थि॒तः॒ । परा᳚म् । प॒रा॒वत॒मिति॑ परा - वत᳚म् । गन्तोः᳚ । यत् । पा॒र्थु॒र॒श्ममिति॑ पार्थु - र॒श्मम् । ब्र॒ह्म॒सा॒ममिति॑ ब्रह्म - सा॒मम् । भव॑ति । अश्व॑स्य । यत्यै᳚ । धृत्यै᳚ । संकृ॒तीति॒ सं - कृ॒ति॒ । अ॒च्छा॒वा॒क॒सा॒ममित्य॑च्छावाक - सा॒मम् । भ॒व॒ति॒ । उ॒थ्स॒न्न॒य॒ज्ञ् इत्यु॑थ्सन्न - य॒ज्ञ्ः । वै । ए॒षः । यत् । अ॒श्व॒मे॒ध इत्य॑श्व - मे॒धः । कः । तत् । वे॒द॒ । इति॑ । आ॒हुः॒ । यदि॑ । सर्वः॑ । वा॒ । क्रि॒यते᳚ । न । वा॒ । सर्वः॑ । इति॑ । यत् । संकृ॒तीति॒ सं - कृ॒ति॒ । अ॒च्छा॒वा॒क॒सा॒ममित्य॑च्छावाक - स॒मम् । भव॑ति । अश्व॑स्य । स॒र्व॒त्वायेति॑ सर्व - त्वाय॑ । पर्या᳚प्त्या॒ इति॒ परि॑ - आ॒प्त्यै॒ । अन॑न्तराया॒येत्यन॑न्तः - आ॒या॒य॒ । सर्व॑स्तोम॒ इति॒ सर्व॑ - स्तो॒मः॒ । अ॒ति॒रा॒त्र इत्य॑ति - रा॒त्रः । उ॒त्त॒ममित्यु॑त् - त॒मम् । अहः॑ । भ॒व॒ति॒ ( ) । सर्व॑स्य । आप्त्यै᳚ । सर्व॑स्य । जित्यै᳚ । सर्व᳚म् । ए॒व । तेन॑ । आ॒प्नो॒ति॒ । सर्व᳚म् । ज॒य॒ति॒ ॥ \textbf{  58} \newline
                  \newline
                      (अह॑र्भवति॒ - वा अश्वो - ऽह॑र्भवति॒ - दश॑ च)  \textbf{(A12)} \newline \newline
\textbf{praSna korvai with starting padams of 1 to 12 anuvAkams :-} \newline
(दे॒वा॒सु॒रास्तेन - र्त॒व्या॑ - रु॒द्रो - ऽश्म॑ - न्नृ॒षदे॒ व - डुदे॑नं॒ - प्राची॒मिति॒ - वसो॒॒र्धारा॑ - म॒ग्निर्दे॒वेभ्यः॑ - सुव॒र्गाय॑ यत्राकू॒ताय॑ - छन्द॒श्चितं॒ - पव॑स्व॒ - द्वाद॑श ) \newline

\textbf{korvai with starting padams of1, 11, 21 series of pa~jcAtis :-} \newline
(दे॒वा॒सु॒रा - अ॒जायां॒ - ॅवै ग्रु॑मु॒ष्टिः - प्र॑थ॒मो दे॑वय॒तामे॒ - तद्वै छन्द॑सा - मृ॒ध्नो - त्य॒ष्टौ प॑ञ्चा॒शत्) \newline

\textbf{first and last padam of fourth praSnam of 5th kANDam} \newline
(दे॒वा॒सु॒राः - सर्वं॑ जयति) \newline 


॥ हरिः॑ ॐ ॥॥ कृष्ण यजुर्वेदीय तैत्तिरीय संहितायां पञ्चमकाण्डे चतुर्थः प्रश्नः समाप्तः ॥
------------------------------------ \newline
\pagebreak
\pagebreak
        


\end{document}
