\documentclass[17pt]{extarticle}
\usepackage{babel}
\usepackage{fontspec}
\usepackage{polyglossia}
\usepackage{extsizes}



\setmainlanguage{sanskrit}
\setotherlanguages{english} %% or other languages
\setlength{\parindent}{0pt}
\pagestyle{myheadings}
\newfontfamily\devanagarifont[Script=Devanagari]{AdishilaVedic}


\newcommand{\VAR}[1]{}
\newcommand{\BLOCK}[1]{}




\begin{document}
\begin{titlepage}
    \begin{center}
 
\begin{sanskrit}
    { \Large
    ॐ नमः परमात्मने, श्री महागणपतये नमः, 
श्री गुरुभ्यो नमः । ह॒रिः॒ ॐ ॥ 
    }
    \\
    \vspace{2.5cm}
    \mbox{ \Huge
    1.5     प्रथमकाण्डे पञ्चमः प्रश्नः - (पुनराधानं)   }
\end{sanskrit}
\end{center}

\end{titlepage}
\tableofcontents

ॐ नमः परमात्मने, श्री महागणपतये नमः, 
श्री गुरुभ्यो नमः । ह॒रिः॒ ॐ ॥ \newline
1.5     प्रथमकाण्डे पञ्चमः प्रश्नः - (पुनराधानं) \newline

\addcontentsline{toc}{section}{ 1.5     प्रथमकाण्डे पञ्चमः प्रश्नः - (पुनराधानं)}
\markright{ 1.5     प्रथमकाण्डे पञ्चमः प्रश्नः - (पुनराधानं) \hfill https://www.vedavms.in \hfill}
\section*{ 1.5     प्रथमकाण्डे पञ्चमः प्रश्नः - (पुनराधानं) }
                                \textbf{ TS 1.5.1.1} \newline
                  दे॒वा॒सु॒रा इति॑ देव - अ॒सु॒राः । संॅय॑त्ता॒ इति॑ सं - य॒त्ताः॒ । आ॒स॒न्न् । ते । दे॒वाः । वि॒ज॒यमिति॑ वि - ज॒यम् । उ॒प॒यन्त॒ इत्यु॑प - यन्तः॑ । अ॒ग्नौ । वा॒मम् । वसु॑ । सम् । नीति॑ । अ॒द॒ध॒त॒ । इ॒दम् । उ॒ । नः॒ । भ॒वि॒ष्य॒ति॒ । यदि॑ । नः । जे॒ष्यन्ति॑ । इति॑ । तत् । अ॒ग्निः । नीति॑ । अ॒का॒म॒य॒त॒ । तेन॑ । अपेति॑ । अ॒क्रा॒म॒त् । तत् । दे॒वाः । वि॒जित्येति॑ वि - जित्य॑ । अ॒व॒रुरु॑थ्समाना॒ इत्य॑व - रुरु॑थ्समानाः । अन्विति॑ । आ॒य॒न्न् । तत् । अ॒स्य॒ । सह॑सा । एति॑ । अ॒दि॒थ्स॒न्त॒ । सः । अ॒रो॒दी॒त् । यत् । अरो॑दीत् । तत् । रु॒द्रस्य॑ । रु॒द्र॒त्वमिति॑ रुद्र - त्वम् । यत् । अश्रु॑ । अशी॑यत । तत् । \textbf{  1} \newline
                  \newline
                                \textbf{ TS 1.5.1.2} \newline
                  र॒ज॒तम् । हिर॑ण्यम् । अ॒भ॒व॒त् । तस्मा᳚त् । र॒ज॒तम् । हिर॑ण्यम् । अ॒द॒क्षि॒ण्यम् । अ॒श्रु॒जमित्य॑श्रु - जम् । हि । यः । ब॒र्॒.हिषि॑ । ददा॑ति । पु॒रा । अ॒स्य॒ । सं॒ॅव॒थ्स॒रादिति॑ सं - व॒थ्स॒रात् । गृ॒हे । रु॒द॒न्ति॒ । तस्मा᳚त् । ब॒र्॒.हिषि॑ । न । देय᳚म् । सः । अ॒ग्निः । अ॒ब्र॒वी॒त् । भा॒गी । अ॒सा॒नि॒ । अथ॑ । वः॒ । इ॒दम् । इति॑ । पु॒न॒रा॒धेय॒मिति॑ पुनः - आ॒धेय᳚म् । ते॒ । केव॑लम् । इति॑ । अ॒ब्रु॒व॒न्न् । ऋ॒द्ध्नव॑त् । खलु॑ । सः । इति॑ । अ॒ब्र॒वी॒त् । यः । म॒द्दे॒व॒त्य॑मिति॑ मत् - दे॒व॒त्य᳚म् । अ॒ग्निम् । आ॒दधा॑ता॒ इत्या᳚ - दधा॑तै । इति॑ । तम् । पू॒षा । एति॑ । अ॒ध॒त्त॒ । तेन॑ । \textbf{  2} \newline
                  \newline
                                \textbf{ TS 1.5.1.3} \newline
                  पू॒षा । आ॒र्द्ध्नो॒त् । तस्मा᳚त् । पौ॒ष्णाः । प॒शवः॑ । उ॒च्य॒न्ते॒ । तम् । त्वष्टा᳚ । एति॑ । अ॒ध॒त्त॒ । तेन॑ । त्वष्टा᳚ । आ॒र्द्ध्नो॒त् । तस्मा᳚त् । त्वा॒ष्ट्राः । प॒शवः॑ । उ॒च्य॒न्ते॒ । तम् । मनुः॑ । एति॑ । अ॒ध॒त्त॒ । तेन॑ । मनुः॑ । आ॒र्द्ध्नो॒त् । तस्मा᳚त् । मा॒न॒व्यः॑ । प्र॒जा इति॑ प्र - जाः । उ॒च्य॒न्ते॒ । तम् । धा॒ता । एति॑ । अ॒ध॒त्त॒ । तेन॑ । धा॒ता । आ॒र्द्ध्नो॒त् । सं॒ॅव॒थ्स॒र इति॑ सं - व॒थ्स॒रः । वै । धा॒ता । तस्मा᳚त् । सं॒ॅव॒थ्स॒रमिति॑ सं-व॒थ्स॒रम् । प्र॒जा इति॑ प्र-जाः । प॒शवः॑ । अनु॑ । प्रेति॑ । जा॒य॒न्ते॒ । यः । ए॒वम् । पु॒न॒रा॒धेय॒स्येति॑ पुनः - आ॒धेय॑स्य । ऋद्धि᳚म् । वेद॑ । \textbf{  3 } \newline
                  \newline
                                \textbf{ TS 1.5.1.4} \newline
                  ऋ॒द्ध्नोति॑ । ए॒व । यः । अ॒स्य॒ । ए॒वम् । ब॒न्धुता᳚म् । वेद॑ । बन्धु॑मा॒निति॒ बन्धु॑ - मा॒न् । भ॒व॒ति॒ । भा॒ग॒धेय॒मिति॑ भाग -धेय᳚म् । वै । अ॒ग्निः । आहि॑त॒ इत्या - हि॒तः॒ । इ॒च्छमा॑नः । प्र॒जामिति॑ प्र-जाम् । प॒शून् । यज॑मानस्य । उपेति॑ । दो॒द्रा॒व॒ । उ॒द्वास्येत्यु॑त् - वास्य॑ । पुनः॑ । एति॑ । द॒धी॒त॒ । भा॒ग॒धेये॒नेति॑ भाग - धेये॑न । ए॒व । ए॒न॒म् । समिति॑ । अ॒र्ध॒य॒ति॒ । अथो॒ इति॑ । शान्तिः॑ । ए॒व । अ॒स्य॒ । ए॒षा । पुन॑र्वस्वो॒रिति॒ पुनः॑ - व॒स्वोः॒ । एति॑ । द॒धी॒त॒ । ए॒तत् । वै । पु॒न॒रा॒धेय॒स्येति॑ पुनः - आ॒धेय॑स्य । नक्ष॑त्रम् । यत् । पुन॑र्वसू॒ इति॒ पुनः॑ - व॒सू॒ । स्वाया᳚म् । ए॒व । ए॒न॒म् । दे॒वता॑याम् । आ॒धायेत्या᳚ - धाय॑ । ब्र॒ह्म॒व॒र्च॒सीति॑ ब्रह्म - व॒र्च॒सी । भ॒व॒ति॒ । द॒र्भैः ( ) । एति॑ । द॒धा॒ति॒ । अया॑तयामत्वा॒येत्यया॑तयाम - त्वा॒य॒ । द॒र्भैः । एति॑ । द॒धा॒ति॒ । अ॒द्भ्य इत्य॑त् - भ्यः । ए॒व । ए॒न॒म् । ओष॑धीभ्य॒ इत्योष॑धि-भ्यः॒ । अ॒व॒रुद्ध्येत्य॑व - रुद्ध्य॑ । एति॑ । ध॒त्ते॒ । पञ्च॑कपाल॒ इति॒ पञ्च॑ - क॒पा॒लः॒ । पु॒रो॒डाशः॑ । भ॒व॒ति॒ । पञ्च॑ । वै । ऋ॒तवः॑ । ऋ॒तुभ्य॒ इत्यृ॒तु - भ्यः॒ । ए॒व । ए॒न॒म् । अ॒व॒रुद्ध्येत्य॑व - रुद्ध्य॑ । एति॑ । ध॒त्ते॒ ॥ \textbf{  4} \newline
                  \newline
                      (अशी॑यत॒ तत्- तेन॒-वेद॑- द॒र्भैः पञ्च॑विꣳशतिश्च)  \textbf{(A1)} \newline \newline
                                \textbf{ TS 1.5.2.1} \newline
                  परेति॑ । वै । ए॒षः । य॒ज्ञ्म् । प॒शून् । व॒प॒ति॒ । यः । अ॒ग्निम् । उ॒द्वा॒सय॑त॒ इत्यु॑त् - वा॒सय॑ते । पञ्च॑कपाल॒ इति॒ पञ्च॑ - क॒पा॒लः॒ । पु॒रो॒डाशः॑ । भ॒व॒ति॒ । पाङ्क्तः॑ । य॒ज्ञ्ः । पाङ्क्ताः᳚ । प॒शवः॑ । य॒ज्ञ्म् । ए॒व । प॒शून् । अवेति॑ । रु॒न्धे॒ । वी॒र॒हेति॑ वीर - हा । वै । ए॒षः । दे॒वाना᳚म् । यः । अ॒ग्निम् । उ॒द्वा॒सय॑त॒ इत्यु॑त् - वा॒सय॑ते । न । वै । ए॒तस्य॑ । ब्रा॒ह्म॒णाः । ऋ॒ता॒यव॒ इत्यृ॑त-यवः॑ । पु॒रा । अन्न᳚म् । अ॒क्ष॒न्न् । प॒ङ्क्त्यः॑ । या॒ज्या॒नु॒वा॒क्या॑ इति॑ याज्या - अ॒नु॒वा॒क्याः᳚ । भ॒व॒न्ति॒ । पाङ्क्तः॑ । य॒ज्ञ्ः । पाङ्क्तः॑ । पुरु॑षः । दे॒वान् । ए॒व । वी॒रम् । नि॒र॒व॒दायेति॑ निः - अ॒व॒दाय॑ । अ॒ग्निम् । पुनः॑ । एति॑ । \textbf{  5} \newline
                  \newline
                                \textbf{ TS 1.5.2.2} \newline
                  ध॒त्ते॒ । श॒ताक्ष॑रा॒ इति॑ श॒त - अ॒क्ष॒राः॒ । भ॒व॒न्ति॒ । श॒तायु॒रिति॑ श॒त - आ॒युः॒ । पुरु॑षः । श॒तेन्द्रि॑य॒ इति॑ श॒त - इ॒न्द्रि॒यः॒ । आयु॑षि । ए॒व । इ॒न्द्रि॒ये । प्रतीति॑ । ति॒ष्ठ॒ति॒ । यत् । वै । अ॒ग्निः । आहि॑त॒ इत्या - हि॒तः॒ । न । ऋ॒द्ध्यते᳚ । ज्यायः॑ । भा॒ग॒धेय॒मिति॑ भाग - धेय᳚म् । नि॒का॒मय॑मान॒ इति॑ नि - का॒मय॑मानः । यत् । आ॒ग्ने॒यम् । सर्व᳚म् । भव॑ति । सा । ए॒व । अ॒स्य॒ । ऋद्धिः॑ । समिति॑ । वै । ए॒तस्य॑ । गृ॒हे । वाक् । सृ॒ज्य॒ते॒ । यः । अ॒ग्निम् । उ॒द्वा॒सय॑त॒ इत्यु॑त् - वा॒सय॑ते । सः । वाच᳚म् । सꣳसृ॑ष्टा॒मिति॒ सं - सृ॒ष्टा॒म् । यज॑मानः । ई॒श्व॒रः । अन्विति॑ । परा॑भवितो॒रिति॒ परा᳚ - भ॒वि॒तोः॒ । विभ॑क्तय॒ इति॒ वि - भ॒क्त॒यः॒ । भ॒व॒न्ति॒ । वा॒चः । विधृ॑त्या॒ इति॒ वि - धृ॒त्यै॒ । यज॑मानस्य । अप॑राभावा॒येत्यप॑रा - भा॒वा॒य॒ । \textbf{  6} \newline
                  \newline
                                \textbf{ TS 1.5.2.3} \newline
                  विभ॑क्ति॒मिति॒ वि - भ॒क्ति॒म् । क॒रो॒ति॒ । ब्रह्म॑ । ए॒व । तत् । अ॒कः॒ । उ॒पाꣳ॒॒श्वित्यु॑प-अꣳ॒॒शु । य॒ज॒ति॒ । यथा᳚ । वा॒मम् । वसु॑ । वि॒वि॒दा॒नः । गूह॑ति । ता॒दृक् । ए॒व । तत् । अ॒ग्निम् । प्रतीति॑ । स्वि॒ष्ट॒कृत॒मिति॑ स्विष्ट- कृत᳚म् । निरिति॑ । आ॒ह॒ । यथा᳚ । वा॒मम् । वसु॑ । वि॒वि॒दा॒नः । प्र॒का॒शमिति॑ प्र - का॒शम् । जिग॑मिषति । ता॒दृक् । ए॒व । तत् । विभ॑क्ति॒मिति॒ वि - भ॒क्ति॒म् । उ॒क्त्वा । प्र॒या॒जेनेति॑ प्र - या॒जेन॑ । वष॑ट् । क॒रो॒ति॒ । आ॒यत॑ना॒दित्या᳚ - यत॑नात् । ए॒व । न । ए॒ति॒ । यज॑मानः । वै । पु॒रो॒डाशः॑ । प॒शवः॑ । ए॒ते इति॑ । आहु॑ती॒ इत्या - हु॒ती॒ । यत् । अ॒भितः॑ । पु॒रो॒डाश᳚म् । ए॒ते इति॑ । आहु॑ती॒ इत्या - हु॒ती॒ । \textbf{  7} \newline
                  \newline
                                \textbf{ TS 1.5.2.4} \newline
                  जु॒होति॑ । यज॑मानम् । ए॒व । उ॒भ॒यतः॑ । प॒शुभि॒रिति॑ प॒शु - भिः॒ । परीति॑ । गृ॒ह्णा॒ति॒ । कृ॒तय॑जु॒रिति॑ कृ॒त - य॒जुः॒ । संभृ॑तसंभार॒ इति॒ संभृ॑त - स॒भां॒रः॒ । इति॑ । आ॒हुः॒ । न । स॒भृंत्या॒ इति॑ सं - भृत्याः᳚ । सं॒भा॒रा इति॑ सं-भा॒राः । न । यजुः॑ । क॒र्त॒व्य᳚म् । इति॑ । अथा॒ इति॑ । खलु॑ । स॒भृंत्या॒ इति॑ सं - भृत्याः᳚ । ए॒व । स॒भां॒रा इति॑ सं-भा॒राः । क॒र्त॒व्य᳚म् । यजुः॑ । य॒ज्ञ्स्य॑ । समृ॑द्ध्या॒ इति॒ सं - ऋ॒द्ध्यै॒ । पु॒न॒र्नि॒ष्कृ॒त इति ॑पुनः - नि॒ष्कृ॒तः । रथः॑ । दक्षि॑णा । पु॒न॒रु॒थ्स्यू॒तमिति॑ पुनः - उ॒थ्स्यू॒तम् । वासः॑ । पु॒न॒रु॒थ्सृ॒ष्ट इति॑ पुनः - उ॒थ्सृ॒ष्टः । अ॒न॒ड्वान् । पु॒न॒रा॒धेय॒स्येति॑ पुनः - आ॒धेय॑स्य । समृ॑द्ध्या॒ इति॒ सं - ऋ॒द्ध्यै॒ । स॒प्त । ते॒ । अ॒ग्ने॒ । स॒मिध॒ इति॑ सं - इधः॑ । स॒प्त । जि॒ह्वाः । इति॑ । अ॒ग्नि॒हो॒त्रमित्य॑ग्नि - हो॒त्रम् । जु॒हो॒ति॒ । यत्र॑य॒त्रेति॒ यत्र॑-य॒त्र॒ । ए॒व । अ॒स्य॒ । न्य॑क्त॒मिति॒ नि - अ॒क्त॒म् । ततः॑ । \textbf{  8} \newline
                  \newline
                                \textbf{ TS 1.5.2.5} \newline
                  ए॒व । ए॒न॒म् । अवेति॑ । रु॒न्धे॒ । वी॒र॒हेति॑ वीर - हा । वै । ए॒षः । दे॒वाना᳚म् । यः । अ॒ग्निम् । उ॒द्वा॒सय॑त॒ इत्यु॑त् - वा॒सय॑ते । तस्य॑ । वरु॑णः । ए॒व । ऋ॒ण॒यादित्यृ॑ण - यात् । आ॒ग्नि॒वा॒रु॒णमित्या᳚ग्नि - वा॒रु॒णम् । एका॑दशकपाल॒मित्येका॑दश - क॒पा॒ल॒म् । अनु॑ । निरिति॑ । व॒पे॒त् । यम् । च॒ । ए॒व । हन्ति॑ । यः । च॒ । अ॒स्य॒ । ऋ॒ण॒यादित्यृ॑ण - यात् । तौ । भा॒ग॒धेये॒नेति॑ भाग - धेये॑न । प्री॒णा॒ति॒ । न । आर्ति᳚म् । एति॑ । ऋ॒च्छ॒ति॒ । यज॑मानः ॥ \textbf{  9} \newline
                  \newline
                      (आ-ऽप॑राभावाय-पुरो॒डाश॑मे॒ते-आहु॑ती॒-ततः॒ -षटत्रिꣳ॑शच्च)  \textbf{(A2)} \newline \newline
                                \textbf{ TS 1.5.3.1} \newline
                  भूमिः॑ । भू॒म्ना । द्यौः । व॒रि॒णा । अ॒न्तरि॑क्षम् । म॒हि॒त्वेति॑ महि-त्वा ॥ उ॒पस्थ॒ इत्यु॒प - स्थे॒ । ते॒ । दे॒वि॒ । अ॒दि॒ते॒ । अ॒ग्निम् । अ॒न्ना॒दमित्य॑न्न - अ॒दम् । अ॒न्नाद्या॒येत्य॑न्न - अद्या॑य । एति॑ । द॒धे॒ ॥ एति॑ । अ॒यम् । गौः । पृश्निः॑ । अ॒क्र॒मी॒त् । अस॑नत् । मा॒तर᳚म् । पुनः॑ ॥ पि॒तर᳚म् । च॒ । प्र॒यन्निति॑ प्र - यन्न् । सुवः॑ ॥ त्रिꣳ॒॒शत् । धाम॑ । वीति॑ । रा॒ज॒ति॒ । वाक् । प॒त॒ङ्गाय॑ । शि॒श्रि॒ये॒ ॥ प्रतीति॑ । अ॒स्य॒ । व॒ह॒ । द्युभि॒रिति॒ द्यु - भिः॒ । अ॒स्य । प्रा॒णादिति॑ प्र - अ॒नात् । अ॒पा॒न॒तीत्य॑प - अ॒न॒ती । अ॒न्तः । च॒र॒ति॒ । रो॒च॒ना ॥ वीति॑ । अ॒ख्य॒त् । म॒हि॒षः । सुवः॑ ॥ यत् । त्वा॒ । \textbf{  10} \newline
                  \newline
                                \textbf{ TS 1.5.3.2} \newline
                  क्रु॒द्धः । प॒रो॒वपेति॑ परा - उ॒पव॑ । म॒न्युना᳚ । यत् । अव॑र्त्या ॥ सु॒कल्प॒मिति॑ सु - कल्प᳚म् । अ॒ग्ने॒ । तत् । तव॑ । पुनः॑ । त्वा॒ । उदिति॑ । दी॒प॒या॒म॒सि॒ ॥ यत् । ते॒ । म॒न्युप॑रोप्त॒स्येति॑ म॒न्यु-प॒रो॒प्त॒स्य॒ । पृ॒थि॒वीम् । अन्विति॑ । द॒द्ध्व॒से ॥ आ॒दि॒त्याः । विश्वे᳚ । तत् । दे॒वाः । वस॑वः । च॒ । स॒माभ॑र॒न्निति॑ सं - आभ॑रन्न् ॥ मनः॑ । ज्योतिः॑ । जु॒ष॒ता॒म् । आज्य᳚म् । विच्छि॑न्न॒मिति॒ वि - छि॒न्न॒म् । य॒ज्ञ्म् । समिति॑ । इ॒मम् । द॒धा॒तु॒ ॥ बृह॒स्पतिः॑ । त॒नु॒ता॒म् । इ॒मम् । नः॒ । विश्वे᳚ । दे॒वाः । इ॒ह । मा॒द॒य॒न्ता॒म् ॥ स॒प्त । ते॒ । अ॒ग्ने॒ । स॒मिध॒ इति॑ सं - इधः॑ । स॒प्त । जि॒ह्वाः । स॒प्त । \textbf{  11} \newline
                  \newline
                                \textbf{ TS 1.5.3.3} \newline
                  ऋष॑यः । स॒प्त । धाम॑ । प्रि॒याणि॑ ॥ स॒प्त । होत्राः᳚ । स॒प्त॒धेति॑ सप्त - धा । त्वा॒ । य॒ज॒न्ति॒ । स॒प्त । योनीः᳚ । एति॑ । पृ॒ण॒स्व॒ । घृ॒तेन॑ ॥ पुनः॑ । ऊ॒र्जा । नीति॑ । व॒र्त॒स्व॒ । पुनः॑ । अ॒ग्ने॒ । इ॒षा । आयु॑षा ॥ पुनः॑ । नः॒ । पा॒हि॒ । वि॒श्वतः॑ ॥ स॒ह । र॒य्या । नीति॑ । व॒र्त॒स्व॒ । अग्ने᳚ । पिन्व॑स्व । धार॑या ॥ वि॒श्वफ्‌स्नि॒येति॑ वि॒श्व - फ्‌स्नि॒या॒ । वि॒श्वतः॑ । परि॑ ॥ लेकः॑ । सले॑क॒ इति॒ स - ले॒कः॒ । सु॒लेक॒ इति॑ सु - लेकः॑ । ते । नः॒ । आ॒दि॒त्याः । आज्य᳚म् । जु॒षा॒णाः । वि॒य॒न्तु॒ । केतः॑ । सके॑त॒ इति॒ स - के॒तः॒ । सु॒केत॒ इति॑ सु - केतः॑ । ते । नः॒ ( ) । आ॒दि॒त्याः । आज्य᳚म् । जु॒षा॒णाः । वि॒य॒न्तु॒ । विव॑स्वान् । अदि॑तिः । देव॑जूति॒रिति॒ देव॑ - जू॒तिः॒ । ते । नः॒ । आ॒दि॒त्याः । आज्य᳚म् । जु॒षा॒णाः । वि॒य॒न्तु॒ ॥ \textbf{  12} \newline
                  \newline
                      (त्वा॒-जि॒ह्वाः स॒प्त-सु॒केत॒स्ते न॒-स्त्रयो॑दश च )  \textbf{(A3)} \newline \newline
                                \textbf{ TS 1.5.4.1} \newline
                  भूमिः॑ । भू॒म्ना । द्यौः । व॒रि॒णा । इति॑ । आ॒ह॒ । आ॒शिषेत्या᳚ - शिषा᳚ । ए॒व । ए॒न॒म् । एति॑ । ध॒त्ते॒ । स॒र्पाः । वै । जीर्य॑न्तः । अ॒म॒न्य॒न्त॒ । सः । ए॒तम् । क॒स॒र्णीरः॑ । का॒द्र॒वे॒यः । मन्त्र᳚म् । अ॒प॒श्य॒त् । ततः॑ । वै । ते । जी॒र्णाः । त॒नूः । अपेति॑ । अ॒घ्न॒त॒ । स॒र्प॒रा॒ज्ञिया॒ इति॑ सर्प -रा॒ज्ञियाः᳚ । ऋ॒ग्भिरित्यृ॑क् - भिः । गार्.ह॑पत्य॒मिति॒ गार्.ह॑ - प॒त्य॒म् । एति॑ । द॒धा॒ति॒ । पु॒न॒र्न॒वमिति॑ पुनः- न॒वम् । ए॒व । ए॒न॒म् । अ॒जर᳚म् । कृ॒त्वा । एति॑ । ध॒त्ते॒ । अथो॒ इति॑ । पू॒तम् । ए॒व । पृ॒थि॒वीम् । अ॒न्नाद्य॒मित्य॑न्न - अद्य᳚म् । न । उपेति॑ । अ॒न॒म॒त् । सा । ए॒तम् । \textbf{  13} \newline
                  \newline
                                \textbf{ TS 1.5.4.2} \newline
                  मन्त्र᳚म् । अ॒प॒श्य॒त् । ततः॑ । वै । ताम् । अ॒न्नाद्य॒मित्य॑न्न - अद्य᳚म् । उपेति॑ । अ॒न॒म॒त् । यत् । स॒र्प॒रा॒ज्ञिया॒ इति॑ सर्प -रा॒ज्ञियाः᳚ । ऋ॒ग्भिरित्यृ॑क् - भिः । गार्.ह॑पत्य॒मिति॒ गार्.ह॑ - प॒त्य॒म् । आ॒दधा॒तीत्या᳚ - दधा॑ति । अ॒न्नाद्य॒स्येत्य॑न्न - अद्य॑स्य । अव॑रुद्ध्या॒ इत्यव॑ - रु॒द्ध्यै॒ । अथो॒ इति॑ । अ॒स्याम् । ए॒व । ए॒न॒म् । प्रति॑ष्ठित॒मिति॒ प्रति॑ - स्थि॒त॒म् । एति॑ । ध॒त्ते॒ । यत् । त्वा॒ । क्रु॒द्धः । प॒रो॒वपेति॑ परा - उ॒वप॑ । इति॑ । आ॒ह॒ । अपेति॑ । ह्नु॒ते॒ । ए॒व । अ॒स्मै॒ । तत् । पुनः॑ । त्वा॒ । उदिति॑ । दी॒प॒या॒म॒सि॒ । इति॑ । आ॒ह॒ । समिति॑ । इ॒न्धे॒ । ए॒व । ए॒न॒म् । यत् । ते॒ । म॒न्युप॑रोप्त॒स्येति॑ म॒न्यु - प॒रो॒प्त॒स्य॒ । इति॑ । आ॒ह॒ । दे॒वता॑भिः । ए॒व । \textbf{  14} \newline
                  \newline
                                \textbf{ TS 1.5.4.3} \newline
                  ए॒न॒म् । समिति॑ । भ॒र॒ति॒ । वीति॑ । वै । ए॒तस्य॑ । य॒ज्ञ्ः । छि॒द्य॒ते॒ । यः । अ॒ग्निम् । उ॒द्वा॒सय॑त॒ इत्यु॑त् - वा॒सय॑ते । बृह॒स्पति॑व॒त्येति॒॒ बृह॒स्पति॑ - व॒त्य॒ । ऋ॒चा । उपेति॑ । ति॒ष्ठ॒ते॒ । ब्रह्म॑ । वै । दे॒वाना᳚म् । बृह॒स्पतिः॑ । ब्रह्म॑णा । ए॒व । य॒ज्ञ्म् । समिति॑ । द॒धा॒ति॒ । वच्छि॑न्न॒मिति॒ वि-छि॒न्न॒म् । य॒ज्ञ्म् । समिति॑ । इ॒मम् । द॒धा॒तु॒ । इति॑ । आ॒ह॒ । संत॑त्या॒ इति॒ सं - त॒त्यै॒ । विश्वे᳚ । दे॒वाः । इ॒ह । मा॒द॒य॒न्ता॒म् । इति॑ । आ॒ह॒ । स॒न्तत्येति॑ सं - तत्य॑ । ए॒व । य॒ज्ञ्म् । दे॒वेभ्यः॑ । अन्विति॑ । दि॒श॒ति॒ । स॒प्त । ते॒ । अ॒ग्ने॒ । स॒मिध॒ इति॑ सं - इधः॑ । स॒प्त । जि॒ह्वाः । \textbf{  15} \newline
                  \newline
                                \textbf{ TS 1.5.4.4} \newline
                  इति॑ । आ॒ह॒ । स॒प्तस॒प्तेति॑ स॒प्त - स॒प्त॒ । वै । स॒प्त॒धेति॑ सप्त - धा । अ॒ग्नेः । प्रि॒याः । त॒नुवः॑ । ताः । ए॒व । अवेति॑ । रु॒न्धे॒ । पुनः॑ । ऊ॒र्जा । स॒ह । र॒य्या । इति॑ । अ॒भितः॑ । पु॒रो॒डाश᳚म् । आहु॑ती॒ इत्या - हु॒ती॒ । जु॒हो॒ति॒ । यज॑मानम् । ए॒व । ऊ॒र्जा । च॒ । र॒य्या । च॒ । उ॒भ॒यतः॑ । परीति॑ । गृ॒ह्णा॒ति॒ । आ॒दि॒त्याः । वै । अ॒स्मात् । लो॒कात् । अ॒मुम् । लो॒कम् । आ॒य॒न्न् । ते । अ॒मुष्मिन्न्॑ । लो॒के । वीति॑ । अ॒तृ॒ष्य॒न्न् । ते । इ॒मम् । लो॒कम् । पुनः॑ । अ॒भ्य॒वेत्येत्य॑भि - अ॒वेत्य॑ । अ॒ग्निम् । आ॒धायेत्या᳚ - धाय॑ । ए॒तान् () । होमान्॑ । अ॒जु॒ह॒वुः॒ । ते । आ॒र्द्ध्नु॒व॒न्न् । ते । सु॒व॒र्गमिति॑ सुवः - गम् । लो॒कम् । आ॒य॒न्न् । यः । प॒रा॒चीन᳚म् । पु॒न॒रा॒धेया॒दिति॑ पुनः - आ॒धेया᳚त् । अ॒ग्निम् । आ॒दधी॒तेत्या᳚- दधी॑त । सः । ए॒तान् । होमान्॑ । जु॒हु॒या॒त् । याम् । ए॒व । आ॒दि॒त्याः । ऋद्धि᳚म् । आर्द्ध्नु॑वन्न् । ताम् । ए॒व । ऋ॒द्ध्नो॒ति॒ ॥ \textbf{  16} \newline
                  \newline
                      (सैतं-दे॒वता॑भिरे॒व-जि॒ह्वा-ए॒तान्-पञ्च॑विꣳशतिश्च )  \textbf{(A4)} \newline \newline
                                \textbf{ TS 1.5.5.1} \newline
                  उ॒प॒प्र॒यन्त॒ इत्यु॑प - प्र॒यन्तः॑ । अ॒द्ध्व॒रम् । मन्त्र᳚म् । वो॒चे॒म॒ । अ॒ग्नये᳚ ॥ आ॒रे । अ॒स्मे इति॑ । च॒ । शृ॒ण्व॒ते ॥ अ॒स्य । प्र॒त्नाम् । अन्विति॑ । द्युत᳚म् । शु॒क्रम् । दु॒दु॒ह्रे॒ । अह्र॑यः ॥ पयः॑ । स॒ह॒स्र॒सामिति॑ सहस्र - साम् । ऋषि᳚म् ॥ अ॒ग्निः । मू॒र्धा । दि॒वः । क॒कुत् । पतिः॑ । पृ॒थि॒व्याः । अ॒यम् ॥ अ॒पाम् । रेताꣳ॑सि । जि॒न्व॒ति॒ ॥ अ॒यम् । इ॒ह । प्र॒थ॒मः । धा॒यि॒ । धा॒तृभि॒रिति॑ धा॒तृ - भिः॒ । होता᳚ । यजि॑ष्ठः । अ॒द्ध्व॒रेषु॑ । ईड्यः॑ ॥ यम् । अप्न॑वानः । भृग॑वः । वि॒रु॒रु॒चुरिति॑ वि - रु॒रु॒चुः । वने॑षु । चि॒त्रम् । वि॒भुव॒मिति॑ वि - भुव᳚म् । वि॒शेवि॑श॒ इति॑ वि॒शे - वि॒शे॒ ॥ उ॒भा । वा॒म् । इ॒न्द्रा॒ग्नी॒ इती᳚न्द्र - अ॒ग्नी॒ । आ॒हु॒वद्ध्यै᳚ । \textbf{  17} \newline
                  \newline
                                \textbf{ TS 1.5.5.2} \newline
                  उ॒भा । राध॑सः । स॒ह । मा॒द॒यद्ध्यै᳚ ॥ उ॒भा । दा॒तारौ᳚ । इ॒षाम् । र॒यी॒णाम् । उ॒भा । वाज॑स्य । सा॒तये᳚ । हु॒वे॒ । वा॒म् ॥ अ॒यम् । ते॒ । योनिः॑ । ऋ॒त्वियः॑ । यतः॑ । जा॒तः । अरो॑चथाः ॥ तम् । जा॒नन्न् । अ॒ग्ने॒ । एति॑ । रो॒ह॒ । अथ॑ । नः॒ । व॒र्द्ध॒य॒ । र॒यिम् ॥ अग्ने᳚ । आयूꣳ॑षि । प॒व॒से॒ । एति॑ । सु॒व॒ । ऊर्ज᳚म् । इष᳚म् । च॒ । नः॒ ॥ आ॒रे । बा॒ध॒स्व॒ । दु॒च्छुना᳚म् ॥ अग्ने᳚ । पव॑स्व । स्वपा॒ इति॑ सु - अपाः᳚ । अ॒स्मे इति॑ । वर्चः॑ । सु॒वीर्य॒मिति॑ सु - वीर्य᳚म् ॥ दध॑त् । पोष᳚म् । र॒यिम् । \textbf{  18} \newline
                  \newline
                                \textbf{ TS 1.5.5.3} \newline
                  मयि॑ ॥ अग्ने᳚ । पा॒व॒क॒ । रो॒चिषा᳚ । म॒न्द्रया᳚ । दे॒व॒ । जि॒ह्वया᳚ ॥ एति॑ । दे॒वान् । व॒क्षि॒ । यक्षि॑ । च॒ ॥ सः । नः॒ । पा॒व॒क॒ । दी॒दि॒वः । अग्ने᳚ । दे॒वान् । इ॒ह । एति॑ । व॒ह॒ ॥ उपेति॑ । य॒ज्ञ्म् । ह॒विः । च॒ । नः॒ ॥ अ॒ग्निः । शुचि॑व्रततम॒ इति॒ शुचि॑व्रत - त॒मः॒ । शुचिः॑ । विप्रः॑ । शुचिः॑ । क॒विः ॥ शुचिः॑ । रो॒च॒ते॒ । आहु॑त॒ इत्या - हु॒तः॒ ॥ उदिति॑ । अ॒ग्ने॒ । शुच॑यः । तव॑ । शु॒क्राः । भ्राज॑न्तः । ई॒र॒ते॒ ॥ तव॑ । ज्योतीꣳ॑षि । अ॒र्चयः॑ ॥ आ॒यु॒र्दा इत्या॑युः - दाः । अ॒ग्ने॒ । अ॒सि॒ । आयुः॑ । म॒ । \textbf{  19} \newline
                  \newline
                                \textbf{ TS 1.5.5.4} \newline
                  दे॒हि॒ । व॒र्चो॒दा इति॑ वर्चः- दाः । अ॒ग्ने॒ । अ॒सि॒ । वर्चः॑ । मे॒ । दे॒हि॒ । त॒नू॒पा इति॑ तनू - पाः । अ॒ग्ने॒ । अ॒सि॒ । त॒नुव᳚म् । मे॒ । पा॒हि॒ । अग्ने᳚ । यत् । मे॒ । त॒नुवाः᳚ । ऊ॒नम् । तत् । मे॒ । एति॑ । पृ॒ण॒ । चित्रा॑वसो॒ इति॒ चित्र॑ - व॒सो॒ । स्व॒स्ति । ते॒ । पा॒रम् । अ॒शी॒य॒ । इन्धा॑नाः । त्वा॒ । श॒तम् । हिमाः᳚ । द्यु॒मन्त॒ इति॑ द्यु- मन्तः॑ । समिति॑ । इ॒धी॒म॒हि॒ । वय॑स्वन्तः । व॒य॒स्कृत॒मिति॑ वयः-कृत᳚म् । यश॑स्वन्तः । य॒श॒स्कृत॒मिति॑ यशः - कृत᳚म् । सु॒वीरा॑स॒ इति सु - वीरा॑सः । अदा᳚भ्यम् ॥ अग्ने᳚ । स॒प॒त्न॒दंभ॑न॒मिति॑ सपत्न - दंभ॑नम् । वर्.षि॑ष्ठे । अधीति॑ । नाके᳚ ॥ समिति॑ । त्वम् । अ॒ग्ने॒ । सूर्य॑स्य । वर्च॑सा ( ) । अ॒ग॒थाः॒ । समिति॑ । ऋषी॑णाम् । स्तु॒तेन॑ । समिति॑ । प्रि॒येण॑ । धाम्ना᳚ ॥ त्वम् । अ॒ग्ने॒ । सूर्य॑वर्चा॒ इति॒ सूर्य॑ - व॒र्चाः॒ । अ॒सि॒ । समिति॑ । माम् । आयु॑षा । वर्च॑सा । प्र॒जयेति॑ प्र - जया᳚ । सृ॒ज॒ ॥ \textbf{  20} \newline
                  \newline
                      (आ॒हु॒वद्ध्यै॒-पोषꣳ॑ र॒यिं-मे॒-वर्च॑सा-स॒प्तद॑श च )  \textbf{(A5)} \newline \newline
                                \textbf{ TS 1.5.6.1} \newline
                  समिति॑ । प॒श्या॒मि॒ । प्र॒जा इति॑ प्र - जाः । अ॒हम् । इड॑प्रजस॒ इतीड॑ - प्र॒ज॒सः॒ । मा॒न॒वीः ॥ सर्वाः᳚ । भ॒व॒न्तु॒ । नः॒ । गृ॒हे ॥ अभंः॑ । स्थ॒ । अम्भः॑ । वः॒ । भ॒क्षी॒य॒ । महः॑ । स्थ॒ । महः॑ । वः॒ । भ॒क्षी॒य॒ । सहः॑ । स्थ॒ । सहः॑ । वः॒ । भ॒क्षी॒य॒ । ऊर्जः॑ । स्थ॒ । ऊर्ज᳚म् । वः॒ । भ॒क्षी॒य॒ । रेव॑तीः । रम॑द्ध्वम् । अ॒स्मिन्न् । लो॒के । अ॒स्मिन्न् । गो॒ष्ठ इति॑ गो-स्थे । अ॒स्मिन्न् । क्षये᳚ । अ॒स्मिन्न् । योनौ᳚ । इ॒ह । ए॒व । स्त॒ । इ॒तः । मा । अपेति॑ । गा॒त॒ । ब॒ह्वीः । मे॒ । भू॒या॒स्त॒ । \textbf{  21} \newline
                  \newline
                                \textbf{ TS 1.5.6.2} \newline
                  सꣳ॒॒हि॒तेति॑ सं - हि॒ता । अ॒सि॒ । वि॒श्व॒रू॒पीरिति॑ विश्व - रू॒पीः । एति॑ । मा॒ । ऊ॒र्जा । वि॒श॒ । एति॑ । गौ॒प॒त्येन॑ । एति॑ । रा॒यः । पोषे॑ण । स॒ह॒स्र॒पो॒षमिति॑ सहस्र - पो॒षम् । वः॒ । पु॒ष्या॒स॒म् । मयि॑ । वः॒ । रायः॑ । श्र॒य॒न्ता॒म् ॥ उपेति॑ । त्वा॒ । अ॒ग्ने॒ । दि॒वेदि॑व॒ इति॑ दि॒वे-दि॒वे॒ । दोषा॑वस्त॒रिति॒ दोषा᳚-व॒स्तः॒ । धि॒या । व॒यम् ॥ नमः॑ । भर॑न्तः । एति॑ । इ॒म॒सि॒ ॥ राज॑न्तम् । अ॒द्ध्व॒राणा᳚म् । गो॒पामिति॑ गो - पाम् । ऋ॒तस्य॑ । दीदि॑विम् ॥ वर्ध॑मानम् । स्वे । दमे᳚ ॥ सः । नः॒ । पि॒ता । इ॒व॒ । सू॒नवे᳚ । अग्ने᳚ । सू॒पा॒य॒न इति॑ सु - उ॒पा॒य॒नः । भ॒व॒ ॥ सच॑स्व ।  नः॒ । स्व॒स्तये᳚ ॥ अग्ने᳚ । \textbf{  22} \newline
                  \newline
                                \textbf{ TS 1.5.6.3} \newline
                  त्वम् । नः॒ । अन्त॑मः ॥ उ॒त । त्रा॒ता । शि॒वः । भ॒व॒ । व॒रू॒थ्यः॑ ॥ तम् । त्वा॒ । शो॒चि॒ष्ठ॒ । दी॒दि॒वः॒ ॥ सु॒म्नाय॑ । नू॒नम् । ई॒म॒हे॒ । सखि॑भ्य॒ इति॒ सखि॑ - भ्यः॒ ॥ वसुः॑ । अ॒ग्निः । वसु॑श्रवा॒ इति॒ वसु॑ - श्र॒वाः॒ ॥ अच्छ॑ । न॒क्षि॒ । द्यु॒मत्त॑म॒ इति॑ द्यु॒मत् - त॒मः॒ । र॒यिम् । दाः॒ ॥ ऊ॒र्जा । वः॒ । प॒श्या॒मि॒ । ऊ॒र्जा । मा॒ । प॒श्य॒त॒ । रा॒यः । पोषे॑ण । वः॒ । प॒श्या॒मि॒ । रा॒यः । पोषे॑ण । मा॒ । प॒श्य॒त॒ । इडाः᳚ । स्थ॒ । म॒धु॒कृत॒ इति॑ मधु - कृतः॑ । स्यो॒नाः । मा॒ । एति॑ । वि॒श॒त॒ । इराः᳚ । मदः॑ ॥ स॒ह॒स्र॒पो॒षमिति॑ सहस्र - पो॒षम् । वः॒ । पु॒ष्या॒स॒म् । \textbf{  23} \newline
                  \newline
                                \textbf{ TS 1.5.6.4} \newline
                  मयि॑ । वः॒ । रायः॑ । श्र॒य॒न्ता॒म् ॥ तत् । स॒वि॒तुः । वरे᳚ण्यम् । भर्गः॑ । दे॒वस्य॑ । धी॒म॒हि॒ ॥ धियः॑ । यः । नः॒ । प्र॒चो॒दया॒दिति॑ प्र-चो॒दया᳚त् ॥ सो॒मान᳚म् । स्वर॑णम् । कृ॒णु॒हि । ब्र॒ह्म॒णः॒ । प॒ते॒ ॥ क॒क्षीव॑न्त॒मिति॑ क॒क्षी - व॒न्त॒म् । यः । औ॒शि॒जम् । क॒दा । च॒न । स्त॒रीः । अ॒सि॒ । न । इ॒न्द्र॒ । स॒श्च॒सि॒ । दा॒शुषे᳚ ॥ उपो॒पेत्युप॑ - उ॒प॒ । इत् । नु । म॒घ॒व॒न्निति॑ मघ - व॒न्न् । भूयः॑ । इत् । नु । ते॒ । दान᳚म् । दे॒वस्य॑ । पृ॒च्य॒ते॒ ॥ परीति॑ । त्वा॒ । अ॒ग्ने॒ । पुर᳚म् । व॒यम् । विप्र᳚म् । स॒ह॒स्य॒ । धी॒म॒हि॒ ॥ धृ॒षद्व॑र्ण॒मिति॑ धृ॒षत् - व॒र्ण॒म् ( ) । दि॒वेदि॑व॒ इति॑ दि॒वे - दि॒वे॒ । भे॒त्तार᳚म् । भ॒ङ्गु॒राव॑त॒ इति॑ भङ्गु॒र-व॒तः॒ ॥ अग्ने᳚ । गृ॒ह॒प॒त॒ इति॑ गृह - प॒ते॒ । सु॒गृ॒ह॒प॒तिरिति॑ सु - गृ॒ह॒प॒तिः । अ॒हम् । त्वया᳚ । गृ॒हप॑ति॒नेति॑ गृ॒ह - प॒ति॒ना॒ । भू॒या॒स॒म् । सु॒गृ॒ह॒प॒तिरिति॑ सु - गृ॒ह॒प॒तिः । मया᳚ । त्वम् । गृ॒हप॑ति॒नेति॑ गृ॒ह - प॒ति॒ना॒ । भू॒याः॒ । श॒तम् । हिमाः᳚ । ताम् । आ॒शिष॒मित्या᳚ - शिष᳚म् । एति॑ । शा॒से॒ । तन्त॑वे । ज्योति॑ष्मतीम् । ताम् । आ॒शिष॒मित्या᳚ - शिष᳚म् । एति॑ । शा॒से॒ । अ॒मुष्मै᳚ । ज्योति॑ष्मतीम् ॥ \textbf{  24} \newline
                  \newline
                      (भू॒या॒स्त॒-स्व॒स्तयेऽग्ने॑-पुष्यासं-धृ॒षद्व॑र्ण॒-मेका॒न्नत्रिꣳ॒॒शच्च॑ )  \textbf{(A6)} \newline \newline
                                \textbf{ TS 1.5.7.1} \newline
                  अय॑ज्ञ्ः । वै । ए॒षः । यः । अ॒सा॒मा । उ॒प॒प्र॒यन्त॒ इत्यु॑प - प्र॒यन्तः॑ । अ॒द्ध्व॒रम् । इति॑ । आ॒ह॒ । स्तोम᳚म् । ए॒व । अ॒स्मै॒ । यु॒न॒क्ति॒ । उपेति॑ । इति॑ । आ॒ह॒ । प्र॒जेति॑ प्र - जा । वै । प॒शवः॑ । उपेति॑ । इ॒मम् । लो॒कम् । प्र॒जामिति॑ प्र - जाम् । ए॒व । प॒शून् । इ॒मम् । लो॒कम् । उपेति॑ । ए॒ति॒ । अ॒स्य । प्र॒त्नाम् । अन्विति॑ । द्युत᳚म् । इति॑ । आ॒ह॒ । सु॒व॒र्ग इति॑ सुवः - गः । वै । लो॒कः । प्र॒त्नः । सु॒व॒र्गमिति॑ सुवः - गम् । ए॒व । लो॒कम् । स॒मारो॑ह॒तीति॑ सं - आरो॑हति । अ॒ग्निः । मू॒र्धा । दि॒वः । क॒कुत् । इति॑ । आ॒ह॒ । मू॒र्धान᳚म् । \textbf{  25} \newline
                  \newline
                                \textbf{ TS 1.5.7.2} \newline
                  ए॒व । ए॒न॒म् । स॒मा॒नाना᳚म् । क॒रो॒ति॒ । अथो॒ इति॑ । दे॒व॒लो॒कादिति॑ देव - लो॒कात् । ए॒व । म॒नु॒ष्य॒लो॒क इति॑ मनुष्य - लो॒के । प्रतीति॑ । ति॒ष्ठ॒ति॒ । अ॒यम् । इ॒ह । प्र॒थ॒मः । धा॒यि॒ । धा॒तृभि॒रिति॑ धा॒तृ - भिः॒ । इति॑ । आ॒ह॒ । मुख्य᳚म् । ए॒व । ए॒न॒म् । क॒रो॒ति॒ । उ॒भा । वा॒म् । इ॒न्द्रा॒ग्नी॒ इती᳚न्द्र - अ॒ग्नी॒ । आ॒हु॒वद्ध्यै᳚ । इति॑ । आ॒ह॒ । ओजः॑ । बल᳚म् । ए॒व । अवेति॑ । रु॒न्धे॒ । अ॒यम् । ते॒ । योनिः॑ । ऋ॒त्वियः॑ । इति॑ । आ॒ह॒ । प॒शवः॑ । वै । र॒यिः । प॒शून् । ए॒व । अवेति॑ । रु॒न्धे॒ । ष॒ड्भिरिति॑ षट् - भिः । उपेति॑ । ति॒ष्ठ॒ते॒ । षट् । वै । \textbf{  26} \newline
                  \newline
                                \textbf{ TS 1.5.7.3} \newline
                  ऋ॒तवः॑ । ऋ॒तुषु॑ । ऐ॒व । प्रतीति॑ । ति॒ष्ठ॒ति॒ । ष॒ड्भिरिति॑ षट् - भिः । उत्त॑राभि॒रित्युत् - त॒रा॒भिः॒ । उपेति॑ । ति॒ष्ठ॒ते॒ । द्वाद॑श । समिति॑ । प॒द्य॒न्ते॒ । द्वाद॑श । मासाः᳚ । स॒म्ॅव॒थ्स॒र इति॑ सं - व॒थ्स॒रः । स॒म्ॅव॒थ्स॒र इति॑ सं - व॒थ्स॒रे । ए॒व । प्रतीति॑ । ति॒ष्ठ॒ति॒ । यथा᳚ । वै । पुरु॑षः । अश्वः॑ । गौः । जीर्य॑ति । ए॒वम् । अ॒ग्निः । आहि॑त॒ इत्या - हि॒तः॒ । जी॒र्य॒ति॒ । स॒॒म्ॅव॒थ्स॒रस्येति॑ सं - वथ्स॒रस्य॑ । प॒रस्ता᳚त् । आ॒ग्नि॒पा॒व॒मा॒नीभि॒रित्या᳚ग्नि - पा॒व॒मा॒नीभिः॑ । उपेति॑ । ति॒ष्ठ॒ते॒ । पु॒न॒र्न॒वमिति॑ पुनः-न॒वम् । ए॒व । ए॒न॒म् । अ॒जर᳚म् । क॒रो॒ति॒ । अथो॒ इति॑ । पु॒नाति॑ । ए॒व । उपेति॑ । ति॒ष्ठ॒ते॒ । योगः॑ । ए॒व । अ॒स्य॒ । ए॒षः । उपेति॑ । ति॒ष्ठ॒ते॒ । \textbf{  27} \newline
                  \newline
                                \textbf{ TS 1.5.7.4} \newline
                  दमः॑ । ए॒व । अ॒स्य॒ । ए॒षः । उपेति॑ । ति॒ष्ठ॒ते॒ । याच्ञै । ए॒व । अ॒स्य॒ । ए॒षा । उपेति॑ । ति॒ष्ठ॒ते॒ । यथा᳚ । पापी॑यान् । श्रेय॑से । आ॒हृत्येत्या᳚ - हृत्य॑ । न॒म॒स्यति॑ । ता॒दृक् । ए॒व । तत् । आ॒यु॒र्दा इत्या॑युः - दाः । अ॒ग्ने॒ । अ॒सि॒ । आयुः॑ । मे॒ । दे॒हि॒ । इति॑ । आ॒ह॒ । आ॒यु॒र्दा इत्या॑युः - दाः । हि । ए॒षः । व॒र्चो॒दा इति॑ वर्चः - दाः । अ॒ग्ने॒ । अ॒सि॒ । वर्चः॑ । मे॒ । दे॒हि॒ । इति॑ । आ॒ह॒ । व॒र्चो॒दा इति॑ वर्चः - दाः । हि । ए॒षः । त॒नू॒पा इति॑ तनू - पाः । अ॒ग्ने॒ । अ॒सि॒ । त॒नुव᳚म् । मे॒ । पा॒हि॒ । इति॑ । आ॒ह॒ । \textbf{  28} \newline
                  \newline
                                \textbf{ TS 1.5.7.5} \newline
                  त॒नू॒पा इति॑ तनू - पाः । हि । ए॒षः । अग्ने᳚ । यत् । मे॒ । त॒नुवाः᳚ । ऊ॒नम् । तत् । मे॒ । एति॑ । पृ॒ण॒ । इति॑ । आ॒ह॒ । यत् । मे॒ । प्र॒जाया॒ इति॑ प्र - जायै᳚ । प॒शू॒नाम् । ऊ॒नम् । तत् । मे॒ । एति॑ । पू॒र॒य॒ । इति॑ । वाव । ए॒तत् । आ॒ह॒ । चित्रा॑वसो॒ इति॒ चित्र॑ - व॒सो॒ । स्व॒स्ति । ते॒ । पा॒रम् । अ॒शी॒य॒ । इति॑ । आ॒ह॒ । रात्रिः॑ । वै । चि॒त्राव॑सु॒रिति॑ चि॒त्र - व॒सुः॒ । अव्यु॑ष्ट्या॒ इत्यवि॑ - उ॒ष्ट्यै॒ । वै । ए॒तस्यै᳚ । पु॒रा । ब्रा॒ह्म॒णाः । अ॒भै॒षुः॒ । व्यु॑ष्टि॒मिति॒ वि - उ॒ष्टि॒म् । ए॒व । अवेति॑ । रु॒न्धे॒ । इन्धा॑नाः । त्वा॒ । श॒तम् । \textbf{  29} \newline
                  \newline
                                \textbf{ TS 1.5.7.6} \newline
                  हिमाः᳚ । इति॑ । आ॒ह॒ । श॒तायु॒रिति॑ श॒त - आ॒युः॒ । पुरु॑षः । श॒तेन्द्रि॑य॒ इति॑ श॒त - इ॒न्द्रि॒यः॒ । आयु॑षि । ए॒व । इ॒न्द्रि॒ये । प्रतीति॑ । ति॒ष्ठ॒ति॒ । ए॒षा । वै । सू॒र्मी । कर्ण॑काव॒तीति॒ कर्ण॑क-व॒ती॒ । ए॒तया᳚ । ह॒ । स्म॒ । वै । दे॒वाः । असु॑राणाम् । श॒त॒त॒र्॒.हानिति॑ शत - त॒र्॒.हान् । तृꣳ॒॒ह॒न्ति॒ । यत् । ए॒तया᳚ । स॒मिध॒मिति॑ सं-इध᳚म् । आ॒दधा॒तीत्या᳚ - दधा॑ति । वज्र᳚म् । ए॒व । ए॒तत् । श॒त॒घ्नीमिति॑ शत - घ्नीम् । यज॑मानः । भ्रातृ॑व्याय । प्रेति॑ । ह॒र॒ति॒ । स्तृत्यै᳚ । अच्छ॑बंट्कार॒मित्यच्छ॑बंट् - का॒र॒म् । समिति॑ । त्वम् । अ॒ग्ने॒ । सूर्य॑स्य । वर्च॑सा । अ॒ग॒थाः॒ । इति॑ । आ॒ह॒ । ए॒तत् । त्वम् । असि॑ । इ॒दम् । अ॒हम् ( ) । भू॒या॒स॒म् । इति॑ । वाव । ए॒तत् । आ॒ह॒ । त्वम् । अ॒ग्ने॒ । सूर्य॑वर्चा॒ इति॒ सूर्य॑ - व॒र्चाः॒ । अ॒सि॒ । इति॑ । आ॒ह॒ । आ॒शिष॒मित्या᳚ - शिष᳚म् । ए॒व । ए॒ताम् । एति॑ । शा॒स्ते॒ ॥ \textbf{  30 } \newline
                  \newline
                      (मू॒र्द्धानꣳ॒॒-षड्वा-ए॒ष उप॑ तिष्ठते-पा॒हीत्या॑ह-श॒त-म॒हꣳ षोड॑श च)  \textbf{(A7)} \newline \newline
                                \textbf{ TS 1.5.8.1} \newline
                  समिति॑ । प॒श्या॒मि॒ । प्र॒जा इति॑ प्र - जाः । अ॒हम् । इति॑ । आ॒ह॒ । याव॑न्तः । ए॒व । ग्रा॒म्याः । प॒शवः॑ । तान् । ए॒व । अवेति॑ । रु॒न्धे॒ । अभंः॑ । स्थ॒ । अभंः॑ । वः॒ । भ॒क्षी॒य॒ । इति॑ । आ॒ह॒ । अभंः॑ । हि । ए॒ताः । महः॑ । स्थ॒ । महः॑ । वः॒ । भ॒क्षी॒य॒ । इति॑ । आ॒ह॒ । महः॑ । हि । ए॒ताः । सहः॑ । स्थ॒ । सहः॑ । वः॒ । भ॒क्षी॒य॒ । इति॑ । आ॒ह॒ । सहः॑ । हि । ए॒ताः । ऊर्जः॑ । स्थ॒ । ऊर्ज᳚म् । वः॒ । भ॒क्षी॒य॒ । इति॑ । \textbf{  31} \newline
                  \newline
                                \textbf{ TS 1.5.8.2} \newline
                  आ॒ह॒ । ऊर्जः॑ । हि । ए॒ताः । रेव॑तीः । रम॑द्ध्वम् । इति॑ । आ॒ह॒ । प॒शवः॑ । वै । रे॒वतीः᳚ । प॒शून् । ए॒व । आ॒त्मन्न् । र॒म॒य॒ते॒ । इ॒ह । ए॒व । स्त॒ । इ॒तः । मा । अपेति॑ । गा॒त॒ । इति॑ । आ॒ह॒ । ध्रु॒वाः । ए॒व । ए॒नाः॒ । अन॑पगा॒ इत्यन॑प-गाः॒ । कु॒रु॒ते॒ । इ॒ष्ट॒क॒चिदिती᳚ष्टक-चित् । वै । अ॒न्यः । अ॒ग्निः । प॒शु॒चिदिति॑ पशु - चित् । अ॒न्यः । सꣳ॒॒हि॒तेति॑ सं - हि॒ता । अ॒सि॒ । वि॒श्व॒रू॒पीरिति॑ विश्व - रू॒पीः । इति॑ । व॒थ्सम् । अ॒भीति॑ । मृ॒श॒ति॒ । उपेति॑ । ए॒व । ए॒न॒म् । ध॒त्ते॒ । प॒शु॒चित॒मिति॑ पशु - चित᳚म् । ए॒न॒म् । कु॒रु॒ते॒ । प्रेति॑ । \textbf{  32} \newline
                  \newline
                                \textbf{ TS 1.5.8.3} \newline
                  वै । ए॒षः । अ॒स्मात् । लो॒कात् । च्य॒व॒ते॒ । यः । आ॒ह॒व॒नीय॒मित्या᳚ - ह॒व॒नीय᳚म् । उ॒प॒तिष्ठ॑त॒ इत्यु॑प - तिष्ठ॑ते । गार्.ह॑पत्य॒मिति॒ गार्.ह॑ - प॒त्य॒म् । उपेति॑ । ति॒ष्ठ॒ते॒ । अ॒स्मिन्न् । ए॒व । लो॒के । प्रतीति॑ । ति॒ष्ठ॒ति॒ । अथो॒ इति॑ । गार्.ह॑पत्या॒येति॒ गार्.ह॑ - प॒त्या॒य॒ । ए॒व । नीति॑ । ह्नु॒ते॒ । गा॒य॒त्रीभिः॑ । उपेति॑ । ति॒ष्ठ॒ते॒ । तेजः॑ । वै । गा॒य॒त्री । तेजः॑ । ए॒व । आ॒त्मन्न् । ध॒त्ते॒ । अथो॒ इति॑ । यत् । ए॒तम् । तृ॒चम् । अ॒न्वाहेत्य॑नु - आह॑ । संत॑त्या॒ इति॒ सं - त॒त्यै॒ । गार्.ह॑पत्य॒मिति॒ गार्.ह॑ - प॒त्य॒म् । वै । अन्विति॑ । द्वि॒पाद॒ इति॑ द्वि॒ - पादः॑ । वी॒राः । प्रेति॑ । जा॒य॒न्ते॒ । यः । ए॒वम् । वि॒द्वान् । द्वि॒पदा॑भि॒रिति॑ द्वि - पदा॑भिः । गार्.ह॑पत्य॒मिति॒ गार्.ह॑ - प॒त्य॒म् । उ॒प॒तिष्ठ॑त॒ इत्यु॑प - तिष्ठ॑ते । \textbf{  33} \newline
                  \newline
                                \textbf{ TS 1.5.8.4} \newline
                  एति॑ । अ॒स्य॒ । वी॒रः । जा॒य॒ते॒ । ऊ॒र्जा । वः॒ । प॒श्या॒मि॒ । ऊ॒र्जा । मा॒ । प॒श्य॒त॒ । इति॑ । आ॒ह॒ । आ॒शिष॒मित्या᳚ - शिष᳚म् । ए॒व । ए॒ताम् । एति॑ । शा॒स्ते॒ । तत् । स॒वि॒तुः । वरे᳚ण्यम् । इति॑ । आ॒ह॒ । प्रसू᳚त्या॒ इति॒ प्र - सू॒त्यै॒ । सो॒मान᳚म् । स्वर॑णम् । इति॑ । आ॒ह॒ । सो॒म॒पी॒थमिति॑ सोम - पी॒थम् । ए॒व । अवेति॑ । रु॒न्धे॒ । कृ॒णु॒हि । ब्र॒ह्म॒णः॒ । प॒ते॒ । इति॑ । आ॒ह॒ । ब्र॒ह्म॒व॒र्च॒समिति॑ ब्रह्म-व॒र्च॒सम् । ए॒व । अवेति॑ । रु॒न्धे॒ । क॒दा । च॒न । स्त॒रीः । अ॒सि॒ । इति॑ । आ॒ह॒ । न । स्त॒रीम् । रात्रि᳚म् । व॒स॒ति॒ । \textbf{  34} \newline
                  \newline
                                \textbf{ TS 1.5.8.5} \newline
                  यः । ए॒वम् । वि॒द्वान् । अ॒ग्निम् । उ॒प॒तिष्ठ॑त॒ इत्यु॑प - तिष्ठ॑ते । परीति॑ । त्वा॒ । अ॒ग्ने॒ । पुर᳚म् । व॒यम् । इति॑ । आ॒ह॒ । प॒रि॒धिमिति॑ परि-धिम् । ए॒व । ए॒तम् । परीति॑ । द॒धा॒ति॒ । अस्क॑न्दाय । अग्ने᳚ । गृ॒ह॒प॒त॒ इति॑ गृह - प॒ते॒ । इति॑ । आ॒ह॒ । य॒था॒य॒जुरिति॑ यथा - य॒जुः । ए॒व । ए॒तत् । श॒तम् । हिमाः᳚ । इति॑ । आ॒ह॒ । श॒तम् । त्वा॒ । हे॒म॒न्तान् । इ॒न्धि॒षी॒य॒ । इति॑ । वाव । ए॒तत् । आ॒ह॒ । पु॒त्रस्य॑ । नाम॑ । गृ॒ह्णा॒ति॒ । अ॒न्ना॒दमित्य॑न्न - अ॒दम् । ए॒व । ए॒न॒म् । क॒रो॒ति॒ । ताम् । आ॒शिष॒मित्या᳚-शिष᳚म् । एति॑ । शा॒से॒ । तन्त॑वे । ज्योति॑ष्मतीम् ( ) । इति॑ । ब्र॒या॒त् । यस्य॑ । पु॒त्रः । अजा॑तः । स्यात् । ते॒ज॒स्वी । ए॒व । अ॒स्य॒ । ब्र॒ह्म॒व॒र्च॒सीति॑ ब्रह्म - व॒र्च॒सी । पु॒त्रः । जा॒य॒ते॒ । ताम् । आ॒शिष॒मित्या᳚ - शिष᳚म् । एति॑ । शा॒से॒ । अ॒मुष्मै᳚ । ज्योति॑ष्मतीम् । इति॑ । ब्रू॒या॒त् । यस्य॑ । पु॒त्रः । जा॒तः । स्यात् । तेजः॑ । ए॒व । अ॒स्मि॒न्न् । ब्र॒ह्म॒व॒र्च॒समिति॑ ब्रह्म - व॒र्च॒सम् । द॒धा॒ति॒ ॥ \textbf{  35} \newline
                  \newline
                      (ऊर्जं॑ ॅवो भक्षी॒येति॒ - प्र -गार्.ह॑पत्यमुप॒तिष्ठ॑ते -वसति॒-ज्योति॑ष्मती॒ - मेका॒न्नत्रिꣳ॒॒शच्च॑)  \textbf{(A8)} \newline \newline
                                \textbf{ TS 1.5.9.1} \newline
                  अ॒ग्नि॒हो॒त्रमित्य॑ग्नि - हो॒त्रम् । जु॒हो॒ति॒ । यत् । ए॒व । किम् । च॒ । यज॑मानस्य । स्वम् । तस्य॑ । ए॒व । तत् । रेतः॑ । सि॒ञ्च॒ति॒ । प्र॒जन॑न॒ इति॑ प्र - जन॑ने । प्र॒जन॑न॒मिति॑ प्र - जन॑नम् । हि । वै । अ॒ग्निः । अथ॑ । ओष॑धीः । अन्त॑गता॒ इत्यन्त॑ - ग॒ताः॒ । द॒ह॒ति॒ । ताः । ततः॑ । भूय॑सीः । प्रेति॑ । जा॒य॒न्ते॒ । यत् । सा॒यम् । जु॒होति॑ । रेतः॑ । ए॒व । तत् । सि॒ञ्च॒ति॒ । प्रेति॑ । ए॒व । प्रा॒त॒स्तने॒नेति॑ प्रातः - तने॑न । ज॒न॒य॒ति॒ । तत् । रेतः॑ । सि॒क्तम् । न । त्वष्ट्रा᳚ । अवि॑कृत॒मित्यवि॑ - कृ॒त॒म् । प्रेति॑ । जा॒य॒ते॒ । या॒व॒च्छ इति॑ यावत् - शः । वै । रेत॑सः । सि॒क्तस्य॑ । \textbf{  36} \newline
                  \newline
                                \textbf{ TS 1.5.9.2} \newline
                  त्वष्टा᳚ । रू॒पाणि॑ । वि॒क॒रोतीति॑ वि - क॒रोति॑ । ता॒व॒च्छ इति॑ तावत्-शः । वै । तत् । प्रेति॑ । जा॒य॒ते॒ । ए॒षः । वै । दैव्यः॑ । त्वष्टा᳚ । यः । यज॑ते । ब॒ह्वीभिः॑ । उपेति॑ । ति॒ष्ठ॒ते॒ । रेत॑सः । ए॒व । सि॒क्तस्य॑ । ब॒हु॒श इति॑ बहु - शः । रू॒पाणि॑ । वीति॑ । क॒रो॒ति॒ । सः । प्रेति॑ । ए॒व । जा॒य॒ते॒ । श्वः श्व॒ इति॒ श्वः - श्वः॒ । भूयान्॑ । भ॒व॒ति॒ । यः । ए॒वम् । वि॒द्वान् । अ॒ग्निम् । उ॒प॒तिष्ठ॑त॒ इत्यु॑प - तिष्ठ॑ते । अहः॑ । दे॒वाना᳚म् । आसी᳚त् । रात्रिः॑ । असु॑राणाम् । ते । असु॑राः । यत् । दे॒वाना᳚म् । वि॒त्तम् । वेद्य᳚म् । असी᳚त् । तेन॑ । स॒ह । \textbf{  37} \newline
                  \newline
                                \textbf{ TS 1.5.9.3} \newline
                  रात्रि᳚म् । प्रेति॑ । अ॒वि॒श॒न्न् । ते । दे॒वाः । ही॒नाः । अ॒म॒न्य॒न्त॒ । ते । अ॒प॒श्य॒न्न् । आ॒ग्ने॒यी । रात्रिः॑ । आ॒ग्ने॒याः । प॒शवः॑ । इ॒मम् । ए॒व । अ॒ग्निम् । स्त॒वा॒म॒ । सः । नः॒ । स्तु॒तः । प॒शून् । पुनः॑ । दा॒स्य॒ति॒ । इति॑ । ते । अ॒ग्निम् । अ॒स्तु॒व॒न्न् । सः । ए॒भ्यः॒ । स्तु॒तः । रात्रि॑याः । अधीति॑ । अहः॑ । अ॒भीति॑ । प॒शून् । निरिति॑ । आ॒र्ज॒त् । ते । दे॒वाः । प॒शून् । वि॒त्त्वा । कामान्॑ । अ॒कु॒र्व॒त॒ । यः । ए॒वम् । वि॒द्वान् । अ॒ग्निम् । उ॒प॒तिष्ठ॑त॒ इत्यु॑प - तिष्ठ॑ते । प॒शु॒मानिति॑ पशु - मान् । भ॒व॒ति॒ । \textbf{  38} \newline
                  \newline
                                \textbf{ TS 1.5.9.4} \newline
                  आ॒दि॒त्यः । वै । अ॒स्मात् । लो॒कात् । अ॒मुम् । लो॒कम् । ऐ॒त् । सः । अ॒मुम् । लो॒कम् । ग॒त्वा । पुनः॑ । इ॒मम् । लो॒कम् । अ॒भीति॑ । अ॒द्ध्या॒य॒त् । सः । इ॒मम् । लो॒कम् । आ॒गत्येत्या᳚ - गत्य॑ । मृ॒त्योः । अ॒बि॒भे॒त् । मृ॒त्युस॑म्ॅयुत॒ इति॑ मृ॒त्यु-स॒म्ॅयु॒तः॒ । इ॒व॒ । हि । अ॒यम् । लो॒कः । सः । अ॒म॒न्य॒त॒ । इ॒मम् । ए॒व । अ॒ग्निम् । स्त॒वा॒नि॒ । सः । मा॒ । स्तु॒तः । सु॒व॒र्गमिति॑ सुवः-गम् । लो॒कम् । ग॒म॒यि॒ष्य॒ति॒ । इति॑ । सः । अ॒ग्निम् । अ॒स्तौ॒त् । सः । ए॒न॒म् । स्तु॒तः । सु॒व॒र्गमिति॑ सुवः - गम् । लो॒कम् । अ॒ग॒म॒य॒त् । यः । \textbf{  39} \newline
                  \newline
                                \textbf{ TS 1.5.9.5} \newline
                  ए॒वम् । वि॒द्वान् । अ॒ग्निम् । उ॒प॒तिष्ठ॑त॒ इत्यु॑प - तिष्ठ॑ते । सु॒व॒र्गमिति॑ सुवः - गम् । ए॒व । लो॒कम् । ए॒ति॒ । सर्व᳚म् । आयुः॑ । ए॒ति॒ । अ॒भीति॑ । वै । ए॒षः । अ॒ग्नी इति॑ । एति॑ । रो॒ह॒ति॒ । यः । ए॒नौ॒ । उ॒प॒तिष्ठ॑त॒ इत्यु॑प - तिष्ठ॑ते । यथा᳚ । खलु॑ । वै । श्रेयान्॑ । अ॒भ्यारू॑ढ॒ इत्य॑भि - आरू॑ढः । का॒मय॑ते । तथा᳚ । क॒रो॒ति॒ । नक्त᳚म् । उपेति॑ । ति॒ष्ठ॒ते॒ । न । प्रा॒तः । समिति॑ । हि । नक्त᳚म् । व्र॒तानि॑ । सृ॒ज्यन्ते᳚ । स॒ह । श्रेयान्॑ । च॒ । पापी॑यान् । च॒ । आ॒सा॒ते॒ इति॑ । ज्योतिः॑ । वै । अ॒ग्निः । तमः॑ । रात्रिः॑ । यत् । \textbf{  40} \newline
                  \newline
                                \textbf{ TS 1.5.9.6} \newline
                  नक्त᳚म् । उ॒प॒तिष्ठ॑त॒ इत्यु॑प - तिष्ठ॑ते । ज्योति॑षा । ए॒व । तमः॑ । त॒र॒ति॒ । उ॒प॒स्थेय॒ इत्यु॑प - स्थेयः॑ । अ॒ग्नी(3)ः । न । उ॒प॒स्थेया(3) इत्यु॑प-स्थेया(3)ः । इति॑ । आ॒हुः॒ । म॒नु॒ष्या॑य । इत् । नु । वै । यः । अह॑रह॒रित्यहः॑ - अ॒हः॒ । आ॒हृत्येत्या᳚ - हृत्य॑ । अथ॑ । ए॒न॒म् । याच॑ति । सः । इत् । नु । वै । तम् । उपेति॑ । ऋ॒च्छ॒ति॒ । अथ॑ । कः । दे॒वान् । अह॑रह॒रित्यहः॑-अ॒हः॒ । या॒चि॒ष्य॒ति॒ । इति॑ । तस्मा᳚त् । न । उ॒प॒स्थेय॒ इत्यु॑प - स्थेयः॑ । अथो॒ इति॑ । खलु॑ । आ॒हुः॒ । आ॒शिष॒ इत्या᳚ - शिषे᳚ । वै । कम् । यज॑मानः । य॒ज॒ते॒ । इति॑ । ए॒षा । खलु॑ । वै । \textbf{  41} \newline
                  \newline
                                \textbf{ TS 1.5.9.7} \newline
                  आहि॑ताग्ने॒रित्याहि॑त-अ॒ग्नेः॒ । आ॒शीरित्या᳚ - शीः । यत् । अ॒ग्निम् । उ॒प॒तिष्ठ॑त॒ इत्यु॑प - तिष्ठ॑ते । तस्मा᳚त् । उ॒प॒स्थेय॒ इत्यु॑प - स्थेयः॑ । प्र॒जाप॑ति॒रिति॑ प्र॒जा - प॒तिः॒ । प॒शून् । अ॒सृ॒ज॒त॒ । ते । सृ॒ष्टाः । अ॒हो॒रा॒त्रे इत्य॑हः - रा॒त्रे । प्रेति॑ । अ॒वि॒श॒न्न् । तान् । छन्दो॑भि॒रिति॒ छन्दः॑-भिः॒ । अन्विति॑ । अ॒वि॒न्द॒त् । यत् । छन्दो॑भि॒रिति॒ छन्दः॑-भिः॒ । उ॒प॒तिष्ठ॑त॒ इत्यु॑प - तिष्ठ॑ते । स्वम् । ए॒व । तत् । अन्विति॑ । इ॒च्छ॒ति॒ । न । तत्र॑ । जा॒मि । अ॒स्ति॒ । इति॑ । आ॒हुः॒ । यः । अह॑रह॒रित्यहः॑ - अ॒हः॒ । उ॒प॒तिष्ठ॑त॒ इत्यु॑प - तिष्ठ॑ते । इति॑ । यः । वै । अ॒ग्निम् । प्र॒त्यङ्ङ् । उ॒प॒तिष्ठ॑त॒ इत्यु॑प - तिष्ठ॑ते । प्रतीति॑ । ए॒न॒म् । ओ॒ष॒ति॒ । यः । पराङ्॑ । विष्वङ्ङ्॑ । प्र॒जयेति॑ प्र - जया᳚ । प॒शुभि॒रिति॑ प॒शु - भिः॒ ( ) । ए॒ति॒ । कवा॑तिर्य॒ङ्ङिति॒ कवा᳚ - ति॒र्य॒ङ्ङ् । इ॒व॒ । उपेति॑ । ति॒ष्ठे॒त॒ । न । ए॒न॒म् । प्र॒त्योष॒तीति॑ प्रति - ओष॑ति । न । विष्वङ्ङ्॑ । प्र॒जयेति॑ प्र - जया᳚ । प॒शुभि॒रिति॑ प॒शु - भिः॒ । ए॒ति॒ ॥ \textbf{  42} \newline
                  \newline
                      (सि॒क्तस्य॑-स॒ह-भ॑वति॒-यो-यत्-खलु॒ वै-प॒शुभि॒-स्त्रयो॑दश च)  \textbf{(A9)} \newline \newline
                                \textbf{ TS 1.5.10.1} \newline
                  मम॑ । नाम॑ । प्र॒थ॒मम् । जा॒त॒वे॒द॒ इति॑ जात - वे॒दः॒ । पि॒ता । मा॒ता । च॒ । द॒ध॒तुः॒ । यत् । अग्रे᳚ ॥ तत् । त्वम् । बि॒भृ॒हि॒ । पुनः॑ । एति॑ । मत् । ऐतो॒रित्या - ए॒तोः॒ । तव॑ । अ॒हम् । नाम॑ । बि॒भ॒रा॒णि॒ । अ॒ग्ने॒ ॥ मम॑ । नाम॑ । तव॑ । च॒ । जा॒त॒वे॒द॒ इति॑ जात - वे॒दः॒ । वास॑सी॒ इति॑ । इ॒व॒ । वि॒वसा॑ना॒विति॑ वि-वसा॑नौ । ये इति॑ । चरा॑वः ॥ आयु॑षे । त्वम् । जी॒वसे᳚ । व॒यम् । य॒था॒य॒थमिति॑ यथा-य॒थम् । वि । परीति॑ । द॒धा॒व॒है॒ । पुनः॑ । ते इति॑ ॥ नमः॑ । अ॒ग्नये᳚ । अप्र॑तिविद्धा॒येत्यप्र॑ति - वि॒द्धा॒य॒ । नमः॑ । अना॑धृष्टा॒येत्यना᳚ - धृ॒ष्टा॒य॒ । नमः॑ । स॒म्राज॒ इति॑ सं - राजे᳚ ॥ अषा॑ढः । \textbf{  43} \newline
                  \newline
                                \textbf{ TS 1.5.10.2} \newline
                  अ॒ग्निः । बृ॒हद्व॑या॒ इति॑ बृ॒हत् - व॒याः॒ । वि॒श्व॒जिदिति॑ विश्व - जित् । सह॑न्त्यः । श्रेष्ठः॑ । ग॒न्ध॒र्वः ॥ त्वत्पि॑तार॒ इति॒ त्वत्-पि॒ता॒रः॒ । अ॒ग्ने॒ । दे॒वाः । त्वामा॑हुतय॒ इति॒ त्वाम् - आ॒हु॒त॒यः॒ । त्वद्वि॑वाचना॒ इति॒ त्वत् - वि॒वा॒च॒नाः॒ ॥ समिति॑ । माम् । आयु॑षा । समिति॑ । गौ॒प॒त्येन॑ । सुहि॑त॒ इति॒ सु - हि॒ते॒ । मा॒ । धाः॒ ॥ अ॒यम् । अ॒ग्निः । श्रेष्ठ॑तम॒ इति॒ श्रेष्ठ॑ - त॒मः॒ । अ॒यम् । भग॑वत्तम॒ इति॒ भग॑वत्-त॒मः॒ । अ॒यम् । स॒ह॒स्र॒सात॑म॒ इति॑ सहस्र - सात॑मः ॥ अ॒स्मै । अ॒स्तु॒ । सु॒वीर्य॒मिति॑ सु - वीर्य᳚म् ॥ मनः॑ । ज्योतिः॑ । जु॒ष॒ता॒म् । आज्य᳚म् । विच्छि॑न्न॒मिति॒ वि - छि॒न्न॒म् । य॒ज्ञ्म् । समिति॑ । इ॒मम् । द॒धा॒तु॒ ॥ याः । इ॒ष्टाः । उ॒षसः॑ । नि॒म्रुच॒ इति॑ नि-म्रुचः॑ । च॒ । ताः । समिति॑ । द॒धा॒मि॒ । ह॒विषा᳚ । घृ॒तेन॑ ॥ पय॑स्वतीः । ओष॑धयः । \textbf{  44} \newline
                  \newline
                                \textbf{ TS 1.5.10.3} \newline
                  पय॑स्वत् । वी॒रुधा᳚म् । पयः॑ । अ॒पाम् । पय॑सः । यत् । पयः॑ ॥ तेन॑ । माम् । इ॒न्द्र॒ । समिति॑ । सृ॒ज॒ ॥ अग्ने᳚ । व्र॒त॒प॒त॒ इति॑ व्रत - प॒ते॒ । व्र॒तम् । च॒रि॒ष्या॒मि॒ । तत् । श॒के॒य॒म् । तत् । मे॒ । रा॒द्ध्य॒ता॒म् ॥ अ॒ग्निम् । होता॑रम् । इ॒ह । तम् । हु॒वे॒ । दे॒वान् । य॒ज्ञियान्॑ । इ॒ह । यान् । हवा॑महे ॥ एति॑ । य॒न्तु॒ । दे॒वाः । सु॒म॒न॒स्यमा॑ना॒ इति॑ सु - म॒न॒स्यमा॑नाः । वि॒यन्तु॑ । दे॒वाः । ह॒विषः॑ । मे॒ । अ॒स्य ॥ कः । त्वा॒ । यु॒न॒क्ति॒ । सः । त्वा॒ । यु॒न॒क्तु॒ । यानि॑ । घ॒र्मे । क॒पाला॑नि । उ॒प॒चि॒न्वन्तीत्यु॑प - चि॒न्वन्ति॑ । \textbf{  45} \newline
                  \newline
                                \textbf{ TS 1.5.10.4} \newline
                  वे॒धसः॑ ॥ पू॒ष्णः । तानि॑ । अपीति॑ । व्र॒ते । इ॒न्द्र॒वा॒यू इती᳚न्द्र - वा॒यू । वीति॑ । मु॒ञ्च॒ता॒म् ॥ अभि॑न्नः । घ॒र्मः । जी॒रदा॑नु॒रिति॑ जी॒र - दा॒नुः॒ । यतः॑ । आत्तः॑ । तत् । अ॒ग॒न्न् । पुनः॑ ॥ इ॒द्ध्मः । वेदिः॑ । प॒रि॒ध॒य॒ इति॑ परि - धयः॑ । च॒ । सर्वे᳚ । य॒ज्ञ्स्य॑ । आयुः॑ । अनु॑ । समिति॑ । च॒र॒न्ति॒ ॥ त्रय॑स्त्रिꣳश॒दिति॒ त्रयः॑ - त्रिꣳ॒॒श॒त् । तन्त॑वः । ये । वि॒त॒त्नि॒र इति॑ वि - त॒त्नि॒रे । ये । इ॒मम् । य॒ज्ञ्म् । स्व॒धयेति॑ स्व - धया᳚ । दद॑न्ते । तेषा᳚म् । छि॒न्नम् । प्रतीति॑ । ए॒तत् । द॒धा॒मि॒ । स्वाहा᳚ । घ॒र्मः । दे॒वान् । अपीति॑ । ए॒तु॒ ॥ \textbf{  46} \newline
                  \newline
                      (अषा॑ढ॒-ओष॑धय-उपचि॒न्वन्ति॒-पञ्च॑चत्वारिꣳशच्च)  \textbf{(A10)} \newline \newline
                                \textbf{ TS 1.5.11.1} \newline
                  वै॒श्वा॒न॒रः । नः॒ । ऊ॒त्या । आ । प्रेति॑ । या॒तु॒ । प॒रा॒वत॒ इति॑ परा - वतः॑ ॥ अ॒ग्निः । उ॒क्थेन॑ । वाह॑सा ॥ ऋ॒तावा॑न॒मित्यृ॒त-वा॒न॒म् । वै॒श्वा॒न॒रम् । ऋ॒तस्य॑ । ज्योति॑षः । पति᳚म् ॥ अज॑स्रम् । घ॒र्मम् । ई॒म॒हे॒ ॥ वै॒श्वा॒न॒रस्य॑ । दꣳ॒॒सना᳚भ्यः । बृ॒हत् । अरि॑णात् । एकः॑ । स्व॒प॒स्य॑येति॑ सु - अ॒प॒स्य॑या । क॒विः ॥ उ॒भा । पि॒तरा᳚ । म॒हयन्न्॑ । अ॒जा॒य॒त॒ । अ॒ग्निः । द्यावा॑पृथि॒वी इति॒ द्यावा᳚-पृ॒थि॒वी । भूरि॑रेत॒सेति॒ भूरि॑ - रे॒त॒सा॒ ॥ पृ॒ष्टः । दि॒वि । पृ॒ष्टः । अ॒ग्निः । पृ॒थि॒व्याम् । पृ॒ष्टः । विश्वाः᳚ । ओष॑धीः । एति॑ । वि॒वे॒श॒ ॥ वै॒श्वा॒न॒रः । सह॑सा । पृ॒ष्टः । अ॒ग्निः । सः । नः॒ । दिवा᳚ । सः । \textbf{  47} \newline
                  \newline
                                \textbf{ TS 1.5.11.2} \newline
                  रि॒षः । पा॒तु॒ । नक्त᳚म् ॥ जा॒तः । यत् । अ॒ग्ने॒ । भुव॑ना । व्यख्य॒ इति॑ वि - अख्यः॑ । प॒शुम् । न । गो॒पा इति॑ गो - पाः । इर्यः॑ । परि॒ज्मेति॒ परि॑ - ज्मा॒ ॥ वैश्वा॑नर । ब्रह्म॑णे । वि॒न्द॒ । गा॒तुम् । यू॒यम् । पा॒त॒ । स्व॒स्तिभि॒रिति॑ स्व॒स्ति - भिः॒ । सदा᳚ । नः॒ ॥ त्वम् । अ॒ग्ने॒ । शो॒चिषा᳚ । शोशु॑चानः । एति॑ । रोद॑सी॒ इति॑ । अ॒पृ॒णाः॒ । जाय॑मानः ॥ त्वम् । दे॒वान् । अ॒भिश॑स्ते॒रित्य॒भि - श॒स्तेः॒ । अ॒मु॒ञ्चः॒ । वैश्वा॑नर । जा॒त॒वे॒द॒ इति॑ जात - वे॒दः॒ । म॒हि॒त्वेति॑ महि - त्वा ॥ अ॒स्माक᳚म् । अ॒ग्ने॒ । म॒घव॒थ्स्विति॑ म॒घव॑त् - सु॒ । धा॒र॒य॒ । अना॑मि । क्ष॒त्रम् । अ॒जर᳚म् । सु॒वीर्य॒मिति॑ सु - वीर्य᳚म् ॥ व॒यम् । ज॒ये॒म॒ । श॒तिन᳚म् । स॒ह॒स्रिण᳚म् । वैश्वा॑नर । \textbf{  48} \newline
                  \newline
                                \textbf{ TS 1.5.11.3} \newline
                  वाज᳚म् । अ॒ग्ने॒ । तव॑ । ऊ॒तिभि॒रित्यू॒ति - भिः॒ ॥ वै॒श्वा॒न॒रस्य॑ । सु॒म॒ताविति॑ सु - म॒तौ । स्या॒म॒ । राजा᳚ । हिक᳚म् । भुव॑नानाम् । अ॒भि॒श्रीरित्य॑भि - श्रीः ॥ इ॒तः । जा॒तः । विश्व᳚म् । इ॒दम् । वीति॑ । च॒ष्टे॒ । वै॒श्वा॒न॒रः । य॒त॒ते॒ । सूर्ये॑ण ॥ अवेति॑ । ते॒ । हेडः॑ । व॒रु॒ण॒ । नमो॑भि॒रिति॒ नमः॑ - भिः॒ । अवेति॑ । य॒ज्ञेभिः॑ । ई॒म॒हे॒ । ह॒विर्भि॒रिति॑ ह॒विः - भिः॒ ॥ क्षयन्न॑ । अ॒स्मभ्य॒मित्य॒स्म - भ्य॒म् । अ॒सु॒र॒ । प्र॒चे॒त॒ इति॑ प्र - चे॒तः॒ । राजन्न्॑ । ए॒नाꣳ॑सि । शि॒श्र॒थः॒ । कृ॒तानि॑ ॥ उदिति॑ । उ॒त्त॒ममित्यु॑त् - त॒मम् । व॒रु॒ण॒ । पाश᳚म् । अ॒स्मत् । अवेति॑ । अ॒ध॒मम् । वीति॑ । म॒द्ध्य॒मम् । श्र॒था॒य॒ ॥ अथ॑ । व॒यम् । आ॒दि॒त्य॒ । \textbf{  49} \newline
                  \newline
                                \textbf{ TS 1.5.11.4} \newline
                  व्र॒ते । तव॑ । अना॑गसः । अदि॑तये । स्या॒म॒ ॥ द॒धि॒क्राव्‌ण्ण॒ इति॑ दधि - क्राव्‌ण्णः॑ । अ॒का॒रि॒ष॒म् । जि॒ष्णोः । अश्व॑स्य । वा॒जिनः॑ ॥ सु॒र॒भि । नः॒ । मुखा᳚ । क॒र॒त् । प्रेति॑ । नः॒ । आयूꣳ॑षि । ता॒रि॒ष॒त् ॥ एति॑ । द॒धि॒क्रा इति॑ दधि - क्राः । शव॑सा । पञ्च॑ । कृ॒ष्टीः । सूर्य॑ । इ॒व॒ । ज्योति॑षा । अ॒पः । त॒ता॒न॒ ॥ स॒ह॒स्र॒सा इति॑ स॒ह॒स्र - साः । श॒त॒सा इति॑ शत - साः । वा॒जी । अर्वा᳚ । पृ॒णक्तु॑ । मद्ध्वा᳚ । समिति॑ । इ॒मा । वचाꣳ॑सि ॥ अ॒ग्निः । मू॒र्धा । भुवः॑ ॥ मरु॑तः । यत् । ह॒ । वः॒ । दि॒वः । सु॒म्ना॒यन्त॒ इति॑ सुम्न - यन्तः॑ । हवा॑महे ॥ एति॑ । तु । नः॒ । \textbf{  50} \newline
                  \newline
                                \textbf{ TS 1.5.11.5} \newline
                  उपेति॑ । ग॒न्त॒न॒ ॥ या । वः॒ । शर्म॑ । श॒श॒मा॒नाय॑ । सन्ति॑ । त्रि॒धातू॒नीति॑ त्रि - धातू॑नि । दा॒शुषे᳚ । य॒च्छ॒त॒ । अधि॑ ॥ अ॒स्मभ्य॒मित्य॒स्म-भ्य॒म् । तानि॑ । म॒रु॒तः॒ । वीति॑ । य॒न्त॒ । र॒यिम् । नः॒ । ध॒त्त॒ । वृ॒ष॒णः॒ । सु॒वीर॒मिति॑ सु-वीर᳚म् ॥ अदि॑तिः । नः॒ । उ॒रु॒ष्य॒तु॒ । अदि॑तिः । शर्म॑ । य॒च्छ॒तु॒ ॥ अदि॑तिः । पा॒तु॒ । अꣳह॑सः ॥ म॒हीम् । उ॒ । स्विति॑ । मा॒तर᳚म् । सु॒व्र॒ताना॒मिति॑ सु - व्र॒ताना᳚म् । ऋ॒तस्य॑ । पत्नी᳚म् । अव॑से । हु॒वे॒म॒ ॥ तु॒वि॒क्ष॒त्रामिति॑ तुवि - क्ष॒त्राम् । अ॒जर॑न्तीम् । उ॒रू॒चीम् । सु॒शर्मा॑ण॒मिति॑ सु - शर्मा॑णम् । अदि॑तिम् । सु॒प्रणी॑ति॒मिति॑ सु - प्रणी॑तिम् ॥ सु॒त्रामा॑ण॒मिति॑ सु - त्रामा॑णम् । पृ॒थि॒वीम् । द्याम् । अ॒ने॒हस᳚म् । सु॒शर्मा॑ण॒मिति॑ सु - शर्मा॑णम् ( ) । अदि॑तिम् । सु॒प्रणी॑ति॒मिति॑ सु - प्रणी॑तिम् ॥ दैवी᳚म् । नाव᳚म् । स्व॒रि॒त्रामिति॑ सु - अ॒रि॒त्राम् । अना॑गसम् । अस्र॑वन्तीम् । एति॑ । रु॒हे॒म॒ । स्व॒स्तये᳚ ॥ इ॒माम् । स्विति॑ । नाव᳚म् । एति॑ । अ॒रु॒ह॒म् । श॒तारि॑त्रा॒मिति॑ श॒त - अ॒रि॒त्रा॒म् । श॒तस्फ्या॒मिति॑ श॒त - स्फ्या॒म् ॥ अच्छि॑द्राम् । पा॒र॒यि॒ष्णुम् ॥ \textbf{  51} \newline
                  \newline
                      (दिवा॒ स-स॑ह॒स्रिणं॒ ॅवैश्वा॑नरा-दित्य॒- तू नो ॑- ऽने॒हसꣳ॑ सु॒शर्मा॑ण॒-मेका॒न्नविꣳ॑श॒तिश्च॑ )  \textbf{(A11)} \newline \newline
\textbf{praSna korvai with starting padams of 1 to 11 anuvAkams :-} \newline
(दे॒वा॒सु॒राः-परा॒-भूमि॒र्-भूमि॑-रुपप्र॒यन्तः॒-सं प॑श्या॒-म्यय॑ज्ञ्ः॒- सं प॑श्या - म्यग्निहो॒त्रं - मम॒ नाम॑-वैश्वान॒र-एका॑दश । ) \newline

\textbf{korvai with starting padams of1, 11, 21 series of pa~jcAtis :-} \newline
(दे॒वा॒सु॒राः-क्रु॒द्धः-सं प॑श्यामि॒-सं प॑श्यामि॒-नक्त॒-मुप॑गन्त॒-नैक॑पञ्चा॒शत् । ) \newline

\textbf{first and last padam of fifth praSnam :-} \newline
(दे॒वा॒सु॒राः-पा॑रयि॒ष्णुं ।) \newline 


॥ हरिः॑ ॐ ॥॥ कृष्ण यजुर्वेदीय तैत्तिरीय संहितायां प्रथमकाण्डे पञ्चमः प्रश्नः समाप्तः ॥ \newline
\pagebreak
1.5.1   Annexure for 1.5\\1.5.11.4 अ॒ग्निर् मू॒र्द्धा>1\\अ॒ग्निर्मू॒र्द्धा दि॒वः क॒कुत् पतिः॑ पृथि॒व्या अ॒यं ।\\अ॒पाꣳ रेताꣳ॑सि जिन्वति । (ट्श् 4-4-4-1)\\\\1.5.11.4 भुवः॑>2\\भुवो॑ य॒ज्ञ्स्य॒ रज॑सश्च ने॒ता यत्रा॑ नि॒युद्भिः॒ \\सच॑से शि॒वाभिः॑ ।\\दि॒वि मू॒र्द्धानं॑ दधिषे सुव॒र्षा जि॒ह्वाम॑ग्ने चकृषे \\हव्य॒वाहं᳚ । (ट्श् 4-4-4-4)\\\\====================================\\
\pagebreak
        


\end{document}
